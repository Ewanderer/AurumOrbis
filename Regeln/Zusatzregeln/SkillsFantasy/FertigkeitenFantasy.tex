\documentclass[a4paper,12pt,oneside]{book}
\usepackage[ngerman]{babel}
\usepackage[utf8]{inputenc}
\usepackage{imakeidx}
\usepackage[hypertexnames=false]{hyperref}
\usepackage[all]{hypcap}
\usepackage{nameref}
\usepackage{ulem}


\hypersetup{
	bookmarks=true,
    colorlinks,
    citecolor=black,
    filecolor=black,
    linkcolor=black,
    urlcolor=black
}

\author{Jordan Eichner}
\title{Fertigkeiten zu Arum Orbis}
\date{}

\begin{document}

\maketitle
\tableofcontents
\section{Übersicht von Fertigkeiten}
\subsection{Waffen}
\begin{description}
\item[Bogen]- STR, BEW, WIS
\item[Schwert]- STR, MUT, BEW
\item[Dolch]- STR, MUT, BEW
\item[Faustkampf]- STR, MUT, BEW
\item[kleine Wurfwaffen]- FF, STR, WIS
\\Nichts kann einen direkten Zweikampf so gut beenden, wie eine kleiner Wurfdolch aus dem Unterarm im Herzen des Feindes. Natürlich sind diese kleinen Projektile auch außerhalb von Kämpfen, sehr nützlich, da sie dennoch eine gewisse Durchschlagskraft besitzen, leise sind und ggf. es auch eine Weile dauert sie im Körper zu finden.
\item[Wurfwaffen]- BEW, STR, WIS
\\Speere
\item[Schleudern]- STR, BEW, WIS
\\Auch in Zeiten von Schusswaffen und Zaubern sollte man nicht die Kraft der herkömmlichen Schleuder unterschätzten. In diese Kategorie fallen auch Waffen, wie Bolas und Wurfnetze.
\item[Hiebwaffen]
\item[Zweihändige Schwerter]- STR, BEW, MUT
\item[Zweihändige Hiebwaffen]
\item[Improvisierte Waffen] - STR, BEW, WIS
\\Wer gewillt ist zu Kämpfen, kann nahezu alles als Waffe benutzen, doch es erfordert doch ein gewisses Talent, um sog. Improvisierte Waffen auch gegen echte Waffen zu verwenden oder überhaupt alle Möglichkeiten für Waffen zu erkennen.
\item[Schilde] - STR, KON, BEW
\\Auch wenn Schilde für sich genommen keine Waffen sind, ist ihre effektive Anwendung doch eine Kunst für sich. Nicht nur gewährt ein Schild, die Möglichkeit eingehende Fernkampfangriffe abzuwehren. Natürlich lassen sich auch mit dem Schild bestimmte Kampftalente anwenden.
\item[Großwaffenkunde] – STR, INT, WIS
\\Hierunter fallen sowohl die Bedienung von schwerer Artillerie, sowie die Abschätzung der Flugbahn, damit die Projektile auch ihren Weg zum Ziel finden.
\item[Kristallprügel] - STR, BEW, WIS
\\Auch wenn Beschwörer mehr für ihre Magischen Fähigkeiten berühmt sind, erhält jeder von ihnen eine Ausbildung im Kampf, wobei sie ihren Kristall als Waffe benutzen. Er er eignet sich zwar nicht als Schild, kann aber dafür auf der anderen Seite je nach Wunsch des Beschwörers mit Zusatzeffekten, wie Entwaffnen benutzt werden. 
\end{description}
\subsection{Soziales}
\begin{description}
\item[Überzeugen]- CHA,WIS,INT
\\Die Rhetorik ist die Hohe Kunst Argumente in schöne Worte zu kleiden, um langfristig die Einstellung seines Gegenübers hin zu seinem eigenem Standpunkt zu bewegen. Der Erfolg hängt einerseits von der Attitüde des Gegenübers ab. Befindet man sich in einer Diskussionsrunde können die Gegenüber ihre eigene Fertigkeit einbringen, da sie die Argumente entkräften. Wird diese Fähigkeiten in großen Menschenmengen benutzt wird die Probe zusätzlich entsprechend der Größe erschwert. Anbei ein paar Tabellen zu den Modifikatoren:
%hier tabelle
Die Wirkung von Überzeugung hält in der Regel mehrere Tage, wobei angesichts veränderte Umstände die Wirkung auch spontan verfliegen kann. Übrig gebliebene Punkte entscheiden daher über den Grad der Überzeugung:
%Hier tabelle
Für jede Komplexitätsklasse über dem Stand des Ziels erhöht die Wirkung um eine Kategorie.
\item[Überreden]- CHA,INT,WIS
\item[Einschüchtern]- CHA, MUT, STR
\item[Lügen]- CHA, INT, MUT
\item[Verspotten]- CHA, INT, MUT
\\Jedes Kind lernt irgendwann eine Reihe von Beschimpfungen, um ihre Abneigung gegenüber anderen auszudrücken, doch das Geheimnis des Verspotten ist, dass Opfer wirklich zu treffen. Außerhalb von Kämpfen kann hiermit der Ruf einer Figur kurzzeitig zerstört werden(Was die eigenen sozialen Fähigkeiten senkt). Im Kampf sorgt ein erfolgreicher Spott, dafür, dass das Ziel ohne Rücksicht auf die eigene Defensive nach dem Ziel ihrer Wut schlagen. 
\item[Menschenkenntnis]- WIS, INT, CHA
\item[Etikette]- INT, MUT, FF
\item[Gassenwissen]- CHA, WIS, INT
\item[Tanzen]- CHA, CHA, BEW 
\item[Singen]- CHA, CHA, WIS
\item[Lyrik]- CHA, WIS, CHA
\item[Musizieren] - WIS, WIS, INT 
\item[Schauspielen]- CHA, WIS, MUT
\item[Verkleiden]- WIS, MUT, INT
\item[Gaukeln]- MUT, BEW, FF
\item[Zechen]- CHA, KON, MTB
\end{description}

\subsection{großes Handwerk}
\begin{description}
\item[Bootsbauer]-WIS, INT, STR
\item[Büchsenmacher]- FF, FF, INT
\item[Grobschmied]- STR, KON, WIS 
\item[Feinschmied]- STR, FF, WIS
\item[Balieren]Edelsteinschleifer
\item[Schreiner]-FF, STR, WIS
\item[Baukunst]-
\\Umsetzten von Plänen
\item[Schlösser]- FF, FF, INT
\\Knacken und Machen
\item[Feinmechanik]- FF, FF, INT
\\Wenn eine Apparatur nur aus genügend großen Teilen besteht, ist eigentlich jeder in der Lage ihn zusammenzufügen. Doch wenn man über Gerätschaften, wie eine Taschenuhr spricht, erfordert sowohl die Planung, die Hand eines Könners. Natürlich kann mit dieser Fertigkeit auch die Funktionsweise eines fremden Mechanismus ergründen, ohne dabei alles zu zerstören.
\item[Alchemie]- FF, INT, WIS
\\Auführen von Rezepten
\item[Lederbearbeitung]- FF, KON, WIS
\end{description}
\subsection{kleines Handwerk}
\begin{description}
\item[Fleischer]
\item[Kürschner]
\item[Weben und Spinnen]
\item[Holzbearbeitung]
\item[Fallen stellen]- FF, FF, INT
\item[Malen/Zeichnen]-WIS, FF, INT
\item[Metallurgie]-WIS, KON, FF
\end{description}


\subsection{Wissen}
Mithilfe von Wissensfertigkeiten können Merkmale von Feinden aufgedeckt werden. Dazu muss in der Regel eine Probe erschwert um die Gefahrenstufe selbigen gemacht werden. Für je 3 übrig gebliebene Punkte wird ein Merkmal aufgedeckt. Diese Erkennen-Probe kann pro Begegnung mit diesem Feind einmal eingesetzt werden und kostet eine Runde.
\begin{description}
\item[Flugdynamik]-STR, INT, WIS
\item[Orientieren]- WIS, INT, CHA
\item[Architektur]-
\\Planen von Gebäuden
\item[Tierkunde]- WIS, WIS, CHA
\\Neben den Völkern und Monstern gibt es noch eine ganze Reihe von animalischen Kreaturen. Über die Pflege von domestizierten Varianten, Verbreitung und Eigenheiten der Wilden Varianten gibt diese Fähigkeit Auskunft. Jeder Wurf wird durch die Gefahrenstufe des Tiers erschwert. Bei Pflege muss um gesundheitliche Probleme zu behandeln ein weitere Wurf auf Heilkunde mit der Erschwernis der Gefahrenstufe, abzüglich der übrigen Punkte aus der Tierkundeprobe.
\item[Alchemistische Ordnung]-INT, INT, WIS
\\Allen Objekten, ob Blut, Minerale, Pflanzen, verfügt laut den Alchemisten über besondere Eigenschaften. Diese zu erkennen, zu extrahieren und mit anderen zu kombinieren sind drei verschiedene Proben, die im letzten Schritt bei Fehlschlägen zur Vernichtung ganzer Stadtviertel führen können. Auch fertige Kreationen dieser Schule können mit dieser Fähigkeiten erkannt werden. Die Probe ist dabei in der Regel 3/4 der zur Herstellung benötigten Probe. In der Regel muss der Anwender auch Zugriff auf ein Labor haben, ansonsten muss er je nach Rezeptur mit heftigen Erschwernissen oder dem garantieren Scheitern rechnen. Für mehr Details siehe hierzu das Kapitel Alchemie.
\item[Kräuterkunde]-INT, INT, WIS
\\Vielen Pflanzen in ganz Aurum Orbis haben von Natur aus Heilkräfte oder schädliche Wirkungen. Für die Erkennung und fachgerechte Verarbeitung werden zwei Proben auf diese Fähigkeit benötigt. Fehlschläge können dabei je nach schwere bis hin zu tödlichen Verwechselungen oder Selbstvergiftung führen. Für eine fachgerechte Verarbeitung muss in der Regel Zugriff auf eine Küche und entsprechendes Werkzeug bestehen, bei echten Laboren wird die Probe zum Verarbeiten um bis zu 5 Punkte erleichtert. 
\item[Götter und Aspekte]- INT, WIS, WIS
\item[Arkanes Wissen]Die Arkanen Künste sind nicht nur auf das Auswendiglernen von Zaubersprüchen beschränkt. Sie alle beruhen letztendlich auf einer Sprache mit eigenem Vokabular und Grammatik. Wer dieser mächtig ist, kann seine Zauber leicht anpassen, um sie den eigenen Wünschen anzupassen. Diese erhöhen natürlich die Komplexität eines Zaubers und erfordern eine Probe auf arkanes Wissens, wobei übrig gebliebende Punkte die anschließende Zauberprobe bis zur ursprünglichen Komplexität erleichtern. Anbei ein paar Möglichkeiten:
Effizientes Zaubern-Die Mana-Kosten für den Zauber werden bis zu 25\% reduziert. Komplexität steigt um die gewählte Prozentzahl.
Zusätzlich kann ein Anwender dieser Fertigkeit ihm nicht bekannte Zaubersprüche erkennen. Hierzu muss eine Probe in Höhe der dem Zauber entsprechendem Komplexität bestanden werden. Magier können nachdem sie diesen erkannt haben, mit einem Gegenzauber abblocken, wobei die Zauberprobe um die übrig gebliebenen Punkte aus der Erkennenprobe vereinfacht wird.
\item[Sagen und Legenden]Schon immer vermengten die Barden und Geschichtenerzähler die Wahrheit mit Elementen ihrer eigenen Fiktion. So ist zwar nicht immer auf solche Erzählungen verlass und doch kann es, wenn die Bücher des Gelehrten versagen, die Worte einer Ballade sein, die über den Kampf gegen uralte Übel und exotische Gefahren entscheiden kann. Mit dieser Fähigkeiten kann das Fehlen andere Wissensfertigkeiten zum Teil ausgeglichen werden, auch wenn der Meister die Informationen in geeigneten Rätseln verpacken sollte.
\item[Adel und Persönlichkeiten]- INT, INT, CHA
\item[Historie \& Geographie]- INT, INT, WIS
\item[Anatomie]
\item[Rechnen]- INT, INT, WIS
\item[Rechtskunde]- INT, WIS, CHA
\item[Schätzen]- INT, INT, WIS
\item[Staatskunde]- INT, WIS, CHA
\item[Kriegskunst]- INT, MUT, WIS
\item[Kryptographie]- INT, INT, WIS
\item[Mechanik]-INT, INT, WIS
\\Verstädnis von Mechanischen Vorgängen
\item[Kristallkunde]- FF, INT, MUT
\end{description}
\subsection{Sonstige}
\begin{description}
\item[Schleichen]- BEW, BEW, WIS; Frei
\\Wann immer eine Figur versucht unbemerkt zu bleiben, während sie sich bewegt, dann ist Schleichen gefragt. Im Kampf nimm ihre Bewegungsgeschwindigkeit auf 1/4 reduziert und sprinten unzulässig. Angriffe mit Nahkampfwaffen heben Schleichen auf.
\item[Verstecken]- WIS, MUT, BEW;Stillstand, Angriffe aus dem Verborgenem. Waffen, die keinen Sound abgeben la
\item[Wahrnehmung]- INT, WIS, WIS
\item[Athletik]- STR, KON, BEW
\\Rennen, Springen 
\item[Akrobatik]- BEW, BEW, STR
\\Ausweichen, Balancieren, Manöver
\item[Schwimmen]- STR, KON, STR
\\Schwimmen, Tauchen
\item[Klettern]- STR, BEW, MUT
\\Klettern
\item[Reiten]- BEW, WIS, MUT
\item[Überleben]- WIS, FF, BEW
\item[Werfen]

\end{description}

\end{document}