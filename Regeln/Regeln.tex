\documentclass[a4paper,12pt,oneside]{book}
\usepackage[ngerman]{babel}
\usepackage[utf8]{inputenc}
\usepackage{imakeidx}
\usepackage[hypertexnames=false]{hyperref}
\usepackage[all]{hypcap}
\usepackage{nameref}
\usepackage{ulem}


\hypersetup{
	bookmarks=true,
    colorlinks,
    citecolor=black,
    filecolor=black,
    linkcolor=black,
    urlcolor=black
}

\author{Jordan Eichner}
\title{Regelwerk zu Arum Orbis}
\date{}

\begin{document}

\maketitle
\part*{Vorwort}
Das nun folgende Regelwerke ist eine von verschiedenen Regelwerken, wie DSA, Opus Anima und Pathfinder, inspirierte Version eines Rollenspiels für die Anwendung in Echtzeit-Action-Szenarien, also vor allem Computerspielen. Neben den Regel wird dieses Werk auch einige Konzepte der auf diesen Regel basierenden Programmumsetzung enthalten. Natürlich sind alle Interessierten eingeladen, die Regeln für eine Verwendung als klassisches P\&P anzupassen.
Jordan Eichner

\tableofcontents
  
\part{Grundlagen}
\chapter{Ein paar Abkürzugen}
W6=6 Seitiger Würfel

\chapter{Strukturierung der Welt}
Die Spielwert setzt sich aus zwei Teilen zusammen: der Statischen Umgebung, wozu vor allem das Terrain zählt und den sogenannten RPG-Objekten. Letztere kann man schließlich noch in folgende Kategorien einteilen:
\begin{description}
\item[Kreaturen:]
Hierzu gehören alle Lebewesen, seien es Spielerfiguren oder angreifende Monster.
\item[Interaktive Objekte:]
Neben Gegenständen, zählen auch alle Objekte mit denen der Spieler interagieren kann.
\end{description}
\chapter{Komplexitäts-Klassen}
Aurum Orbis wird von Kreaturen aller Art bewohnt, angefangen bei einem einfachem Bauer, bis zu einem Drachen. Natürlich sind ihre Möglichkeiten dabei so verschieden, dass man für einen besseren Vergleich auf ein Stufensystem zurückgreifen muss. Alle Aktionen, Kampfstile und Figuren lassen sich daher in sog. Komplexitätsklassen einordnen:
\begin{description}
\item[Mundan:] Hier lässt sich jeder von Tagelöhner bis zum frisch Gebackenen Abenteurer, sowie viele Wildtiere einordnen. Alle Kampfstile auf dem Niveau von einer Bürgermiliz befinden sich auf diesem Niveau, genauso wie die Herstellung eines Großteil von Alltagsgegenständen und alle mit einfachen Professionen verbundenen Tätigkeiten.
\item[Heroisch:] Die kleineren magischen Monster und die größeren Kreaturen, sowie Abenteurer mit jahrelanger Erfahrung und Angehörige von weiterentwickelten Professionen, wie dem Bestienjäger, gehören in diese Kreaturen. Einige wenige Alltagsgegenstände, sowie verbesserte Ausrüstung und einfache exotischen Waffen kommen aus dieser Kategorie, Kampfstile die im Militär gelehrt werden sind diese Kategorie.
\item[Episch:]Die meisten magischen Monster, sowie einfache Diener von Göttern und wirkliche Plagen aus der Gegenwelt, sowie wirkliche Berühmtheiten können sich zu dieser Kategorie zählen. Aufwendige Konstruktionen und Rezepturen und verbesserte einfache Exotische Waffen, sowie seltene Exotische Waffen und die Techniken von ...
\item[Legendär:]
\item[Ultimativ:]Götter und die ältesten der Drachen, sowie die Herrscher über Teile der Gegenwelt spielen in dieser Liga, also eigentlich nur Unsterbliche. Ein Sterblicher vermag sich zwar daran versuchen etwas vom Niveau dieser Giganten zu erschaffen, doch selbst einem Meisterlichem Handwerker hat für das einfachster aus dieser Kategorie nur eine 5\% Erfolgschance und häufig fatalen Folgen bei einem Fehlschlag.
\end{description}
\section{Einstufungen von Proben}
Die Einstufung einer Probe erfolgt entweder durch statische Einteilungen, wie bei Handwerkrezepten, oder durch die persönlichen Fertigkeiten der ausführenden Figur. Bei konkurrierenden Proben wird dabei die Kategorie durch den Herausforderer bestimmt. Ansonsten gilt die Regel, dass die Kategorie in der Regel dem Wunsch des Spielers entsprechen kann. Diese Fertigkeiten werden als Frei markiert.
\section{Mathematik zu Schranken für Spielerfortschritt}
Um ein geeignetes Balanceing zu erzielen hier ein paar Gedanken.
\begin{itemize}
\item Wir entwickeln zunächst ein sauberes System für alle nicht Kampf-Bezogenen Elemente und passen die Formel zu Berechnung von Kampf-Werten an das bestehende System an.
\item Die Anzahl aller nicht Waffenfertigkeiten sei: $S$. 
\item Wir gehen davon aus, dass jeder Spieler im durchschnitt Meisterschaft in einer bestimmen Anzahl an Fertigkeiten(keine Waffen) haben: \(S_{3}\)
\item Zusätzlich gibt es noch zwei weitere Arten von Fertigkeiten die jeweils grundlegend(1) und durchschnittlich(2):$S_{1}, S{2}$. Von allen anderen Fertigkeiten gehen wir davon aus, dass diese ungelernt sind und Ordnen sie $S_{0}$ zu.
\item Eine Figur ist zusätzlich noch bis zu einem gewissen Grad optimiert, also $S_{1..3}$ bauen auf den gleichen Attributen auf. Der Grad dieser Einigkeit sei: $O$
\item Wir arbeiten nun nacheinander die Komplexitätsklasse $K_{1..5}$ ab.
\item Zu Beginn unserer Untersuchung gehen wir davon aus, dass alle Fertigkeiten aus $S_{3}$ auf 75\% des untersuchten $K$ liegt. ($K_{1}$ =20 * 0.75 => 15) Für $S_2$ bei 50 und für $S_1$ bei 25\% liegt. $S_{0}$ bleiben immer bei 0. Diese drei Grenzen nennen wir $R_{1...3}$.
\item In der für $K$ erstellen wir nun für jede Gruppe von Fertigkeiten $S_{0..3}$ jeweils die gewünschte Erfolgswahrscheinlichkeit für das äußerste $Q_{komplex}$ und das einfachste $Q_{simpel}$ für den Fall, dass $O>80\%$ und für den Fall, dass $O<20\%$ ist.
\item Nun Suchen wir nach einer geeigneten Verteilung der 10 Attribute damit wir für das untersuchte K, die Wahrscheinlichkeiten erfüllen. Nun isolieren wir die Attribute in eine geeignete Menge($A_{c}=3$) an Gruppen($A_{1..c}$), in denen jeweils die am dichtesten Attribute liegen und bestimmen für den Mittelwert dieser Gruppen$A_{m, 1..c}$.
\item Nun erstellen wir aus den beiden Profilen $A_{m}$ zu O den Mittelwert, dieser entspricht dann geeigneten Schranken für die Attribute: $A_b,1...c$.
\item Nachdem wir alle Komplexitätsklassen auf diese Weise abgearbeitet haben, untersuchen wir, ob die berechneten Attributsverteilungen im Gesamtbild stimmig sind. Dazu gehört, dass mit einem Profil aus der niedriegeren Komplexitätsklasse, selbst für Proben zu Fertigkeiten aus $S_3$ aus $Q_{simpel,1}$ nur eine Wahrscheinlichkeit von 25\% besteht.
\item Falls dies nicht der Fall ist ändern wir ggf. $R_{1..3}, A_{c}$ und beginnen erneut mit der Berechnung für alle Komplexitätsklassen.
\item Sind wir mit dem $A_b$ für alle $K$ zufrieden, beginnen wir nun mit der Optimierung für Kampfsituationen.
\item Alle Kampffertigkeiten seien 
\end{itemize}

\chapter{Proben}
\section{Übersicht}
Wann immer der Ausgang einer Aktion, sei es von einem Spieler oder einer Kreatur entscheidet eine sogenannte Probe über diese. In der Regel besteht eine Probe aus einer Reihe von Würfen mit 100-seitigen Würfeln. Anbei eine Reihe über die Abläufe verschiedener Proben. 
\section{Arten von Proben}
\begin{description}
\item[Attributsprobe:]Auch wenn man nur in Ausnahmefällen auf ein Attribut würfelt(In der Regel, wenn keine passende Fertigkeit zur Verfügung steht), ist dieser Wurf die Grundlage für Fertigkeitsproben. Mit einem W100, muss man einen Wert kleiner gleich dem gefordertem Attributswert erzielen. Dabei kann je nach Situation eine sog. Erschwernis(die auf den Würfelwurf addiert wird) oder Erleichterung(die vom Würfelwurf subtrahiert wird) erteilt werden. 
\item[Fertigkeitsprobe:]
Sie ist eigentlich nur eine erweiterte Attributsprobe, nur dass man auf die drei, mit der Fertigkeit assoziierten Attributen, eine Probe machen muss. Dabei dienen die Fertigkeitspunkte als zusätzliche Erleichterung, mit der man Ausgleichen kann. Wie bei der Attributsprobe gibt es Erschwernis und Erleichterung, wenn die Erschwernis die Fertigkeitspunkte übersteigt, werden die übrigen Punkte als Erschwernis bei allen drei Würfen eingerechnet werden muss.
\item[Offene Probe:]
In der Regel benötigt man nur Auskunft darüber, ob der Probe bestanden wurde, doch in bestimmten Fällen möchte man über die Qualität des Erfolges, bzw. der Niederlage Bescheid wissen(z.B. beim Handwerk). In diesem Fall würfelt man ganz normal seine Probe betrachtet anschließend die übrig gebliebenen, bzw. fehlenden Punkte. Man kann dabei nie mehr Punkte übrig behalten, wie man ohne Situationsabhängige Erleichterung zur Verfügung hatte(im Fall einer klassischen Fertigkeitsprobe niemals mehr als man Punkte in der Fertigkeit hat).
\item[konkurrierende Probe:]
In bestimmten Situationen, versucht man nicht einfach eine Aufgabe zu erledigen, sondern sie besser als jemand anders zu machen. In einem solchen Fall kommt es zu einer konkurrierenden Probe. Dabei macht der Herausforderer zunächst eine offene Probe, anschließend der Herausgeforderte mit einer Erschwernis, bzw. bei gescheiterte Probe Erleichterung in Höhe der übrig gebliebenen, bzw. fehlenden Punkte. Je nachdem wie beide Proben ausgegangen sind und wie viele Punkte der Herausgeforderte, hat die Probe unterschiedliche Ausgänge. 
\end{description}
\section{Besonderheiten}
Bei jeder Probe kommen der 1 und der [20,40,60,80,100] eine Besonderheit zu:
\begin{description}
\item[Alles Einser:]
In diesem Fall spricht man von einem kritischen Erfolg, welcher nicht nur einen automatischen Erfolg der Probe sondern auch noch einen zusätzlichen positiven Seiteneffekt hat. Dazu kann zum Beispiel eine erhöhte Ausführungsgeschwindigkeit der mit der Probe verbunden Tätigkeit gehören.
\item[Mehr als die Hälfte eine Eins:]
Dies zählt als automatischer Erfolg, unabhängig davon ob man aufgrund von Erschwernis die Probe eigentlich nicht erfolgreich abschließen konnte.
\item[Mehr als die Hälfte eine Maximalwert]
Es handelt sich um einen Patzer, einem automatischen Misserfolg, egal wie viele Punkte man zum Ausgleich besitzt.
\item[Alles Maximalwerte]
Das Glück ist nicht mit einem. In diesem Fall spricht man von einem kritischem Fehlschlag, welcher neben dem automatischen Misserfolg, eine negative Wirkung mit sich bringt, wie Selbstverletzung.
\end{description}

\chapter{Sagenpunkte}
Jede Handlung hat Konsequenzen, manchmal sogar für den Rest der Welt. Sagenpunkte sind eine Mischung aus Eben diesem Einfluss, seinem Ruf und der generellen Lebenserfahrung einer Kreatur. Die Möglichkeiten an denen man sie sammelt und an denen man sie Ausgeben kann sind so groß, dass nun zwei eigene Abschnitte folgen:
\section{Verdienstmöglichkeiten}
\begin{itemize}
\item Durch das Erreichen von einfachen Erfolgen beim Kampf gegen Monster, angefangen bei den ersten Hundert erschlagenen Dunkelelfen, über eine Horde magischer Bestien, bis hin zum Erschlagen eines feindseligen Jungdrachens. Allgemein kann man sagen, dass man 1/100 der SP des erschlagenen Feindes erhält.
\item Beim Abschließen von Geschichtssträngen, in Abhängigkeit der eigenen Beteiligung(Meisterentscheid).
\item Durch die Teilnahme an einem historischem Ereignis, wie einem Krieg oder einer Seuche, auch hier abhänig von der Beteiligung(Meisterentscheid).
\item Beim Erkunden und Ausräumen von Horten.
\item Als Handwerker, einmal für die Fertigung, wobei natürliche Artefakte wesentlich lokrativer sind und außerdem rückwirkend, wenn der Gegenstand zum Einsatz kommt. Wenn also das Schwert benutzt wird, um einen Drachen zu erschlagen, so wird in Abhängigkeit, wie weit der Weg des Schwertes vom Amboss zum Helden war(Anzahl der Weiterverkäufe), eine gewisse Menge an SP an den Schmied ausgeschüttet. In der Regel erhält man 1/500 der SP für den erschlagenden Feind, geteilt durch die Anzahl der Weiterverkäufe, nach Abgabe vom Schmied an den ersten Besitzer.
\item Die Kreation eines Epos zu einem historischen Moment aus den Beschreibungen eines Augenzeugen. Auch hier gibt es eine Rückwirkende SP-Gewinnung. (Diese Möglichkeit wird zunächst nicht im Spiel integriert sein und ist für eine spätere Entwicklung evt. geplant)

\end{itemize}
\section{Ausgabemöglichkeiten}
\begin{itemize}
\item Einen Titel erkaufen oder zumindest die Möglichkeit freischalten, diesen zu erfahren. In der Regel besteht diese Möglichkeit in einer Aufgabe, bei der sich die Figur beweisen muss.
\item In der Regel kann man jede allgemein verfügbare Spezialisierung durch das bezahlen eines Lehrmeisters erlernen. Doch um z.B. bei den Lehrmeistern einer Geheimorganisation sich die Grundzüge anzueignen, muss man erst einmal über diese stoßen, ähnlich wie bei den zuvor genannten Titel, wird also eine besondere Aufgabe freigeschaltet an deren Ende ein Lehrmeister wartet.
\item Viele Aufgaben im Leben einer Figur sind reine Fleißarbeit, dazu gehört Muskeltraining, Studium in der Bibliothek oder die Erforschung einer neuen Technologie oder Rezeptur. Diese Aufgaben werden im Groben durch SP erledigt, einerseits um den Zeitaufwand im Spiel zu reduzieren, andererseits damit man sich beispielsweise das Erstellen einer vollständigen Bibliothek ersparen kann.
\end{itemize}
\section{Anmerkungen}
Das Startguthaben einer Figur besteht aus SP, welche für die in den ersten Lebensjahren gesammelte Erfahrung steht. Bei der Charaktererstellung können diese noch zusätzlich für direkte Attributssteigerung oder Vermögen und evt. weitere Dinge ausgeben.

\chapter{Erfahrungspunkte}


\part{Kreaturen}
\setcounter{chapter}{0}
\chapter{Übersicht}
Neben einer Bezeichnung(und beim Spiel einer Internen eindeutigen ID) ist jede Kreatur Teil einer Art an, von dieser erhält sie in der 
\chapter{Attribute}
Die Attribute symbolisieren die Ausprägen der geistigen und köperliche Kapazitäten der Kreatur. Attributswerte können dabei einen Wert zwischen 0 und 100 besitzen, auch wenn in der Regel die Werte bei 1 oder höher liegen. Anbei eine kurze Beschreibung zu jedem Attribut:
\begin{description}
\item[Stärke(STR)]
Dieses Attribut ließe sich ohne weiteres in weitere Unterkategorien einteilen, aber im Grunde geht es um die Muskelkraft und maximale Belastbarkeit des Körpers, bis es diesen zerreißt. Wenn eine Tür einzutreten gilt oder man schwere Lasten heben muss, dann ist dieses Attribut in erster Instanz gefragt. Stärke kann durch gezieltes Muskeltraining und eine gezielte Diät angehoben werden, wobei natürlich auch körperliche Arbeit genauso fördernd für diese ist und dazu auch noch profitabler.
\item[Mut(MUT)]
Sich als Mensch gegen einen Drachen zu werfen würden viele als blanken Wahnsinn bezeichnen, andere als Mut. Dieses Attribut gibt Auskunft über das Selbstvertrauen einer Figur in ihre Fähigkeiten oder die mentale Belastbarkeit gegen Furcht. Dabei sind nicht nur direkte Konfrontationen mit einem Feind, sondern ebenso manche Soziale Interaktionen(Wie das Betören oder Lügen) Gelegenheiten an dem man seinen Mut beweisen muss. Dass Attribut wächst in der Regel mit jeder erfolgreich gemeisterten Konfrontation, kann aber ebenso bei langanhaltendem Pech sinken.
\item[Beweglichkeit(BEW)]
Flexibilität und Gleichgewicht können im Angesicht wirbelnder Klingen oder um sich durch enge Passagen zu Kämpfen. Insgesamt beschreibt Gewandtheit die allgemeinen Körperkoordinaten und Balance. Damit fließt es oft auch in athletische Talente ein. Im Allgemeinen sind Dehnübungen ein ausgezeichneter Weg um an diesem Attribut zu arbeiten.
\item[Fingerfertigkeit(FF)]
Im Gegensatz zur zuvor erläuterten Beweglichkeit ist dieses Attribut spezialisiert auf die Feinmotorik der Hände und das Feingefühl in diesen. Handwerkliche Talente für kleine Gegenstände oder der Einsatz von kleinen Objekten, sei es im Kampf oder medizinischen Eingriffen. Ständiges Training zahlt sich auch hier aus.
\item[Konstitution(KON)]
Die allgemeine Ausdauer, aber auch die Körpereigenen Abwehrkräfte gegen Gifte und Krankheiten wird durch Konstitution bestimmt. Bei Ausdauersport oder auch einfach auf der Flucht vor einer heranstürmenden Gefahr, kann Konstitution nie schaden. Eben diese Tätigkeiten steigern gleichzeitig das Attribut.
\item[Metabolismus(MTB)]

\item[Intelligenz(INT)]
\item[Weisheit(WIS)]
\item[Charisma(CHA)]
\item[Aussehen(AUS)]
Dieses Attribut beschreibt die Physische Erscheinung und ihre Wirkung auf andere. Dabei ist zu erwähnen, dass Schönheitsideale von Volk zu Volk unterschiedlich sein können. Auch wird dieses Talent weder gesteigert noch sonst irgendwie erhöht, vielmehr wird bei der Charaktererstellung ein Wert zwischen 1-100 bestimmt, der je nach Physiognomie des Volkes mit zunehmenden Alter modifiziert wird. Weitere Informationen zu Aussehen werden in den folgenden Kapitel besprochen.
\end{description}
\chapter{Steigern und Fortschritt}
Erfahrungspunkte in einer Fähigkeit werden gleichmäßig auf die verwendeten Attribute verteilt. Die Formel zur Bestimmung der Kosten für eine Stufe lautet:
\\$9X+11X*(\frac{X}{20})$
\\Für Fertigkeiten mit einem Lernmodifikator ungleich 1, multipliziere Wert mit dem Kehrwert des Modifikator und runde auf.
\\
\\Zuästzlich können Sagenpunkte im Verhältnis 1:10 investiert werden können.
\chapter{Sekundäre Attribute}
Die hier aufgeführten Werte, berechnen sich aus den Basisattributen und können daher nicht direkt durch Training verbessert werden.
\\Vitalität=Konstitution+Boni
\\Trefferpunkte=Vitalität*5
\\Regnerationsrate=Metabolismus/(Konstitution*2)+1+Boni
\\Kondition=Konstitution*2+Boni
\\Ausdauerpunkte=Kondition*5
\\Gefahren-Level(Monster/Tiere)=Summe aller Attribute / 10
%K
\chapter{Fertigkeiten}
\section{Einführung}
Alle Tätigkeiten, die nicht auf einem Attribut basiert, sondern eine Mischung von Erfahrung und Attributen werden durch sogenannte Fertigkeiten beschrieben. Fertigkeiten lassen sich dabei in die Kategorien: Waffen(B), großes Handwerk(B), Wissen(C), Soziales(D), kleines Handwerk(D), Sonstiges(E). Die Kategorie ist dabei ausschlaggebend für die Lernkosten. Zusätzlich hat jede Fertigkeit einen eigenen, von den Professionen und der sog. Komplexität(A=0, B=0.25, C=0.5, D=0.75, E=1, F=1.25...) abhängigen Lernmodifikator. Eine Fertigkeit kann zudem in sog. Fokussierungen auftreten. Diese sind in der Regel Teil einer Profession und erlaubt in bestimmten Situationen mehr Punkte in eine Probe einzubringen.
\section{Erfahrung \& Steigern}
Bei jeder Probe werden EP für die genutzte Fertigkeit ausgegeben. Dazu gilt Komplexitätsklasse * 3 EP * Lernmodifikator * (Prozentchance für Misserfolg/100, Minimum 0.05). Agiert man außerhalb der eigenen Komplexitätsklasse so werden die Basis-EP auf die eigene Klasse abgewertet. Für Aktionen unterhalb der eigenen Komplexitätsklasse wird der Mittelwert abgerundet gebildet.
\\Die Formel zur Kostenbestimmung für einen Rang:
\\$7x+9x*\frac{x}{10}$
\\Für Fertigkeiten mit einem Lernmodifikator ungleich 1, multipliziere Wert mit dem Kehrwert des Modifikator und runde auf.
\\
\\Hat eine Fertigkeit die Komplexität A kann nur durch einen Lehrmeister/Trainingpartner gelernt werden. Nähere Regel hierzu folgen dann.
\\Für eine bessere Übersicht und Rechnung können diese Werte mit 100 multipliziert werden und bleibende Nachkommastellen fallen lassen, außerdem sollte man die Auswertung immer auf die nächste Rast verschieben.
\section{Probe}
Auch wenn die genaue Probenart je nach Situation und Fertigkeit variieren kann, so dienen die drei mit der Fähigkeit verknüpften Attribute als Schwellen und der Fertigkeits-Rang als positiver Modifikator.
\section{Charisma und Aussehen}
Aussehen kann bei bestimmten Sozialen Fertigkeiten einen Modifikator, in der Höhe von AUS/10 erzeugen. Beachte die Verhältnisse bei anderen Völkern und Geschlechtern.
\\\textsuperscript{c+}, bedeutet Positiver Modifikator
\\\textsuperscript{c-}, `` negativer ``

\chapter{Effekte}
\section{passive Eigenschaften}
Eine passive Eigenschaft besteht aus seiner Bezeichnung, einem Wert, sowie einer Angabe über die Herkunft.
\subsection{Modification}
Eine Modifikation wird auf den bezeichneten Basiswert gerechnet, wobei jeweils für einen Source-Typ nur der höchste positive und niedrigste negative Wert beachtet wird. Hierbei ist zu beachten, dass ein * am Ende der Bezeichnung einen Multiplikator bedeutet, für die andere Regeln gibt. Diese werden erst nach Addition der übrigen Werte angerechnet. Negative Werte werden hier als 1/abs(Wert) gewertet und die 0 hat Vorrang über negative Werte und kann nur durch einen besonderen Block abgewehrt werden.
\subsection{Block}
Sperren schützen vor den Einflüssen von Modifikationen. Die Bezeichnung bedeutet, dass der bezeichnete Basiswert nur durch eine Modifikation von höherem Wert(negativ natürlich retrospekt negativ) als dem Block bearbeitet werden kann. Ein Block mit 0, kann einen negativen Modifikator. Ein Block mit * am Ende seiner Bezeichnung bedeutet, weist diese als Source aus, die andere Modifikationen/Blocks unter den Tisch fallen lassen, die kleinerer Order haben, bei Multiplikativen werden nur Absolute betrachtet. Mit * gezeichnete Blöcke interagieren nicht untereinander.
\subsection{Counter}
Kommen erst bei konkurrierenden Proben zum Einsatz. Die Bezeichnung steht dafür für die Probe/Fertigkeit, bei der dieser Counter relevant sind. Die Source hebt alle Modifikationen auf, mit selber Source, die geringeren Value(absoluter Wert) haben.

\part{Ergänzung für Charaktere}
\setcounter{chapter}{0}
\chapter{Spezialisierungen/Professionen}
\section{Aufbau einer Profession/Spezialisierung}
Jede Profession bringt sog. favorisierte und entlegene Fertigkeiten mit, bei denen jeweils der Lernmodifikator entsprechend erhöht und verringert wird. Bevorzugte Fertigkeiten werden um 2 Kategorien einfacher, während vernachlässigte Fertigkeiten um 3 schwerer werden. Weitere Professionen ändern bei Übereinstimmungen diesen Modifikator nur um 1.
Dazu gibt es verteilt auf bis zu 6 Ränge, beginnend mit 0, Talente, Fertigkeiten-Foki und passive Vorteile.
Pro Rang gibt es bis zu 4 mit Professionspunkten bezahlbare solcher Inhalte, sowie ein kostenloser, sobald der Rang durch Erfahrung freigeschaltet wurde. Höhere Ränge können dabei nicht eingesehen werden.  

\chapter{Merkmale}
\section{Einführung}
Nicht jeder wird gleich geboren. Dies kann sich sowohl in Makeln, wie auch besonderen Fähigkeiten oder anderer Eigenarten äußern. Neben der Möglichkeit dies alles über Rollenspiel zu definieren, sei hier auch ein Rahmen für feste Merkmale gegeben. Alle Merkmale lassen sich in eine von 3 Kategorien einteilen: Den Vorteilen, Nachteilen oder Neutralen. Letztere schränken den Charakter in seinem Handlungen ein, belohnen ihn dafür ebenfalls. Bei der Charaktererstellung bringt standardmäßig jeder Nachteil oder Neutrale zusätzliche GP, während Vorteile selbige Kosten. Zusätzlich kann nur eine gewisse Menge an GP durch Nachteile eingekauft werden können und die Zahl an verschiedenen Nachteilen kann ggf. ebenfalls beschränkt werden. Bestimmte Professionen und Rassen bringen ihre eigenen Vor- und Nachteile mit sich.
\section{Alternativregeln}
Es folgen einige Vorschläge Merkmale anders einzuführen, als oben beschrieben:
\begin{description}
\item[Niemand ist Perfekt]Unabhängig von den GP würfelt man bei der Charerstellung 3 negative Eigenschaften aus.
\end{description}
\section{Generelle Merkmale}
Wie so oft, variieren die genauen Merkmale von Welt zu Welt, weshalb hier beispielhaft eigentlich allgemeingültige Merkmale aufgeführt werden. Der Meister hat bei der Wahl von Merkmalen immer das letzte Wort.
\subsection{Vorteile}
\begin{description}
\item[Ausgeprägter Sinn]1+[1..4]GP
\\Wahrnehmungsproben mit diesem Sinn erhalten einen Bonus von 2*Stufe des Merkmals.
\item[Ausdauernd] (Athletik/Kampf/Körperarbeit/Marsch) Ausdauerkosten für gewählte Tätigkeit werden halbiert.
\item[Hohe Regeneration] Bei Rast wird doppelte Regenerationsrate angerechnet.
\item[Talent für Fertigkeit] Lernklasse wird eine besser.
\item[Talent für Fertigkeitengruppe] Lernklasse wird 1 besser
\item[Eisern] Wundschwelle steigt
\end{description}

\subsection{Nachteile}
\begin{description}
\item[Eingeschränkter Sinn]1+[1..4]GP
\\Wahrnehmungsproben mit diesem Sinn erhalten einen Malus von 2*Stufe des Makels.
\end{description}

\subsection{Neutrale}
\begin{description}
\item[Strenger Tagesplan] Muss bestimmte Tätigkeitsphasen einhalten, wie Essenszeit, Schlafenszeit oder Freizeit, Verletzungen dieser führen zu Malus. Auf der anderen Seite sind Versuche die Person von ihrem Plan abzubringen entsprechend erschwert. Umgewöhnung Möglich, bei Letzterer oder Verletzung gehen Vorteile für 1 Monat verloren.
\end{description}    

\chapter{Gesundheit und Schwere Wunden}
Die wichtigsten Werte für eine Figur ist die eigene Gesundheit. Diese wird durch die sog. Vitalität und Trefferpunkte verkörpert.
\\Vitalität - Dieser Wert ist entscheidend für das Überleben einer Figur. Sinkt sie auf 0 stirbt die Figur unweigerlich.
\\Trefferpunkt - Bevor eine Figur lebensbedrohliche und schwere Verwundungen einsteckt, geht jeder Schaden gegen diesen Wert. Ist er aufgebraucht wird Vitalität im Verhältnis 1:5 aufgebraucht.
\\Während die Regeneration von Trefferpunkten bereits während einer ersten echten Rast(8 Stunden) vollständig abgeschlossen ist, wobei Trefferpunkte nicht die aktuelle Vitalität übersteigen können, sieht es im Falle der Vitalität anders aus. In der Regel generiert eine Figur pro Rast ihre Vitalität mit der Regenerationsrate. Durch medizinische Behandlung können beide Regenetationen angehoben werden: 1h intensiver medizinischer Behandlung stellt alle Trefferpunkte wieder her. Für die Vitalität muss die Behandlung auf 12 ausgeweitet werden, wenn auch nicht zwangsläufig so intensiv, siehe hierfür die Fertigkeiten zur Heilkunde und regeneriert dabei die Hälfte der Vitalität.
Schwere Verletzungen, wie fehlende Gliedmaßen reduzieren dauerhaft die maximale Vitalität, sofern sie nicht wiederhergestellt werden. Anbei eine Tabelle zur Übersicht der Auswirkungen einer derartigen Verletzung:
%Tabelle

\chapter{Kondition und Erschöpfung}
Jede körperliche Aktivität fordert früher oder später ihren Tribut in Form von Erschöpfung. Der Grad der aktuellen Erschöpfung wird hierbei durch die Werte Ausdauer und Kondition verkörpert:
\\Ausdauer - Alle Körperlichen was über normale Gehen ohne Gepäck hinaus geht kostet in der Regel Ausdauer.
\\Kondition - Im Fall das einer Figur die Puste ausgeht oder sie einer langfristigen Belastung durch beispielsweise einen Gewaltmarsch mit Gepäck durch die Wüste ausgesetzt ist, sinkt die Kondition. Ausdauer kann im Verhältnis 1:5 durch Kondition ersetzt werden.
\\Erhält eine Figur die Gelegenheit für Atem zu schöpfen, so regeneriert sich pro Minute ein Zehntel der maximalen Ausdauer. Diese wird begrenzt durch die aktuelle Kondition. Kondition hingegen erholt sich vollständig über eine 8 Stunden Rast, wobei auch kleinere Pausen von einer Mindestdauer von 1h die Kondition wiederherstellen.
Eine niedrige Kondition belegt eine Figur außerdem mit zusätzlichen Mali auf seine Angriffs-, Parade-, sowie gekennzeichnete Fertigkeiten.
50\% der maximalen Kondition erteilen einen Malus von 5, während 25\% einen Malus von 10 erteilen. Bei 5\% schließlich wird der Malus auf 20 angehoben.
\part{Ausrüstung \& anderen Gegenständen}
\setcounter{chapter}{0}
\chapter{Waffen}
\section{Nahkampfwaffen}

\section{Fernkampfwaffen}
\chapter{Rüstungen}



\part{Herstellung von Objekten}
\setcounter{chapter}{0}
\chapter{Grundlagen}
Jedes Objekt setzt sich aus sog. Komponenten zusammen, was durch ein sog. Rezept verkörpert wird. Im Rezept sind sog. Slots verankert. Für jeden Slot muss das dort angegebene Material bereitgestellt werden, ggf. in einer bestimmten Qualitätsgüte. Anschließend muss für jeden Slot eine Handwerksprobe abgeschlossen werden, mit der für den Slot angegeben Erschwernis. Die Komplexitätsklasse hängt hierbei von der Angabe des Slots und der Qualität ab. Je besser diese Probe abhängt desto höher ist die Verarbeitungsqualität des Slots und die Gesamtqualität der Durchschnitt aller.
\chapter{Rezepte}
Ein Rezept gibt immer an wie sich die Qualität auf das Endprodukt auswirkt.
\section{Slot}
\begin{description}
\item[Materialtyp]Gibt an was benötigt wird.
\item[Min-Qualität]Gibt an von welcher Qualität das Material sein muss.
\item[Max-Qualität]Gibt an, wie hoch die Verarbeitungsqualität durch das Material angehoben werden kann. Natürlich kann der Wert angehoben werden für reiche Schnösel, aber naja es wird dadurch nicht besser, natürlich muss man auch das notwendige Talent zur Verabeitung besitzen. Für magische Rituale kann es natürlich interessant sein, dass das Objekt einen Mindestwert hat.
\item[Standard-Materialqualität]Wenn das gewählte Material minderwertig ist, wird die Verarbeitungskomplexität runter gezogen, aber auch die Qualität.
\item[Verarbeitungskomplexität]Kann durch hochwertiges Material nach oben gezogen werden.
\item[Min-Erfolg]Bei der Probe muss dieser Wert übrig bleiben, damit die Verarbeitung gelingt. Fehlende Punkte machen prozentual das Material kaputt?
\item[Qualitätsschritte]Alle so viele Punkte steigt die Qualität um eine Kategorie.
\item[Standard-Verabeitungsqualität]Dieser wird wird von der erwürfelten Verabeitungsqualität abgezogen, um auch minderwertige Produktionen zu erlauben.
\item[Max-Verabeitungsqualität]Begrenzt die Bonus-Qualität nach oben.
\item[Qualitätsmodifikator]Gibt wie weit diese Qualität am Ende zum Endprodukt beiträgt.
\end{description}
Die Verarbeitungsqualität ist (Materialqualität+Verabeitungsqualität-Standardverarbeitungsqualität)*Qualitätsmodifikator.
\section{Endqualität \& Wertigkeit}
Summe aller Verarbeitungsqualität und dann Einteilen. Der Wert hängt von Verabeitungsqualität ab.


\part{Kampf}
\setcounter{chapter}{0}
\chapter{Ablauf}
%In Kampfsituationen wird das Kampfgeschehen in Runden eingeteilt. Zu Beginn jeder Runde wird der sog. PAI bestimmt. PAI steht dabei für: Parade, Attacke und Initiative. %Initiative wird im Programm einmalig bei Ausrüsten einer Waffe berechnet und ohne das zusätzliche Punkt darauf verteilt werden können.
%Diese Werte werden über den Grundwert von Waffe, Attacke-Nahkampf/Fernkampf, Parade-Basiswert. In einer Runde stehen einer Figur 2 Aktionen zur Verfügung. Zusätzlich hat jede Figur noch min. eine Freie Aktion, wobei z.B. das Verlieren von Konzentration nicht als solche gilt.
%Runden bestehen aus 2 Phasen: Ansage und Ausführung.
%Während der Ansage, die bei der Figur mit der geringsten Initiative beginnt, müssen die Aktionen bestimmt werden. Die Ausführung selbiger Aktionen folgt anschließend, beginnend mit der Figur mit der höchsten Initiative.
%Bei einem klassischem Angriff wird eine Probe auf das Waffentalent gemacht, modifiziert durch die Attacke. ist der Angriff erfolgreich steht dem Ziel, falls es eine Parade offen hat, diese zu oder es kann als freie Aktion Ausweichen. In beiden Fällen muss eine entsprechende Gegenprobe gemacht werden, die allerdings nicht konkurrierend ist. Scheitert diese Probe war der Angriff erfolgreich und basierend auf der Waffe Schaden ausgewürfelt. Für 3 Punkte die beim Angriff übrig geblieben sind, erleidet das Ziel einen weiteren Punkt Schaden. 

%RUNDENABLAUF FOR REALS
%Abhandeln der Runden: Z=0. Ein Kombatant mit PAI hat eine Aktion frei zum Ansagen, wenn Z seine INI erreicht. Danach folgen drei Phasen für jeden Kombatant, die jeweils die Länge von INI/3 haben. In der ersten kann die angesagte Aktion frei gewählt und gewechselt werden, in der zweiten wird bei jedem Wechsel die Aktion mit einem Malus von 4 belegt, während in der dritten Phase bei Wahl oder Wechsel ein Malus von 8 erteilt wird. Sobald Z ein Vielfaches von 100 erreicht, werden alle Aktionen in der Reihenfolge ihrer Belegung auf Zeitachse ausgeführt, angefangen mit der höchsten. Liegen 2 Aktionen auf demselben Marker entscheidet Initiative über die erste. Alle Aktionen die ihre Gültigkeit verlieren, werden ersatzlos gestrichen, sofern der Spieler keine Malus 8 in Kauf nimmt, um die Aktion anzupassen. Bei überhängenden Rahmen, bei hohen INI Werten, wird dieser Überhang halbiert und anschließend erhält man für 10 Punkte eine weitere Aktion in der Runde(max. 3) die man beliebig innerhalb des eigenen Intiativ-Rahmens verwenden kann. Alternativ können damit die Mali für verspätete Aktionen in Phasen 2+3 negiert werden. Im Computer bedeuten Zusatzaktionen einen Boost, welcher beim Laufen geleert wird und damit die Füllgeschwindigkeit nicht verringert wird, bzw. der Aktionsbalken sofort gefüllt werden. Der Boost lädt sich selbst langsam auf.

%

Kämpfe finden in Runden statt. Zu Beginn jeder Runde wird der P(arade)A(ngriff)I(nitiative) ermittelt. Dazu wird der Kampfwert, welcher aus den drei Hauptattributen der Waffenfertigkeit plus die Ränge in der Waffenfertigkeit besteht. Der Kampfwert kann nun nach eigenem ermessen und ggf. durch den Waffentyp beschränkt auf den PAI verteilt werden, abschließend kommen noch die PAI-Werte der Waffe auf diese Werte und ein W6 pro Komplexitätsklasse, welche sich nach dem Kampfwert richtet, auf Initiative. Nun beginnt derjenige mit der geringsten Initiative mit der Ansage. Er plant seine Aktion und deckt dabei seinen PAI auf. Sobald alle Aktionen geplant wurden, werden diese im umgekehrter Order ausgeführt, beginnend mit demjenigen der die höchste Initiative hat. Zu jedem Zeitpunkt kann eine Figur eine freie Aktion durchführen. Die Anzahl freier Aktionen pro Runde kann beschränkt sein.
\chapter{Aktionen}
Folgende Aktionen stehen zur Verfügung:
\begin{description}
\item[Bewegen*]Jede Figur kann basierend auf der Art wie sie sich Fortbewegt mit der entsprechenden Geschwindigkeit bewegen. Während einer Bewegung können andere Aktionen eingebunden werden, auch wenn sie ggf. Abzüge erhalten.
\item[Rennen*]Die Figur bewegt sich mit 2,5-facher Geschwindigkeit bewegen, dabei können keine anderen Aktionen eingebunden werden. 
\item[Nahkampfangriff]Das Ziel macht eine Probe gegen den eigenen Angriffswert, bei Erfolg steht dem Ziel ggf. eine Probe auf seine Parade zu. Basierend auf der Differenz wird Schaden gerollt. Alternativ kann man seinen Angriffswert auf mehrere nebeneinander stehende Gegner oder für mehrere Nahkampfangriffe auf ein Ziel aufteilen.
\item[Kampfmanöver]Eine konkurrierende Probe mit dem eigenem Waffentalente und dem gegnerischen Waffentalente.
\item[Fernkampfangriff*]Sofern die Waffe geladen ist kann man eine offene Probe auf das Waffentalent machen und basierend auf dem Restwert Schaden rollen. Weiß das Ziel um die nahende Gefahr kann es versuchen auszuweichen. Handelt es sich um eine Waffe mit Schnellfeuer können entsprechend dieses Wertes und dem Restmagazin mehrere Schüsse abgegeben werden, mit einer Erleichterung von 1 für jeden Folge Schuss und einer Erschwernis von 2 für jedes weitere Ziel.
\item[Volle Defensive]Für jeden parierten Angriff verliert man in der Regel 3 Punkte in seiner Parade, mit dieser Aktion kann man für jeden eingehenden Angriff eine Probe auf seine Defensiv-Waffe machen. Falls diese erfolgreich ist, verliert man keinen Paradewert und kann zusätzlich die Differenz zwischen eigenem Paradewert mit gegnerischem Angriff mit übrigen Punkten im Verhältnis 1:3 ausgleichen. 
\item[Ausweichen*]Falls man durch eine Überzahl zu überwältigt werden droht oder unter Beschuss steht kann man eine Akrobatik-Probe machen. Ist diese erfolgreich negiert man den Angriff, verliert jedoch für den Rest der Runde seinen PAI. Ein kritischer Erfolg negiert diesen Effekt. Ausweichen kann eine freie Aktion sein, allerdings verliert man dann unabhängig des Ergebnisses seinen PAI.
\item[Nachladen*]Eine leere Nahkampfwaffe muss für die auf der Waffe angegeben Anzahl von Runden nachgeladen werden. Zusätzlich kann! eine Waffenprobe gemacht werden, die wenn erfolgreich die Waffe in halber Zeit nachlädt, während ein Fehlschlag die Dauer um die Hälfte anhebt. Diese Probe kann pro Nachladen nur einmalig gemacht werden. Beträgt die Nachladedauer 0 bis <1, kann die Waffe in derselben Runde benutzt werden.
\item[Fertigkeiten und andere Proben*]Natürlich kann man jederzeit eine Fertigkeitsprobe machen.**
\end{description}
*Eine Aktion, die z.B. die gesamte Konzentration benötigt, lässt einen den PAI verlieren.
\\**Sofern die Probe z.B. Wissenfertigkeit zur Analyse eines Feindes ist kann man natürlich Ausnahmen machen.
\\Zusätzlich gibt es noch einen Haufen freier Aktionen, die bis zu einem gewissen Maß jederzeit eingesetzt werden können.
\begin{description}
\item[Sprechen]Naja reden, schreien, all das ist eine freie Aktion, auch wenn natürlich die Anzahl der Wörter begrenzt ist.
\item[Waffe fallen lassen]Der PAI verfällt für den Rest der Runde, auch wenn eine neue Waffe gezogen wird.
\end{description}
\chapter{Ohne PAI}
Wenn eine Figur ihren PAI im Laufe der aktuellen Runde verliert, bleibt ihre Initiative bestehen und sie können ggf. noch eine Aktion ausführen. Ansonsten gilt das jeder Figur, sofern sie sich eingehender Angriffe bewusst ist Ausweichen als freie Aktion zur Verfügung steht. Alle Aktionen gegen einen PAI-losen, die Proben verlangen, müssen immer noch gewürfelt werden, wenn sie auch um 3 erleichtert sind.
\chapter{Ungültige Aktionen und Planänderungen}
Im Falle, dass z.B. ein Ziel nicht mehr vorhanden ist, verfällt die Aktion ersatzlos. Zu Beginn ihrer Aktion kann sich jede Figur zu einer Planänderung entscheiden. Falls die neue Aktion keine Probe erfordert, muss der Erfolg durch eine Angemessene(Angriff=Waffenprobe) entschieden werden. In jedem Fall sind diese Proben um 4 erschwert.
\chapter{Zusätzliche Regeln}
\begin{itemize}
\item Werden mehrere Waffen geführt erhält man einen Abzug auf Angriff und Parade von 2 und Initiative von 1. Dieser Malus steigt mit jeder Größenkategorie der Zweitwaffe über klein.
\item Bei Nahkampfkonfrontationen mit unterschiedlichen Komplexitätsklassen muss derjenige mit der niedrigeren, für jede Differenz eine zusätzliche Probe machen.
\item Fällt ein PAI in einer Komplexitätsklasse, so bleiben die Kosten bestehen.
\item Die Ausdauerkosten steigen aufgrund von Komplexitätsklasse steigen bereits 5 Punkte unter der Schwelle. Dieser virtuelle Zuschlag kann für 5 Punkte im Waffentalent-Komplexitätsklasse*20 gesenkt werden. Beispiel 15 Waffentalent erlaubt freien Angriff/Parade bis 19 ohne Zuschlag.
\end{itemize} 
%Programmnotiz: Alle Aktions-Befehle führt das Programm basierend auf der Initiative mit einer Verzögerung aus(während man also auf seinen Zug wartet), innerhalb dieser Zeit kann die Aktion durch eine neue ersetzt werden, wodurch diese jedoch einen Malus erhält. Aktionen werden hierbei evt. in einer Warteschlange abgelegt, wobei ungültige Befehle verfallen. Zusätzlich lassen sich aus der Warteschlange kommende befehle innerhalb eines gewissen Zeitfensters löschen und Ersetzten. HUD-Mäßig haben wir einen sog. Ini-Balken, der in drei Abschnitte aufgeteilt ist. Im ersten Drittel lassen sich Befehle beliebig ersetzten ohne Abzüge, im zweitem Drittel werden bei jedem Wechsel Abzüge erteilt. Im letzten drittel schließlich wird auch die erst gewählte Aktion mit Mali belegt. Ist der Balken voll wird die Aktion ausgeführt und der Balken zurückgesetzt. Aktive Bewegung und andere nicht freien Aktionen reduzieren die Geschwindigkeit mit der sich der Balken füllt. Während der Ausführung abrechenbarer Aktionen, z.B. beim Zaubern füllt der Balken sich nicht. Außerhalb von Kämpfen ist der Balken deaktiviert. Falls Überrascht lässt sich im ersten Balken keine Aktion unterbringen. Fenster für die Zeit bis der Balken sich füllt 1-2 Sekunden.
\part{Meister \& Hintergründe}
\setcounter{chapter}{0}
\part{Zusätzliche Regeln}
\setcounter{chapter}{0}
\phantomsection

\part{Sortierendes}

\chapter{Ausdauerkosten}
Angriffe/Parade 3 * Komplexitätsklasse.


\chapter{Erstellung}
Startwert Attribute ist 8.
Zusätzlich erhält jeder Char 530XP jeweils für F. und A. zum Verteilen.

\chapter{Sagenpunkt Anwendungen}
Erwerben von Professionspunkten. Für den Rang bezahlt man jeweils Erlernungskosten*Rang, sofern nicht anders angegeben. Die Pakete in einem Zweig kosten jeweils die Hälfte der Kosten für den Rang. Knoten in einem Fähigkeitennetz können verbunden sein. Falls so werden die Kosten für jeden bereits erhaltenen Knoten um 1/10 gesenkt.
\\50 Sagenpunkte können genutzt werden, um einen NPC von einem gehört zu haben. Ein W6 entscheidet über das Ergebnis(1->Bewunderer, 2-4->Positiv gestimmt, 5->Unwirksam, 6->Negativ)
\\Attribute und Fertigkeiten steigern
\\Beim Umschulen dürfen Punkte im Verhältnis 1:(10*Rang der Abgebaut werden) modifiziert mit der Ähnlichkeit zur neuen Profession umgetauscht werden. 
\\Insgesamt darf in Professionen nur X Punkte eingesetzt werden.


\end{document}