\documentclass[a4paper,12pt,oneside]{book}
\usepackage[ngerman]{babel}
\usepackage[utf8]{inputenc}
\usepackage{imakeidx}
\usepackage[hypertexnames=false]{hyperref}
\usepackage[all]{hypcap}
\usepackage{nameref}
\usepackage{ulem}


\hypersetup{
	bookmarks=true,
    colorlinks,
    citecolor=black,
    filecolor=black,
    linkcolor=black,
    urlcolor=black
}

\title{Aurum Orbis - Alchemistische Ordnung}
\author{Jordan Eichner, Christoph Schmidt}
\date{}
\setcounter{secnumdepth}{-1}
\setcounter{tocdepth}{10}

\begin{document}

\maketitle
\tableofcontents

\part{Grundlagen}

\part{Runen}

\chapter{Elementare}
Alle Runen aus dieser Kategorie sind für sich schon einfache Zauber, erfordern sofern nicht deklariert keine Zauberprobe und sind Magiern aller Schulen und Zirkel bekannt.

\begin{description}
\item[Lux,] Lucis; 1M/h; Komponente: -
\\Die Hand des Magiers wird in einen Lichtschein mit der Helligkeit einer Fackel gehüllt. Durch Berührung kann dieser Lichtschein auf ein berühtes Objekt übertragen werden. Der Magier kann die Farbe bei Erschaffung anpassen.
\item[Ignis,] Ignis; 5M; Komponente: S
\\Durch einen Fingerzeig wird ein Objekt mit der Brennbarkeit von trockenen Holzscheiten entflammt.
\\item[Aqua,] Aquae; 
\end{description}

\part{Zauber}

\end{document}