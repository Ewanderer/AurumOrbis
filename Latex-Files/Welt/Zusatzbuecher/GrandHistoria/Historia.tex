\documentclass[a4paper,12pt,oneside]{book}
\usepackage[ngerman]{babel}
\usepackage[utf8]{inputenc}
\usepackage{imakeidx}
\usepackage[hypertexnames=false]{hyperref}
\usepackage[all]{hypcap}
\usepackage{nameref}
\usepackage{ulem}


\hypersetup{
	bookmarks=true,
    colorlinks,
    citecolor=black,
    filecolor=black,
    linkcolor=black,
    urlcolor=black
}

\title{Aurum Orbis - Grand Historia}
\author{Jordan Eichner}
\date{}
\setcounter{secnumdepth}{-1}
\setcounter{tocdepth}{10}

\begin{document}

\maketitle
\tableofcontents
[Axiomaten und Götter]
Nachdem das Göttertrio mit der Flora und Fauna zufrieden waren, kamen sie erneut zusammen, um über ihre nächsten Schritte zu beraten. Es stand außer Frage, dass man nicht auf intelligente, beseelte Völker verzichten wollte, schon allein, um die eigenen Kraftreserven über die Zeit wieder zu befüllen, um evt. Nachzügler aus der Götterwelt vertreiben zu können. 
\\Da man es aber nicht auf einen direkten Konflikt untereinander abgesehen hatte, ging der eigentlichen Schöpfung viel Planung voraus. Zunächst schuf man gemeinschaftlich die Drachen, um den Pakt zu besiegeln. Sie begannen ihre Existenz in einem entfernten Winkel der Welt, wo einige bekanntermaßen nach einigen Jahrhunderten die Magie an einer Anomalie adaptierten.

\\\\Diese ersten magisch begabten Drachen unterschieden sich in ihrem Wesen nicht viel von den Menschen. Arroganz und Egoismus sorgten dafür, dass man sich gegenseitig an die Kehle ging, um den anderen zu unterwerfen. So ging es einige Jahre lang, bevor einige wenige Jungdrachen auf der Suche nach mehr Macht, sich schließlich mit den Naturgesetzen und anderen Geheimnissen ihrer Existenz beschäftigten.
\\Neben einigen Antworten fanden sie aber vor allem mehr Fragen, die ihr Bewusstsein um eine Ebene erweiterten, vor der ihre vorherigen Ziele nach Herrschaft verblassten und sie die aktuellen Herrscher nur stürzten, um zusammen mit Gleichgesinnten in Ruhe nach Antworten zu suchen. Die übrigen begabten Drachen schlossen sich wie von selbst an und auch in den folgenden Generationen sollte jeder Drache zur selben Mentalität kommen, da die Neugier die oberste Charaktereigenschaft eines Drachen war.
\\Die Enklave der Wahrheitssucher fand ihren Weg schließlich zum Hauptkontinent, da es dort wesentlich mehr Anomalien und damit potenzielle Antworten gab. Man ließ sich im unwirklichen und unbewohnten Feuerland nieder. Im Zuge ihrer Studien der anderen Völker aus dem Verborgenen heraus wurde ihre Aufmerksamkeit auf die Aspekte und damit die Götter gelenkt.
\\Die erste Konfrontation der Drachen erwischte die Götter völlig kalt und ihre Vorstöße brachten sie bald in arge Bedrängnis, da sie auf die Fragen keine Antworten wussten und den Drachen nicht aus dem Weg gehen konnten. Da ihre Macht den Drachen keinen direkten Schaden mehr zuzufügen vermochte, wandten sich die Götter verzweifelt der einzigen Kunst zu, die ihnen blieb: Der Schöpfung.
\\Als Resultat dieses zweiten direkten Zusammenschlusses, von Wissen, Handwerkskunst und Wildheit, entstanden aus den neu geschöpften Göttermetallen, an einem der Erzseen, der später Herz der Zwerge genannt wurde, das Volk der Zwerge. Dank des Herzens konnten die Zwerge sich in Windeseile vermehren und schwärmten schließlich an die Oberfläche, um die Drachen zu vernichten.
\\Der nachfolgende Krieg war blutig und schien zu Gunsten der Zwerge auszugehen, bis ein Stamm Dunkelelfen zwischen die Fronten der letzten Schlacht in den Feuerlanden geriet und das Blatt wendeten, da die Zwerge sie nicht zu verletzten suchten. In der Kurzen Atempause, die den Drachen vergönnt war, bedankten sie sich bei den gegen sie anrennenden Dunkelelfen, indem sie aus ihren Herzen des Hass nahmen und so den Fluch ihrer böswilligen Schöpferin zu brechen, wie es die Drachen nannten.
\\Dies war die Geburtstunde der Elfen und einer Freundschaft, die die Äonen überdauern sollte. Gemeinsam machte man sich auf den Weg im Schutz des Mantels der Unsichtbarkeit, bis zum Herz der Zwerge, wo sie mit einem Schlag das Herz eroberten und so die Produktion neuer Zwerge unterbanden. Ein paar Elfen ließ man anschließend zurück, während man mit den Aufräumarbeiten im Rest des Kontinentes begann. Nach Abschluss dieses Genozid, kehrte man zum Herzen zurück, wo inzwischen eine stattliche Enklave aufgebaut wurde. 
\\Die Drachen ließen die Elfen an ihrer Macht und ihrem Wissen teilhaben und die Elfen verpflichteten sich, das Wissen um die Drachen zu wahren, auch vor ihren Nachkommen. Von diesen ersten Nachkommen zog es viele an die Oberfläche zu den übrigen Völkern, da sie die Welt sehen wollten. Was folgte war die blutigste Katastrophe und beinahe die Vernichtung der Elfen. Mancherorts wurden sie natürlich mit ihren verwandten den Dunkelelfen verwechselt, andernorts offenbarten die Elfen ihre magischen Kräfte und weckten die Neugier Machthungriger Tyrannen. Allerorts starben Elfen, für ihr Blut oder einfach, weil sie anders waren. Schließlich verfolgte man ihre Spuren zurück bis zu den unterirdischen Reichen, wo die Drachenelfen traurig auf den Tod ihrer Kinder machtlos, da sie ihren Auftrag wahren mussten.
\\Doch mit dem Eindringen in die Hallen war eine Grenze überschritten und die Drachenelfen begannen systematisch mit der Auslöschung sämtlicher Bewohner im Umland ihrer Stollen. Auch wenn man dort Magie zum Teil adaptiert hatte, war man Chancenlos. Es bestand noch nicht einmal die Zeit, damit sich die Kunde von diesem Massaker ausweiten konnte. Was blieb waren einige wenige Individuen mit Manablut, ein paar lose Geschichten und ein Gebirge aus Leichen unter der Erde.
\\Die wenigen zurückkehrenden Elfen wurden mit offenen Armen empfangen, auch wenn die Gemeinschaft sich entschloss ihr Verhalten zu überdenken, um das Bestehen ihres Volkes zu sichern. Am Ende dieser Entwicklung standen dann die Kasten, in denen die Elfen eine Beschäftigung fanden. Erst im Laufe der nächsten Jahrhunderte begann man wieder mit der aktiven Kontaktaufnahme zu übrigen Völkern begann.

\\\\Wie bereits erwähnt, war man darauf bedacht, zwischen den Völkern keinen Krieg zu provozieren. Daher wurden Niju's Schöpfungen strikt von den Menschen getrennt. Letztere wurden verteilt auf 3 Städte, um Vielfalt zu fördern. So gingen die ersten Jahre ereignislos ins Land, bis zu jenem Schicksalhaften Tag, an dem sich das erste Mal Menschen aus verschiedenen Städten trafen. Man sollte meinen, dass nach Jahrhunderte langer Übung ein Gott ein Händchen dafür haben sollte, die Entwicklung der eigenen Schöpfung nach seinen Vorstelllungen vorherbestimmen zu können. Stattdessen kam es bald zu Anfeindung zwischen diesen drei Stämmen, woraus die ersten großen Kriege begannen. Es standen sich gegenüber die Sommerfelder, unter der Führung einer Monarchie, im Osten, die wilden Nordlinge und die parlamentarischen Mittelländer. Wo zuvor die Reiche relativ schnell gewachsen sind, sorgte der Krieg zu einem Stillstand. 

\end{document}