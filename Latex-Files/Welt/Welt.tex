\documentclass[a4paper,12pt,oneside]{book}
\usepackage[ngerman]{babel}
\usepackage[utf8]{inputenc}
\usepackage{imakeidx}
\usepackage[hypertexnames=false]{hyperref}
\usepackage[all]{hypcap}
\usepackage{nameref}
\usepackage{ulem}
\usepackage{Zusaetze}

\hypersetup{
	bookmarks=true,
    colorlinks,
    citecolor=black,
    filecolor=black,
    linkcolor=black,
    urlcolor=black
}


\makeindex[name=Stichworte,title=Stichwortverzeichnis]
\makeindex[name=Geographie,title=Geographieglossar]
\makeindex[name=Kreaturen,title=Monsterindex]

\title{Aurum Orbis}
\author{Jordan Eichner, Simon Hornisch, Till Markusch}
\date{}
\setcounter{secnumdepth}{-1}
\setcounter{tocdepth}{10}

\begin{document}

\maketitle
\tableofcontents

\part{Geschichte}
\chapter{Die Dunkle Zeit}
Die Ereignisse aus diesem Zeitraum sind aufgrund ihrer Natur und den anfangs nur mündlichen Überlieferung nur unvollständig bis gar nicht überliefert. Selbst Legenden aus dieser Zeit sind meist frei erfunden und magische oder göttliche Hilfsmittel können ebenfalls nur wenige brauchbare Informationen liefern, Götter weigern sich oder sind selbst unwissend, während Zauber aufgrund von Resonanzen mit den magischen Energien aus dieser Zeit bis zur Unkenntlichkeit verzerrt werden.

\section{Die Weltbauer}
Am Anfang gab es die \uline{Axiomaten}\index[Stichworte]{Axiomat}. Wesen die in das Chaos des Multiversums Ordnung zu bringen suchten und die Absolute Gewalt über die Realitäten innerhalb der Raum-Zeit-Kontinuum verfügten.   
Und sie taten für Ewigkeiten nichts anderes als alles in ein statisches Gleichgewicht zu bringen und als sie ihr Ziel erreichten und die Ordnung perfekt war, gerieten sie in eine Krise. Sie waren, soweit man als Außenstehender es beurteilen konnte sehr geduldig, doch die Aussicht für immer einfach nur auf eine Perfekte Ordnung hinabzublicken verlor irgendwann für einige ihren Reiz. Und so verwarfen sie ihre ursprüngliche Philosophie und wurden zu den Weltenbauern. Sie begannen Veränderung zu erlauben und die Universen erwachten aus ihrem Stillstand. Doch relativ schnell vergingen viele dieser Welten, als sie aufgrund von Instabilität zerfielen. Man lernte dazu und mit wachsender Erfahrung, ließen sich auch die ersten einfachen toten Welten bauen. Der nächste Schritt bestand aus der Einführung von Leben und anderen Faktoren.

\section{Der große Streit}
Einst trat ein großer Schöpfergeist, der im nachfolgenden als Alpha benannt werden soll, an eine tote Welt heran und hauchte ihr eine neue Existenz ein. Alpha schuf sie sehr statisch mit einem großem Planeten und ein paar Monden, sowie einer Sonne, vor einem Hintergrund aus Sternenhimmel. Wie immer flossen Ewigkeiten in die Gestaltung der Details, bevor er überhaupt in Erwägung zog, Zeit zuzulassen. Als sich Alpha daran machen wollte an seinen ersten intelligenten Bewohnern zu arbeiten, trat Omega, ein anderer Weltbauer ein.
\\ Unter gewöhnlichen Umständen gehen sich die Weltenbauer aus dem Weg, um genau zu sein, seit der letzten Zusammenkunft der Axiomaten, wo sie ihre Zukunft als Weltbauer entschieden, kam es nie wieder zu einem Zusammentreffen. Omega war mehr aus Zufall über Alphas Schöpfung gestolpert und ihm durch sein Loch am Ende des Universums gefolgt und er war nicht zufrieden. Für Omega standen Alphas Konzepte im krassen Gegensatz zu seinem eigenem Verständnis einer Welt. Es folgte ein Streit, in dem sie das Gefüge der Welt mit Anomalien als Zeichen für ihre Vorstellungen einer perfekten Welt zeichneten.
\\ Am Ende lief es auf eine finale Konfrontation am Raum-Zeitriss, indem beide Existenzen vernichteten und sie eine tote, verwundete Welt zurückließen. Aurum Orbis war geboren.

\section{Neue Götter}
Eine Ewigkeit war es still in Aurum Orbis, während an den Rändern des Raumlochs, das Gefüge der Welt zerfaserte und sich so immer weitete. In der Zwischenzeit in einer anderen Realität:
\\Ein Pantheon von Götter herrschten hier über eine Schöpfung, führten untereinander Kriege auf einem weltlichen Schlachtfeld mit ihren eigenen Schöpfungen, die durch ihre religiöse Verehrung, den Göttern weitere Stärke gaben. Über die Jahrhunderte führte dieser Zyklus aus Göttlicher Intervention und Verehrung zur Entstehung eines gewaltigen Energieüberschusses zugunsten der mächtigsten Götter, die im Gegenzug immer brutaler Gegeneinander vorgingen. Am Ende kam es zu einer Apokalypse, in der das Raum-Zeit-Gefüge selbst in Stücke gerissen wurde und in dem viele der niederen Götter vergingen, doch einem kleinem Pantheon der Mächtigsten gelang es sich auf eine Scholle aus Raum und Zeit zu retten, mit der sie durch das Nichts zwischen den Welten trieben, bis sie angezogen von dem Sog der von Aurum Orbis Loch ausging in diesen hinein gesogen wurden. Bei ihrer Ankunft zerschellte ihre Scholle und die Fragmente, stabilisierten das Gefüge genug, dass sich die Realität nicht noch weiter zerfaserte, wenn sie es auch nicht ganz schloss.

\section{Die Götterkriege}
Nachdem sich die Götter es sich in Aurum Orbis eingerichtet hatten, beschlossen sie mit erstem Leben zu füllen, auch wenn sie beschlossen sich nicht aktiv in die neuen Gesellschaften einmischen wollten. Diese ersten Schöpfungen waren die Drachen. Ausgestattet mit einem großem Intellekt und eigenen gottähnlichen Kräften begannen sie bald nach ihren Ursprüngen zu graben und stießen dabei mehr oder weniger zufällig über die von den Anomalien der Schöpfer ausgehende Energie, welche sie jahrelang studierten und schließlich adaptierten. Später sollte sie als Magie bekannt sein. In der Zwischenzeit hatten sich die Götter anderen Spezies zugewandt, wobei sie angesichts ihrer eigenen schwindenden Kräfte, nun weitaus einfacherer Geschöpfe formten. Dies war der Geburt der sterblichen Rassen. Als sie schließlich die Früchte ihrer Arbeit betrachteten, fiel ihr Blick wieder auf die Drachen und sie schreckten zurück, als ihre Schöpfung zurückschaute. Die Drachen auf der Suche nach immer mehr Wissen um die neuentdeckte Magie noch weiter nutzten zu können, hatten die Drachen den Schleier durchbrochen hinter dem sich die Götter zurückgezogen hatten. Aufgrund von Uneinigkeit zwischen beiden Parteien über die wahre Natur der Welt, kam es zu einem schrecklichen Krieg, der jedoch weites gehend unbemerkt von den jungen sterblichen Rassen von statten gingen. Schnell zeigte sich, dass die Götter in einem direkten Machtkampf nicht viel ausrichten können würden, da ihre Fähigkeiten in der Gestaltung der Welt lag, wo die Drachen jedoch einen Vorteil besaßen und so erschufen sie in aller Eile und mit den letzten Resten ihrer Macht die Zwerge. Tötungsmaschinen mit nur einem Auftrag: Die Drachen zu vernichten. Ihre Zahl war denen der Drachen um ein vielfaches überlegen und dieser Vorteil führte zu einer Wende des Blattes. 

\subsection{Der Aufstieg der Elfen}
Es wäre der Untergang der Drachen gewesen, wenn nicht ein Stamm Dunkelelfen gewesen wäre. Sie gerieten zufällig zwischen die Fronten eines kleineren Scharmützels zwischen 3 Drachen und einer Horde Zwerge. Keine der beiden Seiten schenkte ihnen Beachtung, die Drachen, weil sie zu sehr mit Überleben beschäftigt waren und die Zwerge, weil sie diese auf göttliche Anweisung in Ruhe lassen sollten, wie auch die anderen sterblichen Rassen. Doch gemäß ihrer Natur stürzten sich die Dunkelelfen in die Schlacht und sie trieben einen Keil zwischen die Kontrahenten, der es den Drachen erlaubte die auf Nahkampf spezialisierten Zwerge mit ihrem Feuerodem zu Asche zu verarbeiten, während diese mehr oder weniger hilflos, aufgrund der an ihnen klebenden Dunkelelfen dies über sich ergehen lassen mussten. Nach der schicksalshaften Schlacht kamen die Drachen und Dunkelelfen zusammen. Letztere wollten die Drachen natürlich ungeachtet der Tatsache, dass sie Zeugen ihrer Macht waren, töten, aber all ihre Versuche scheiterten Kläglich. Die Drachen bald von den hoffnungslosen Versuchen gelangweilt, machten dem Stamm ein Geschenk: Frieden. Sie zogen den tief sitzenden Stachel des Hasses aus ihrem Geist, wo er bei ihrer Schöpfung gepflanzt worden war. Den Dunkelelfen war es, als öffneten sie zum ersten Mal die Augen und sie sahen sich selbst und die Welt um sich und fielen, als das Leben selbst mit all seiner Schönheit in sie einströmte, auf die Knie und weinten. Anschließend bedankten sie sich bei ihren Errettern, die selbst nicht weniger Dank zeigten. Die Elfen, wie sie sich nach diesem Erlebnis nannten, entschieden sich dazu den Drachen für ihr Geschenk zu danken, indem sie ihnen bei der Vernichtung der Zwerge beistanden. Wie eine lebende Mauer stellten sie sich zwischen die Drachen und ihre Feinde, die unter dem Odem der Drachen fielen und zurück unter die Erde, bis zum Herz der Zwerge in deren Feuern die Götter sie geschaffen hatten und schließlich gab es keine Zwerge mehr. 
\\Was folgte war die Schmiedung eines Paktes. Die Drachen gewährten ihren Rettern ihr Blut, welches sie befähigte selbst Magie zu nutzen und teilten auch einen Teil von ihrem Verständnis der Welt mit ihnen, wodurch die Elfen unermessliche Macht erhielten. Im Gegenzug versprachen die Elfen die Existenz der Drachen vor den anderen Sterblichen, ja auch vor weiten Teilen ihrer eigenen Nachkommen geheimzuhalten, damit diese in der Nähe des Herzen der Welt, welches für die Drachen eine gewaltige Magiequelle darstellte, weiter an ihrem Verständnis der Welt arbeiten konnten. So kam es zur Entstehung des Elfenvolkes und ihren stillen Wächtern den Drachenelfen. Die Götter währenddessen beschlossen, sich in Zukunft von den Drachen fernzuhalten, ein Verhaltem, dem die Drachen folgten.

\subsection{Währenddessen bei den anderen Völkern}
Nach ihrer Schöpfung herrschte unter den sterblichen Völkern, wenn man von den Dunkelelfen absieht, weitestgehend eine Art Waffenstillstand, während dem sich die Völker über die Welt verteilten und erste Siedlungen gründeten, sowie die Welt selbst erkundeten. Auch wenn die Götter sich alle Mühe gaben, sich aus den Angelegenheiten ihrer Schöpfungen raus zuhalten, wenn man von ihrem Krieg mit den Drachen absah, entstanden eine ganze Reihe von Götterkulten und wie es in der Natur der Götter lag, mussten sie dem Ruf ihrer neuen Anhänger folge leisten. Um die Reihen ihres Pantheon zu füllen, divergierten sie in eine ganze Reihe von persönlichkeitsfreien, auf einzelne Elemente spezialisierte Entitäten, die man später nur noch als Aspekte bezeichnete. 
Nebenbei stolperte man allerorts über die prähistorischen Anomalien und widmete sich der Ergründung ihrer Geheimnisse. Dabei stießen sie, ähnlich wie die Drachen auf die Magie der Schöpfer, wenn auch ihre sterbliche Natur unfähig war, sie in selber Weise zu verstehen und nutzbar zu machen, wie es den Drachen gelungen war. Doch über die Jahre gelang es ihnen die schöpferische Energie als Mana zu destillieren und erste Anwendungen zu finden. Dazu gehörten leider nicht nur friedliche, denn über die Jahre war der stumme Friede zerbröckelt und zwischen den ersten kleineren Reichen waren Kriege ausgebrochen. 

\section{Die ersten Manakriege}
Trotz ihres geringen Verständnis für das neuentdeckte Mana, setzten es die oberirdischen Völker immer wieder häufiger in seiner Rohform als Waffe ein. Bei einer gezielten Explosion von Mana in der Nähe des großem Arkeum, einer der größeren Anomalien, kam es schließlich zu einer Katastrophe, als dieses zerbrach und Unmengen an roher Energie freisetzte, die die Stadt Thalum und alle ihre Bewohner ausradierte und ein widernatürliches Land hinterließ, aus dem sich von Magie verzerrte Kreaturen erhoben und in die Welt flüchteten, diese wurden später allgemein als magische Monster bezeichnet. \\Dieses Ereignis führte später immer wieder an verschiedensten Orten zu Eruptionen von chaotischem Mana, welches die betroffene Umwelt radikal veränderte und weitere magische Monster hervorbrachte. Und dennoch schreckte dieses Ereignis die Völker nur geringfügig ab, bis eine ganze Reihe weitere Katastrophen, die unter anderem zur Entstehung des zersplitterten Kontinents führten,  alle Beteiligten zwangen, angesichts der Gefahr, die Welt durch den Einsatz roher Manabomben(siehe Seite \pageref{Manabombe}) noch weiter zu zerstören, bei ihren zukünftigen Konflikten auf Mana zu verzichten. Eine Tradition die sich auch über die Grenzen der Dunklen Zeit hielt, auch wenn ihr Ursprung durch den Einfluss der Mächtigen, die ihre Fehler zu vertuschen zu suchten, verloren ging.

\section{Die Gegenwelt}\index[Stichworte]{Gegenwelt}
\subsection{Vorgeschichte}
Unter den unzähligen Gescheiterten Welten, die die Axiomaten erschufen, gehört die Gegenwelt sowohl zu den ältesten dauerhaft existierenden Welten, als auch den größten Entäuschung für die Axiomaten. Ihr Schöpfer hat sie im Sinne seiner Ursprünglichen Aufgabe perfekt Geordnet und ausbalanciert, bevor er alles ins Gegenteil verkehrte und eine Welt des puren Chaos und der Gegensätze schaffte, wo es weder feste Regel noch irgendeine Form von definierten Grenzen, zwischen Gegensätzlichen Elementen gab. Gut und Böse, Tod und Leben, Oben und Unten, das und noch alles weitere wahren nicht mehr als Illusion oder eher Illusionen von Illusionen. Im Grunde genommen war sie nicht mehr als ein Wrack, dass allerorts Löcher hinaus ins Nichts besaß und stellenweise kollabiert war. Doch kein Axiomat, obwohl sie ihrer Bestimmung schon lange entsagt hatten, konnte sich auch nur in die Nähe dieses von Chaos regierten Ortes begeben, ohne sich innerlich zu krümmen. Als seine Nachbar-Realitäten kollabierten wurden, verschlang das Chaos diese und ihm wuchsen Metaphorisch gesehen Tentakel aus Raumzeit, die sich daraufhin willkürlich zwischen den anderen Realitäten wanden, kollabierte verschlangen und so immer weiter wuchsen.

\subsection{Die Ankunft des Pantheon in der Gegenwelt}
Die Götter von Aurum Orbis, wähnten sich bei ihrer Flucht aus ihrer Heimatrealität, als die einzigen Überlebenden, doch sie lagen falsch. Ein paar ihrer Brüder und Schwestern, gelang es wie ihnen sich auf eine Reihe von kleineren Scherben aus Raum-Zeit zu retten, mit der sie durch das Nichts glitten. Sie gerieten schließlich nacheinander durch Zufall in die `Fänge` der Gegenwelt. Viele von ihnen vergingen in dem Chaos, weil sie sich nicht an die chaotische Natur ihrer neuen Heimat gewöhnen konnten, doch einigen radikalen Göttern gelang es sich einzurichten, neue Schöpfungen hervorzubringen und sich ihr persönliches Reich zu errichten, in dem das Chaos zumindest bis zu einem gewissen Grad unter Kontrolle gehalten wurde.

\subsection{Erster Kontakt}
Irgendwann war es dann so weit, dass ein Tentakel der Gegenwelt seinen Weg nach Aurum Orbis und seinem Loch fand, der Kontakt sollte zwar Aurum Orbis Raum-Zeit Gefüge nicht weiter stören, doch es wurde eine schmale Brücke in das Herz des Chaos geöffnet, durch die die ersten Gegenwelter ihren Weg nach Aurum Orbis fanden.
Dem Gebiet um das Weltloch wurde lange Zeit nur wenig Beachtung geschenkt, da es hier weder wertvolle Ressourcen, noch Manareserven gab. Zudem war die Landschaft unwirklich und unfruchtbar, daher blieb die Ankunft der ersten Gegenweltler vorerst unbemerkt und der erste Kontakt war ein wahrer Schock, vor allem auf Seiten der angestammten Bewohner. 
\\Wie von selbst schmiedete die Menschen, Elfen, ja später sogar einige Stämme der Dunkelelfen gegen die neue Bedrohung zu einer großen Fraktion zusammen.

\subsection{Der Erste Gegenweltkriege}
Dennoch waren die schnell darauf entstehenden Kriege beinahe das Ende für die Völker in Aurum Orbis, wenn sie nicht über eine so mächtige Waffe, wie die Magie verfügt hätten, eine Macht die den Gegenweltlern glücklicherweise verschlossen blieb, wenn man von einigen Unglücklichen Zwischenfällen absah, aus denen einige gegenweltlerische magische Monster hervor gingen. Die größte Bedrohung ging vom gegenwelter Götterpantheon aus, die sich bei einer Begegnung mit ihren alten Brüdern und Schwestern, die nur noch als Aspekte existierten, einen neuen richtigen Götterkrieg, wie in den guten alten Zeiten, mit diesen versprachen. Doch den Aspekten lag nichts an diesen Konflikten, im Gegenteil, von den alten Göttern war so wenig übrig geblieben, dass sie die Herausforderung gar nicht wahrnahmen und einfach still zusahen, wie die Horden aus Gegenweltern Aurum Orbis überrannten. Es waren schließlich die Drachen, die mehr aus Neugier, als um wirklich zu helfen, sich ihren Weg in Richtung Gegenwelt schlugen, im Zuge ihrer noch immer weitergehenden Suche nach neuen Antworten auf ihre Fragen. Sie kamen allerdings nie dort an, denn das Gebiet um das Weltloch, erwies sich für sie als unüberwindbare Barriere aus Antimagie(siehe Seite \pageref{Antimagie}), die ihre Existenz auszulöschen drohte. Doch dort angekommen, in Begleitung einiger treuer Drachenelfen, machten sie der Welt ein Geschenk, sie errichteten eine hermetische Barriere aus Magie, die das Vorankommen der Gegenweltler stoppte, bevor sie sich genauso unbemerkt, wie sie gekommen, zogen sie sich auch wieder zurück in ihr Refugium am Herzen der Welt zurück. Kurz darauf konnten das Bündnis der Völker sich der letzten Gegenweltler entledigen und es folgte eine kurze Atempause, während sich alle von den Strapazen erholten.

\subsection{Der Fall der großen Barriere}
Die Barriere der Drachen sollte allerdings nur einen gewissen Aufschub gewähren und das war auch einigen aus den Reihen des großen Paktes klar, der jedoch kurz nach Abwendung der Katastrophe weitestgehend wieder zerfiel. Um sich auf einen zweiten Krieg vorzubereiten, begannen die letzten Mitglieder des Paktes eine Befestigung um das Gelände der Barriere zu errichten und mit der Ausbildung neuer Truppen zu beginnen. Dies war die Geburtsstunde der \uline{letzten Front}\index[Stichworte]{Letzte Front}, jener Einheit, die sich schwor, allzeit, auch über ihren Tod hinaus in Namen von Aigis, dem Schutzherren, Aurum Orbis vor jeder Bedrohung von außen und später auch von innen zu bewachen. Schließlich brachte eine der vielen Magieeruptionen des Arkeum die Mauer zum fallen: Der zweite Gegenweltkrieg war eingeläutet.

\subsection{Der Zweite Gegenweltkrieg}
Diesmal wurden die Horden aus der Gegenwelt mit gezückten Waffen erwartet, was zunächst auf einen Sieg oder gar einen Gegenschlag ins Herz der Gegenwelt hoffen ließ. Doch leider erwies sich der Gegner als zahlenmäßig überlegen, denn in der Zwischenzeit war es in Aurum Orbis erneut zu inneren Kriegen gekommen, die sich diesmal nicht beilegen lassen wollten und nur ein Bruchteil der alten Verbündeten unterstützte die Letzte Front bei ihrem Kampf. Doch neue Verbündete sollten kommen. Neben den vielen Bestien kamen auch einige aus der Gegenwelt, die mehr den Völkern aus Aurum Orbis glichen. Einige kamen als Flüchtlinge auf der Suche nach einer neuen, besseren Heimat. Andere um alten Feinden aus der Gegenwelt den Kampf anzusagen. Anfangs wurden sie zurückgewiesen, doch als die Lage immer aussichtsloser wurde, ergriff man die Hand der neuen Verbündeten und mit ihrem Wissen über die Geheimnisse der Gegenwelt, Magie und Götterkraft gelang es abermals die Gegenwelt abzuschotten(siehe hierzu das Tal des Zwielichts aus Seite \pageref{Tal des Zwielichts}) und die Bedrohung abzuwenden, auch wenn weite Teile von Aurum Orbis in Trümmern lagen. 

\chapter{Die Jahre nach der Dunklen Zeit}
\section{Der Neuanfang}
In den Jahren nach dem Gegenweltkrieg, hoben die letzte Überlebenden die Trümmer ihrer Vorfahren auf und begannen mit dem Aufbau, Seite an Seite mit einigen Verbündeten aus dem zweiten Gegenweltkrieg, dessen Schrecken man schnell verdrängte, um sich wieder den üblichen Streitereien und anderen Alltagsproblemen zu widmen. Und dennoch wollten einige verhindern, dass sich die Vergangenheit wiederholte und man stattdessen für die Zukunft lernte. So begann die erste wirkliche und dauerhafte Geschichtsschreibung in Form von Dutzenden Tempeln und Orden, die ihre Lehren und Philosophien an nachkommende Generationen weitertragen sollten.

\part{Völker}
\setcounter{chapter}{0}
\chapter{Kinder von Aurum Orbis}

\section{Menschen}

\section{Elfen}\label{Elfen}\index[Stichworte]{Elfen}
Auch wenn sie in ihrer jetzigen Form nicht direkt durch einen Gott geschaffen wurde, so sind sie doch die Verwandten der Dunkelelfen und werden als angestammtes Volk von Aurum Orbis betrachten. Ihre Heimat sind große unterirdische Reiche, in denen sie mit Magie Gärten anlegen. Ihre Gesellschaft gliedert sich streng in Kasten, die zwar miteinander arbeiten und leben, aber jeweils ihre Geheimnisse und Rituale für sich bewahren. In der Regel stehen sie als geschlossene Einheit gegenüber Fremden, denen sie zunächst neutral bis freundlich begegnen, solange sie sich innerhalb ihrer Reiche benehmen und ihre Nase nicht zu tief in die Angelegenheit der Kasten stehen. Um hierfür zu sorgen, existiert sogar eine eigene Kaste.

\subsection{Spezifikationen}
\begin{description}
\item[Elfische Perfektion:]
Aufgrund ihrer Langlebigkeit existieren in der elfischen Gesellschaft andere Maßstäbe für alle Bereiche ihres Lebens. Doch dies hat seinen Preis, die Kosten für das Erlernen und Verbessern von Fertigkeiten sind erheblich erhöht. Auf der anderen Seite erhalten sie einen einzigartigen Bonus auf einige von diesen:
\begin{itemize}
\item{Handwerk}
\\Elfische Handwerksprodukte sind filigran und von einer eigenartigen Ästhetik. Eine elfisch gefertigter Gegenstand ist immer um die Hälfte leichter ohne an Stabilität zu verlieren. Händler, die auf Ästhetik wert legen, sind in der Regel bereit 25\% mehr zu bezahlen. Außerdem qualifizieren sich nur elfische Waffen für elfischen Kampfstil.
\item{Kampf}
\\Mit der Zeit verfeinert ein Elf seinen Kampfstil bis zu dem Punkt, an dem er eins mit seiner Waffe wird und eine nahezu unüberwindbare Verteidigung erzielt. Einen Fehler in ihrer Parade muss erst bestätigt werden, ansonsten wird dieser als ein Grad besser behandelt.
\item{Soziales}
\\Auch in ihrem Umgang miteinander, legen Elfen eine ganz besondere Sorgfalt an den Tag. Ihre Sprache ist immer höchst melodiös. Vor allem bei anderen Völkern bleiben daher ihre Worte und ihr Gesang länger im Gedächtnis. Sprachliche Effekte haben daher die doppelte Wirkungsdauer. Auf der anderen Seite überlagert die innere Melodie die um sie herum und sie erhalten eine Erleichterung auf ihre Resistenz-Proben.
\item{Kochen}
\\Elfische Nahrung ist fast genauso zeitlos wie ihre Schöpfer und in der Regel doppelt so lange haltbar, verlieren dabei nie ihr Aroma oder ihr Aussehen, außerdem kann man elfische Speisen nicht unbemerkt vergiften.
\end{itemize}
\item[Drachenblut:]
Elfen haben im Zuge des Paktes ihr Blut mit dem der Drachen vermengt. Zwar ist es in ihren Adern nicht so potent, wie das ihrer Gönner, dennoch gewährt es ihnen ihre außergewöhnlich lange Lebenspanne, sowie eine natürliche Begabung für Magie. Elfen riskieren in der Regel keine Überdosierung, außer die Quelle übersteigt ihre eigene Manakapazität um ein 10-faches. Ansonsten besitzen sie eine gewisse Resistenz gegen Krankheiten die durch verdorbenes Mana weitergegeben werden.
\end{description}

\subsection{Kasten}
Auch wenn es überall unterirdische Elfenreiche gibt, die untereinander nur selten Kontakt haben, so sind die Kasten stammesübergreifend in ihren Traditionen und Lehren absolut identisch. Untereinander stehen sich diese sogar nähe als zu ihren stammesverwandten Kasten. Die hier angegebenen Kastennamen mögen seltsam wirken. Es handelt sich hierbei um Pseudo-Übersetzungen der elfischen Bezeichnungen.

\subsubsection{Hain der Felsblumen}
Der Ursprung der Elfen lag in einem Leben in der Natur und auch wenn sie nicht mehr viel mit den Dunkelelfen gemein haben, so fühlen sie sich unsagbar fest mit der Natur verbunden. Um sich also in ihrer neuen Heimat wohl fühlen zu können, legen die Elfen unterirdische Gärten oder sogar Wälder an. Um jedoch dies auf so unfruchtbarem Untergrund wie den Höhlen zu erreichen, haben die Mitglieder dieser Kasten eine Technik entwickelt um ein noch tieferes Verständnis von Pflanzen zu erhalten. Kern dieser Tradition ist die magische Symbiose mit einer persönliche Pflanze. 

\subsubsection{Herz der Felsen}
Neben ihren Gärten müssen natürlich auch die Tunnel und Kavernen gewartet werden. Diese Kaste verziert darüber hinaus auch die Felswände und überwacht die innersten Tunnel. Sie sind also eine Mischung aus Steinmetzen und Kämpfern, die nur sehr wenig Magie einsetzten.

\subsubsection{Wissen des Stammes}
Diese Kaste ist relativ jung und entstand, nachdem im Zuge einiger Krieger, einige Traditionen und Kulte der anderen Kasten drohten und teilweise wirklich verloren gingen. Diese Kaste genießt damit einen Sonderstatus, da die anderen Kasten ihre Geheimnisse mit dieser teilen. Allerdings verschafft dieses Wissen der Kaste nur begrenzt neue Möglichkeiten, da ihr Fokus auf der sicheren Konservierung des Wissen liegt. Sie selbst haben deshalb eine Geheimsprache und eine Reihe von Verschlüsselungsmethoden für verschiedene Überlieferungsformen entwickelt: Tänze, Gesänge, Malereien und natürlich auch Gravuren, dienen also nicht nur zur Aufwertung des Lebensstandard, sondern enthalten oft auch eine tiefere Bedeutung, die nur einem Eingeweihten klar wird. Im begrenzten Maße gibt diese Kaste auch einen Teil ihres Wissen an andere Kasten ab, um nicht selbst Opfer des Vergessens zu werden, dabei handelt es sich aber in der Regel um Schlüssel und nicht Verschlüsselungsmethoden. 

\subsubsection{Elfische Garde}
Auch wenn alle Elfen bis zu einem gewissem Grad eine Kampfausbildung erhalten, sind die Mitglieder andere Kasten in der Regel zu sehr mit der Pflege ihrer Tradition beschäftigt, um eigene Wachen aufzustellen. Die Garde erfüllt hier den Zweck, als Soldaten und Wächter ihren Stamm vor Gefahren von Außen zu schützen und über Besucher von Außen zu wachen. Sie werden vor allem für ihre Verschwiegenheit und Diskretion von den anderen Kasten geschätzt, während man sie wegen ihrem Kampfstil der eine Mischung aus Magie und Nahkampf darstellt gefürchtet werden. Ihre Magische Spezialisierung ist hierbei die Manipulation von Magnetfeldern, mit denen sie Gegner entwaffnen und Rüstungsträger bewegungsunfähig machen, während sie sich, selbst als schwer gepanzerte, leicht wie eine Feder durch Reihen von Gegnern schlagen. 

\subsubsection{Freunde in der Fremde}
Diplomatie war schon immer einer der wichtigsten Strategien der Elfen. Dieser Umstand geht zurück auf ihre Ursprünge als kriegerische Dunkelelfen zurück von denen sie sich gegenüber der anderen Völker distanzieren wollten, damit ihre neuen Siedlungen nicht ständig von Armeen überrannt werden. Aus diesem Grund studierte diese Kaste die Bräuche und Traditionen ihrer Nachbarn und war stets bemüht freundschaftliche Beziehung mit diesen zu pflegen. Mit der Zeit, als die Wirtschaft der Elfen immer größer wurde und sich ihre Reiche unterirdisch immer weiter ausdehnten, übernahm diese Kaste auch die Funktion als Händler und Führer von Fremden durch die sicheren Tunnel, um gefährliche Gebiete zu durchkreuzen. Dabei bewahrten sie aber stets die Neutralität ihres Stammes und reglementieren die Reisen, um in Kriegszeiten nicht zwischen die Fronten geraten zu können.

\subsubsection{Drachenelfen}
Im eigentlichem Sinne werden die Drachenelfen nicht als Kaste bezeichnet, vielmehr sind sie die Fadenzieher hinter dem sog. Drachenkult, einer von ihnen gegründeten Religion, in der sie als Nachfahren der Drachen auftreten. Man wird nicht in diese Kaste hineingeboren, sondern im Nachhinein auserwählt und eingeweiht. Drachenelfen sind hinter den Kulissen der Draht zu den Drachen und sorgen dafür, dass ihre Existenz ein Mythos bleibt. Sie leben in der Regel in einzelnen ihnen geweihten Tempel und werden von Mitgliedern andere Kasten oder Verstoßenen, die so ihre Schuld abarbeiten, umsorgt. Ihnen ist das direkte Wissen der Drachen zuteil geworden, weshalb sie über eine Unzahl mächtiger Zauber verfügen, mit denen sie ihren göttlichen Status aufrecht erhalten. Wenn sie nicht gerade ein paar ihrer Geheimnisse zur Entwicklung neuer Kastentraditionen nutzen oder in Notzeiten Rat und Magie zur Verfügung stellen, halten sie sich aus den Angelegenheiten des Stammes weites gehend heraus und meiden auch Besucher von Außerhalb.

\section{Dunkelelfen}\index[Stichworte]{Dunkelelf}\label{Dunkelelf}
Obwohl die Dunkelelfen bereits seit den ersten Tage in Aurum Orbis wandeln, kennen viele Bewohner dieses feindselige Volk nur aus Legenden und Geschichten. Ihre Heimat ist die tiefste Wildnis, in der sie in Stämmen als Nomaden von Ort zu Ort ziehen und auf ihrem Weg alles und jeden töten, was kein Dunkelelf ist. Sie sind gefürchtete Kämpfer, die selbst im Angesicht eines weit überlegenden Feindes keine Sekunde zögern. Aufgrund ihrer Kampfkraft und weil man die Wanderrouten der Dunkelelfen kennt und sich so vor ihren Schlachtzügen in Sicherheit bringen kann, wurden noch keine Versuche unternommen diese Monster, wie sie in Kindergeschichten genannt werden, zu erlegen. Die Motivation dabei ist ein uralter, tief in ihrem Geist verwurzelte Hass, der ihnen bei ihrer Schöpfung von Niju, auf alles was kein Dunkelelf ist. Im Kontrast zu diesem Hass ist ihre Liebe zum Stamm und ihre Opferbereitschaft um seinen Fortbestand zu gewährleisten grenzenlos. Bei ihren Reisen durch Aurum Orbis bündelt der Stamm alle Erfahrungen über das Land und seine Bewohner, die sie trotz ihrer Aversion im Sterben und danach ausgiebig studieren, in ihren Traditionen, sodass jede neue Generation noch besser angepasst an ihren ständigen Kampf mit der Welt ist. Nur selten bricht aus den Stämmen ein Dunkelelf auf, um in der Zivilisation zu leben. Diese Aussteiger sind dabei stets Exilanten, die im Zuge eines Konfliktes oder aufgrund eines eigenen Versäumnis verstoßen wurden. Man nimmt sie aufgrund vieler Vorurteile nur ungern auf und meist leben sie mit anderen Verstoßenden in den Elendsvierteln, wo sie entweder es schaffen sich mit ihrer Erfahrung im Kampf in einer Bande zu integrieren oder sie degenerieren zu einem Haufen Elend der seine Vergangenheit im Alkohol zu ertränken sucht.

\subsection{Eigenschaften von Dunkelelfen}
\begin{description}
\item[Widerborstigkeit]
Jahrhunderte in den abgelegenen Teilen der Welt haben den Dunkelelfen die Fähigkeiten gelehrt, selbst unter widrigsten Umständen zu überleben. Ein Dunkelelf kann, solange er bei Bewusstsein ist, in jeder natürlichen Umgebung, die nicht sofort tötet(z.B. Im Innern eines Vulkans oder Eiswasser aus hoher See ohne irgendetwas) überleben und sogar ihre Wunden und andere Gebrechen behandeln. Und selbst in absolut tödlichen Umgebungen überleben sie doppelt so lange, wie sonst üblich. Sie können diesen Bonus nicht auf andere Übertragen, da z.B. ein Dunkelelf auch ungenießbare bis giftige Nahrung verspeisen können.

\item[Dilettantischer Jäger]
Jeder Dunkelelf bekommt die Grundlagen der Jagd gelehrt. Sie erhalten zu Beginn die Fertigkeit Pirschen mit der Lernkomplexität A. Außerdem erhalten sie einen Bonus auf die Fertigkeiten: Fallen stellen, Wild verarbeiten und Spurenlesen.

\item[Unüberwindbare Aversion]
Auch wenn die Stämme der Dunkelelfen in Notzeiten den Wert von Zusammenarbeit mit anderen Völkern verstehen, nachdem sie ihn, in vielen Kriegen, auf die harte Tour lernen mussten, gibt sind Religion und Magie zwei Dinge, denen sie eine tiefer Aversion entgegenbringen und die ihrem ganzen Wesen widersprechen. Kein Dunkelelf kann ein Geweihter werden, bzw. wird wenn er sie automatisch annimmt(z.B. durch Vampirismus) nicht nutzen. Ebenso widerstrebt es einem Dunkelelf ein Manablütiger zu werden und sein Körper wird von selbst das Gift bekämpfen und ggf. daran verenden. Doch ihr Widerstreben macht es genauso schwer sie auf diesen Wegen zu manipulieren. Gegen sie gerichtete Zauber haben eine geringe Chance von 5\% an der, den Dunkelelf wie eine Mauer, umgebenden Ablehnung zu zerschellen. Dunkelelfen erhalten stets einen zweiten Versuch um den Effekt einer göttlichen Intervention zu überwinden, wobei sie auch ihnen positiv gesinnte Effekte abzuschütteln versuchen.
\end{description}

\chapter{Freunde aus der Gegenwelt}

\section{Nephilim}\label{Nephilim}\index[Stichworte]{Nephilim, die}
Einst waren sie Teil der Garde von \uline{\hyperref[Eron]{Eron}}, ultimativem Richter der Gegenwelt, den \uline{\hyperref[Engel]{Engeln}}. Doch auch Engel können Fehler machen, was jedoch unter Eron nicht toleriert wird oder den Sinn ihres Daseinszweck hinterfragen, wonach sie selbst entscheiden sich von ihrem Schöpfer abzuwenden. Mit dem Verlassen tritt eine schlagartige Veränderung auf: Ihre Flügel verbrennen, bis nur noch Bruchstücke oder gar nichts zurück bleibt, das Insignium von Eron auf ihrer Stirn wird unter Schmerzen entfernt, was zum Teil eine Narbe zurück lässt und ihre von Eron gegeben Waffen und Rüstungen zersplittern und zerfallen anschließend zu Staub, wobei dem Engel einige hässliche Narben zugefügt werden können. Anschließend sind sie Gejagte, sowohl von Engel, die den Makel in der Welt entfernen wollen, als auch denjenigen, die Eron einen Gefallen tun wollen, da über jedem Nephilim ein Todesurteil mit zusätzliche Belohnung ausgesetzt ist. Nephilim waren damit eine der Ersten, die sich nach Aurum Orbis flüchteten. Im zweitem Gegenweltkrieg kämpften bereits viele Inkognito in den Reihen der letzten Front und sorgten dafür, dass die zweite Welle Nephilim ohne Probleme überlaufen konnten, da sie wertvolle Informationen über die Gegenwelt und ihrer Bewohner, die für die Bekämpfung essenziell waren, mitbrachten. Nach der Schlacht zerstreuten sich die Nephilim und gliederten sich relativ schnell in die restliche Zivilisation ein, auch wenn es natürlich immer wieder zu den üblichen rassistischen Anfeindungen kam.
\subsection{Eigenschaften von Nephilim}
\begin{description}
\item[Heiliges Fleisch:]
Auch wenn die Nephilim nicht länger Gesandte Erons sind, so konnte dieser ihnen nicht nehmen, dass sie aus reinster Essenz bestehen. Durch diese kann ein Nephilim besonders leicht göttliche Macht kanalisieren, weshalb geweihte Nephilim einen Bonus auf Weihen-, Ritual- und Gabenproben erhalten. Auf der Anderen Seite läuft das Konzept der Magie ihrer Natur zu wider und es ist ihnen unmöglich manablütige zu werden.
\item[Vitalität]
Von Engel wurde oft erwartet, dass sie bei der Ausführung ihrer Pflicht häufig verletzt wurden und mit Krankheiten oder Giften in Kontakt kamen, weshalb ihre Körper unnatürlich schnell regenerieren. Diese Fähigkeit langsam auch ohne aktive Behandlung ihre Wunden zu heilen ist den Nephilim erhalten geblieben. Zudem erhalten sie einen Bonus auf alle Resistenzproben gegen Gifte oder Krankheiten.
\end{description}

\part{Geographie}
\setcounter{chapter}{0}

\part{Magie}
\setcounter{chapter}{0}
\chapter{Ursprung von Magie}
Der wahre Charakter ist allen Völkern, ja sogar den Göttern ein Mysterium. Schließlich ist die Magie selbst ein Nebenprodukt des Einflusses der Axiomaten. In der Regel räumen die Axiomaten hinter sich auf und neutralisieren sämtliches Magierauschen, weshalb in anderen Welten diese Form der Energie nicht bekannt ist. Aufgrund der unglücklichen Geschehnisse beim großen Streit wurde jedoch schlampig gearbeitet, sodass die vielen Anomalien, ja selbst weite Teile der gewöhnlichen Schöpfung über ein Magierauschen verfügen.

\chapter{Mana}\index[Stichworte]{Mana}\label{Mana}
In ihrer Rohform als energetisches `Rauschen`, welches für Magie empfängliche  Wesen wahrnehmen können oder das bei ihrer Weiterverarbeitung entsteht, ist Magie nicht nutzbar, da es selbst in direkter Nähe zu einer Anomalie, wie dem Arkeum, nur in einer geringen Konzentration auftritt und ständig voneinander weg strebt. Um sie nutzbar zu machen, muss Magie zu Mana destilliert werden. Drachen haben ihre Körper soweit umgestaltet, dass sie die Magie direkt destillieren können, während sich unter den sterblichen Völkern unzählige Methoden gebildet haben, von denen jede ihre Vor- und Nachteile besitzt. Trotz der vielen Variation gelten für jede Form von Mana ein paar konstante Regeln:
\begin{itemize}
\item Mana kann immer in den klassischen Aggregatzuständen: Fest(Kristallin), Flüssig und Gasförmig auftreten. 
\item Jedoch ist seine Dichte nicht konstant. So können bereits mit geringen Mengen Mana große Kristalle konstruiert werden, die man im Nachhinein mit Mana füllen kann.
\item Untereinander sind die verschiedenen Mana-Formen kompatible und können ihre Energie aneinander abgeben.
\item Treffen sich zwei Manareserven, so beginnt ein Manastrom in Richtung der Manareserve mit dem geringeren Mana-Volumen Verhältnis. Diese Eigenschaft ist vor allem für das Auffrischen von Manareserven oder dem Konzentrieren von Mana-Nebeln einiger Destillationsverfahren wichtig. Der Strom bricht erst ab, wenn eine Manareserve bis zum Existenz-Minimum verbraucht ist, gilt nur für Kristalline Vorkommen oder der physische Kontakt unterbrochen wird.
\item Ähnlich wie bei der Destillation, existieren auch eine Unzahl von Methoden, um dieses Verhalten zu verändern. So kann man den Fluss umpolen, den Fluss so begrenzen, dass es nicht zu einer Überdosierung kommt oder die Geschwindigkeit des Flusses verringern.
\item Wird Mana stark überdosiert erhält man eine \uline{Manabombe} \index[Stichworte]{Manabombe} \label{Manabombe}
, die durch ein weiteres hinzufügen von Mana, schließlich kollabiert, wodurch in einem großen Radius die Realität einer chaotischen Veränderung unterworfen ist, welche Lebensformen in der Regel tötet und tote Materie entweder radikal umstrukturiert oder neuen Naturgesetzen unterwirft. Aufgrund des zerstörerischen Potenzials wird streng darauf geachtet, dass vor allem Magier nur passend dosierte Mana-Mengen erhalten. Hält sich ein Magier nicht an seine Dosierung, kann eine Destabilisierung seines körpereigenen Manas, ihn vergiften oder gar töten, in diesem Fall handelt es sich um eine Form von \uline{Manabrand}. \index[Stichworte]{Manabrand!Überdosierung}
\item Mana wird bei seiner Anwendung wieder in ein `Rauschen` umgesetzt, dieses gezielte Rauschen, führt in seinem Ursprung(oft das Blut des Magiers) zu Resonanzen, die ab einer gewissen Stärken, zu einer Pseudo-Manabombenexplosion führt, welche in sehr begrenztem Umfang(im Falle eines Magiers, sein Körperinneres) schweren Schaden zufügen. Im Falle eines Magiers handelt es sich um eine Form von \uline{Manabrand}. \index[Stichworte]{Manabrand!Resonaz} 
\end{itemize} 
Als Mana destillierte Magie, kann nun durch die Verbindung mit einem Bewusstsein dazu genutzt werden, um in das Realitätsgefüge einzugreifen. Die Möglichkeiten der Veränderung sind dabei nur durch zwei Faktoren begrenzt:
\\Zum einem durch das Verständnis des Anwenders von der Realität, was bei einem rudimentären mehr auf Intuition und Erfahrung Einsatz in Form von magischen Techniken, über das mehr oder weniger Ausgeprägte Verständnis eines durchschnittlichen akademischen Magiers in Form von variablen Zaubern, bis hin zum über Jahrhunderte gesammelte Erfahrungs- und Wissensgrad eines Drachen in Form von wundergleichen Effekten. Doch auch den Drachen sind Grenzen gesetzt, denn nur der umfassende Geist eines Axiomaten ist in der Lage, die Realität von Grund auf neu zu schreiben.
\\Zum Anderen die Stabilität des Verwendeten Manas, um einem Manabrand zu entgehen.

\chapter{Anwendungen von Magie}
Magie ist ein Werkzeug, dass, vorausgesetzt man verfügt über die Notwendigen Kenntnisse für alles verwendet werden. Dies ist gleichzeitig aber auch die Prämisse bei der ganzen Sache. Während es noch vergleichsweise einfach ist unbelebte Materie oder Naturgesetze 

\chapter{Magieanwender}
Um Mana zu verwenden, bedarf es Zugriff auf eine Manaquellen, die im direkten Kontakt zum Bewusstsein steht. Bei den meisten Kreaturen wird diese Manaquelle durch eine Portion flüßigen Manas im Blut realisiert. Zusätzlich benötigt man in der Regel ein gewisses Verständnis über die Anwendung. Magische Monster besitzen aufgrund ihrer Natur einen natürliche intuitiven Draht zu ihrer Magie, weshalb sie keine Ausbildung benötigen, auch wenn sie nicht in der Lage sind ihr Wissen weiterzugeben. Alle Anderen müssen entweder trainieren oder Selbststudium betreiben. Im folgenden eine Übersicht über die verschiedenen Kategorien von Magieanwendern.

\section{Kategorien von Magieanwendern}

\subsection{Intuitive}
Wie bereits zuvor angesprochen, benötigen magische Monster keine Ausbildung, sondern nutzen ihre Kräfte intuitiv. Dabei sind diese in der Regel streng limitiert in ihrer Anwendung, wobei sie häufig zusätzlich in der Lage sind selbst Mana zu destillieren. In einigen wenigen Ausnahmen sind auch gewöhnliche Magier in der Lage ihr Mana auf diese Weise zu nutzen, auch wenn hier die gleichen Beschränkungen gelten. Ein Intuitiver Nutzer kann seine Technik nicht weitergeben und sie nicht anpassen.

\subsection{Ritualisiten}
Diese Bezeichnung haben nicht immer mit umständlichen und eigentlich nur dekorativen Ritualen zu tun, sondern bezeichnet ihre Art ihr Mana einzusetzen, anstatt sich das gesamte Gefüge der Realität ins Bewusstsein zu rufen, nutzen sie eine ihnen beigebrachte Technik, bei der sie einfach einen Hebel einsetzten und so einen bestimmten Effekt erzeugen. Sie gehören zu den am weitesten verbreiteten kultivierten Magieanwendern, auch wenn viele sich selbst nicht als solche bezeichnen würden. Denn meist ist ihr Verständnis von Magie, geprägt durch Halbwissen und ominöse Kulte und Praktiken, die meist sehr eng mit der Spezialisierung ihrer Schule/Organisation zu tun haben. Ritualisten können ihre Techniken weitergeben, allerdings können sie nicht flexible angepasst werden.

\subsection{Gelehrte}
Der wahre Einsatz von Magie erfordert ein Verständnis von Realität. Während die beiden oben genannten dieses teilweise intuitiv oder verzerrt und beschränkt durch eine Weltanschauung, sind Gelehrte diejenigen, mit dem größten Potenzial. Und gleichzeitig ist es ihr Wunsch nach vollständigem Verständnis, welches sie oft einschränkt, weshalb einige Ritualisten einen bestimmten Effekt leichter und vor allem mit weniger Aufwand erzielen können, als ein Gelehrter. Diesen Nachteil machen sie durch ihre Vielseitigkeit und ihre Möglichkeit, einen Zauber ihren Bedürfnissen anzupassen, wieder wett. Sie können zwar nicht die Zauber direkt von anderen Gelehrten oder gar die Techniken eines Ritualisten übernehmen, aber ihr Wissen/Zauber mit anderen teilen. Allerdings kann es auch hier einige Schulen/Organisationen geben, die sich auf bestimmtes Wissen spezialisieren und ihre Geheimnisse nicht an Außenstehende weitergeben.

\section{Manablut}
Wie bereits oben erwähnt, ist es bei allen Magieanwendern üblich, dass sie eine Portion flüssigen Manas in ihrem Blut tragen, um den Kontakt zwischen diesem und ihren Gedanken herzustellen. Die Variante des Manas ist dabei für das Potenzial des Magiers bis auf wenige Ausnahmen irrelevant, sondern ist vielmehr für die Natur des mit der Vergiftung durch das Mana ausschlaggebend. Drachen haben bei ihrer Entdeckung der Magie von ihren durch die Götter verliehene Macht ihren Körper und auch ihr Blut angepasst. Drachenblut ist ein besonderer Magiespeicher, der den Drachen nicht nur einen Nachteilfreien Umgang mit Mana ermöglichen, sondern darüber hinaus Manabrand nur in Ausnahmefällen zulässt. Gepaart mit der Langlebigkeit der Drachen und dem daraus resultierenden Wissensschatz zählen zu den mächtigsten Wesen in Aurum Orbis. Als sie den Pakt mit den Elfen abschlossen vermengte sich ein Teil ihres Blutes mit denen der Drachen, was jeden Elfen von Natur aus zur Magie befähigt. Ihr Blut ist zwar nicht so potent wie das ihrer Verbündeten, allerdings sind sie zumindest bis zu einem gewissen Grad gegen eine Überdosierung geschützt und Resonanzen beim Zauberwirken sind halbiert.

\section{Manabrand}\index[Stichworte]{Manabrand}
Ein Magieanwender bekommt bei seiner Initiation eine ganze Reihe von Verhaltensregeln eingegeben. Während die meisten eher ethischer Natur sind und je nach Überzeugung seiner Schule/Organisation variieren, gibt es im allgemeinen einen Block an Schulen-übergreifenden Sicherheitsvorschriften. Diese dienen dazu den Magieanwender vor einem Manabrand zu schützen. Dieser ist Folge einer Destabilisierung des körpereigenes Mana, welche entweder durch Überdosierung, Aufzehrung aller körpereigenen Mana oder durch Resonanz, hervorgerufen werden. Die Symptomen reichen von Übelkeit über Fieber und Krämpfe, bis hin zu tödlichen inneren Blutungen oder noch viel grässlicheren tödlichen Deformationen der Innereien.

\chapter{Antimagie}\label{Antimagie}\index[Stichworte]{Antimagie}
Die Letzte Schlacht von Alpha und Omega aus den Tagen des großen Streites führte bei ihrer Vernichtung, aufgrund des Kontrastes zwischen den beiden Entitäten zur Entstehung einer ganz eigenen Anomalie: Antimagie. Es ist ein magisches Rauschen, welches so gerichtet ist, dass andere Zauber innerhalb dieser nicht zugelassen werden und sich einfach einordnen. Mana ergeht es ähnlich, es kehrt schlagartig in seinen Grundzustand zurück. Im Gegensatz zu gewöhnlichem Magierauschen, breitet sich Antimagie zum Glück nicht aus, sondern vernichtet einfach alle überschüssige Magie, die aus den übrigen Anomalien austritt, was sehr zur Stabilisierung der Welt beiträgt. Antimagie lässt sich nicht zu Mana destillieren oder in sonst irgendeiner Weise konserviert werden, um sie beispielsweise als Waffe gegen Magier einzusetzen und seit der Errichtung des \nameref{Tal des Zwielicht} ist der Zugang zu ihrer sehr erschwert worden. 

\chapter{Anomalie}
Von den Axiomaten sind die Anomalien wohl die deutlichsten, wenn auch für die Bewohner rätselhaftesten, Spuren. Es handelt sich um meist um Orte, wo die Gesetze der restlichen Welt ignoriert oder völlig verkehrt wurde. Manche von ihnen haben entweder nur geringen Nutzen oder sind gar gefährlich, aber sie alle sind Quellen von Magierauschen, weshalb sich strategisch wichtige Posten meist in der Nähe dieser befinden. Die wenigen wirklich nützlichen Anomalien sind in dieser Hinsicht oft mit ganzen Festungen und einer gewaltigen Metropolen umgeben.

\section{Übersicht der Anomalien}

\subsection{Das Arkeum}\index[Stichworte]{Arkeum}\label{Arkeum}
Geordnetes Chaos, das wäre wohl die eheste Beschreibung für die mächtigste den sterblichen Völkern bekannte Anomalie. An sich handelt es sich beim Arkeum um ein überdimensionales gläsernes Tetraeder, das zu jedem Zeitpunkt als Schatten einen perfekten Regenbogen wirft. Das besondere offenbart sich erst bei Nähere Betrachtung der Oberfläche hinter der man eine gewaltige Menge von Zeichen erkennt. Lange Zeit vermuteten die Völker, besonders die Drachen in diesen Zeichen eine Botschaft, ein gewaltiges Kompendium des Wissens zu erkennen und verbrachten lange Zeit mit dem Studium der Abschriften der Oberflächen, die ganze Bibliotheken füllten, doch wie man es auch drehte und wendete, niemanden sollte und würde es gelingen Sinn in diese Zeichen zu bringen, denn ihre Ordnung ist das Chaos. Sie enthalten gleichzeitig alles und nichts und nur ein Axiomat wäre in der Lage sie zu verstehen, wenn er ihre Botschaft auch scheut, da sie von der Ebenbürtigkeit des Chaos zur Ordnung singt. 
\\Bei den Völkern hingegen führte sie bei jenen, die sich zu lange mit ihnen beschäftigten entweder dazu, dass sie nach Jahren aufgaben oder sie entwickelten eine krankhafte Besessenheit, die sie meist in ihren Untergang führte, vor allem nach den Manakriegen. Während dieser hatte man das Arkeum umgebende Thalum, welches Hauptstadt des damaligen Reiches war, mit Manabomben attackiert. Eine der Manabomben verfehlte ihr Ziel und traf das Arkeum, wo die Explosion ein Loch in das ansonsten unzerstörbare Arkeum riss, mit fatalen Folgen: Das ausströmende Chaos verband sich mit der Magie und kontaminierte den gesamten Landstrich um das Arkeum mit verdrehtem magischen Rauschen, das alle Materie und alle Lebewesen einer Umstrukturierung und teilweise neuen Naturgesetzen unterwarf. Viele Lebewesen überlebten diesen Prozess nicht, andere sollten fortan als magische Monster die restliche Welt unsicher machen.
\\Noch immer kann sich seitdem den Bereich um das Arkeum niemand nähern, ohne zu riskieren, selbst ein Opfer einer solchen Verwandlung zu werden. Die Völker haben am Rand der Quarantäne gewaltige Manakristalle aufgestellt, um eine Ausbreitung zu verhindern. Dieser Ring stellt die größte Manareserve im gesamten Aurum Orbis da, dennoch steht das gesamte Gebiet unter einem Waffenstillstand, da niemand dafür verantwortlich sein möchte, dass sich das Chaos, welches sie im wahrsten Sinne des Wortes eindämmen, noch mehr ausbreitet. Mit der Vernichtung von Thalum, wo man die Abschriften aufbewahrt hatte, wurde das Projekt zur Übersetzung dieser eingestellt.

\subsection{Die Erzseen}
Aurum Orbis wurde in seinen Grundfesten als ein statisches Univerium gedacht, alles sollte nur bis zu einem gewissen Punkt verfügbar sein, wozu vor allem Metalle zählen. Der zweite Schöpfer brach diese Regel indem er diese unteriridische Magmaseen schuf, die stets bis zum Rand mit Erzen gefüllt sein sollte. Insgesamt gibt es 7 bekannte Erzseen, von denen 6 allerdings keine Edelmetalle, wie Gold und Silber und keine von den Göttern nachträglich eingeführten Stoffe produzieren. Nur der siebte, das Herz der Zwerge, wie sein elfischer Name übersetzt ist, hat diese Eigenschaft, seitdem die Götter selbst sich an dieser Quelle bedienten, um ihre Zwerge erschufen und ihnen einige Erzbrocken aus ihrer Schöpfung in diesen fielen. \\Um alle Erzseen hat sich eine rege Industrie mitsamt Städten und Militärischen Einrichtungen, zum sicheren Abschöpfen der Lava und Gewinnung von Erzen aus den Magmabrocken. Sie alle sind stehts heftig umkämpfte Postionen von besondere strategische Bedeutung, da die Kontrolle über einen Erzsee die Möglichkeit Waffen und Kriegsgerät in Massen zu produzieren. Auch hier bildet das Herz der Zwerge eine Ausnahme, da die Elfen, hier in ihre erste Stadt nach dem Fall der Zwerge gründeten, den Standort und die Existenz vor anderen Völkern geheim halten, eine Unternehmung bei der eine Hundertschaft Drachenelfen und sogar ein Drache im Verborgenem mithilft.
\\Es gibt zwar noch vereinzelt in den unerforschten und vor allem nur schwer zugänglichen Teilen der Welt, über die auch eine ganze Reihe von Gerüchten und Legenden existieren, aber diese wurden noch nicht entdeckt.

\subsection{Das Weltherz}
Im Zentrum von Aurum Orbis, lediglich den Drachen und ihren engsten vertrauten Drachenelfen und dem ersten Götterpantheon, liegt das Herz der Welt. Es ist eine gewaltige Maschinerie aus einem unbekannten goldenem Material, welchen allen Versuchen es zu schmelzen oder herauszubrechen widersteht, die ununterbrochen tickt und arbeitet. Es handelt sich um den Ursprung der Zeit und reguliert ihren Fluss, auch wenn sich dies den Drachen, die ihre Horte hier anlegten und es seither studieren nur in Ansätzen erschlossen hat. Das einzige war sie mit absoluter Gewissheit sagen können ist, dass seine Beschädigung noch katastrophalere Auswirkung haben würde als der Schaden am Arkeum, weshalb sie das Herz mit einer ständigen Abschirmung ähnlich der, die einmal die Gegenwelt abschirmte, sodass nicht einmal die Götter Kenntnis von diesem Ort mehr erhalten konnten, geschweige denn ihn erreichen. Das erste Götterpantheon hat zwar durch ihre Zwerge Kenntnis von dem Herz erlangen können, allerdings haben sie sich stillschweigend mit den Drachen darauf geeinigt, dass sie dieses Geheimnis mit ins Grab nehmen würden.

\part{Götter}
\setcounter{chapter}{0}
\chapter{Die Aspekte}

\section{Ursprung}
Der Ursprung der Aspekte besteht aus dem Pantheon von drei Göttern, die im Nachhinein ihre Existenz als Individuen zu großen Teilen Aufgaben, um besser auf die Anbetung durch ihre neuen Schöpfungen eingehen zu können. Nur eine kleine Auswahl an Geheimkulten beten die ursprünglichen Götter an, diese Götter werden nun aufgelistet:

\subsection{Niju, Herrin über Land}\label{Niju}\index[Stichworte]{Niju}
Ihre Domäne ist die Natur und sie ist vor allem für die Flora und Fauna in Aurum Orbis zuständig. Sie ist launisch und hat sich nur selten für die Konflikte der anderen Götter interessiert. Sie ist die Schöpferin der \nameref{Dunkelelf}en, bei denen sie es eher aus Neugier, als aus Routine, gehandelt hat. Und ihre Entscheidung sich der Absprache mit ihren Mitgöttern, kein kriegerisches Volk zu erschaffen, war Folge ihres Wissens, dass sie später nicht mehr viel mit ihren animalischen oder pflanzlichen Schöpfungen spielen dürfen würde. Auch wenn die anderen Götter ihren Vorstoß wohl oder übel hinnahmen mussten, wollten sie nicht in ihre alten Muster zurückfallen, sorgten sie dafür, dass die Flora und Fauna von den anderen Völkern anvisiert wurde und man den Dunkelelfen keinen Zugriff auf die später entstehenden Aspekte gewährte, was diese aber nicht wirklich bemerkten oder interessierten.
\\\\\large Niju-Kulte
\\\normalsize Niju selbst wird von ihrer eigenen Schöpfung nicht verehrt, allerdings haben über die Jahre einige, in die Wildnis verbannte, Individuen zu ihr gefunden. Niju selbst kann ihren wenigen Anhängern nur eines bieten, freies Geleit durch selbst die tiefste und feindlichste Wildnis. Im Austausch hierfür verlangt sie lediglich, dass ihre Anhänger ein kurzes Dankesgebet für jedes erlegte Wildtier und für jedes Hindernis(wie Flüsse oder Pässe), welches sie überwunden haben.

\subsection{Ak-Mol, Hüter von Esse und Webstuhl}\label{Ak-Mol}\index[Stichworte]{Ak-Mol}
Seine Domäne sind alle Formen des Handwerks. Er steht für Stabilität und liebt die künstlerische Gestaltung von Handwerksobjekten, auch wenn er dabei nie den eigentlichen Zweck aus den Augen verliert. Er hat ganz nach seinen Vorstellungen die\nameref{Mensch}en erschaffen, die mit seiner Zielstrebigkeit die Dominanz in den ersten Zeitaltern übernehmen konnten. Als Aurum Orbis später durch Kriege und Katastrophen schwer gebeutelt wurde trat Ak-Mol an einzelne Individuen an und gab ihnen in entschiedenen Momenten eine Eingebung. So sind einige der mächtigsten Artefakte aus den Gegenweltkriegen, durch sein Wirken entstanden. Als sein Eingreifen jedoch von den Gegenwelt-Göttern zu Bemerken begannen und ihre Kriegstreiberei verstärkten zog sich Ak-Mol in die Verborgenheit hinter den Aspekten zurück. Erst nach den Gegenwelt-Kriegen begann er erneut auf die sterbliche Welt hinab zu sehen. Wie auch die anderen des Göttertrios, vermied auch er es all zu sehr unter den sterblichen Völkern aufzufallen, sondern sie den Aspekten zu überlassen.  Und dennoch konnte er nicht umhin die kunstvollsten Schöpfungen zu bewundern und ihren Schöpfern respekt zu Zollen. So entstand ein ausgewählter Kreis von Ak-Mol Geweihten.
\\\\\large Ak-Mol-Kult
\\\normalsize Ak-Mol macht sich nicht viel aus Verehrung und Gebeten sondern achtet nur handwerkliches Geschick. Seine Gaben bestehen auch nicht direkt aus Macht, sondern vielmehr aus Wissen. Einem Geweihtem des Ak-Mol kann von den Erfahrungen und Geheimnissen seiner Vorgängern profitieren, um seine Künste auf eine noch höhere Ebene zu bringen. In seltenen Fällen ist Ak-Mol von den draus resultierenden Objekten so begeistert, dass er ihnen einen finalen Schliff verleiht, wodurch ein Artefakt entsteht.

\section{Aufbau der Aspekt-Kulte}
Die Tradition der Aspekte ist bei allen Völkern gemeinsam entstanden, weshalb es innerhalb eines bestimmten Aspekt-Kultes weltweit keine Unterschiede gibt. Ein Aspekt verkörpert im Gegensatz zu den unbekannten Göttern lediglich sein Element ohne dabei einen eigenen Charakter zu besitzen, vielmehr stellt jeder Geweihte, so werden die Anhänger eines Aspektes genannt, diesen bis zu einem gewissen Punkt selbst da. Aus diesem Umstand gewinnen sie dann die Kraft für ihr Wirken. Es entwickelt sich ein Wechselspiel, aus Verehrung des Aspektes in den sog. Weihen, aus welcher sie göttliche Kraft ziehen, die sie bei einem Ritual wieder verbrauchen. Jeder Kult besitzt unter Umständen eine eigene Hierarchie der teilnehmenden Geweihten und teilweise eine Reihe von Orden und Organisationen, die jeweils die Anwendungsmöglichkeiten für ihre göttliche Kraft erweitern.

\section{Übersicht über die Aspekte}

\subsection{Aigis, der Schutzherr}
Er gehört wohl zu den am meisten verehrten Aspekten, denn er hält seine Hand schützend über jene, die sich nicht aus eigener Kraft verteidigen können. Seine Tempel, überall selbst an den entlegsten Orten vorhanden, sind nicht nur Orte der Andacht, sondern dienen auch der Ausbildung von Wächtern, sowie der Unterbringung von Schutzlosen und Wandernden. 
\\\\Weihen:
\begin{itemize}
\item In Not geratene Beschützen
\\Sei es durch Räuber oder Wölfe, wenn ein Geweihter einen Wanderer oder eine Karawane vor einem mehr oder weniger großem Übel beschützt wird dies Belohnt.
\item Lebensband
\\Der Geweihte bindet sein Leben an eine andere Kreatur, für deren Schutz er fortan zuständig ist. Stirbt sein Schützling eines nicht natürlichen Todes, so stirbt der Geweihte und der Schützling ersteht an einem sicherem Ort wieder auf. Der Schwur endet erst mit dem natürlichen Tod eines der Beiden und versorgt den Geweihten jeden Mond mit einer gewissen Menge Energie.
\item Finales Opfer
Im Angesichts der Entscheidung zwischen seinem Leben und dem vieler, kann ein Geweihter sein Leben über das der Übrigen stellen. Im Gegenzug für die völlige und endgültige Vernichtung seiner eigenen Existenz, erhält der Geweihte in seinen letzten Momenten eine gewaltige Menge an Energie.
\end{itemize}
Rituale:
\begin{itemize}
\item Geteilter Schild
\\Der Geweihte verschanzt sich hinter seinen Schild und spricht eine kurzes Gebet. Daraufhin erscheint ein Geisterhafter Schutzschirm aus seinem Schild, der die neben ihm stehenden und ihm deckt, als trügen sie einen ähnlichen Schild der nächsthöheren Größenkategorie. Würde es sich hierbei um einen Turmschild handeln, so erhalten alle eine volle Deckung.
\end{itemize}
\subsubsection{Die Wächter}\index[Stichworte]{Aigis!Wächter}
Aurum Orbis ist ein gefährlicher Ort, so war es schon immer und so wird es auch bleiben. Und genauso wird es auch immer die geben, die nicht über die Stärke verfügen sich selbst zu verteidigen. Die Wächter sind ein Orden von Kämpfern, die sich unerbittlich zwischen die Gefahren der Welt und den Bedürftigen stellen. Sie halten Wache an den entlegensten Pässen wichtiger Handelsrouten oder streifen in kleinen Gruppen durch die Welt, um nach heranziehenden Gefahren Ausschau zu halten. Reisende Wächter gelten als höchst respektabel und finden in der Regel bei einfachem Volk immer eine einfache Unterkunft und eine warme Mahlzeit. Zu ihren Spezialitäten gehört eine besondere Kampfaufstellung, die sie zu einer nahezu undurchdringlichen Mauer werden lässt, die durch den göttlichen Beistand ihr Defensivpotenzial noch einmal gewaltig steigern können.
\subsubsection{Die Letzte Front}\index[Stichworte]{Aigis!Letzte Front}
Bevor man im großen Stil mit dem Aufstellen der Wächter begann, kam es nach dem ersten Gegenweltkrieg zur Bildung der letzten Front, die heute als legendäre Behüter von Aurum Orbis gelten. Sie stehen ganz im Zeichen von Aigis und gelten als das letzte, ultimative Bollwerk. Jeder einzelne hat einen Schwur, gesegnet durch Aigis selbst, wie man sagt, der ihn über den Tod hinaus an seinen Dienst bindet, geleistet. Ihre erste und einzige Schlacht zu Beginn des zweiten Gegenweltkrieges, hat nur ein einziger Herold überlebt, der sein Vermächtnis, ein Banner und ein Schlachthorn an seine Nachkommen weitergereicht hat, auf das in Zeiten größter Gefahr, mit diesen Zeichen die letzte Front wiederauferstehen lassen, um die Gefahr abzuwenden. Doch bis dahin werden die Artefakte und das Erbe streng behütet vor den Augen der Welt, damit diese sich nicht auf der Gewissheit, im schlimmsten Fall errettet zu werden, ausruht.

\subsection{Ixania, das Licht der Welt}\index[Stichworte]{Ixania}\label{Ixania}
Sonne und Mond sind zwei Lebenswichtige Dinge für fast alle Bewohner von Aurum Orbis. Auch wenn das Mondlicht nur eine sekundäre Rolle für das Pflanzenwachstum spielt, so ist es dennoch Wegweisend und hat für viele Rituale eine hohe Bedeutung. Für die Geweihten von Ixania, besteht daher kein Unterschied zwischen beiden. Trotzdem gilt zu beachten, dass für bestimmte Rituale(z.B. Erschaffung von Zwielicht) doch eine Grenze zwischen Sonnen- und Mondlicht gezogen werden. 
\\\\Weihen:
\begin{itemize}
\item Morgentau fangen
\\Wie der Name schon sagt, sammelt der Geweihte den Morgentau bei Sonnenaufgang in einem geeigneten Behältnis, anschließend wird er im Zuge eines kurzen Gebetes zu Ixania getrunken. Alternativ kann es zur Destillierung von Zwielicht verwendet werden oder zur späteren Vervollständigung der Weihe aufbewahrt werden. Morgentau muss dabei innerhalb eines Sonnenzyklus verbraucht werden.
\item Segen der verbrannten Sonnen
\\Dieses Ritual kann mit dem untergehen der Sonne begonnen werden. Zunächst wird ein Feuer entzündet, dabei ist nur frischestes Holz zu verwenden. Sobald die Mitte der Nacht vorüber gezogen ist, kann man mit der Phase zwei beginnen. Die Reine Asche, ohne Reste von Ruß oder verkohltem Holz muss mit Wasser aus einer unterirdischen Quelle vermengt werden, bis eine silbrige Paste entsteht. Diese wird anschließend in einer Silberschüssel ausgestrichen und ins Mondlicht gehalten. Nach einer Stunde wird die Paste im Zuge eines Gebetes auf der Haut des Geweihten verrieben, worauf sich die Asche zu Ruß wandelt, den man erst nach Sonnenaufgang abwaschen darf. Alternativ kann man die Paste, sofern man sie nicht dem Sonnenlicht aussetzt für einen Mondzyklus aufbewahrt werden, um sie später zu benutzen oder sie mit geweihtem Morgentau zu Zwielicht zu destillieren.
\end{itemize}
%\\\\Rituale:
\subsubsection{Zwielicht}\index[Stichworte]{Ixania!Zwielicht}\label{Zwielicht}
Eines der ältesten und wahrscheinlich einfachsten Rituale für einen Geweihten von Ixania, ist die Destillierung von Mond und Sonnenlicht zu einer Schemenhaften Substanz, namens Zwielicht. Es verfügt über eine gewisse Abschirmwirkung gegen göttliche Interventionen, weshalb es oft zusammen mit magischen Barrieren zur Abschirmung von Quellen potenzieller Gefahr genutzt wurde. Allerdings hat Zwielicht eine berauschende Wirkung, bei Kontakt, die sogar abhängig machen kann, mit schrecklichen Entzugserscheinungen. Weil es inzwischen bessere Möglichkeiten gibt, um göttliche Intervention zu blockieren, wurde der allgemeine Gebrauch des Zwielicht-Ritus streng reglementiert. Einer der wenigen Orte, an denen noch Gebrauch von diesem Ritus gemacht wird, ist das \uline{\nameref{Tal des Zwielicht}}.

\chapter{Pantheon der Gegenwelt}

\section{Einführung}

\section{Wirken außerhalb der Gegenwelt}
Das \uline{\nameref{Tal des Zwielicht}} leistet ganze Arbeit bei der Abschirmung göttlicher Interventionen aus der Gegenwelt. Wenn man von wenigen Ausnahmen absieht, die sich fest in Aurum Orbis niedergelassen haben, sind die Götter der Gegenwelt völlig machtlos, was vor allem ihren direkten Gesandten zu schaffen macht, deren Existenz bis zu einem gewissen Grad von der göttlichen Macht abhängig ist. Dennoch können Kulte in Aurum Orbis mit ihrer Verehrung die Gegenwelt erreichen und die Götter besitzen dabei zumindest ein Mindestmaß an Möglichkeiten um zu antworten, meist in Form von Visionen. Auf beiden Seiten der Barriere kämpfen Anhänger um die Zerstörung dieser, um erneut einen Kontakt mit der anderen Seite herzustellen. Doch zum Glück scheiterten all diese Versuche auf Seiten von Aurum Orbis an der Verworfenheit der einzelnen Kulte untereinander.

\section{Übersicht über die Götter}

\subsection{Eron, der Gerechte}\index[Stichworte]{Eron}\label{Eron}
Man sagt, dass Erons Streben nach Gerechtigkeit geprägt war von dem höherem Ziel, Frieden unter den Göttern zu schaffen, weswegen er oft ein Diplomat oder Berater war, im Zuge von Ragnarök, in dem alle seine unschuldigen Schöpfungen umkamen, verloren gingen. In der Gegenwelt ist er zumindest nur noch als ein Fanatiker und Streiter für eine Ordnung geworden, deren Maßstäbe unvereinbar sind mit der Natur sterblicher Wesen, bekannt. Er ist in der Gegenwelt das ultimative Gericht, der eine sehr überspitzte Form von Gerechtigkeit vertritt: Ein Mörder muss mit dem Tod bestraft werden, ein Geretteter steht stets in der Schuld seines Helfer und muss diesem im Zweifel bis zum Tod gehorsam dienen. Selbst eine im Spaß verteilte Schelle oder ein böses Wort werden streng geahndet. Nicht selten mussten sich sogar die Verteidigen, die nach Gerechtigkeit schrien, da sie in den Augen Erons, selbst Schuld an ihrem Elend sein, ja sogar sich sogar selbst mit Schuld beluden, weil sie voreilig den Finger erhoben. Und da die Gegenwelt eine Welt ständigen Chaos ist, hat seine persönliche Elite-Schöpfung die \uline{\nameref{Engel}} immer etwas zu tun und Erons Macht wächst mit jedem Tag. Bei dem Kontakt mit Aurum Orbis waren seine Diener unter den ersten, mit der Mission die Gerechtigkeit in diese neue Welt zu bringen. Er selbst ließ sich dabei stets von dieser neuen Welt berichten, die in seinen Augen, fast eine noch größere Katastrophe war als die Gegenwelt, da man ihm solchen Widerstand entgegen brachte und man sich einer Macht(Magie) bediente, die unverdient in jedermanns Kontrolle fallen konnte und als die Drachen den ersten Gegenweltkrieg mit ihrer Barriere beendeten, hat Eron Jahrzehntelang seine Engel auf der Innenseite patrouillieren lassen, auf der Suche nach einer Schwachstelle, während er die wenigen Agenten, die noch in Aurum Orbis waren antrieb mit ihrer Mission fortzufahren. Im zweitem Gegenweltkrieg sorgte er dafür, dass seine Engel wesentlich besser gerüstet in die Schlacht zogen und trieb von allen Göttern die Eroberungsversuche am beständigsten fort. Nach seiner erneuten Abtrennung durch das \uline{\nameref{Tal des Zwielicht}}, kehrte er bis auf weiteres zurück zu seinem üblichen Kampf um Gerechtigkeit in der Gegenwelt und der Jagd auf seine gefallenen Kinder, die \uline{\nameref{Nephilim}}.
\subsubsection{Eron Verehrung}
Eron lebt vom Streben nach Gerechtigkeit. Jedes Urteil, welches nach den Maßstäben seiner Gerechtigkeit getroffen wurde, stärkt ihn und darüber hinaus verlangt es Eron nach keiner Verehrung. Wer ein von Eron oder seinen Gesandten gesprochenes Urteil vollzieht hat damit bei Eron etwas gut. Man kann dies entweder zum Begleichen einer Schuld die man später auf sich lädt geltend machen oder auf eine Reihe niederer Gaben Zugreifen:\\\\Gaben \& Rituale:
\begin{description}
\item[Ruf nach Gerechtigkeit:]
Nach der Gerechtigkeit durch Eron oder seiner Engel zu rufen, ist eine gefährliche Sache, da man entweder selbst Opfer des Urteils wird oder einfach bestraft wird die Zeit und Energie für Nichtigkeiten verschwendet zu haben. Man kann jedoch einen Gefallen bei Eron einfordern, sodass man zumindest für letzteres nicht bestraft wird und sich die Engel wirklich Zeit für den Fall nehmen. 
\end{description}

\subsection{Askon, der Gefallene Gott}\index[Stichworte]{Askon}\label{Askon}
Askons Geschichte reicht zurück bis zu den ersten Tagen der Götter, lange bevor Ragnarök ihre Welt zerstörte. Damals waren die Götter bereits in ihren ewigen Schlachten um die Position des obersten Gottes am Kämpfen und während die meisten von ihnen, dies mit purer Stärke suchten, war Askon wahrscheinlich der Listigste. Er hatte sich die Domäne der Nacht und Schatten erwählt und seine Kreaturen strichen dann durch die Welt, wenn andere schliefen. Zwar konnten sie nicht selbst die Macht an sich reißen oder ihre Widersacher ausschalten, aber in Ruhe ihrer Verehrung für ihren Vater nachgehen. Askon schließlich griff mit seiner Macht nach der Sonne selbst, um sie für immer zu zerstören und die Welt in eine ewige Nacht zu hüllen, die niemand, außer seinen Kreaturen hätte überleben können. Nur durch Zufall und im allerletzten Moment erkannten die anderen Götter die drohende Gefahr und handelten sofort und ohne die geringste Uneinigkeit untereinander. Sie nutzten einen Großteil ihrer Macht und belegten Askon mit einem schrecklichen Fluch, der ihn auf alle Zeit in die Schatten verbannte, außer Stande sich der Sonne zu nähern und ihr Schaden zufügen zu können, der Fluch sollte sich auch an seine Schöpfungen übertragen und sie so zu einem ewigen Leben im Schatten zwingen. Außerdem einigte man sich darauf eine Auge auf Askon zu werfen und beim kleinsten Anzeichen einer Intrige seine Kräfte gegen ihn und seine Schöpfungen zu richten. Askon wurde regelrecht aus dem Pantheon verstoßen und musste Unterschlupf in den Schatten bei den Sterblichen zu suchen, wo er lauerte, wartend auf eine Gelegenheit sich zu rächen und alle zu übertrumpfen. Auch nach seiner Ankunft in der Gegenwelt sollte sich aber erst einmal nichts an seiner Situation ändern, denn das Pantheon der Gegenwelt achtete weiterhin sorgsam auf jeden seiner Schritte.\\Doch schließlich ergab sich eine Gelegenheit für ihn als Charon an ihn herantrat. Charon selbst war wenig erfolgreich mit seinen Untoten, was nicht zuletzt an seiner persönlichen Beschränktheit lag, durch die er als zu oft von seinen Widersachern ausmanövriert wurde. Er selbst war bei Charons Verbannung dabei gewesen, wenn er damals noch nur als Wächter für das Equilibrium gewesen war und war Zeuge von seiner Raffinesse geworden und erhoffte sich von dem in seinen Augen gebrochenem Askon einen wertvollen Verbündeten zu finden. So trat er, als es eigentlich seine Aufgabe sein sollte, ein Auge auf Askon zu werfen, an jenen heran und bot ihm an, hinter dem Rücken der anderen an einer neuen Schöpfung zu arbeiten. Askon nahm die Gelegenheit wahr und schuf mit Charon den Fluch des Vampirismus. Doch als Charon seine ersten Vampire schuf, um sie in einem Gefecht auszuprobieren, tat Askon das Selbe, nur das er seinen Vampiren mit Rat und Tat zur Seite stand. Was folgte war die langsame Abschlachtung von Charons Vampirbrut, bis nur noch Askons Schöpfungen verblieben, die, bevor die anderen Götter noch reagieren konnten, sich mit der Gesellschaft vermengten und sich dem direkten Zugriff durch die Götter entzogen. Charon wurde für seinen Fehler schwer bestraft, als man allgemein die Nekromantie aus allen Gesellschaften verbannte, doch Askon und seine Vampire konnten der Jagd nach ihnen entgehen und entkamen schließlich im Zuge der Gegenweltkriege nach Aurum Orbis. Askon, inzwischen Müde von den ganzen Götterkriegen beschränkte sich daraufhin nur noch mit der Pflege seiner letzten und einzigen Schöpfung, den Vampiren. 
\subsubsection{Askon Verehrung}
Askon selbst interessiert sich nicht für Verehrung abgesehen von denen der Vampire(siehe auf Seite \pageref{Vampire:AskonGaben}). Da diese manchmal aber selbst ihre Beute in ebensolche organisieren, hat Askon eine Reihe von seinen alten Weihen an die Vampire zur Weiterverbreitung gegeben, ein Geweihter, selbst ein Vampir kann für diese Form der Verehrung in der Regel keine Gegenleistung erwarten.
\\\\Weihen:
\begin{itemize}
\item Kodex der Nacht
\\Der Geweihte, isst, arbeitet, sprich lebt nur zwischen Sonnenuntergang und Sonnenaufgang und verbringt die Zeit tagsüber nur innerhalb der Schatten, wo er meditiert oder schläft. Er darf während des Tages kein Wort sprechen. Eine Sonnenfinsternis gilt als Nachtzeit. Jede vollendete Woche in der der Kodex ungebrochen befolgt wurde, zählt als Verehrung.
\item Opfer für die Schatten
\\Der Geweihte verzichtet im Zuge eines feierlichen Schwurs, der in einer Neumondnacht abgelegt werden muss, auf seinen Schatten und ihren Schutz. Bis zur nächsten Neumondnacht darf sich der Geweihte nicht mehr im Schatten befinden, sondern stets von einer Lichtquelle beleuchtet werden, ansonsten beginnt er sich aufzulösen. Zusätzlich wirft er für diese Zeit keinen Schatten und verfügt auch über kein Spiegelbild. Löst sich der Geweihte aufgrund von Kontakt mit Schatten auf, ist er unwiederbringlich verloren und seine Essenz geht komplett zu Askon über. Dieser Schwur wurde vor seinem Fall häufig von seinen humanoiden Agenten abgelegt, bevor sie sich auf eine Infiltrationsmission unter die anderen Völker begeben haben, um bei Enttarnung sich selbst am Reden hindern zu können.
\end{itemize}


\subsection{Charon, der Fährmann}\index[Stichworte]{Charon}\label{Charon}
Die Seelen der von den Göttern geschaffenen Kreaturen enthalten ein gewaltiges Potenzial an Macht für einen Gott, allerdings hat dies eine entscheidende Konsequenz: Ein diamantene Kern der Seele wird unweigerlich zu einem eigenem neuem Gott, wodurch der Konkurrenzkampf im Pantheon nur noch mehr verstärkt wird. Damit diese Praxis nicht Überhand nehmen konnte, erschufen die Götter einen Sammelplatz für Seelen, wo sie, bis zu ihrem Zerfall durch Vergessen, gelagert werden sollten, das Equilibrium und postierten einen Wächter vor dessen Eingang, betraut mit der Aufgabe die Seelen der Verstorbenen aus der irdischen Welt dorthin zu tragen und dafür zu sorgen, dass niemand unbemerkt in das Equilibrium eindringen konnte. Doch der Plan sollte nicht ganz aufgehen. Die Götter übersahen, dass jede Seele bei ihrem Zerfall eine geringe Menge Energie abgibt, nicht genug um auf den ersten Blick bemerkt zu werden. Charon bemerkte sie auch nur durch Zufall und sammelte sie, mehr weil er besorgt war sonst nicht seinem Auftrag gerecht zu werden. Schließlich aber bemerkte er seine neue Macht. Es sollte Geburtsstunde eines neuen Gottes sein, als Charon begann sich eingehender mit den ihn überantworteten Seelen beschäftigten. Und über die Zeit eignete er sich wertvolles Wissen an. Ein Gott vermochte vielleicht nicht eine Seele als Energiequelle zu nutzen ohne einen neuen Gott zu erschaffen, weil er aufgrund seiner Macht viel zu grob zu Werke ging, doch Sterbliche besaßen in dieser Hinsicht ein gewisse Geschicktheit, die sie zu einer solchen Leistung befähigte. Charon selbst war zwar nicht in der Lage wie andere Götter durch Verehrung Macht zu sammeln, aber er besaß nun seinen eigenen Weg um sich mit einer Stufe zu den Göttern zu gesellen. Er begann hinter ihrem Rücken widernatürliche Rituale zur Zersetzung von Seelen zu entwickeln, deren Energie direkt an ihn fließen sollte. Mit diesem Wissen trat er an jene heran, die ihre Götter verlassen hatten und wie nach Charon nach Macht strebten. Dies war die erste Stunde der Nekromantie, als die ersten Seelen Charons neuen Ritualen zum Opfer fielen und er mit der neu gewonnen Macht seine Helfer mit untoten Dienern ausstattete. Als die Götter sein Treiben bemerkten und ihn zur Rede stellen wollten, hatte Charon bereits genug Macht gesammelt, um sich gegen sie zu behaupten. Er sorgte dafür, dass ihnen das Geheimnis hinter seinen Ritualen verborgen blieb und verteidigte seinen Seelenhort, der ihn neben den Nekromanten, auf die die einzelnen weltlichen Helfer der übrigen Götter Jagd machten, mit Energie versorgten. Nach seiner Ankunft in der Gegenwelt gelang es ihm zwar nicht, ein eigenes neues Equilibrium zu erschaffen, aber seine Lehren der Nekromantie waren immer noch sehr begehrt. Doch Charon selbst konnte sich nie groß behaupten, was vor allem daran lag, dass man ihn bei seiner Schöpfung nur mit einem Mindestmaß an Intellekt geschaffen hatte, welche ihn sowohl als Strategen, noch als kreativen Schöpfer ausschieden ließ und er auf das Potenzial seiner Diener angewiesen war. Nach der Episode mit \uline{\hyperref[Askon]{Askon}} geriet er zusehends in Bedrängnis und nutzte seine erste Gelegenheit um aus der Gegenwelt nach Arum Orbis zu schlüpfen. Hier gelang es ihm schließlich, wieder ein Equilibrium zu errichten und die Lehren Nekromantie zu verbreiten. Seitdem genießt er mehr oder weniger seine neugewonnene Ruhe, um an sich selbst zu feilen und seinen Seelenhort zu pflegen.
\subsubsection{Nekromantie}\label{Nekromantie}\index[Stichworte]{Nekromantie}
Charon selbst kann nicht durch Verehrung an Macht gewinnen und es gibt keine Charon-Geweihten, stattdessen stellt Charon eine Reihe von Ritualen zur Verfügung und die Ausführenden dieser werden allgemein hin Nekromanten genannt. Zu Beginn seiner Karriere beginnt ein jeder Nekromant mit einem Ritual der Opferung:
\begin{itemize}\label{Nekromantie:Ritual der Opferung}\index[Stichworte]{Nekromantie!Ritual der Opferung}
\item Vorbereitung auf das Ritual, der Seelenstein
\\Um die Seele aus dem Körper zu ziehen und sie für die Weiterverarbeitung zu konservieren, benötigt es einen zur Seele affinen Seelenstein. Dabei vermengt man 5 Unzen Blut vom Opfer oder eines anderen angehörigen seines Volkes mit einer Unze Materials, welches dem Schöpfer des Opfers geweiht ist. Einfaches Weihwasser ist dabei schon ausreichend. Die Mischung muss dann bei Mondlicht, welches Brücke für Charon zwischen Irdischer Welt und seinem Equilibrium ist, zusammen mit einem Metall, für ungefähr eine Münze, in einen Schmelztiegel gegeben werden. Das geschmolzene Metall wird daraufhin in einem Wasserbecken ausgekühlt. Der Klumpen wird anschließend noch mit einer dünnen Schicht Blut vom Beschwörer bestrichen, während er einen Treueschwur zu Charon leistet. Dies verhindert, dass andere Götter sich an der Energie aus der Seele des Opfers vergreifen können. 
\item Beschaffung der anderen Utensilien:
\\Für das Ritual werden außerdem noch folgende Dinge benötigt: Ein Pinsel, ein Messer oder vergleichbare Klinge, einen Eimer mit Wasser und einen Lappen, sowie Nadel und Faden. Außerdem sollte eine Esse mit Amboss und Hammer bereit stehen.
\item Das Opfer wird zunächst mit ausgestreckten Gliedmaßen auf dem Rücken fixiert. Ein Knebel oder eine Betäubung werden empfohlen, um das Opfer am Schreien zu hindern, allerdings ist dies nicht notwendig.
\item Anschließend muss aus dem Blut des Opfers ein Kreis gezogen werden. Ein zweiter Kreis sollte um Esse und Amboss gezogen werden, falls man diese nicht in den ersten Kreis mit einschließen kann. Sobald das Blut vollständig getrocknet ist und keine Lücken  mehr nachgezogen werden müssen, kann mit dem zweiten Schritt angefangen werden.
\item Mit dem Messer wird nun der Brustkorb aufgeschnitten und der Seelenstein auf das Herz gelegt, alles bis auf das Herz selbst kann dabei ruhig zerstört werden. Anschließend muss man den Brustkorb wieder verschließen, indem man das Fleisch mit Nadel und Faden zusammenflickt. Dieser Schritt muss beendet werden, solange das Opfer noch lebt!
\item Nun muss man ohne den Kreis zu verlassen auf den Tod des Opfers warten, sollte man zu gut gearbeitet haben, darf Gift gespritzt werden oder das Opfer erwürgt werden, allerdings sollte möglichst wenig Blut vergossen werden und das wenige darf unter keinen Umständen den Kreis durchbrechen.
\item Sobald das Opfer verstorben ist kann der Brustkorb wieder geöffnet werden und der Seelenstein entnommen werden.
\item Anschließend muss der Seelenstein komplett vom Blut des Opfers befreit werden und erneut mit dem Blut des Nekromanten bestrichen werden, der Treueschwur an Charon sollte wiederholt werden. Unter Nekromanten ist man sich über diesen Schritt nicht ganz einig, aber doppelt hält besser.
\item Anschließend muss der Seelenstein in der Esse zum Glühen gebracht werden.
\item Der Stein muss anschließend mit einem Hammer zerteilt werden.
\item Die einzelnen Bruchstücke müssen nun abkühlen, bevor sie aus dem Blutkreis entfernt werden dürfen.
\item Von diesem Punkt kann der Nekromant die einzelnen Bruchstücke zum Bezahlen der Seelenkosten für andere Nekromantie-Rituale verwenden.  
\end{itemize}
\paragraph{Rituale}


%ABSCHNITT KREATUREN---------------------------------------
\part{Kreaturen}
\setcounter{chapter}{0}
\chapter{Flora und Fauna von Aurum Orbis}

\begin{description}

\item[Jungdrache,]der \index[Kreaturen]{Jungdrache}
\newline Nicht alle Drachen haben damals nach ihrer Entdeckung der Magie, sich diese neue Kraft angeeignet und verfolgten das Ziel, die Wahrheit über das Wesen der Welt zu ergründen. Sie genossen ihr einfaches Leben, weshalb sie in den Götterkriegen auch von den Zwergen verschont wurden. Über die Jahrhunderte, in denen sie sich paarten und wenn sie ihre Zeit für gekommen sahen, ihr Leben beendeten, degenerierte dieser Schlag von Drachen mit jeder Generation, wodurch sie kleiner und vor allem weniger göttlich wurden, als die Drachen bei ihrer Schöpfung waren. Dennoch sind sie immer noch imposante und mächtige Kreaturen, deren Atem alles zu Asche verbrennen konnten und dem kein natürliches Material der Welt stand halten konnte. Doch ähnlich wie ihre philosophischen Verwandten tief unter der Erde, meiden Jungdrachen den Kontakt mit der restlichen Zivilisation. Sie leben dabei entweder in einem Hort, um interessante Objekte, wozu meist Schätze gehören, aufzutürmen oder streifen durch die Wildnis, auf der Suche nach Abenteuern, einen guten Kampf oder um die Schönheit der Welt zu erkunden.

\item[Zwerg],der\index[Kreaturen]{Zwerg}
Eine Waffe der Götter, die einst im Krieg gegen die Drachen geschmiedet hatte, als diese versuchten hinter den Göttern nach der tieferen Wahrheit zu suchen. Erschaffen im Herz der Zwerge aus von den Göttern geschaffenen Materialien, vermehrten sich die Zwerge durch den Zugang zu unerschöpflichem Erz wie die Fliegen und schwärmten bald aus, um den rebellischen Drachen ein Ende zu bereiten. Ihr Siegeszug sollte Enden als sie aufgrund ihrer Programmierung andere Lebewesen als ihre Ziele zu ignorieren und nicht zu verletzten überlistet werden konnten. Eigentlich sollte es keine Zwerge mehr geben, aber bei ihren Streifzügen durch die jungen Reiche anderer Völker hinterließen diese Titanen eine Spur der Verwüstung, etwas was diese nicht hinnehmen konnten, weshalb sie ihren eigenen sehr einseitigen Kampf mit den Zwergen führten. Diese waren ausgestattet mit ungeheurer Stärke und lediglich das durch Magie verstärkte Feuer war in der Lage ihre Panzerung zu zerschmelzen, dafür waren sie nur mit einem Minimum an Intelligenz ausgestattet, da sie über eine telepathische Verbindung im Kollektiv dachten. Gemeinsam schaffte man es ein paar der Zwerge unter großen Felslawinen zu begraben oder man Übergoss sie mit flüssigem Metall und stellte sich die im Erz gefangenen Zwerge als Statuen. Doch natürlich starben diese wenigen Exemplare, etwas unter 1000, nie und über die Jahrhunderte standen sie still, während in ihrem inneren langsam die Programmierung zerbröselte. Als man schließlich bei einer Ausgrabung einen der verschütteten Zwerge befreite, begann er als Mordmaschine über alles und jeden herzufallen. Anschließend begab sich dieser auf den Weg seine Geschwister zu befreien, um ihren neuen Auftrag, die Auslöschung aller Lebewesen zu beginnen. Zwar blieb den Göttern diese Entwicklung nicht verborgen, doch sie waren inzwischen nicht mehr in der Lage etwas dagegen zu unternehmen. Die Drachen hingegen beschlossen abzuwarten, um sich die Genugtuung zu gewähren, den Göttern ihre Unzulänglichkeit vor die Augen zu führen und nur im Notfall zu handeln. Die Zwerge haben, nachdem sie alle beisammen waren sich irgendwo in die Wildnis zurück gezogen, wo sie auf der Suche nach Erzen für weitere Zwerge sind nur unglückliche Abenteurer überraschen und angreifen.


\end{description}

\chapter{Magische Monster}

\begin{description}
\item[Verlorene Maid,]die\index[Kreaturen]{Verlorene Maid}
Die Legende sagt, dass diese magische Kreatur ihren Ursprung in Jungfrauen mit gebrochenem Herzen findet, die sich mit ihrem Kummer in die Wälder zurück gezogen haben, um dort ihren tragischen Leben ein Ende zu bereiten. Und die Verlorene Maid sei die rastlose Seele, der diese Welt nicht verlassen kann, weil ihr Herz immer noch an ihrem Geliebten hängt, der als einzige in der Lage wäre ihr Frieden zu schenken. Eine verlorene Maid hat stets die Erscheinung einer Frau, manchmal jung, manchmal fast schon Greisenhaft, in eine schlichte Toga aus purem Mondlicht gewoben, wie man sagt. Sie leben auf Waldlichtungen, in den Tiefen der Wildnis, die sie über die Jahre mit den eigenen Händen zu kleinen Schreinen umarbeiten in deren Zentrum sie sich bei Nacht niederlassen und einen traurigen Gesang anstimmen. Wer diesem Klagelied folgt wird die Maid meist auf einem Stein oder einem  

\end{description}

\chapter{Plagen aus der Gegenwelt}
Die Gegenwelt ist das Herz von Chaos und Gegensätzen. In dieser unwirtlichen Welt werden neue Arten so schnell geboren, wie sie auch wieder aufgrund der Radikalität ausgelöscht werden. Nur die härtesten und anpassungsfähigen schaffen dauerhaft Fuß zu fassen, auch wenn ihr Kampf ums Überleben nie aufhört. Mit der Ankunft in Aurum Orbis ging für einige der Traum nach einem ruhigen Leben in Erfüllung. Doch die Mehrzahl waren Bestien, die alle ihre eigenen Ziele verfolgen, denen meistens die angestammten Bewohner im Weg stehen, wenn nicht sie selbst das Ziel sind. 
\begin{description}
\item[Werkreatur,]die\index[Kreaturen]{Werkreatur}
\\Als Werkreaturen werden die Opfer einer Krankheit, welche man in Aurum Obis Therianthropie nennt und die aus den Phfulen der Gegenwelt stammt. Sie erzeugt ein Mutagen, welches den Infizierten in unregelmäßigen Abständen, einer rapiden Mutation in eine schrecklich deformierte Bestie unterwirft. Bei dieser Transformation wird der Körper stark beansprucht und verbrennt einen großen Teil der Nährstoffe, wodurch das entstehende Monstrum einen gewaltigen ungezügelten Hunger auf Fleisch verspürt, der, wenn es ihm nicht schnell nachkommt, tötet. Nach einer gewissen Dauer, meistens 6-8 Stunden findet eine schlagartige Rückverwandlung in die Ursprungsform statt, dabei werden verlorene oder vom Kampf und Leben erlittene Lädierungen rückgängig gemacht(Also sofern sie nicht bereits ohne Arme geboren wurden, werden diese Nachwachsen). Auch sonst ist eine Werkreatur immun gegen Krankheiten und seine Lebenserwartung kann sich teilweise verdoppeln, da bei jeder Verwandlung ein zeitweiliger Verjüngungsprozess eintritt.
Mit der Ankunft der ersten Werkreaturen auf Aurum Orbis unterlag ihre Erkrankung einer gewissen Veränderung. Zum einem ereigneten sich jetzt die Transformationen zum großen Teil in Zyklen, die sich sehr am Mond orientieren. In Neumondnächten ist eine Werkreatur sicher, um dieses Datum herum steigt die Wahrscheinlichkeit in jeder Woche um 25\%, sofern der Mond für die Werkreatur sichtbar ist und führt in einer Vollmondnacht immer zu einer Verwandlung. Die zweite große Veränderung ist, dass die Bestiengestalten nicht bei jeder Verwandlung unterschiedlich ist, sondern nach ihrer ersten Manifestation bleibt, zudem haben die Bestiengestalten von Neuinfizierte oft große Ähnlichkeit zu Raubtieren(Wölfe) aus Arum Orbis und werden anhand dieser in Subkategorien(z.B. Werwölfe) eingeordnet.
\\\textbf{Heilung}
\\Obwohl es schon oft versucht wurde, der Therianthropie Herr zu werden und sie zu behandeln, gelang es noch niemanden diesen Erreger aus einem Infizierten auszutreiben. Sogar Magie und göttliche Intervention scheiterten an dieser Aufgabe und hatten meist nur eine außerplanmäßige Transformation zur Folge.
Lediglich die Erweckung als Untoter oder die Verwandlung in einen Vampir kann der Therianthropie Einhalt gebieten, da diese im Totem Leib verendet.
\\\textbf(Verbreitung)
\\Zum Glück stehen sich Verbreitung und Natur bis zu einem gewissem Punkt im Wege: Therianthropie, wird durch Körperflüssigkeiten, die in der Bestiengestalt produziert werden, weitergegeben. Dies ist meistens der Speiche während die Bestie ihr Opfer frisst, nur diejenigen, die diese Begegnung überleben sind danach ebenfalls Werkreaturen. Ansonsten setzten sich nur alle, die mit den Überresten einer frisch erlegten Bestie agieren dem Risiko aus, infiziert zu werden. Kinder von Infizierten neigen dazu sich teilweise im Mutterleib zu transformieren und ihre Mutter oder ihr Ei(Je nach Art) von innen aufzufressen. Bei einer Rückverwandlung sterben sie dann meistens.
\\\textbf{Beziehungen mit Anderen}
\\Aufgrund ihrer brutalen und mörderischen Natur sind Werkreaturen Einzelgänger untereinander. Von der restlichen Gesellschaft werden sie gefürchtet und verfolgt, weshalb sie sich entweder in die Wildnis zurückziehen oder sich in den Schatten der Gesellschaft zu bewegen. Eine ganz besondere Beziehung hegen ''zivilisierte'' Werkreaturen mit den Vampiren, da sich beide ihr Jagdgebiet teilen müssen. Durch die unkontrollierten Ausbrüche sehen die Vampire in Werwölfen eine Bedrohung ihrer eigenen Sicherheit und versuchen sie durch Intrigen auffliegen zu lassen, während letztere die Vampire einfach nur zu töten suchen, was ihnen in ihrer Bestiengestalt, in der sie gegen die betäubende Wirkung des Vampirismus immun sind, sogar gelingt. 
\\\textbf{Eigenschaften}
\\Es folgt eine Liste der Standardmäßigen Fertigkeiten einer Werkreatur. Diese Liste kann je nach Sub-Typ noch um natürliche Eigenschaften der assoziierten Gestalt erhalten(Ein Wervogel kann beispielsweise Fliegen). Bezieht sich eine Eigenschaft auf die Bestie, so gilt diese nur wenn die Werkreatur transformiert ist.
\begin{description}
\item[Instinkte der Bestie]
Eine Werkreatur entwickelt bei seiner ersten Transformation einen schärferen Geruchs- und Hörsinn. Sie kann essbares Fleisch  auf 150m erschnüffeln und kann blind nur auf Basis seines Gehörs kämpfen, sofern sein Ziel Geräusche macht. Proben auf Wahrnehmung(Gehör) sind entsprechend erleichtert. Nach ihrer Rückverwandlung behält die Werkreatur ihre neue Sinnesschärfe. Außerdem entwickelt die Bestie einen 6. Sinn für Gefahr, der ihr eine 15\% Chance darauf gibt, gegen sie gerichteten Angriffen oder Umgebungszaubern auszuweichen, wenn sie die Quelle nicht erkannt hat. Alle Talente die auf dem Einsatz von Intuition basieren erhalten einen bestialischen Bonus.
\item[Unstet]
Der Trieb zu Fressen überlagert in einer Werkreatur alle anderen Empfindungen. Eine Bestie ist immun gegen Betäubung, Angst und Gedankenkontrolle aller Art oder Erschöpfung, sofern sie nicht natürlichen Ursprungs ist. Auf der anderen Seite verliert die Bestie jedwede Möglichkeit von Sozialen Fertigkeiten Gebrauch zu machen. Alle Aktionen die Konzentration benötigen sind ebenfalls unmöglich.
\item[Metabolismus]

\end{description}
\end{description}

\chapter{Schöpfungen fremder Götter}

\begin{description}

\item[Engel,] der\label{Engel}\index[Kreaturen]{Engel}\index[Stichworte]{Eron!Engel}
\\Um seine Urteile durchzusetzen, hat \uline{\hyperref[Eron]{Eron, der Gerechte}}, eine sehr stark an ihn gebundene Art geschaffen, die Engel. Ausgestattet mit Flügeln und dem Insignium von Eron auf der Stirn streifen sie unermüdlich durch die Welt um Recht zu sprechen. Dabei sind sie ebenso kurzzeitig und hochmütig wie ihr Schöpfer und schlagen ohne zu zögern gegen jeden, der sich ihren Urteilen zu widersetzen sucht oder Kritik an ihrem Gerechtigkeitssinn übt. Neben ihrem engen Kontakt zu Eron, der ihren Körpern die Möglichkeit gegeben hat, dass sie selbst eine geringe Menge göttlicher Macht für eigene Interventionen speichern können, weshalb sie auch trotz der Barriere weiterhin Interventionen ausführen können. Durch die Ausführung ihrer Urteile erhalten sie dabei einen Teil der göttlichen Energie, der ihren Vorrat auffüllt. 
\\\textbf{Eigenschaften von Engeln}
\begin{description}
\item[Gesandte Erons:]
Engel besitzen eine natürliche Verbindung zu Eron, dem sie stets Geweiht sind, über die sie mit diesem über ihre Urteile diskutieren und er ihr Verhalten überwacht. Gleichzeitig speichert ihr Körper einen Teil der göttlichen Energie, die bei der Ausführung von Erons Urteilen entsteht zur eigenen Verwendung in ihren Körpern. Diese können sie ähnlich wie andere Geweihte für Rituale und Gaben verwenden, nur dass sie selbst Ursprung dieser Intervention sind, weshalb sie auch in Aurum Orbis diese durchführen können.
\item[Flügel:]
Jeder Engel trägt ein paar Flügel aus Energie auf den Rücken, mit diesen können Engel unabhängig der Umgebung frei fliegen(Perfekte Manövrierbarkeit).
\end{description} 

\item[Vampir,] der\label{Vampir}\index[Kreaturen]{Vampir}\index[Stichworte]{Askon!Vampir}
\\Nur wenige Kreaturen werden so sehr gefürchtet wie der Vampir. Nicht nur gelten sie perfekte Jäger der Nacht, sondern verfügen in der Regel auch über einen Intellekt, der es ihnen erlaubt sich relativ frei unter ihrer Beute zu bewegen. Dazu bringt jeder Vampir noch eine Reihe von Möglichkeiten mit, seine Beute in williges Vieh zu verwandeln und so seine Herrschaft über die Lebenden noch weiter auszubauen. Lediglich ein alter Fluch, der auf einem ihrer finsterem Erschaffer, \uline{\hyperref[Askon]{Askon, dem gefallenem Gott}}, lastet und das Licht der Sonne für sie zur Todesfalle werden lässt, verhindert, dass sie ihr Imperium auf die gesamte lebendige Welt ausbreiten können. Über ihre Herkunft wissen nur die ersten und ältesten Vampire Bescheid und sie hüten dieses Wissen wie einen Schatz, genauso wie die geheimen Techniken die sie bei ihrer Schöpfung gelehrt bekommen haben.
\\\\\textbf{Vampirismus}
\\Der Fluch des Vampir(Vampirismus) ist das Ergebnis einer unheiligen Verbindung zwischen \uline{\hyperref[Charon]{Charon, Herr dem Fährmann}} und \uline{\hyperref[Askon]{dem gefallenem Askon}}, in der sie versuchten die Schwachpunkte ihrer Schöpfungen auszugleichen. Und auch wenn Askon seinem Partner später um die Kreation betrog, konnte Charon sich der Vampire nicht mehr entledigen. Der Fluch des Vampirismus tötet seine Opfer, durchwirkt sie mit seinem Gift, bevor er sie als Untote Kreaturen wiederauferstehen lässt. Fortan müssen sie sich von der göttlichen Essenz aus dem Blut andere Schöpfungen ernähren. Im Gegenzug winkt ihnen ein Leben in Unsterblichkeit und Macht, sofern sie das Sonnenlicht meiden. Denn Askon verleiht im Gegenzug für die Opferung von der gesammelten Lebenskraft eines Vampires diesen außergewöhnliche Fähigkeiten. Auch wenn Vampirismus einen de Fakto zu einem Untoten macht, interessieren sie sich nicht für Nekromantie, ja verachten diese sogar. 
\\\\\textbf{Eigenschaften und Fähigkeiten eines Vampirs}\\
\begin{description}
\item[Der Biss:]
Der Biss eines Vampirs dient diesem zum Aussagen seines Opfers. Um den Prozess dabei zu vereinfachen, injiziert der Vampir eine sehr, sehr geringe Menge seines eigenen Blutes, wodurch sein Opfer in einen Zustand purer Euphorie und Entspannung verfällt und nach Beendigung des Absaugen, das Opfer ohne Erinnerung an dieses Ereignis zurück lässt. Je nachdem unter welchen Umständen der Biss eingeleitet wurde(z.B. vor dem geplanten Akt oder bei einem amourösen Kuss), kann das Opfer danach mit einer emotionalen Bindung zum Vampir zurück gelassen werden, was spätere Bisse wesentlich einfacher macht. 
\item[Der Kuss:]
Ein Vampir kann nicht nur sein Opfer blutleer saugen, sondern ihm anschließend einen Teil seines eigen vom Vampirismus verdorbenen Blutes schenken, wodurch dieses sich wenig später ebenfalls als Vampir erheben wird. Durch den Kuss entsteht ein Band zwischen beiden Vampiren, ähnlich der Bindung zwischen Eltern und ihrem Kind, auch wenn vor allem bei korrumpierten Seelen, dieses Überwunden werden kann. Es gilt außerdem zu beachten, dass der Fluch mit jeder Generation eines neuen Vampir schwächer wird, weshalb ältere Vampire in der Regel über den jüngeren stehen.
\item[Der niedere Kuss:]
Ein Vampir kann, wenn sein Opfer noch genug eigenes Blut in seinen Adern hat, eine Portion, die wesentlich geringer als beim Kuss ist, seines Blutes injizieren. Dies hat zur Folge dass der Fluch des Vampirismus das Opfer nicht komplett tötet, allerdings eine eigene Lebenskraft langsam verzehrt. Das Opfer, welches unter Vampiren als Famulus bezeichnet wird, fühlt sich zunächst ähnlich wie beim Biss euphorisch, was durch ein Gefühl von Macht, da man einen Teil der Stärke und Ausdauer eines Vampirs erhält, noch verstärkt wird. Doch gegen Ende des ersten Mondes nach der ersten Injektion werden erste Nebenwirkungen auftreten, rapide wird das Opfer altern und gebrechlich werden, sowie einen schrecklichen Hunger verspüren, den nichts, auch nicht der Genuss von gewöhnlichem Blut stillten kann. Der Vampir kann in diesem Zustand sich seines Famulus annehmen und ihn durch eine erneute Injektion wiederherstellen. Das Opfer wird dabei für die Zukunft durch das Wissen von Schmerz bei Ausbleiben einer neuen Injektion und des alles übersteigenden Glückes nach einer Injektion zu einem Sklaven seines Gönners. Oft geht dieses Verhältnis mit einer abgöttischen Verehrung für diesen einher, auch wenn es Ausnahmen gibt, die bis hin zu abgrundtiefen Hass reichen können. Ein Famulus, der für zwei Monde kein neues Blut von seinem Gönner bekommt, geht unweigerlich zu Grunde, wobei Körper und Seele vollständig vernichtet werden.
\item[Vampirblut:]
Mit der Transformation zu einem Vampir, zerfrisst der Fluch jeden Tropfen von Restblut und einen Großteil der inneren Organe und füllt die Blutbahnen mit einer Tinten-ähnlichen Flüssigkeit, die bei Kontakt mit der Luft jedoch sofort Blutrot wird. Dieses Blut verliert alle Eigenschaften bezüglich der Ursprungsrasse, wie die Affinität zu Mana von Drachenblut. Zwar gehört der Vampir immer noch seiner Ursprünglichen Rasse an, wenn es um die Benutzung von göttlichen Artefakten geht, allerdings verdirbt ihr Einfluss diese, wodurch seine Wirkung oft ins Gegenteil verkehrt wird.
\\Zwar ist der Vampirismus eine göttliche Schöpfung, was viele Vampire dennoch nicht davon abhält sich die Magie anzueignen. Aufgrund ihrer Fähigkeit sich sehr schnell von Wunden zu erholen und sie selbst de Fakto tot sind, verträgt sich ihr Körper außergewöhnlich gut mit Mana aller Art. Sie sind doppelt so Resistent gegen Manabrand, wie vergleichbare nicht-vampirische Manablütige(Bei einem Elfischen Vampir betrachtet man das menschliche Gegenstück).
\\Einem Vampir wird bei seiner Transformation automatisch zu einem Askon-Geweihten und kann diese auch nicht wieder ändern. Er erhält darüber hinaus Zugriff auf eine Reihe nur Vampiren vorbehaltenen Ritualen, die er in der Regel mit seinem eigenem Blut bezahlen muss.
\item[Schatten ihrer Selbst:]
Eine der größten Schwächen eines Vampirs ist das Sonnenlicht. Sobald sie in Kontakt brennt des Sonnenlicht in einem Augenblick Löcher dort wo es den Vampir berührt, es bleibt nichts zurück, wenn ein Vampir komplett bestrahlt wird. Nur Dicke Mauern oder andere Lichtundurchlässige Barrieren können sie gegen dieses Abschirmen, Kleidung, außer aus speziellem Material, kann sie nicht schützten. Selbst wenn es ihnen gelingt sich vollständig gegen das Sonnenlicht abzuschirmen, so schirmt Sonnenlicht alle Kräfte das Vampirs, wie seine Telepathie ab. Eine weitere Folge ihres Zustands ist das Fehlen eines Spiegelbildes oder Schattens, beides Merkmale an denen kundige einen Vampir erkennen können. Durch Rituale erschaffenes Sonnenlicht zählt als vollwertiges Sonnenlicht, gleiches gilt für Morgentau aus der entsprechenden Weihe der Ixania.
\item[Perfekter Jäger:]
Vampire können wenn sie es wünschen, jedes Geräusch ihrer Schritte oder Ausrüstung eliminieren. Sie erhalten die Fähigkeit potenzielle Beute über ihren Geruchssinn aufzuspüren, wogegen nur sog. \uline{\nameref{Vampirbann}} hilft. Ein Vampir kann den Geruch studieren, um sich über den gesundheitlichen Zustand seines Opfers zu informieren, außerdem können besonders erfahrene Vampire, Informationen, wie Rasse, Geschlecht, Alter, Manablut und unter Umständen sogar Stand und Herkunft abschätzen. Vampire besitzen außerdem eine natürliche Begabung, im direkten Handgemenge einen Biss zu platzieren und ihr Opfer als Schild gegen Angriffe einzusetzen. Sie erhalten einen Bonus auf das Einleiten eines Handgemenge, zum Beginn eines Bisses und auf ihre Verteidigung während sie ihr Opfer aussaugen.
\item[Schattenaffin:]
Ein Vampir ist eine Kreatur der Dunkelheit und erhält in magischer oder völligen Dunkelheit eine perfekte Wahrnehmung von allem innerhalb dieser Dunkelheit, solange es sich nicht mehr als 30m von ihm entfernt befindet. Er kann auch Lichtquellen innerhalb dieses Radius lokalisieren, wenn auch nichts innerhalb ihrer Lichtkreise. Der Körper des Vampir scheint, mit der Dunkelheit selbst eins zu werden. Auch Kreaturen mit Dunkelsicht haben daher eine 50\% Chance den Vampiren zu verfehlen. Ein Vampir hat außerdem eine doppelte Bewegungsreichweite innerhalb von Dunkelheit und seine Sprünge haben immer die doppelte Reichweite und zählen immer so, als hätte der Vampir Anlauf genommen.
\item[Band des Geistes:]
Ein Vampir kann durch Blickkontakt mit einem Opfer ein telepathisches Band zwischen beiden knüpfen. Über dieses können beide miteinander kommunizieren, wobei der Vampir seine Ausführungen mit anderen Sinneseindrücken unterlegen kann, weshalb es ihm möglich ist Soziale Fertigkeiten einzusetzen, wie Überreden oder Einschüchtern, wogegen sich selbst ein unwilliger Geist nicht wehren kann. Ein Vampir kann auf diese Weise nur eine Konversation gleichzeitig führen, auch wenn er auf Wunsch jederzeit, auch ohne Blickkontakt das Band zwischen sich und einem seiner Famuli oder Kinder/Erzeuger, aufnehmen kann. Ein anderer Vampir kann diese Form des geistigen Kontaktes unterbinden. 
\item[Totes Fleisch:]
Der Körper eines Vampirs kann sich nicht auf natürlichem Wege heilen, allerdings behindern Wunden einen Vampir nicht, sofern der Körper noch immer ausreichend zusammengehalten wird. Um einen Vampiren zu vernichten, muss er daher dem Sonnenlicht ausgesetzt werden, ansonsten kann der Vampir, wenn er noch über ausreichend Energie verfügt selbst wiederherstellen, selbst als verbrannter Aschehaufen, oder mit der Hilfe durch einen anderen Vampiren oder Askon selbst über einen Zeitraum von 1 Woche wiederhergestellt werden.
\end{description}
\textbf{Askons Gaben}\label{Vampire:AskonGaben}
\\Wie bereits zuvor erwähnt hat jeder Vampir Zugriff auf eine eigene Reihe von Ritualen und Gaben, die der Vampir mit einem kleinem Blutopfer bezahlen muss.
\begin{description}
\item[Stimme des Verführers:]
Ein Vampir kann Askon, der selbst ein Meister der Lügen, falschen Versprechungen und schöner Worte ist, an seiner Stelle reden lassen, wodurch der Vampir einen ungemeinen Bonus auf eine Lügen, Verführen, Überreden, Überzeugen oder vergleichbare Probe erhält. Er kann diesen Bonus auch über sein Band des Geistes zur Geltung bringen.
\item[Vision der ewigen Nacht:]
Der Vampir umgibt sich mit einer Aura von Dunkelheit, die seinen Gegnern einen Eindruck des ihnen bevorstehendes Untergangs vermittelt. Alle nicht-schattenaffine Kreaturen werden Opfer eines Furchteffektes. Der Vampir kann diese Aura nicht für seine Schattenaffinität nutzen, allerdings kann er durch Aufbringen zusätzlicher Blutopfer Sonnenlicht negieren.
\item[Umarmung der Nacht:]
Mit Askons Segen vervollständigt der Vampir seine Verbindung mit der Dunkelheit und kann für eine gewisse Zeit zwischen einer körperlosen aus Dunkelheit, die es ihm ermöglicht innerhalb eines Schattens jede beliebige Form anzunehmen, um beispielsweise Gitter zu passieren oder an Wänden und Decken zu bewegen, und seiner gewöhnlichen Form wechseln. Er kann in seiner Formlosen Gestalt durch physische Effekte keinen Schaden nehmen und auch die meisten Zauber die auf den Körper zielen, verlieren für die Dauer seiner Transformation ihre Wirkung. Endet die Wirkung von Umarmung der Nacht oder wird der Schatten, in dem sich der Vampir befindet, ausgeleuchtet, während er sich in seiner Schattenform befindet, nimmt er im nächsten geeignetem Raum wieder seine physische Gestalt an.
\item[Schatten zu Fleisch:]
Der Körper eines Vampirs kann bekanntermaßen nicht heilen. Um sich dennoch störender Wunden oder von Sonnenlicht gebrannte Löcher zu entledigen, kann ein Vampir sein Blut für die Heilung wiederherstellen.
\end{description}


\end{description}

\phantomsection
\addcontentsline{toc}{chapter}{Kreaturenindex}
\printindex[Kreaturen]

\part{Technologie}
\setcounter{chapter}{0}
G\newpage

\part{Zu Sortierendes}

%Geht in Geographie
\chapter{Tal des Zwielichts}\label{Tal des Zwielicht}\index[Geographie]{Tal des Zwielicht}
Nachdem man feststellen musste, dass magische Barrieren um das Weltenloch niemals einen dauerhaften Schutz vor der Gegenwelt bieten würden, wandten sich die Bewohner von Aurum Orbis der zweiten großen Machtquelle zu, den Aspekten. Sie waren bereits zuvor von großer Hilfe im Kampf gewesen und nun war man soweit, dass man eine Lösung für einen dauerhaften, göttlichen Schutzschirm um den Zugang zur Gegenwelt spannen konnte. Das Weltloch lag verborgen in einem, von einem Gebirge abgeschlossenen, Tal, das den Gegenweltlern zuvor eine ideale Verteidigung und Deckung geboten hatte. Als man mit den neuen Verbündeten aus der Gegenwelt die feindlichen Kräfte zurück in jenes trieben konnte, begann eine ganze Reihe von Geweihten der Ixania mit einem gewaltigen Ritual, an dessen Ende das gesamte Tal in \uline{\nameref{Zwielicht}} getaucht war. Daraufhin konnte keine Armee aus der Gegenwelt das Tal verlassen, ohne wegen des Entzugs innerhalb weniger Tage zugrunde zu gehen, da der Ritus diesen nicht bekannt war. In den nachfolgenden Jahren, nach Ende des zweiten Gegenweltkrieges, errichteten die Ixania-Geweihten und viele Helfer, einen Tempel zum Auffangen des ersten Morgengrauen auf der Außenseite der Berge und einen Tempel zum Einfangen von Mondlicht auf der Innenseite. Dazu kamen eine Reihe von kleineren Schreinen, die jeweils die Tempel bei ihrer Aufgabe unterstützen. Zusammen entsteht ein konstanter Strom von Zwielicht, der das Tal flutet und welches, zusammen mit der \uline{\nameref{Antimagie}}, ein Durchbrechen dieser Barriere unmöglich macht und die Welt schützt.






\cleardoublepage
\phantomsection
\addcontentsline{toc}{part}{Stichwortverzeichnis}
\printindex[Stichworte]
\phantomsection
\addcontentsline{toc}{part}{Geographieglossar}
\printindex[Geographie]


\end{document}
