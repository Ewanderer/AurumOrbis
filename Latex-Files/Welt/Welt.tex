\documentclass[a4paper,12pt,oneside]{book}
\usepackage[ngerman]{babel}
\usepackage[utf8]{inputenc}
\usepackage{imakeidx}
\usepackage[hypertexnames=false]{hyperref}
\usepackage[all]{hypcap}
\usepackage{nameref}
\usepackage{ulem}
\usepackage{Zusaetze}

\hypersetup{
	bookmarks=true,
    colorlinks,
    citecolor=black,
    filecolor=black,
    linkcolor=black,
    urlcolor=black
}


\makeindex[name=Stichworte,title=Stichwortverzeichnis]
\makeindex[name=Geographie,title=Geographieglossar]
\makeindex[name=Kreaturen,title=Monsterindex]
\makeindex[name=Profession,title=Professionen\&Spezialisierungen]
\title{Aurum Orbis}
\author{Jordan Eichner, Simon Hornisch, Till Markusch}
\date{}
\setcounter{secnumdepth}{-1}
\setcounter{tocdepth}{10}

\begin{document}

\maketitle
\tableofcontents

\part{Geschichte}
\chapter{Die Dunkle Zeit}
Die Ereignisse aus diesem Zeitraum sind aufgrund ihrer Natur und den anfangs nur mündlichen Überlieferung nur unvollständig bis gar nicht überliefert. Selbst Legenden aus dieser Zeit sind meist frei erfunden und magische oder göttliche Hilfsmittel können ebenfalls nur wenige brauchbare Informationen liefern, Götter weigern sich oder sind selbst unwissend, während Zauber aufgrund von Resonanzen mit den magischen Energien aus dieser Zeit bis zur Unkenntlichkeit verzerrt werden.

\section{Die Weltbauer}
Am Anfang gab es die \uline{Axiomaten}\index[Stichworte]{Axiomat}. Wesen die in das Chaos des Multiversums Ordnung zu bringen suchten und die Absolute Gewalt über die Realitäten innerhalb der Raum-Zeit-Kontinuum verfügten.   
Und sie taten für Ewigkeiten nichts anderes als alles in ein statisches Gleichgewicht zu bringen und als sie ihr Ziel erreichten und die Ordnung perfekt war, gerieten sie in eine Krise. Sie waren, soweit man als Außenstehender es beurteilen konnte sehr geduldig, doch die Aussicht für immer einfach nur auf eine Perfekte Ordnung hinabzublicken verlor irgendwann für einige ihren Reiz. Und so verwarfen sie ihre ursprüngliche Philosophie und wurden zu den Weltenbauern. Sie begannen Veränderung zu erlauben und die Universen erwachten aus ihrem Stillstand. Doch relativ schnell vergingen viele dieser Welten, als sie aufgrund von Instabilität zerfielen. Man lernte dazu und mit wachsender Erfahrung, ließen sich auch die ersten einfachen toten Welten bauen. Der nächste Schritt bestand aus der Einführung von Leben und anderen Faktoren.

\section{Der große Streit}
Einst trat ein großer Schöpfergeist, der im nachfolgenden als Alpha benannt werden soll, an eine tote Welt heran und hauchte ihr eine neue Existenz ein. Alpha schuf sie sehr statisch mit einem großem Planeten und ein paar Monden, sowie einer Sonne, vor einem Hintergrund aus Sternenhimmel. Wie immer flossen Ewigkeiten in die Gestaltung der Details, bevor er überhaupt in Erwägung zog, Zeit zuzulassen. Als sich Alpha daran machen wollte an seinen ersten intelligenten Bewohnern zu arbeiten, trat Omega, ein anderer Weltbauer ein.
\\ Unter gewöhnlichen Umständen gehen sich die Weltenbauer aus dem Weg, um genau zu sein, seit der letzten Zusammenkunft der Axiomaten, wo sie ihre Zukunft als Weltbauer entschieden, kam es nie wieder zu einem Zusammentreffen. Omega war mehr aus Zufall über Alphas Schöpfung gestolpert und ihm durch sein Loch am Ende des Universums gefolgt und er war nicht zufrieden. Für Omega standen Alphas Konzepte im krassen Gegensatz zu seinem eigenem Verständnis einer Welt. Es folgte ein Streit, in dem sie das Gefüge der Welt mit Anomalien als Zeichen für ihre Vorstellungen einer perfekten Welt zeichneten.
\\ Am Ende lief es auf eine finale Konfrontation am Raum-Zeitriss, indem beide Existenzen vernichteten und sie eine tote, verwundete Welt zurückließen. Aurum Orbis war geboren.
\subsection{Die Hundertjährige Stadt}
Neben den Anomalien ist noch ein anderes Relikt aus dieser Zeit erwähnenswert, die erste Zivilisation von Aurum Orbis. Über diese ist nichts bekannt, doch waren sie in der Perlwüste angesiedelt und sollte auf Kristallenergie fußen. Doch sollte es Omega sein, der ihrer Existenz ein Ende bereiten sollte und nichts zurück ließ als ein paar Monolithen und natürlich die Kristallenergie.

\section{Neue Götter}
Eine Ewigkeit war es still in Aurum Orbis, während an den Rändern des Raumlochs, das Gefüge der Welt zerfaserte und sich so immer weitete. In der Zwischenzeit in einer anderen Realität:
\\Ein Pantheon von Götter herrschten hier über eine Schöpfung, führten untereinander Kriege auf einem weltlichen Schlachtfeld mit ihren eigenen Schöpfungen, die durch ihre religiöse Verehrung, den Göttern weitere Stärke gaben. Über die Jahrhunderte führte dieser Zyklus aus Göttlicher Intervention und Verehrung zur Entstehung eines gewaltigen Energieüberschusses zugunsten der mächtigsten Götter, die im Gegenzug immer brutaler Gegeneinander vorgingen. Am Ende kam es zu einer Apokalypse, in der das Raum-Zeit-Gefüge selbst in Stücke gerissen wurde und in dem viele der niederen Götter vergingen, doch einem kleinem Pantheon der Mächtigsten gelang es sich auf eine Scholle aus Raum und Zeit zu retten, mit der sie durch das Nichts zwischen den Welten trieben, bis sie angezogen von dem Sog der von Aurum Orbis Loch ausging in diesen hinein gesogen wurden. Bei ihrer Ankunft zerschellte ihre Scholle und die Fragmente, stabilisierten das Gefüge genug, dass sich die Realität nicht noch weiter zerfaserte, wenn sie es auch nicht ganz schloss.

\section{Die Götterkriege}
Nachdem sich die Götter es sich in Aurum Orbis eingerichtet hatten, beschlossen sie mit erstem Leben zu füllen, auch wenn sie beschlossen sich nicht aktiv in die neuen Gesellschaften einmischen wollten. Diese ersten Schöpfungen waren die Drachen. Ausgestattet mit einem großem Intellekt und eigenen gottähnlichen Kräften begannen sie bald nach ihren Ursprüngen zu graben und stießen dabei mehr oder weniger zufällig über die von den Anomalien der Schöpfer ausgehende Energie, welche sie jahrelang studierten und schließlich adaptierten. Später sollte sie als Magie bekannt sein. In der Zwischenzeit hatten sich die Götter anderen Spezies zugewandt, wobei sie angesichts ihrer eigenen schwindenden Kräfte, nun weitaus einfacherer Geschöpfe formten. Dies war der Geburt der sterblichen Rassen. Als sie schließlich die Früchte ihrer Arbeit betrachteten, fiel ihr Blick wieder auf die Drachen und sie schreckten zurück, als ihre Schöpfung zurückschaute. Die Drachen auf der Suche nach immer mehr Wissen um die neuentdeckte Magie noch weiter nutzten zu können, hatten die Drachen den Schleier durchbrochen hinter dem sich die Götter zurückgezogen hatten. Aufgrund von Uneinigkeit zwischen beiden Parteien über die wahre Natur der Welt, kam es zu einem schrecklichen Krieg, der jedoch weites gehend unbemerkt von den jungen sterblichen Rassen von statten gingen. Schnell zeigte sich, dass die Götter in einem direkten Machtkampf nicht viel ausrichten können würden, da ihre Fähigkeiten in der Gestaltung der Welt lag, wo die Drachen jedoch einen Vorteil besaßen und so erschufen sie in aller Eile und mit den letzten Resten ihrer Macht die Zwerge. Tötungsmaschinen mit nur einem Auftrag: Die Drachen zu vernichten. Ihre Zahl war denen der Drachen um ein vielfaches überlegen und dieser Vorteil führte zu einer Wende des Blattes. 

\subsection{Der Aufstieg der Elfen}
Es wäre der Untergang der Drachen gewesen, wenn nicht ein Stamm Dunkelelfen gewesen wäre. Sie gerieten zufällig zwischen die Fronten eines kleineren Scharmützels zwischen 3 Drachen und einer Horde Zwerge. Keine der beiden Seiten schenkte ihnen Beachtung, die Drachen, weil sie zu sehr mit Überleben beschäftigt waren und die Zwerge, weil sie diese auf göttliche Anweisung in Ruhe lassen sollten, wie auch die anderen sterblichen Rassen. Doch gemäß ihrer Natur stürzten sich die Dunkelelfen in die Schlacht und sie trieben einen Keil zwischen die Kontrahenten, der es den Drachen erlaubte die auf Nahkampf spezialisierten Zwerge mit ihrem Feuerodem zu Asche zu verarbeiten, während diese mehr oder weniger hilflos, aufgrund der an ihnen klebenden Dunkelelfen dies über sich ergehen lassen mussten. Nach der schicksalshaften Schlacht kamen die Drachen und Dunkelelfen zusammen. Letztere wollten die Drachen natürlich ungeachtet der Tatsache, dass sie Zeugen ihrer Macht waren, töten, aber all ihre Versuche scheiterten Kläglich. Die Drachen bald von den hoffnungslosen Versuchen gelangweilt, machten dem Stamm ein Geschenk: Frieden. Sie zogen den tief sitzenden Stachel des Hasses aus ihrem Geist, wo er bei ihrer Schöpfung gepflanzt worden war. Den Dunkelelfen war es, als öffneten sie zum ersten Mal die Augen und sie sahen sich selbst und die Welt um sich und fielen, als das Leben selbst mit all seiner Schönheit in sie einströmte, auf die Knie und weinten. Anschließend bedankten sie sich bei ihren Errettern, die selbst nicht weniger Dank zeigten. Die Elfen, wie sie sich nach diesem Erlebnis nannten, entschieden sich dazu den Drachen für ihr Geschenk zu danken, indem sie ihnen bei der Vernichtung der Zwerge beistanden. Wie eine lebende Mauer stellten sie sich zwischen die Drachen und ihre Feinde, die unter dem Odem der Drachen fielen und zurück unter die Erde, bis zum Herz der Zwerge in deren Feuern die Götter sie geschaffen hatten und schließlich gab es keine Zwerge mehr. 
\\Was folgte war die Schmiedung eines Paktes. Die Drachen gewährten ihren Rettern ihr Blut, welches sie befähigte selbst Magie zu nutzen und teilten auch einen Teil von ihrem Verständnis der Welt mit ihnen, wodurch die Elfen unermessliche Macht erhielten. Im Gegenzug versprachen die Elfen die Existenz der Drachen vor den anderen Sterblichen, ja auch vor weiten Teilen ihrer eigenen Nachkommen geheimzuhalten, damit diese in der Nähe des Herzen der Welt, welches für die Drachen eine gewaltige Magiequelle darstellte, weiter an ihrem Verständnis der Welt arbeiten konnten. So kam es zur Entstehung des Elfenvolkes und ihren stillen Wächtern den Drachenelfen. Die Götter währenddessen beschlossen, sich in Zukunft von den Drachen fernzuhalten, ein Verhalten, dem die Drachen folgten.

\subsection{Währenddessen bei den anderen Völkern}
Nach ihrer Schöpfung herrschte unter den sterblichen Völkern, wenn man von den Dunkelelfen absieht, weitestgehend eine Art Waffenstillstand, während dem sich die Völker über die Welt verteilten und erste Siedlungen gründeten, sowie die Welt selbst erkundeten. Auch wenn die Götter sich alle Mühe gaben, sich aus den Angelegenheiten ihrer Schöpfungen raus zuhalten, wenn man von ihrem Krieg mit den Drachen absah, entstanden eine ganze Reihe von Götterkulten und wie es in der Natur der Götter lag, mussten sie dem Ruf ihrer neuen Anhänger folge leisten. Um die Reihen ihres Pantheon zu füllen, divergierten sie in eine ganze Reihe von persönlichkeitsfreien, auf einzelne Elemente spezialisierte Entitäten, die man später nur noch als Aspekte bezeichnete. 
Nebenbei stolperte man allerorts über die prähistorischen Anomalien und widmete sich der Ergründung ihrer Geheimnisse. Dabei stießen sie, ähnlich wie die Drachen auf die Magie der Schöpfer, wenn auch ihre sterbliche Natur unfähig war, sie in selber Weise zu verstehen und nutzbar zu machen, wie es den Drachen gelungen war. Doch über die Jahre gelang es ihnen die schöpferische Energie als Mana zu destillieren und erste Anwendungen zu finden. Dazu gehörten leider nicht nur friedliche, denn über die Jahre war der stumme Friede zerbröckelt und zwischen den ersten kleineren Reichen waren Kriege ausgebrochen. 

\section{Die ersten Manakriege}
Trotz ihres geringen Verständnis für das neuentdeckte Mana, setzten es die oberirdischen Völker immer wieder häufiger in seiner Rohform als Waffe ein. Bei einer gezielten Explosion von Mana in der Nähe des großem Arkeum, einer der größeren Anomalien, kam es schließlich zu einer Katastrophe, als dieses zerbrach und Unmengen an roher Energie freisetzte, die die Stadt Thalum und alle ihre Bewohner ausradierte und ein widernatürliches Land hinterließ, aus dem sich von Magie verzerrte Kreaturen erhoben und in die Welt flüchteten, diese wurden später allgemein als magische Monster bezeichnet. \\Dieses Ereignis führte später immer wieder an verschiedensten Orten zu Eruptionen von chaotischem Mana, welches die betroffene Umwelt radikal veränderte und weitere magische Monster hervorbrachte. Und dennoch schreckte dieses Ereignis die Völker nur geringfügig ab, bis eine ganze Reihe weitere Katastrophen, die unter anderem zur Entstehung des zersplitterten Kontinents führten,  alle Beteiligten zwangen, angesichts der Gefahr, die Welt durch den Einsatz roher Manabomben(siehe Seite \pageref{Manabombe}) noch weiter zu zerstören, bei ihren zukünftigen Konflikten auf Mana zu verzichten. Eine Tradition die sich auch über die Grenzen der Dunklen Zeit hielt, auch wenn ihr Ursprung durch den Einfluss der Mächtigen, die ihre Fehler zu vertuschen zu suchten, verloren ging.

\section{Die Gegenwelt}\index[Stichworte]{Gegenwelt}
\subsection{Vorgeschichte}
Unter den unzähligen Gescheiterten Welten, die die Axiomaten erschufen, gehört die Gegenwelt sowohl zu den ältesten dauerhaft existierenden Welten, als auch den größten Entäuschung für die Axiomaten. Ihr Schöpfer hat sie im Sinne seiner Ursprünglichen Aufgabe perfekt Geordnet und ausbalanciert, bevor er alles ins Gegenteil verkehrte und eine Welt des puren Chaos und der Gegensätze schaffte, wo es weder feste Regel noch irgendeine Form von definierten Grenzen, zwischen Gegensätzlichen Elementen gab. Gut und Böse, Tod und Leben, Oben und Unten, das und noch alles weitere wahren nicht mehr als Illusion oder eher Illusionen von Illusionen. Im Grunde genommen war sie nicht mehr als ein Wrack, dass allerorts Löcher hinaus ins Nichts besaß und stellenweise kollabiert war. Doch kein Axiomat, obwohl sie ihrer Bestimmung schon lange entsagt hatten, konnte sich auch nur in die Nähe dieses von Chaos regierten Ortes begeben, ohne sich innerlich zu krümmen. Als seine Nachbar-Realitäten kollabierten wurden, verschlang das Chaos diese und ihm wuchsen Metaphorisch gesehen Tentakel aus Raumzeit, die sich daraufhin willkürlich zwischen den anderen Realitäten wanden, kollabierte verschlangen und so immer weiter wuchsen.

\subsection{Die Ankunft des Pantheon in der Gegenwelt}
Die Götter von Aurum Orbis, wähnten sich bei ihrer Flucht aus ihrer Heimatrealität, als die einzigen Überlebenden, doch sie lagen falsch. Ein paar ihrer Brüder und Schwestern, gelang es wie ihnen sich auf eine Reihe von kleineren Scherben aus Raum-Zeit zu retten, mit der sie durch das Nichts glitten. Sie gerieten schließlich nacheinander durch Zufall in die `Fänge` der Gegenwelt. Viele von ihnen vergingen in dem Chaos, weil sie sich nicht an die chaotische Natur ihrer neuen Heimat gewöhnen konnten, doch einigen radikalen Göttern gelang es sich einzurichten, neue Schöpfungen hervorzubringen und sich ihr persönliches Reich zu errichten, in dem das Chaos zumindest bis zu einem gewissen Grad unter Kontrolle gehalten wurde.

\subsection{Erster Kontakt}
Irgendwann war es dann so weit, dass ein Tentakel der Gegenwelt seinen Weg nach Aurum Orbis und seinem Loch fand, der Kontakt sollte zwar Aurum Orbis Raum-Zeit Gefüge nicht weiter stören, doch es wurde eine schmale Brücke in das Herz des Chaos geöffnet, durch die die ersten Gegenwelter ihren Weg nach Aurum Orbis fanden.
Dem Gebiet um das Weltloch wurde lange Zeit nur wenig Beachtung geschenkt, da es hier weder wertvolle Ressourcen, noch Manareserven gab. Zudem war die Landschaft unwirklich und unfruchtbar, daher blieb die Ankunft der ersten Gegenweltler vorerst unbemerkt und der erste Kontakt war ein wahrer Schock, vor allem auf Seiten der angestammten Bewohner. 
\\Wie von selbst schmiedete die Menschen, Elfen, ja später sogar einige Stämme der Dunkelelfen gegen die neue Bedrohung zu einer großen Fraktion zusammen.

\subsection{Der Erste Gegenweltkriege}
Dennoch waren die schnell darauf entstehenden Kriege beinahe das Ende für die Völker in Aurum Orbis, wenn sie nicht über eine so mächtige Waffe, wie die Magie verfügt hätten, eine Macht die den Gegenweltlern glücklicherweise verschlossen blieb, wenn man von einigen Unglücklichen Zwischenfällen absah, aus denen einige gegenweltlerische magische Monster hervor gingen. Die größte Bedrohung ging vom gegenwelter Götterpantheon aus, die sich bei einer Begegnung mit ihren alten Brüdern und Schwestern, die nur noch als Aspekte existierten, einen neuen richtigen Götterkrieg, wie in den guten alten Zeiten, mit diesen versprachen. Doch den Aspekten lag nichts an diesen Konflikten, im Gegenteil, von den alten Göttern war so wenig übrig geblieben, dass sie die Herausforderung gar nicht wahrnahmen und einfach still zusahen, wie die Horden aus Gegenweltern Aurum Orbis überrannten. Es waren schließlich die Drachen, die mehr aus Neugier, als um wirklich zu helfen, sich ihren Weg in Richtung Gegenwelt schlugen, im Zuge ihrer noch immer weitergehenden Suche nach neuen Antworten auf ihre Fragen. Sie kamen allerdings nie dort an, denn das Gebiet um das Weltloch, erwies sich für sie als unüberwindbare Barriere aus Antimagie(siehe Seite \pageref{Antimagie}), die ihre Existenz auszulöschen drohte. Doch dort angekommen, in Begleitung einiger treuer Drachenelfen, machten sie der Welt ein Geschenk, sie errichteten eine hermetische Barriere aus Magie, die das Vorankommen der Gegenweltler stoppte, bevor sie sich genauso unbemerkt, wie sie gekommen, zogen sie sich auch wieder zurück in ihr Refugium am Herzen der Welt zurück. Kurz darauf konnten das Bündnis der Völker sich der letzten Gegenweltler entledigen und es folgte eine kurze Atempause, während sich alle von den Strapazen erholten.

\subsection{Der Fall der großen Barriere}
Die Barriere der Drachen sollte allerdings nur einen gewissen Aufschub gewähren und das war auch einigen aus den Reihen des großen Paktes klar, der jedoch kurz nach Abwendung der Katastrophe weitestgehend wieder zerfiel. Um sich auf einen zweiten Krieg vorzubereiten, begannen die letzten Mitglieder des Paktes eine Befestigung um das Gelände der Barriere zu errichten und mit der Ausbildung neuer Truppen zu beginnen. Dies war die Geburtsstunde der \uline{letzten Front}\index[Stichworte]{Letzte Front}, jener Einheit, die sich schwor, allzeit, auch über ihren Tod hinaus in Namen von Aigis, dem Schutzherren, Aurum Orbis vor jeder Bedrohung von außen und später auch von innen zu bewachen. Schließlich brachte eine der vielen Magieeruptionen des Arkeum die Mauer zum fallen: Der zweite Gegenweltkrieg war eingeläutet.

\subsection{Der zweite Gegenweltkrieg}
Diesmal wurden die Horden aus der Gegenwelt mit gezückten Waffen erwartet, was zunächst auf einen Sieg oder gar einen Gegenschlag ins Herz der Gegenwelt hoffen ließ. Doch leider erwies sich der Gegner als zahlenmäßig überlegen, denn in der Zwischenzeit war es in Aurum Orbis erneut zu inneren Kriegen gekommen, die sich diesmal nicht beilegen lassen wollten und nur ein Bruchteil der alten Verbündeten unterstützte die Letzte Front bei ihrem Kampf. Doch neue Verbündete sollten kommen. 
\\Neben den vielen Bestien kamen auch einige aus der Gegenwelt, die mehr den Völkern aus Aurum Orbis glichen. Einige kamen als Flüchtlinge auf der Suche nach einer neuen, besseren Heimat, andere, um alten Feinden aus der Gegenwelt den Kampf anzusagen. Anfangs wurden sie zurückgewiesen. Doch als die Lage immer aussichtsloser wurde, ergriff man die Hand der neuen Verbündeten und mit ihrem Wissen über die Geheimnisse der Gegenwelt, Magie und Götterkraft gelang es abermals die Gegenwelt abzuschotten (siehe hierzu das \hyperref[Tal des Zwielichts]{Tal des Zwielichts}) und die Bedrohung abzuwenden, auch wenn weite Teile von Aurum Orbis in Trümmern lagen. 

\section{Der zweiten Manakriege}
Magie war im Laufe der Zeit immer mehr zu einem Werkzeug der Völker geworden. Seine allumfassenden nur von der Kreativität und der Gefahr von Manabränden eingeschränkten Einsatzmöglichkeiten, brachten Aurum Orbis und seine Bewohner allerdings in den Jahren, nachdem der gemeinsame Feind aller, die Gegenwelt, vertrieben wurde, in große Gefahr. Durch die radikalen Änderungen an der Realität, die mit dem Aufkommen neuer Konflikte immer häufiger zu Paradoxen führten, zerrten an dem Gerüst von Raum und Zeit selbst. In der Hitze des Gefechtes litten nicht nur diejenigen die sich der Magie verschrieben, sondern unter anderem auch die verhassten Dunkelelfen, die in einem Pogrom eines Feldherren beinahe ausgelöscht wurden.
\\Schließlich waren es die Drachen, die die bevorstehende Katastrophe abwandten und das \uline{Interdikt der Drachen}\index[Stichworte]{Interdikt der Drachen} der Welt aufzwangen. Als Folge zerfielen viele Großreiche, als die Magie ihren früheren Herrschern den Dienst verweigerten. Es folgten die Dunkelsten Jahren in der bisherigen Geschichte von Aurum Orbis, als jeder im Kampf um das Überleben über seine nächsten herfiel, da viele morbide Flüche das Land unfruchtbar machten und die Wasser vergifteten, sodass es an Nahrung fehlte. Am Ende sollte das weitere Wirken der Drachen aus dem Verborgenen heraus, das Land wieder auf den Weg der Besserung führten, wobei sie hofften das diese Katastrophe den Völkern für alle Zeiten eine Lehre sein würde.

\chapter{Die Jahre nach der Dunklen Zeit}
\section{Der Neuanfang}
Zerrüttet von den Kriegen mit der Gegenwelt und den letzten Manakriegen standen die Völker vor den Trümmern ihrer früheren Welt. Mühsam begann man mit dem Wiederaufbau der Städte, die Bevölkerungszahlen erholten sich langsam und die Drachen lockerten mit Blick auf die sich stauenden Manamengen ihr Interdikt, sodass Magie nun auch in beschränkten, dafür harmloseren Formen den Völkern wieder zur Verfügung stand. Auch wenn man bald wieder in die alten Muster aus Krieg und Intrige zurück fiel, blieb insgesamt die Lage innerhalb von Aurum Orbis für viele Jahre stabil. Einige besannen sich auf die dunklen Jahre zurück und beschlossen, um zukünftige Katastrophen dieser Art zu verhindern, Maßnahmen zu ergreifen. Die erste Bestand aus dem Schaffen der großen Bibliotheken und Archive um alles Wissen zu konservieren, sodass spätere Generationen aus den Fehlern ihrer Vorfahren lernen würden. Man musste allerdings feststellen, dass die dunklen Jahre einen schrecklichen Preis gefordert hatten: Vergessen. Bis auf wenige Bruchstücke und Legenden, in denen mehr Fantasie als Wahrheit steckten, ließ sich nichts über die Vergangenheit zusammen tragen. Selbst die Götter hüteten sich auf die Fragen ihrer Geweihten einzugehen, mit Blick auf die Drachen, die gegen Ende der Dunklen Zeit ihr wahres Potenzial und ihren Willen dieses auch einzusetzen und da sie die Gefahr dieses Wissen sahen. So gaben sie ihnen lediglich Mythen und Rätselhafte Andeutungen, um sie zumindest auf die Spur alter und neuer Gefahren aus der Gegenwelt zu bringen.

\part{Völker}
\setcounter{chapter}{0}
\chapter{Kinder von Aurum Orbis}

\section{Menschen}\label{Menschen}\index[Stichworte]{Menschen}
Bewundert, belächelt, gehasst. Über kein Volk herrschen so unterschiedliche und gegensätzliche Meinungen, wie das der Menschen. Durch ihre Emsigkeit haben sie trotz ihrer geringen Lebensspanne und Zerbrechlichkeit sich am meisten verbreitet in Aurum Orbis, was sie in ständige Kontakt mit allen anderen Völkern gebracht haben. Zu ihren größten Eigenschaften gehört ihre Anpassungsfähigkeit mit der sie selbst unwirtlichstes Gebiet bewohnbar machen und dabei ständig über neue Wunder, aber auch Gefahren stolpern. Ihre Unstetigkeit und vor allem ihr Streben nach Größerem sind dabei im Laufe der Zeit zu einem immer größerem Übel für andere und sich selbst geworden, weshalb man jeden Tag mit einem neuem Krieg rechnen muss.
\subsection{Menschliche Unterarten}
Jahrelanges Leben in den äußersten Winkel der Welt sorgte dafür, dass sich die Menschen angepasst haben und daraus sind sog. Unterarten entstanden:
\subsubsection{Frostbärte}
Die Eroberung ihrer Heimat in den Frostzähnen hat ihnen viel abverlangt und selbst nach vielen Jahren sind sie immer wieder im Krieg mit den Trollen. Dieser Umstand hat die Frostbärte gestärkt und sie bekannt für ihre Kampfkraft gemacht. Damit das Feuer ihres Herzen nicht zufriert, brennt es heißer und kann ihr Blut sehr leicht zum Kochen bringen, auch wenn sie in der Regel einen gewissen Sinn für Humor haben und sich Forstbärte nur zum Spaß mit anderen Prügeln. Lediglich wenn es um ihre Ehre oder die ihrer Familie geht, ist mit ihnen nicht zu spaßen und ein Falsches Wort kann für Jahrzehnte lange Fehden voller Blut und Intrigen führen. Auch wenn Frostbärte in der Regel gut untereinander auskommen, hegen sie in der Regel Misstrauen gegen Fremde Völker oder andere Menschenschläge, da sie um den Besitz ihrer Heimat fürchten. Diese Vorurteile sind allerdings nur lokal verbreitet und diejenigen, die es in die Fremde zieht, legen diese Unfreundlichkeit schnell ab. Außerhalb ihrer Heimat sind diese Hünen in der Regel für ihre Trinkfestigkeit, Kampfwut, im besonderen den Berserkern und ihrem Widerstand gegen die Kälte.
\subsubsection{Mittelländer}
Der ursprüngliche Menschenschlag, wenn man ihn so nennen möchte, der einst das Zentrum von Sechsstein bewohnt haben, ehe dieses im Zuge der Absplitterung östlicher Landmassen zu großen Teilen überflutet wurde und heute als die Grauen Marschen bekannt ist. Ihre neue Heimat fanden damit viele in Sommerfeld, wo sie dank der goldenen Küste bald wieder die Reihen der Adeligen und Eliten bevölkerten. Jeder Mittelländer ist gewitzt und geschickt in der Handarbeit, weshalb viele von ihnen sich als Händler oder Handwerker verdingen. Aufgrund des Makels, den sich ihre Vorfahren aufgeladen haben, wird ihre Herrschaft immer wieder in Frage gestellt, wodurch die Baronie Hortens bereits zusammengebrochen ist.
\subsubsection{Nebelheimer}
Die Menschen, die damals ihre Heimat in den grauen Marschen nicht aufgaben, als die Fluten des Meeres sich über ihr Land ergossen, werden heute abfällig als Nebelheimer bezeichnet. Klein, wenn auch gedrungen, gehen sie nicht so schnell im Moor unter und können sich ihrer Haut gegen Monster oder Banditen erwehren. Die Nähe zum Sumpfgas und der ständige Nebel haben ihnen eine blasse Haut beschwert und Einlagerung von Chemikalien in Haut, Augen oder Haaren, auch wenn diese außer der Verfärbung keine Auswirkung auf die Körperliche Gesundheit haben. Auf der geistigen Seite ist man sich nicht sicher ob es die Gase, der Mangel an Sonnenlicht oder die besonderen Umstände ihres Lebens in einem gewaltigen Sumpf geplagt von Monstern und Banditen ist, welcher für die vielen Merkwürdigkeiten im Gemüt von Nebelheimern verantwortlich ist. Letzter scheinen völlig willkürlich zu sein, auch wenn sie irgendwie familiär weitergegeben werden.

\section{Elfen}\label{Elfen}\index[Stichworte]{Elfen}
Auch wenn sie in ihrer jetzigen Form nicht direkt durch einen Gott geschaffen wurde, so sind sie doch die Verwandten der Dunkelelfen und werden als angestammtes Volk von Aurum Orbis betrachten. Ihre Heimat sind große unterirdische Reiche, in denen sie mit Magie Gärten anlegen. Ihre Gesellschaft gliedert sich streng in Kasten, die zwar miteinander arbeiten und leben, aber jeweils ihre Geheimnisse und Rituale für sich bewahren. In der Regel stehen sie als geschlossene Einheit gegenüber Fremden, denen sie zunächst neutral bis freundlich begegnen, solange sie sich innerhalb ihrer Reiche benehmen und ihre Nase nicht zu tief in die Angelegenheit der Kasten stehen. Um hierfür zu sorgen, existiert sogar eine eigene Kaste.

\subsection{Spezifikationen}
\begin{description}
\item[Elfische Perfektion:]
Aufgrund ihrer Langlebigkeit existieren in der elfischen Gesellschaft andere Maßstäbe für alle Bereiche ihres Lebens. Doch dies hat seinen Preis, die Kosten für das Erlernen und Verbessern von Fertigkeiten sind erheblich erhöht. Auf der anderen Seite erhalten sie einen einzigartigen Bonus auf einige von diesen:
\begin{itemize}
\item{Handwerk}
\\Elfische Handwerksprodukte sind filigran und von einer eigenartigen Ästhetik. Eine elfisch gefertigter Gegenstand ist immer um die Hälfte leichter ohne an Stabilität zu verlieren. Händler, die auf Ästhetik wert legen, sind in der Regel bereit 25\% mehr zu bezahlen. Außerdem qualifizieren sich nur elfische Waffen für elfischen Kampfstil.
\item{Kampf}
\\Mit der Zeit verfeinert ein Elf seinen Kampfstil bis zu dem Punkt, an dem er eins mit seiner Waffe wird und eine nahezu unüberwindbare Verteidigung erzielt. Einen Fehler in ihrer Parade muss erst bestätigt werden, ansonsten wird dieser als ein Grad besser behandelt.
\item{Soziales}
\\Auch in ihrem Umgang miteinander, legen Elfen eine ganz besondere Sorgfalt an den Tag. Ihre Sprache ist immer höchst melodiös. Vor allem bei anderen Völkern bleiben daher ihre Worte und ihr Gesang länger im Gedächtnis. Sprachliche Effekte haben daher die doppelte Wirkungsdauer. Auf der anderen Seite überlagert die innere Melodie die um sie herum und sie erhalten eine Erleichterung auf ihre Resistenz-Proben.
\item{Kochen}
\\Elfische Nahrung ist fast genauso zeitlos wie ihre Schöpfer und in der Regel doppelt so lange haltbar, verlieren dabei nie ihr Aroma oder ihr Aussehen, außerdem kann man elfische Speisen nicht unbemerkt vergiften.
\end{itemize}
\item[Drachenblut:]
Elfen haben im Zuge des Paktes ihr Blut mit dem der Drachen vermengt. Zwar ist es in ihren Adern nicht so potent, wie das ihrer Gönner, dennoch gewährt es ihnen ihre außergewöhnlich lange Lebenspanne, sowie eine natürliche Begabung für Magie. Elfen riskieren in der Regel keine Überdosierung, außer die Quelle übersteigt ihre eigene Manakapazität um ein 10-faches. Ansonsten besitzen sie eine gewisse Resistenz gegen Krankheiten die durch verdorbenes Mana weitergegeben werden.
\end{description}

\subsection{Kasten}
Auch wenn es überall unterirdische Elfenreiche gibt, die untereinander nur selten Kontakt haben, so sind die Kasten stammesübergreifend in ihren Traditionen und Lehren absolut identisch. Untereinander stehen sich diese sogar nähe als zu ihren stammesverwandten Kasten. Die hier angegebenen Kastennamen mögen seltsam wirken. Es handelt sich hierbei um Pseudo-Übersetzungen der elfischen Bezeichnungen.

\subsubsection{Hain der Felsblumen}
Der Ursprung der Elfen lag in einem Leben in der Natur und auch wenn sie nicht mehr viel mit den Dunkelelfen gemein haben, so fühlen sie sich unsagbar fest mit der Natur verbunden. Um sich also in ihrer neuen Heimat wohl fühlen zu können, legen die Elfen unterirdische Gärten oder sogar Wälder an. Um jedoch dies auf so unfruchtbarem Untergrund wie den Höhlen zu erreichen, haben die Mitglieder dieser Kasten eine Technik entwickelt um ein noch tieferes Verständnis von Pflanzen zu erhalten. Kern dieser Tradition ist die magische Symbiose mit einer persönliche Pflanze. 

\subsubsection{Herz der Felsen}
Neben ihren Gärten müssen natürlich auch die Tunnel und Kavernen gewartet werden. Diese Kaste verziert darüber hinaus auch die Felswände und überwacht die innersten Tunnel. Sie sind also eine Mischung aus Steinmetzen und Kämpfern, die nur sehr wenig Magie einsetzten.

\subsubsection{Wissen des Stammes}
Diese Kaste ist relativ jung und entstand, nachdem im Zuge einiger Krieger, einige Traditionen und Kulte der anderen Kasten drohten und teilweise wirklich verloren gingen. Diese Kaste genießt damit einen Sonderstatus, da die anderen Kasten ihre Geheimnisse mit dieser teilen. Allerdings verschafft dieses Wissen der Kaste nur begrenzt neue Möglichkeiten, da ihr Fokus auf der sicheren Konservierung des Wissen liegt. Sie selbst haben deshalb eine Geheimsprache und eine Reihe von Verschlüsselungsmethoden für verschiedene Überlieferungsformen entwickelt: Tänze, Gesänge, Malereien und natürlich auch Gravuren, dienen also nicht nur zur Aufwertung des Lebensstandard, sondern enthalten oft auch eine tiefere Bedeutung, die nur einem Eingeweihten klar wird. Im begrenzten Maße gibt diese Kaste auch einen Teil ihres Wissen an andere Kasten ab, um nicht selbst Opfer des Vergessens zu werden, dabei handelt es sich aber in der Regel um Schlüssel und nicht Verschlüsselungsmethoden. 

\subsubsection{Elfische Garde}
Auch wenn alle Elfen bis zu einem gewissem Grad eine Kampfausbildung erhalten, sind die Mitglieder andere Kasten in der Regel zu sehr mit der Pflege ihrer Tradition beschäftigt, um eigene Wachen aufzustellen. Die Garde erfüllt hier den Zweck, als Soldaten und Wächter ihren Stamm vor Gefahren von Außen zu schützen und über Besucher von Außen zu wachen. Sie werden vor allem für ihre Verschwiegenheit und Diskretion von den anderen Kasten geschätzt, während man sie wegen ihrem Kampfstil der eine Mischung aus Magie und Nahkampf darstellt gefürchtet werden. Ihre Magische Spezialisierung ist hierbei die Manipulation von Magnetfeldern, mit denen sie Gegner entwaffnen und Rüstungsträger bewegungsunfähig machen, während sie sich, selbst als schwer gepanzerte, leicht wie eine Feder durch Reihen von Gegnern schlagen. 

\subsubsection{Freunde in der Fremde}
Diplomatie war schon immer einer der wichtigsten Strategien der Elfen. Dieser Umstand geht zurück auf ihre Ursprünge als kriegerische Dunkelelfen zurück von denen sie sich gegenüber der anderen Völker distanzieren wollten, damit ihre neuen Siedlungen nicht ständig von Armeen überrannt werden. Aus diesem Grund studierte diese Kaste die Bräuche und Traditionen ihrer Nachbarn und war stets bemüht freundschaftliche Beziehung mit diesen zu pflegen. Mit der Zeit, als die Wirtschaft der Elfen immer größer wurde und sich ihre Reiche unterirdisch immer weiter ausdehnten, übernahm diese Kaste auch die Funktion als Händler und Führer von Fremden durch die sicheren Tunnel, um gefährliche Gebiete zu durchkreuzen. Dabei bewahrten sie aber stets die Neutralität ihres Stammes und reglementieren die Reisen, um in Kriegszeiten nicht zwischen die Fronten geraten zu können.

\subsubsection{Drachenelfen}
Im eigentlichem Sinne werden die Drachenelfen nicht als Kaste bezeichnet, vielmehr sind sie die Fadenzieher hinter dem sog. Drachenkult, einer von ihnen gegründeten Religion, in der sie als Nachfahren der Drachen auftreten. Man wird nicht in diese Kaste hineingeboren, sondern im Nachhinein auserwählt und eingeweiht. Drachenelfen sind hinter den Kulissen der Draht zu den Drachen und sorgen dafür, dass ihre Existenz ein Mythos bleibt. Sie leben in der Regel in einzelnen ihnen geweihten Tempel und werden von Mitgliedern andere Kasten oder Verstoßenen, die so ihre Schuld abarbeiten, umsorgt. Ihnen ist das direkte Wissen der Drachen zuteil geworden, weshalb sie über eine Unzahl mächtiger Zauber verfügen, mit denen sie ihren göttlichen Status aufrecht erhalten. Wenn sie nicht gerade ein paar ihrer Geheimnisse zur Entwicklung neuer Kastentraditionen nutzen oder in Notzeiten Rat und Magie zur Verfügung stellen, halten sie sich aus den Angelegenheiten des Stammes weites gehend heraus und meiden auch Besucher von Außerhalb.

\section{Dunkelelfen}\index[Stichworte]{Dunkelelf}\label{Dunkelelf}
Obwohl die Dunkelelfen bereits seit den ersten Tage in Aurum Orbis wandeln, kennen viele Bewohner dieses feindselige Volk nur aus Legenden und Geschichten. Ihre Heimat ist die tiefste Wildnis, in der sie in Stämmen als Nomaden von Ort zu Ort ziehen und auf ihrem Weg alles und jeden töten, was kein Dunkelelf ist. Sie sind gefürchtete Kämpfer, die selbst im Angesicht eines weit überlegenden Feindes keine Sekunde zögern. Aufgrund ihrer Kampfkraft und weil man die Wanderrouten der Dunkelelfen kennt und sich so vor ihren Schlachtzügen in Sicherheit bringen kann, wurden noch keine Versuche unternommen diese Monster, wie sie in Kindergeschichten genannt werden, zu erlegen. Die Motivation dabei ist ein uralter, tief in ihrem Geist verwurzelte Hass, der ihnen bei ihrer Schöpfung von Niju, auf alles was kein Dunkelelf ist. Im Kontrast zu diesem Hass ist ihre Liebe zum Stamm und ihre Opferbereitschaft um seinen Fortbestand zu gewährleisten grenzenlos. Bei ihren Reisen durch Aurum Orbis bündelt der Stamm alle Erfahrungen über das Land und seine Bewohner, die sie trotz ihrer Aversion im Sterben und danach ausgiebig studieren, in ihren Traditionen, sodass jede neue Generation noch besser angepasst an ihren ständigen Kampf mit der Welt ist. Nur selten bricht aus den Stämmen ein Dunkelelf auf, um in der Zivilisation zu leben. Diese Aussteiger sind dabei stets Exilanten, die im Zuge eines Konfliktes oder aufgrund eines eigenen Versäumnis verstoßen wurden. Man nimmt sie aufgrund vieler Vorurteile nur ungern auf und meist leben sie mit anderen Verstoßenden in den Elendsvierteln, wo sie entweder es schaffen sich mit ihrer Erfahrung im Kampf in einer Bande zu integrieren oder sie degenerieren zu einem Haufen Elend der seine Vergangenheit im Alkohol zu ertränken sucht.

\subsection{Eigenschaften von Dunkelelfen}
\begin{description}
\item[Widerborstigkeit]
Jahrhunderte in den abgelegenen Teilen der Welt haben den Dunkelelfen die Fähigkeiten gelehrt, selbst unter widrigsten Umständen zu überleben. Ein Dunkelelf kann, solange er bei Bewusstsein ist, in jeder natürlichen Umgebung, die nicht sofort tötet(z.B. Im Innern eines Vulkans oder Eiswasser aus hoher See ohne irgendetwas) überleben und sogar ihre Wunden und andere Gebrechen behandeln. Und selbst in absolut tödlichen Umgebungen überleben sie doppelt so lange, wie sonst üblich. Sie können diesen Bonus nicht auf andere Übertragen, da z.B. ein Dunkelelf auch ungenießbare bis giftige Nahrung verspeisen können.

\item[Dilettantischer Jäger]
Jeder Dunkelelf bekommt die Grundlagen der Jagd gelehrt. Sie erhalten zu Beginn die Fertigkeit Pirschen mit der Lernkomplexität A. Außerdem erhalten sie einen Bonus auf die Fertigkeiten: Fallen stellen, Wild verarbeiten und Spurenlesen.

\item[Unüberwindbare Aversion]
Auch wenn die Stämme der Dunkelelfen in Notzeiten den Wert von Zusammenarbeit mit anderen Völkern verstehen, nachdem sie ihn, in vielen Kriegen, auf die harte Tour lernen mussten, sind Religion und Magie zwei Dinge, denen sie eine tiefe Aversion entgegenbringen und die ihrem ganzen Wesen widersprechen. Kein Dunkelelf kann ein Geweihter werden, bzw. wird wenn er sie automatisch annimmt(z.B. durch Vampirismus) nicht nutzen. Ebenso widerstrebt es einem Dunkelelf ein Manablütiger zu werden und sein Körper wird von selbst das Gift bekämpfen und ggf. daran verenden. Doch ihr Widerstreben macht es genauso schwer sie auf diesen Wegen zu manipulieren. Gegen sie gerichtete Zauber haben eine geringe Chance von 5\% an der, den Dunkelelf wie eine Mauer umgebenden, Ablehnung zu zerschellen. Dunkelelfen erhalten stets einen zweiten Versuch um den Effekt einer göttlichen Intervention zu überwinden, wobei sie auch ihnen positiv gesinnte Effekte abzuschütteln versuchen.
\end{description}

\section{Mekka}


\chapter{Freunde aus der Gegenwelt}

\section{Nephilim}\label{Nephilim}\index[Stichworte]{Nephilim, die}
Einst waren sie Teil der Garde von \uline{\hyperref[Eron]{Eron}}, ultimativem Richter der Gegenwelt, den \uline{\hyperref[Engel]{Engeln}}. Doch auch Engel können Fehler machen, was jedoch unter Eron nicht toleriert wird oder den Sinn ihres Daseinszweck hinterfragen, wonach sie selbst entscheiden sich von ihrem Schöpfer abzuwenden. Mit dem Verlassen tritt eine schlagartige Veränderung auf: Ihre Flügel verbrennen, bis nur noch Bruchstücke oder gar nichts zurück bleibt, das Insignium von Eron auf ihrer Stirn wird unter Schmerzen entfernt, was zum Teil eine Narbe zurück lässt und ihre von Eron gegeben Waffen und Rüstungen zersplittern und zerfallen anschließend zu Staub, wobei dem Engel einige hässliche Narben zugefügt werden können. Anschließend sind sie Gejagte, sowohl von Engel, die den Makel in der Welt entfernen wollen, als auch denjenigen, die Eron einen Gefallen tun wollen, da über jedem Nephilim ein Todesurteil mit zusätzliche Belohnung ausgesetzt ist. Nephilim waren damit eine der Ersten, die sich nach Aurum Orbis flüchteten. Im zweitem Gegenweltkrieg kämpften bereits viele Inkognito in den Reihen der letzten Front und sorgten dafür, dass die zweite Welle Nephilim ohne Probleme überlaufen konnten, da sie wertvolle Informationen über die Gegenwelt und ihrer Bewohner, die für die Bekämpfung essenziell waren, mitbrachten. Nach der Schlacht zerstreuten sich die Nephilim und gliederten sich relativ schnell in die restliche Zivilisation ein, auch wenn es natürlich immer wieder zu den üblichen rassistischen Anfeindungen kam.
\subsection{Eigenschaften von Nephilim}
\begin{description}
\item[Heiliges Fleisch:]
Auch wenn die Nephilim nicht länger Gesandte Erons sind, so konnte dieser ihnen nicht nehmen, dass sie aus reinster Essenz bestehen. Durch diese kann ein Nephilim besonders leicht göttliche Macht kanalisieren, weshalb geweihte Nephilim einen Bonus auf Weihen-, Ritual- und Gabenproben erhalten. Auf der Anderen Seite läuft das Konzept der Magie ihrer Natur zu wider und es ist ihnen unmöglich manablütige zu werden.
\item[Vitalität]
Von Engel wurde oft erwartet, dass sie bei der Ausführung ihrer Pflicht häufig verletzt wurden und mit Krankheiten oder Giften in Kontakt kamen, weshalb ihre Körper unnatürlich schnell regenerieren. Diese Fähigkeit langsam auch ohne aktive Behandlung ihre Wunden zu heilen ist den Nephilim erhalten geblieben. Zudem erhalten sie einen Bonus auf alle Resistenzproben gegen Gifte oder Krankheiten.
\end{description}

\section{Drakling}\label{Drakling}\index[Stichworte]{Drakling, der}
*hier körperliche Beschreibung*
Das Volk der Draklinge war in seiner Heimat eines der unzähligen Heimatlosen, die ständig im Schatten stärkerer Ungetüme herumkrochen, wenn es ihnen auch gelang ihre Freiheit zu bewahren. Das war vor allem ihrer Anpassungsfähigkeit zu verdanken, die es ihnen erlaubte sich an den wenigen Orten aufzuhalten, wo andere zum Tode verurteilt wären. Es war mehr ein Zufall, der einen Stamm von Draklingen nach Aurum Orbis verschlug. Sie wurden, wie auch die übrigen Gegenweltler, mit Misstrauen konfrontiert. Doch ihre zurückgezogene Natur sorgte dafür, dass man sie passieren lies und im Laufe der Jahre nach den Gegenweltkriegen war man schließlich von der Ungefährlichkeit der Draklinge überzeugt. Doch wie schon in ihrer Heimatwelt blieben die meisten Draklinge unter sich und vegetierten am Rande der zivilisierten Welt, wo sie in einem besonderem Verhältnis mit der Natur leben.
\subsection{Eigenschaften von Draklingen}
\begin{description}
\item[Auf wilden Pfaden wandeln:]
Wie viele Tiere haben auch Draklinge ein besonderes Gespür für natürliche Schneisen und sichere Wege durch das Unterholz. Selbiges und anderes natürliche Gelände reduziert ihre Bewegungsgeschwindigkeit nur um die Hälfte und wenn sie es wünschen können sie keine Fährte hinterlassen, solange sie auf Rennen verzichten.

\end{description}

\section{Dämon}\label{Daemon}\index[Stichworte]{Dämon, der}
Keine Rasse ist in sich so gegensätzlich wie das Volk der Dämonen und dennoch sie stehen als ein Volk zusammen. Ursprung dieses Zusammengehörigkeitsgefühl ist ihre Vergangenheit in der Gegenwelt. Dort waren sie Sklaven größerer Scheusale und standen damit nur ein Stück über den vogelfreien Draklingen. Sie bildeten in den Gegelweltkriegen die erste Front von Angreifern und waren bei ihrer Wiederkehr deshalb gefürchtet und konnten nur durch eine langsame Unterwanderung von Aurum Orbis Fuß fassen und wurden nach ihrer Offenbarung gegen Ende des zweiten Gegenweltkrieges auch nur teilweise akzeptiert, weshalb bis in die heutige Zeit Verfolgung zum Alltag gehören. In den wenigen Orten wo Dämonen sich niederlassen konnten bilden sie verschiedenste Gesellschaften von anarchistisch bis monarchisch.
\subsection{Unterformen von Dämonen}
\begin{description}
\item[Geodum]Groß und stämmig, gehörnt und sehr kantigen Gesichtern waren sie früher Soldaten und Arbeiter. Ihr Körper erinnert an den von Kakerlaken mit 3 Beinpaaren, wobei sie in der Lage sind sich aufzurichten und ihr vorderstes als Armpaar nutzen können. Sie können gewaltige Lasten heben und widerstandsfähig gegen Angriffe oder Schutt. Als Akt der Verzweiflung oder des Hasses kann ein Geodum aus seinen vorderen Beinpaar zwei Unterarm lange Klingen herausschießen lassen. Diese sind von Natur aus vergiftet und stahlhart. Greift ein Geodum zu dieser Waffe, so verendet er in der Regel innerhalb eines gewissen Zeitfensters an einer Selbstvergiftung, verfällt aber dafür zuvor in einen Berserkerrausch, in dem sie sich in einen unaufhaltsamen Sturm aus Klingen und Zähnen verwandeln. Seit ihrem Exodus ist es Brauch rituell diese Klingen, bzw. den für ihre Entwicklung zuständigen Giftsack zu zerstören, womit das Risiko eines Berserkerganges verhindert wird. Geoden sind immer männlich.
\item[Cheylin]Von allen Dämonen ähneln diese Wesen den Humanoiden am meisten, weshalb es ihnen als erste gelang, in Aurum Orbis Fuß zu fassen. Letztere erinnern jedoch mehr Hufen und sind darüber von einem dichten, strubbeligen Fell, bzw. Flaum bedeckt, der sich bis auf Brusthöhe erstreckt. Viele Cheylin bevorzugen es daher ganz auf Kleidung zu verzichten, bzw. in Gesellschaft andere Völker eine Schärpe um die Blöße ihres Afters zu bedecken. Ansonsten fehlen innen sämtliche sichtbaren Geschlechtsmerkmale und wirken daher auf andere sehr androgyn. Nur ein Kenner kann die Geschlechter anhand ihrer Gesichtsform und bestimmter Untertöne in der Stimme identifizieren oder es handelt sich um einen Dämon, welche derartige Fragen ihrem Geruchs, bzw. Geschmackssinn überlassen. Weibliche Cheylin sind die einzigen Dämonen, die von nicht-dämonischen Völkern die Saat des Lebens empfangen können. Allgemein erfüllen Cheylin die Aufgabe eines lebenden Genpools, aus dem sich neue Dämonenarten züchten ließen. Allerdings fehlen seit dem, Auszug aus der Gegenwelt wichtige Komponente, um diesen Prozess fortzuführen. Sie können sowohl männlich, als auch weiblich sein.
\item[Snagger]Für viele nicht mehr als ein Mythos oder eine Schattenhafte Erscheinung. Dieser Zweig ist innerhalb von Aurum Orbis größtenteils unbekannt, was vor allem daran liegt, dass sie sich aufgrund ihrer Gestalt im Verborgenen halten. Im Wesentlichen ist es ein Knoten mit vier Auswüchsen auf denen sie sich bewegen können, mit denen sie auch sehr viel Fingerspitzengefühl beweisen. Ihre Wahrnehmung ist auf Geruch und Sonar beschränkt. Im Kampf verlassen sie sich auf einen Stachel, den sie aus einem ihrer Arme ausfahren können und der problemlos einen Menschen aufspießen kann. Dieser Stachel verfügt gleichzeitig über eine kleine Öffnung unter der Spitze, durch welche sie ihre Nahrung aufnehmen. Snagger besitzen seit dem Exodus ihrer Rasse kein Geschlecht mehr und konnten zuvor auch nur in Notzeiten als männliche Dämon einspringen.
\subsection{Gemeinsamkeiten}
Trotz ihrer Gegensätzlichkeiten sind alle Dämonenrassen untereinander sexuell Kompatibel. Die Nachkommen können dabei unabhängig ihrer Erzeuger auch der dritten Subgruppe angehören. Aller Nachwuchs von Dämonen wächst in Eiern heran, auch die Abkömmlinge mit nicht-dämonischen Arten. Damit dient die Fortpflanzung eher dem Erhalt der Rasse als der Weiterentwicklung, weshalb man dem Volk der Dämonen keine Neigungen zum Sex, des Sexes wegen nachweisen kann und sie gegenüber allen Versuchen, sie aus Basis von Reizen zu verführen nahezu unempfänglich sind. Des weiteren verfügt jede Dämonenrasse über eine ganze Reihe von Pseudo-Fertigkeiten, die jedoch nur mit Materialien aus der Gegenwelt ausführbar sind. Untereinander können sich alle Dämonen über ihren Geruchssinn erkennen und auch Verständigen, die dabei verwendeten Chemikalien sind für alle anderen Lebensformen nicht wahrnehmbar. 
 \end{description}  

\section{Lorkin}
Beseelt von einer unvergleichbaren Freude am Schabernack und gesegnet mit den Gaben ihres Schöpfers Umos erfüllen sie ohne Absicht den Dienst ihres Schöpfers. In ihrer ursprünglichen Gestalt nicht mehr als geduckte affenähnliche Kreaturen, die sich jedoch innerhalb weniger Augenblicke ihre Gestalt ändern können. Dies nutzen sie vor allem um andere zu imitieren und an ihre Stelle zu treten, auch wenn sie eine derartige Macht mehr für ihre Spielereien als Machtinteressen einsetzen. Dabei kennen letztere keine Grenzen, selbst vor den Göttern fehlt ihnen jeder Respekt. So haben sie einst Valery, zu einem Wettkampf mit ihren Sirenen herausgefordert. Ihr Sieg brachte ihnen das Geschenk der perfekten Stimme, jedoch auch als Racheakt der Göttin einen Fluch, der ihre Emotionen in Form leuchtender Muster auf ihrem Körper offenbaren. Über die Jahre entwickelt jeder Lorkin jedoch die Fähigkeit eine innere Ruhe zu finden und so ihre Maske zu perfektionieren. Bei Verlust durch Emotion muss ein Lorkin für mehrere Stunden meditieren.

\chapter{Mutanten und Halbblüter}
\section{Halbdämonen}
Geboren aus einem Ei einer Cheylin, gezeugt mit einem Fremden Volk stellt jeder Halbdämon in sich eine neue Untergruppe von Dämonen dar, allerdings ist ihr Blut nur zum Teil vom Dämonischen Erbe beherrscht.
Ein Halb-Geodum beispielsweise hat stellenweise einen im Licht schillernden Panzer und ihre Gesichter wirken seltsam deformiert, auch sind sie für ihre Statur ungewöhnlich kräftig, was ihnen im Kampf ein gewisses Überraschungsmoment gibt. 
\\Cheylinblut wiederum verleiht etwas von der seltsam abstrakten Schönheit und Anmut, sowie stärkeren Haarwuchs.
\\Ein Abkömmling im Sinne der Snagger wiederum hat blass-blaue, teilweise schwarze Haut und einen eigenen Stachel in einem ihrer Hände und ist sehr agil und hat ein gewisses Talent dafür sich mithilfe ihres Gehörs zu orientieren.
\\Kein Halbdämon erbt dafür den seltsamen Geruchssinn ihrer Erzeugerin, auch wenn ihnen eine gewisse Duftnote anhaftet, an der sich diese orientieren können. Die Geschlechtspräferenzen verfallen ebenfalls. Je nachdem in welchem Umfeld diese Kinder aufwachsen nehmen sie etwas von der Kultur und den Fertigkeiten ihrer Eltern mit. Im Gegensatz zu anderen Halbblütern ist der innere Konflikt der Zugehörigkeit nicht so stark ausgeprägt, da Dämonen ein Halbblut immer als ihres gleichen behandeln, während ihre starke Ähnlichkeit mit ihrem väterlichem Volk, sie nahezu perfekt mit der Gesellschaft verschmelzen lässt. Kinder von Halbblutdämonen nehmen nichts von dämonisches mit sich.

%GEOGRAPHIE!!!!!!!!!!!!!!!!!!!!
\part{Geographie}
\setcounter{chapter}{0}
\chapter{Übersicht}
Aurum Orbis verfügt über eine ganze Vielzahl von Kontinenten die durch das große Meer voneinander getrennt werden. Das Zentrum der Welt bildet dabei Sechsstein und die ätherischen Splitter, die einst östlicher Teil der Landmasse waren.
\chapter{Sechsstein}
Der erste von den Völkern bewohnte Kontinent hatte im Laufe der Geschichte viele Namen. Sein aktueller nimmt Bezug auf die geographische Aufteilung in 6 große Gebiete mit eigenen klimatischen Verhältnissen. Diese sind im Norden die Feuerlande, welches dominiert wird von einer Reihe gigantischer Vulkane. Ihre Hitze sorgt für eine Meeresströmung Richtung Süden, vorbei am gemäßigtem Sommerfeld und des von seiner Hitze ausgetrockneten Perlwüste. Der Süden und Osten wird dominiert durch das eisige Frostzahn und die grauen Marschen. Schließlich befindet sich im Nordosten das mysteriöse von Magieerruptionen geplagte verlorene Land, welches sich wie eine eiternde Wunde bis ins Herz des Kontinents erstreckt. Der Sechsstein ist die Geburtsstätte aller Völker und Quell von 3 Erzseen, darunter das Herz der Zwerge. Außerdem befinden sich sowohl das Arkeum, als auch das Tal des Zwielichts, womit es stets Zentrum großer Geschichtlicher Ereignisse war und sein wird.
\section{Feuerlande}
Die Himmel in diesen heißen Region ist immer grau und bewölkt und wo man auch geht und steht fällt Asche aus den Vulkanen herab. Atemluft ist an manchen Tagen fast ebenso ein knappes Gut, wie Wasser. Das Gesicht dieses Landstriches ändert sich ständig, das Lavaströme und Beben regelmäßig Material umschichten. Nur in einigen Arealen gibt es so etwas wie Zusammenhalt und dort haben sich rege Städte entwickelt. Dabei liegen der Fokus auf Landwirtschaft und Metallindustrie, was vor allem an den Lavaseen liegt.
\section{Verlorenes Land}
Einst blühendes Land, wenn auch teilweise unwegsam, war aufgrund seiner Lage und unglücklicher Zustände Zentrum aller Manakriege und den Gegenweltkriegen und dies hat Spuren hinterlassen. Wüste Landstriche die von wilden Bestien heimgesucht werden sind allgegenwärtig und machen es einfachem Volk nahezu unmöglich sich längere Zeit lebend dort aufzuhalten ohne einem der vielen Gefahren zum Opfer zu fallen. Zusätzlich zu den lebenden Gefahren kommen noch die Spätfolgen von Magieeruptionen, welche häufiger gefährlicher sind, als die wilden Einwohner. 
\section{Perlwüste}
Der Name dieses Landstriches kommt nicht von ungefähr. Im Gegensatz zu anderen Fels- oder Sandwüsten ist dieser Landstrich bedeckt mit einem Meer von in der Sonne schimmernde Kugel von der Größe von Sandkörnern bis hin zu kleinen Bällen, wenn auch die Mehrheit einer gewöhnlichen Perle ähneln. Lange Zeit gemieden von den meisten Völkern aufgrund mangelnder Bodenschätze und ihrer Unfruchtbarkeit geriet sie erst mit der Entdeckung der Kristallenergie, die den Perlen innewohnt, wieder in deren Fokus. 
\subsection{Phänomene des Perlsandes}
In ihrer Gesamtheit produzieren die Perlen Energie, sofern sie durch Druck belastet werden. Diese Grundform von Kristallenergie sorgt in erster Linie dafür, dass die Perlen zusammenhalten, sodass die Perlwüste problemlos zu Fuß überquert werden kann. Regelmäßig steigen aus den tieferen Schichten Energieblasen auf, die an der Oberfläche zu Stürmen eruptieren, wobei Dünen entstehen. Diese auf den ersten Blick an Sandstürme erinnernde Wetter ist jedoch wesentlich ungefährlicher, da die Perlen aufgrund ihrer beinahe Schwerelosigkeit von dem leichtesten Winden aufgehoben werden und damit nur sehr wenig kinetische Energie entwickeln.
\subsection{Die Monolithen}

\section{Sommerfeld}
Nachdem man das verlorene Land in den Ruin getrieben hatte zogen viele in die saftigen Hügel und Ebenen von Sommerfeld. In den Götterkriegen kahl gemacht durch die Götterkriege präsentierte es sich gewärmt durch den Warmen Strom der Feuerlande vor der Küste, als Paradies. Die Zivilisation blühte mit neuen bunten Städten und Dörfern auf. Dennoch geht es nicht immer friedlich zu, vor allem in den entlegeneren Winkel halten Namenlose Schrecken, tyrannische Herrscher oder dem üblichen Gesindel von Räubern und Wegelagerern.
\subsection{Goldene Küste}
Aufgrund seines gemäßigten Klimas, der relativ ebenen und trockenen Straßen, im Vergleich zu den sumpfigen Grauen Marschen, eignet sich die Westküste von Sechstein durch Sommerfeld als ideale Handelsstraße. Exquisiten aus allen Ecken der Welt laufen hier durch und durch die kräftigen Zölle und den fruchtbaren Boden gedeihen hier eine Reihe von Städten. Aktuell sieht die Situation so aus, dass sich drei mittelgroße Reiche den Bereich der goldenen Küste teilen. Die Goldstraße, so der Name der Handelsroute ist vor allem bei einfache Volk beliebt. Da sich die Gefahr durch Räuber in Grenzen hält, man nicht auf teure Navigatoren angewiesen ist und was vor allem für Saisonarbeiter ein Vorteil ist, billig zu nutzen ist, sofern man keine wertvollen Besitztümer mit sich führt. Zahlreiche Gilden und Söldnergruppen verdingen sich hier ihr Geld indem sie die wenigen Scheusale aus dem Osten fernhalten, die ihren Weg in dieses Gebiet gefunden haben.
\subsubsection{Baronie Lethe}
Der Norden befindet sich in fester, wenn auch wiesen Hand seinen Herrschafthauses und strebte stets nach Fortschritt und einem besseren Lebensstandard für alle Bürger. Weshalb Prunk und Festlichkeiten zur Tagesordnung gehören nur unterbrochen von den Phasen hitziger Kraftanstrengungen, die ganze Städte in Anspruch nehmen, um das aktuelle Ziel der Lethe zu beendigen. Die Baronie ist für ihre fortschrittliche Technologie bekannt, wozu vor allem die mächtigen Luftschiffe zählen, die unter vereinter Flagge, die größte Armee des Kontinents darstellt, wenn es Lethe mehr um den Ausbau seines Landesinneren, als die Expansion im Form von Kriegen geht. Dennoch sorgt die Baronie durch ihre Präsenz für genug Abschreckung auf die radikaleren Nachbarn, wie das Haus Rhone, um einen stummen Waffenstillstand zu erzwingen.
\paragraph{Maa}
Als Technologisches Zentrum vereint Maa, die größte Werft für Luftschiffe und die Universität der Baronie aus deren Hallen die Baupläne für diese aberwitzigen Gefährte stammen. Zwar gibt es auch einen kleinen Palast für den Baron, der jedoch ledeglich zu besonderen Anlässen, wie der Fertigstellung eines Luftschiffes, von diesem bezogen wird. Aufgrund ihrer Lage fernab der Goldstraße bleibt den Stadtbewohnern der ständige Durchzug allerlei Gesocks erspart, was zur Kultivierung eines für Außenstehende merkwürdigen Lebenstil aus heitere Geschäftigkeit geführt, da alle stets irgendwie am aktuellen Bauprojekt beteiligt sind und voll in ihrer Arbeit aufgehen.
\paragraph{Bofos}
Mehr Markt als Stadt wird Bofos dominiert von den Reisenden und Händlern, die mehr oder weniger unbeobachtet ihren Geschäften nachgehen können, sofern jede seinen Anteil abtritt. Keine Mauern aus Stein, sondern aus Zelten, umgibt die drei zentralen Märkte und die wenigen, stets übervollen Gasthöfe. Bewohner der Baronie meiden das Gebiet weiträumig, weil sie das Chaos nicht ausstehen können und so bleiben nur die Wachen, Zollbeamten und höfischen Händler, welche Materialien für die Industrie einkaufen.
\paragraph{Burg Höhenstolz}
Auf dem höchsten Gipfel, von dessen Spitze man beinahe die gesamte Goldene Küste mit Ausnahme des Hinterlandes überblicken kann, steht die Festung Höhenstolz. Hier residiert die Familie Lethe umgeben von einem Geschwader Luftschiffe, die aufgrund der Höhe bequem Andocken können und die gesamte Anlage in eine uneinnehmbare Festung verwandeln. Über das höfische Treiben ist nur wenig bekannt, außer, dass hier die besten Ingenieure ihre Fähigkeiten vereinen, um die nächsten Projekte für die Werkstätten und Werften zu entwickeln.
\paragraph{Port Silberstreif}
Während sich Maa nur mit dem Bau von Luftschiffen und anderen Technologieprodukten beschäftigt, ist die Hafenstadt Silberstreif, zwar auch mit einer Schiffswerft ausgestattet, doch im wesentlichen werden hier Waren verladen und sie über den Kontinent zu verteilen. Auch die Luftschiffflotte ist hier untergebracht, sofern sie nicht gewartet werden muss. Somit ist die Stadt Anlaufstelle für jeden Reiselustigen und Sammelpunkt von Matrosen. 
\subsubsection{Baronie Rhone}
Der Name Rhone löst bei vielen Besuchern einen unwillkürlichen Schauer aus. Gefüttert von den Einnahmen aus Handel und Landwirtschaft sitzt hier wie eine Made im Speck die Familie Rhone, ein ganz übler Schlag aus manischen Tyrannen, deren Willkür und Hang zu blutigen Exempel sich schon in den Bannern über ihren Städten äußert. Obwohl die Nachbarn in der goldenen Küste wesentlich gemäßigter und liberaler sind, wagte es bis jetzt noch niemand einen Schlag gegen das Herrscherhaus zu führen, was vor allem dem Ring der Tränen angerechnet wird, zudem seit jeher eine enge Bindung bestehen soll. Auch wenn sie relativ wohlhabend sind, ist die Forschung stark eingeschränkt, aus Angst vor zu viel revolutionärem Gedankengut. Daher sind Dampfmaschienen, wenn auch betrieben mit Sumpfgas, Technischer Höchsstand und die akademische Elite besteht im wesentlichen aus den wenigen Wartungsarbeitern.
\paragraph{Die Trasse}
Um den Kontakt mit der Außenwelt so gering wie möglich zu halten und dem Schmuggel entgegen zu wirken, ist der Abschnitt der Goldstraße eingemauert und gespickt mit Wachposten und Zollstationen. Zusätzlich zu der einfachen Straße existiert auch ein kleines Schienennetz für zahlende Kundschaft. Die einzige Abzweigung bildet eine Ader Richtung Meer zum großen Hafen Corva, vorbei an Groph und den Fabriklagern.
\paragraph{Groph}
Um die Waren von der isolierten Goldstraße ins Landesinnere zu schaffen oder andersherum, existiert in Groph ein gewaltiger Markt der rund um die Uhr besetzt und überwacht ist. Die Stadt ist außerdem Heimat der besser betuchten Leute, damit das Bild der Baronie nicht ganz so blutig herüberkommt. Bei Grophs Stadtplanung merkt man sofort, dass man sowohl auf einen Angriff von innen wie von außen geplant hat. Der Ring um den Markt ist zu beiden Seiten schwer befestigt und die 3 Hauptstraßen in das Umland sind ebenfalls ummauert und überwacht.
\paragraph{Corva}
Wer einen Blick auf die Karte von Corva wirft, merkt schnell wie schwer die Paranoia der Herrscherfamilie, vor fremden Einflüssen ist. Erbaut auf einer Klippe, an deren Fuß künstlich ein Hafen geschaffen wurde, wo durchreisende Schiffe untergebracht sind. Nur durch eine Schleuse, die schwer befestigt ist, kann man das auf der Klippe errichtete Hafenbecken erreichen und hier kommen nur Schiffe, mit der Absicht, echten Handel zu treiben herein. Ähnlich wie in Groph ist alles, was im Kontakt mit der Außenwelt steht, schwer bewacht. Corvas Bevölkerung sind Handwerker und Zollmeister. Seeleute kommen in der Regel durch andere Schiffe oder über den Landweg herein.

\subsubsection{Baronie Hortens}
Dekadenz und schwache Führung haben aus dem einst unter dem Herrschaftshaus Horten geeinten Süden einen Splitterstaat werden lassen, der jedoch bis heute es geschafft hat, seine Eigenständigkeit zu bewahren und der dank vieler Verträge seinen Teil an der goldenen Straße weiterhin zu einer lukrativen Einnahmequelle für alle gemacht hat. 
\paragraph{Kaltweihe}
Südlich vom Ghuile an der Küste in einer kleinen Bucht im Schatten des Passes nach Frostzahn, weshalb regelmäßig Schnee fällt, liegt die ehemalige Akademie der Baronie, an der in diesen Tagen Naturkundler, Alchemisten und eine Hand Magier zusammen ihre Feldstudien vorantreiben. Der Name der Akademie ist aufgrund ihrer geringen Größe und der hohen Streuung über die gesamte Provinz ist ihr Ruf unter Magiern eher lachhaft, wohingegen ihren Arbeiten in Naturwissenschaften und Geschichte sehr viel Beachtung und auch zahlende Gönner finden. Die Akademie steht in der Regel nur zahlenden Schülern offen, außer es mangelt an wissentschaftlichen Mitarbeitern, welche dann aus der Provinz rekrutiert werden und für die ihre Einschreibung oft als Ehre und Glücksfall gewertet werden.
\paragraph{Alt-Hortens}
Der ehemalige Hof des niedergegangen Herrschaftshauses Hortens ist nun Behausung der ehemaligen Hofbediensteten und ehemaliger Militärs mit ihrem Banner und nur noch ein Schatten seiner selbst. Man lebt von dem Gnadenbrot aus den Zolleinnahmen und etwas Landwirtschaft auf den Gehöften in der Umgebung. Die wenigen Gelegenheiten, zu denen etwas Leben in dieses Relikt vergangener Zeiten wiederkehrt, sind diplomatische Gipfel des Staatenverbundes. Aufgrund seiner Lage abseits der goldenen Straße im Nordosten der Provinz werden dort nie Gäste erwartet und die wenigen verirrten Wanderer und Karawanen abgewiesen, was vor allem an den geheimnisvollen Kavernen unterhalb seiner Mauern liegen soll, den sog. Varrentiefen.
\subparagraph{Varrentiefen}
Benannt nach ihrem Entdecker Varren Eisenherz spannt sich dieses Höhlennetz wie ein Labyrinth unter der gesamten Baronie Hortens, wenn nicht darüber hinaus. In den Dunklen Zeiten lebten nach dem Fall der Sommerfeldischen Allianz ganze Stämme hier und bildeten den Menschenschlag der Untergründigen. Ihre Existenz konnte später im hellen Zeitalter, als man mehr durch Zufall die ersten Eingänge unterhalb des Schlosses fand, nicht mehr nachgewiesen werden, bzw. wurde nicht mal angenommen. Auch wenn sich nahe der Oberfläche die ersten Spuren der Untergründigen finden, welche aber inzwischen weit tiefer leben, immer auf dem Rückzug vor der Bedrohung durch die Oberfläche, vor der sie Generationen zuvor geflohen sind. Aufgrund der vielen Zugänge, die man im Geheimen verteilt über die ganze Region aufgetan hat, ist eine unbekannte Anzahl von neuen Bewohnern in die Tiefen eingedrungen und haben es sich dort heimisch gemacht. Man vermutet dort unten inzwischen ganze Städten und Nekropolen, aber auch verlorene Schätze aus Gold und Wissen. 
\paragraph{Ghuile}
Neues wirtschaftliches Zentrum der Baronie, werden hier die Waren umgeschlagen und die Zölle kassiert. Damit Guile die übrigen Ministaaten betrügen kann, besteht der Stadtrat und die Reihen der Zollbeamten zur Hälfte aus Vertretern von außerhalb. Für Junge Leute aus dem Umland ist Ghuile die erste Anlaufstelle auf der Suche nach einer Ausbildung oder Aufträgen als Söldner, weshalb die Stadt ein bunter Trubel aus den unterschiedlichsten Bewohners ist und man in der Regel alles für den eigenen Bedarf finden kann. Da die Region vor allem für Viehzucht bekannt ist, ist der Fleisch und Viehmark erste Anlaufstelle für Gerüchte. Die örtlichen Tavernen sind das Kerkerloch, welches nur Abschaum beherbergt, die 
\subsection{Hinterland}
Das Vorgebirge von Frostzahn im Südosten ist aufgrund der wenigen Übergänge zu den umliegenden Ländern ziemlich isoliert und ist seit jeher gefangen in einem Stadium zwischen vollständiger Anarchie und selbst-erzwungener Ordnung. Tyrannen, Banden oder Parlamente, ja sogar der ein oder andere Nekromant hat sich hier selbst verwirklicht. Und dennoch hat es das Land und die Bevölkerung immer gut überstanden und letztere ist inzwischen an die Herrschaftswechsel gewohnt und ist viel zufrieden mit dem einfachen Leben als Bauer und anderer Landsleute, als dass es sie groß in die prunkvollen Reiche der goldenen Küste ziehen würde. Diese wiederum haben genauso wie die übrigen Nachbarn in den Grauen Marschen oder in Frostzahn kein Interesse an diesem Landstrich und aufgrund der bereits erwähnten Unwegsamkeiten sind die potenziellen Gefahren nur ein minimales Risiko für sie. Das Hinterland gehört damit trotz seiner zentralen Lage, zu einem der wenigen Orte über deren aktuelle Situation am wenigsten bekannt ist, weshalb vor allem durchreisende immer auf der Hut sein sollten.
\subsubsection{Freital}
Ein Richtung Goldener Küste geöffneter Berghang mit nur einer Stadt und einem dutzend kleiner Gehöfte, sind die Wälder eine idealer Rückzugsort für Banditen und anderes Pack, die sich an der goldenen Straße im Westen bereichern, weshalb regelmäßig Söldner zum Durchforsten dieser eingesetzt werden. 
\paragraph{Carnes}
Eine seit jeher Freie Stadt welche den südwestlichen Pass zum Hinterland dominiert. Sie lebt von dem wenigen Handel der über die unsicherer, aber dafür von Rhone nicht überwachten Route über die grauen Marschen und stellt vor allem für Händler mit für den Zoll in Rhone zu exquisiten Waren, die einzige Alternative dar. Carnes ist gleichzeitg Teil der Baronie Hortens und besitzt dennoch die Eigenständigkeit der Hinterländer, was vor allem daran liegt, das die von ihr Kontrollierten Gegenden nicht Teil an der Goldenen Straße haben und sie daher in den meisten Verhandlungen über den Zusammenhalt der ehemaligen Baronie außen vorgelassen wurde. Regiert wird die Stadt seit jeher durch einen Rat aus Magistraten, der sich aus Händlern, Dorfältesten und Spirituellen Führern und Gelehrten aus Freital zusammen setzt.
\subsubsection{Mondtal}
Durchzogen vom namensgebenden Mondstrom, ist dieses Tal mit Sorn, Öllepo und der Festung Nimmermorgen, einigen kleineren Höfen mit weiten Weiden und wenigen kleinen Forsten sehr gut Überschaubar trotz der Größe des Tals.
\paragraph{Öllepo} 
Nach Nimmermorgen die größte Stadt, welche das Zentrum für den Handel mit der Außenwelt darstellt. Hier treffen die teilweise seltsamen Normen der aktuellen Herrscher auf die meist gemäßigten Ansichten der Außenwelt, was nicht selten zu Spannungen führt. In Öllepo trifft die Straße nach Carnes auf jene aus dem südöstlichem Frostzahn und der aus den grauen Marschen. Der Mondstrom Richtung Nimmermorgen dient als schneller Wasserweg, um von den Herrschern in Öllepo erstandene Waren zu ihnen zu liefern. Aus diesem Grund ist das Fischen nördlich der Stadt verboten.%Yay random Facts!
\paragraph{Nimmermorgen}
Grau steht die Festung im Schatten einer Bergflanke, sodass das Licht der Sonne nie ihre Felsen berührt. Hier auf einer Anhöhe, neben der sich der Mondstrom in ein unterirdischen Flussbett ergießt, residieren für gewöhnlich die Herrscher von Mondtal und blicken auf ihr Herrschaftsgebiet herab. Im Laufe der Zeit wurde die ehemalige Festung immer wieder um neue Anbauten erweitert, welche nach einem Machtwechsel entweder abgerissen und durch neue ersetzt wurden oder einfach verfallen lassen wurden. 
\paragraph{Sorn}
Dieses kleine Dorf ist ein Vorposten und erste Anlaufstelle, um sich über den aktuellen Zustand der Route durch das Hinterland zu informieren. 
\section{Graue Marschen}
Dieses eigentlich voller Leben strotzende Land ist seitdem sich die westlichen Landmassen erhoben haben und nun den Splitterkontinent unter den Fluten des Meeres begraben worden, womit es nun mehr Sumpf als Wiesen gab. Das unnachgiebige Pflanzenleben, welches angetrieben von uralter Magie, verwelkt, bevor es sich richtig aus dem Boden erheben konnte. Als Resultat liegt unter dem Wasser ein Haufen verrottender Pflanzen, die durch Bakterien in sog. Sumpfgas umgesetzt wird. Letzteres wird durch große Glocken abgefangen, kompressiert und anschließend über den restlichen Kontinent verteilt, wo es als billiger Treibstoff für Lichtanlagen, Motoren und andere Zwecke verwendet wird. Aufgrund der Unwegsamkeiten und häufigen Nebelsuppen unterliegt das Gebiet keiner dauerhaften Überwachung und wird regelmäßigen von den unterschiedlichsten Bestien, aber auch Räubern und Saboteuren heimgesucht.
\subsection{Gefahren}
Auch wenn die Marschen auf den ersten Blick, wie ein großer Sumpf wirken, gilt es acht zu geben, da die besonderen Eigenschaften des Landes neue Gefahren auftun.
\subsubsection{Gasnebel}
Nebel ist allgegenwärtig in den Grauen Marschen, doch wenn eine kalte Strömung aus den Fjorden Frostzahns auf einen warmen Wind aus Sommerfeld trifft kann es passieren, dass die Luft so sehr mit Wasser gesättigt wird, dass das ausströmende Sumpfgas sich nicht mehr in höheren Luftschichten verlieren kann und ein toxisches Level in Bodennähe erreicht.
\subsubsection{Wechselfurten}
Teilweise besteht der Untergrund aus treibenden Pflanzenresten, die je nach Wasserstand soliden Boden bilden, doch bei Fluten zur Todesfalle werden.
\subsection{Der alte Pfad}
Vor der Flutung der Grauen Marschen, verlief hier eine Handelroute ähnlich der Goldstraße im Sommerfeld. Auch wenn von diesem Weg nur noch Reste übrig sind und man teilweise weite Umwege in Kauf nehmen muss, um überflutete Senken zu umgehen, wird die Straße immer wieder von Händlern und Schmugglern benutzt, da sie auf dem Land das Gebiet der Baronie Rhone umgeht, welche für ihre willkürliche Beschlagnahmung von allzu kostbaren Waren bekannt ist. Der Dienst eines Kundschafters oder Führers durch das Gelände ist daher sehr lukrativ und oft die Fahrkarte für die Jungen aus ihren Dörfern heraus.
\section{Frostzahn}
Das eisige Gebirge im Süden von Sechsstein trägt seinen Namen nicht nur aufgrund der Ähnlichkeiten zu dem Unterkiefer eines Trolls, sondern auch wegen der großen Populationen dieser Biester. Die großen Forste und Erzvorkommen sind wahrscheinlich der ursprüngliche Grund gewesen, weshalb man hier Fuß gefasst hat.
\subsection{Tal Vinura}
Das größte bewohnte Tal liegt im Westen und lässt sich beinahe ebenen Fußes durch die Baronie Hortens betreten werden. Wo kein Dorf oder eine Straße ist, befindet sich in der Regel ein dichter Nadelwald, der Heimat vieler Wildtiere ist. Durch seine Anbindung an Numidas Tal und dem unterirdischen Erzsee um südlichen Gebirge herrscht ein vergleichsweise reger Handel, der jedoch regelmäßig durch die Banditen und Stämme von Wilden behindert wird.
\subsection{Tal Numidas}
Hoch im Gebirge in einem kleinem Kessel entspringt Numidas Träne. Dieser See ist bekannt für den nimmer enden Strom der ihm entspringt und sich ins Hinterland und zu großen Teilen die graue Marschen ergießt. Zusätzlich ist der See mit gewaltigen Fischschwärmen gesegnet, eine Nebenwirkung göttlichen Einflusses auf diese Anomalie. Der größte Nachteil für die Bewohner sind die sog. Trollwinter, wenn der Strom vollständig zufriert und so den Weg für die Trolle öffnet.
\subsection{Trollheim}
Der letzte Winkel von Frostzahn in den die Zivilisation noch nicht vorgedrungen ist, wird von Trollen regiert. Eisiger und karger hat es daher noch kein Vorstoß gegen die Trolle diese Plage nicht beseitigen können, sondern vielmehr fielen die Trolle bei jeder Gelegenheit über die restlichen Täler her und hinterließen nur brennende Ruinen. Aus diesem Grund ist über dieses Gebiet nur wenig bekannt. 
\subsubsection{Festung Trollschild}
Auf einer Landzunge, die im Falle eines Trollwinters erste Anlaufstelle für diese sind, haben die Frostbärte irgendwann eine Mächtige Festung errichtet, an deren Ausläufern sich sonst die Wellen brechen. Besetzt mit schwerer Artillerie mit der sie weiträumig das Gebiet um sich mit Speeren und Brandmunition eindecken können, hat es schon häufig die ersten Vorstöße der Trolle gebrochen und so dem Festland genug Zeit gegeben sich vorzubereiten. Für viele Junge Männer aus der Region ist es daher eine Ehre auf den Mauern zu dienen. WIe oft man allerdings die Stadt wiedererrichten musste, da man nicht bis zum Schmelzen der Seeoberfläche die Ungeheuer zurück halten konnte und die Stadt daher auf ihrem Zug verwüsteten und hunderte ihr Leben kostete, will niemand sagen.

\part{Magie}
\setcounter{chapter}{0}

\chapter{Ursprung von Magie}
Der wahre Charakter ist allen Völkern, ja sogar den Göttern ein Mysterium. Schließlich ist die Magie selbst ein Nebenprodukt des Einflusses der Axiomaten. In der Regel räumen die Axiomaten hinter sich auf und neutralisieren sämtliches Magierauschen, weshalb in anderen Welten diese Form der Energie nicht bekannt ist. Aufgrund der unglücklichen Geschehnisse beim großen Streit wurde jedoch schlampig gearbeitet, sodass die vielen Anomalien, ja selbst weite Teile der gewöhnlichen Schöpfung über ein Magierauschen verfügen. Die Adaption durch die Drachen war ein langwieriger Prozess, deren Anfang ihre Interesse an den Anomalien war. Über die Entwicklung von Destillationsverfahren unter den übrigen Völkern ist nur wenig bekannt, auch wenn man munkeln kann, dass die  Elfen durch ihre Vermengung mit anderen Völkern die Geheimnisse weitergegeben haben, was angesichts der Resultate wohl ein Fehler war und man munkelt, dass die Drachen deshalb so sehr daran interessiert sind, die Konsequenzen der Manakriege so sehr zu bereinigen.  

\chapter{Mana}\index[Stichworte]{Mana}\label{Mana}
In ihrer Rohform als energetisches `Rauschen`, welches für Magie empfängliche  Wesen wahrnehmen können oder das bei ihrer Weiterverarbeitung entsteht, ist Magie nicht nutzbar, da es selbst in direkter Nähe zu einer Anomalie, wie dem Arkeum, nur in einer geringen Konzentration auftritt und ständig voneinander weg strebt. Um sie nutzbar zu machen, muss Magie zu Mana destilliert werden. Drachen haben ihre Körper soweit umgestaltet, dass sie die Magie direkt destillieren können, während sich unter den sterblichen Völkern unzählige Methoden gebildet haben, von denen jede ihre Vor- und Nachteile besitzt. Trotz der vielen Variation gelten für jede Form von Mana ein paar konstante Regeln:
\begin{itemize}
\item Mana kann immer in den klassischen Aggregatzuständen: Fest(Kristallin), Flüssig und Gasförmig auftreten. 
\item Jedoch ist seine Dichte nicht konstant. So können bereits mit geringen Mengen Mana große Kristalle konstruiert werden, die man im Nachhinein mit Mana füllen kann.
\item Untereinander sind die verschiedenen Mana-Formen kompatible und können ihre Energie aneinander abgeben.
\item Treffen sich zwei Manareserven, so beginnt ein Manastrom in Richtung der Manareserve mit dem geringeren Mana-Volumen Verhältnis. Diese Eigenschaft ist vor allem für das Auffrischen von Manareserven oder dem Konzentrieren von Mana-Nebeln einiger Destillationsverfahren wichtig. Der Strom bricht erst ab, wenn eine Manareserve bis zum Existenz-Minimum verbraucht ist, gilt nur für Kristalline Vorkommen oder der physische Kontakt unterbrochen wird.
\item Ähnlich wie bei der Destillation, existieren auch eine Unzahl von Methoden, um dieses Verhalten zu verändern. So kann man den Fluss umpolen, den Fluss so begrenzen, dass es nicht zu einer Überdosierung kommt oder die Geschwindigkeit des Flusses verringern.
\item Wird Mana stark überdosiert erhält man eine \uline{Manabombe} \index[Stichworte]{Manabombe} \label{Manabombe}
, die durch ein weiteres hinzufügen von Mana, schließlich kollabiert, wodurch in einem großen Radius die Realität einer chaotischen Veränderung unterworfen ist, welche Lebensformen in der Regel tötet und tote Materie entweder radikal umstrukturiert oder neuen Naturgesetzen unterwirft. Aufgrund des zerstörerischen Potenzials wird streng darauf geachtet, dass vor allem Magier nur passend dosierte Mana-Mengen erhalten. Hält sich ein Magier nicht an seine Dosierung, kann eine Destabilisierung seines körpereigenen Manas, ihn vergiften oder gar töten, in diesem Fall handelt es sich um eine Form von \uline{Manabrand}. \index[Stichworte]{Manabrand!Überdosierung}
\item Mana wird bei seiner Anwendung wieder in ein `Rauschen` umgesetzt, dieses gezielte Rauschen, führt in seinem Ursprung(oft das Blut des Magiers) zu Resonanzen, die ab einer gewissen Stärken, zu einer Pseudo-Manabombenexplosion führt, welche in sehr begrenztem Umfang(im Falle eines Magiers, sein Körperinneres) schweren Schaden zufügen. Im Falle eines Magiers handelt es sich um eine Form von \uline{Manabrand}. \index[Stichworte]{Manabrand!Resonaz} 
\end{itemize} 
Als Mana destillierte Magie, kann nun durch die Verbindung mit einem Bewusstsein dazu genutzt werden, um in das Realitätsgefüge einzugreifen. Dabei kann eine Veränderung von einem subtilen Eingriff in bestehende Gesetze, eine Kunst, die vor allem die Drachen gemeistert haben, bis zur Holzhammer-Methode bei der man bestehende Gesetze durch eine eigene Realität überschreibt, was jedoch auf Dauer die Welt in einen instabilen Flickenteppich verwandelt, weshalb die Drachen diesem Verhalten einen Riegel vorgelegt haben. Ein Limit, wie weit man mit einer dieser Methoden kommt hängt überdies von der Stabilität des verwendeten Manas ab, um nicht Opfer eines Manabrandes zu werden.

\section{Manablut}
Wie bereits oben erwähnt, ist es bei allen Magieanwendern üblich, dass sie eine Portion flüssigen Manas in ihrem Blut tragen, um den Kontakt zwischen diesem und ihren Gedanken herzustellen. Die Variante des Manas ist dabei für das Potenzial des Magiers bis auf wenige Ausnahmen irrelevant, sondern ist vielmehr für die Natur des mit der Vergiftung durch das Mana ausschlaggebend. Drachen haben bei ihrer Entdeckung der Magie von ihren durch die Götter verliehene Macht ihren Körper und auch ihr Blut angepasst. Drachenblut ist ein besonderer Magiespeicher, der den Drachen nicht nur einen Nachteil-freien Umgang mit Mana ermöglichen, sondern darüber hinaus Manabrand nur in Ausnahmefällen zulässt. Gepaart mit der Langlebigkeit der Drachen und dem daraus resultierenden Wissensschatz zählen zu den mächtigsten Wesen in Aurum Orbis. Als sie den Pakt mit den Elfen abschlossen vermengte sich ein Teil ihres Blutes mit denen der Drachen, was jeden Elfen von Natur aus zur Magie befähigt. Ihr Blut ist zwar nicht so potent wie das ihrer Verbündeten, allerdings sind sie zumindest bis zu einem gewissen Grad gegen eine Überdosierung geschützt und Resonanzen beim Zauberwirken sind halbiert.

\section{Manabrand}\index[Stichworte]{Manabrand}
Ein Magieanwender bekommt bei seiner Initiation eine ganze Reihe von Verhaltensregeln eingegeben. Während die meisten eher ethischer Natur sind und je nach Überzeugung seiner Schule/Organisation variieren, gibt es im allgemeinen einen Block an Schulen-übergreifenden Sicherheitsvorschriften. Diese dienen dazu den Magieanwender vor einem Manabrand zu schützen. Dieser ist Folge einer Destabilisierung des körpereigenes Mana, welche entweder durch Überdosierung, Aufzehrung aller körpereigenen Mana oder durch Resonanz, hervorgerufen werden. Die Symptomen reichen von Übelkeit über Fieber und Krämpfe, bis hin zu tödlichen inneren Blutungen oder noch viel grässlicheren tödlichen Deformationen der Innereien.

\chapter{Anwendungen von Magie}
In seiner Rohform und vor dem \nameref{Interdikt der Drachen} bedurfte es einfach eines Gedachten, wie man sich die neue Realität vorstellte und das Körpermana wurde durch den Willen geprägt und anschließend freigesetzt. Je nachdem wie viel Widerstand in Form von Naturgesetzten und gegensätzlichen Zaubern überwunden werden muss, steigt der Manaverbrauch, bis hin zum Punkt, dass der Versuch bereits unmöglich ist für jeden Sterblichen. Jedoch lassen sich die Kosten drastisch reduzieren, wenn man die Magie mehr wie einen Hebel, als einen Hammer verwendet und durch geschickte Nutzung der bereits bestehenden Zustände die gewünschte Wirkung erzielt. Eine Kunst die die Drachen in Jahren des Studiums Perfektioniert haben. Doch dies setzt umfassendes Wissen über die Funktionsweise des Universums voraus, weshalb auch den Drachen manche Unterfangen, wie die Reise in die Vergangenheit verschlossen bleiben.

\chapter{Interdikt der Drachen}\label{Interdikt der Drachen}\index[Stichworte]{Interdikt der Drachen}
Um den immerwährenden radikalen Einsatz von Magie, vor allem durch Gewohnheitsmagier, die selbst bei einfachsten Tätigkeiten am Gefüge der Welt selbst herumschraubten, auf ein Minimum zu reduzieren ersannen die Drachen einen eigenen Zauber, eine neue Regel, die sie entgegen ihrer langjährigen Tradition, auf die Hammermethode beim Zaubern zu verzichten, in Form eines Willenszaubers in einem Gemeinschaftlichen Akt auf die Welt drückten. Im Groben verbietet er jedem Magieanwender, der nicht den Segen der Drachen empfangen hatten, wozu anfangs nur sie selbst und die \nameref{Drachenelfen} zählten, nicht mehr dem Mana ihren Willen auferlegen konnten, sondern dem von einem gesegneten etablierten Weg Magie zu wirken erneut beschreiten mussten. Wenn sich ein Gesegneter einen Gesang beim Zaubern anstimmte, so konnte jeder Manablütige durch Rezitieren, den Effekt kopieren. Etablierte Zauberrituale blieben damit einmalige Regeländerungen an der Realität, sodass jeder andere Zauberer nur den theoretischen Hebel dieser Regel zog. Dieser Schachzug sollte die Welt vor einem Zerfall bewahren, doch das Mana begann sich in den Jahren seines Nichtverbrauches immer weiter zu stauen und so segneten die Drachen ausgewählte Individuen im Geheimen, sodass sie weitere Rituale oder umgangssprachlich Zauber erschufen, die der Rest der Welt übernahm. Bei den wenigen Auserwählten handelte es sich in der Regel um Menschen die Entweder durch ihre Moral oder teilweise durch ihren eigenen Wahnsinn für ihre Zauber ein gang eigenes System erschufen, das stets seine Vor- und Nachteile, Beschränkungen und Legenden besaß, sodass es später keinem Gelehrten gelingen würde, einen gemeinsamen Nenner zu finden.



\chapter{Magieschulen}
Im Laufe der Zeit gab es viele Gesegnete, die oft unwissentlich den Auftrag der Drachen ausführten, den Bewohnern der Welt einen ungefährlichen Zugang zu Magie zu geben und sie alle sollen hier aufgeführt werden.

\section{Arkane Künste}
Einer der ersten Neu-Gesegneten war Vorkel, ein mutierter Student des Arkeums, der seit Jahrhunderten die verlassenen Gänge von Thalum. Sein selbst entworfenes Runensystem mit dem sich jeder Vorgang und jedes Objekt der Welt beschreiben lässt, wurde als erstes von den Drachen erwählt, um den Völkern Zugang zur Magie zu gewähren. Es zeichnet sich durch seine Vielfältigkeit aus, da sich nahezu jedes gewünschtes Ziel in diesem Beschreiben lässt. Jedoch ist die Anzahl der Runen sehr unübersichtlich und die Komplexität und damit Länge von Zaubern steigt extrem schnell an, da viele Begriffe sich aus durchschnittlich 3 Grundbegriffe zusammensetzt. Zauber können dabei in Form von Runen oder ihrer vokalen Form gewirkt werden, wobei ergänzende Handbewegungen z.B. zum Zielen erforderlich sind. Im Laufe der Zeit ist die Grundlegende Grammatik und der Wortschatz mehr Bücherkram geworden, stattdessen werden eine Vielzahl von fertigen Zaubern unter Anhängern dieser Disziplin verbreitet: 
\begin{description}
\item[Nox]Schlagartig wird, solange der Zauber aufrecht erhalten wird, alles Licht bis zur völligen Schwärze gedämmt.
\item[Flamma]Mit einem Fingerzeig entzündet sich die Luft im Bereich.
%wie man schon ahnen kann sind die Runen an das Lateinische Angelehnt
\end{description}

\section{Alchemistische Ordnung}
Wieder einmal war es ein wahnsinniger Geist, der hinter diesem System steckt. In den Kerkern der Festung Neumon hat eine Gruppe von Wissenschaftlern für ihren Fürsten an Waffen gearbeitet. Doch im Zuge ihrer Forschung wurden jedoch allen möglichen Dingen sog. Astrale Eigenschaften zugeordnet, die unter Zuhilfenahme von verschiedenen Werkzeugen und in langwierigen Prozessen miteinander kombiniert werden, um verschiedene Tränke, Pülverchen, Bomben oder Talismane zu erstellen. Die Drachen haben es trotz ihrer Abneigung gegen die Methoden der \''Grundlagenforschung\'', dieses System zu einer funktionierenden Magie-schule erhoben, auch wenn sich Alchemisten niemals als Magier bezeichnen würden. Alchemie findet in dunklen Laboren, voller seltsamer und äußerst Filigraner Apparaturen. In der Regel können sich daher nur besser gestellte Personen ein derartiges Unterfangen leisten, auch wenn es nicht unüblich ist, dass sich solche Exzentriker Gehilfen leisten, die im Laufe der Zeit sich das Notwendigste zusammen suchen, um ihre eigenen Forschungen voranzutreiben. Im Folgendem soll eine Einführung folgen:
\subsection{Grundlagen}
Jeder Pflanze, jedem Mineral/Erz, ja sogar jeder Faser lebendem Materials, vom Haut, Haaren und Blut, aller Wesen wohnt Energie, in Form von Astraler Eigenschaften inne. In ihrer Grundform zeigen nur wenige Ingredienzien ihre verborgene Macht, sondern müssen erst extrahiert und konzentriert werden. Die Extraktion erfolgt dabei durch eine Reihe von Arbeitsschritten, die jedoch alle ab einem gewissen Punkt, den sog. Sternenkarte einbinden. Diese Platte ist der eigentliche Ursprung aller vermeintlichen Astralen Energien. Die Ingredienzien liefern hierbei nur den Auslöser oder Fokus, auch wenn dies natürlich keinem Alchemisten bewusst ist. Für den Alchemisten ist die Arbeit vielmehr ein kunstvolles Arrangieren und Vermengen von einzelnen Komponenten. Auf dem Bord gelten ein paar Grundlegende Prinzipien:
\begin{enumerate}
\item Gegensätze ziehen sich an. Liegen sie zwei gegensätzliche Komponente im äußeren Ring, werden sich ihre Essenzen im Zentrum treffen.
\item Gleiches stößt sich ab. Liegt eine ähnliche Komponente im Nebenarm, wird ein solcher Strom unterbrochen.
\item 
\end{enumerate}

\section{Segen der fünf Elemente}
Das Große Problem in vielen Magieschulen ist, dass große Wirkung oft nur mit viel Manaaufwand einhergeht. Die Lösung für dieses Problem sind die Beschwörer, welche die Energie für ihre Zauber nicht aus dem Nichts erschaffen, sondern vielmehr auf die in der Welt vorhanden Energiereserven, wie dem feurigem Kern unter ihren Füßen. Und auch wenn man sich fragt, weshalb die Drachen einer Schule mit solchem Zerstörungspotenzial zugestimmt haben, findet seine Antwort, wenn man die Wirkung eines gewaltigen Feuerballs, der langfristig keine Spuren hinterlässt mit denen von Manabomben vergleicht, die weniger vorhersehbar und deren Kausalitäten bis heute sichtbar sind(Stichwort Arkeum). Der Preis für eine derartige Macht ist in vielerlei Hinsicht hoch: Auch wenn die meisten Zauber langwierige Rituale sind, kann ein Emotionsausbruch jederzeit die vom Beschwörer gebundenen Mächte freisetzten und somit viel ungewolltes Leid für ihn selbst und andere verursachen. Zusätzlich behagt es niemanden, wenn jemand mit der Macht kleinere Städte den Erboden gleich zu machen frei umherläuft, weshalb viele Beschwörer von Herrschern versklavt und an der Leine gehalten werden. Und schließlich die physische Bürde seine schweren Machtkristalle jederzeit mit sich zu schleppen und vor Diebstahl zu beschützen.
\\Es folgt ein kleiner Einblick in die Lehre von den fünf Elementen. Diese sind Feuer, Erde, Wasser und Luft, sowie Lyrium, welches den Geist verkörpert. Jeder Beschwörer wird im Zuge seiner Ausbildung mit einem der ersten vier Elemente vermählt, wodurch sie dieses überall und als Teil einer Beschwörung einbringen können, ohne ihre Ritualzeit zu verlängern. Andere Elemente müssen durch aufwendige Rituale gesammelt werden, auch wenn sie sich für die spätere Anwendung in einem sog. großen Fokuskristall abgelegt werden, den ein Beschwörer im Laufe seiner Ausbildung erhält und ihn danach stets mit sich tragen muss, da ein Großteil seiner Zauberkräfte in diesem Gebunden sind. Maximal kann ein Beschwörer zwei solcher Kristalle besitzen. Das Mysteriöse Lyrium kommt in der Regel erst bei hochgradigen Beschwörungen von semi- bis vollbewussten Elementaren zum Einsatz, um ihnen Charakter und Willen zu verleihen, womit wir auch zu der Liste beispielhafter Beschwörungen kommen.
\begin{description}
\item[Kleiner Fokus]Diese erste Beschwörungen erlauben es eines der vier Elemente zu materialisieren. Feuer als eine über Stunden brennende Flamme, Wind als keinen Wirbel auf dem leichte Objekte getragen werden, Wasser als Perle, die Trinkwasser produziert und schließlich Erde in Form eines kleinen Steins mit immensem Gewicht.
\item[Elementarer Impuls]Dieses vergleichsweise kurze Ritual dient dem Beschwörer als Zauber für den Nahkampf. Seine Wirkung ähnelt den grundlegenden Kampfzaubern aus den arkanen Künsten und ist ähnlich effektiv.
\item[Barriere]Der Beschwörer spannt einen Schild auf, der Problemlos eine ganze Stadt einhüllen kann, welche auf dem gegensätzlichen Element basierten Effekte abwehrt und entsprechend seiner Natur zusätzliche Eigenschaften haben. Feuer verbrennt alles Passierende, Erde verhindert physische Durchquerung, Wasser und Luft haben eine extrem hohe Spannweite und können kleine Partikel etc. binden.
\item[Herbeirufung niederen Elementares]Beschwört ein Irrlicht(Feuer), Briesling(Wind) um einen einfachen Dienst zu verrichten.
\end{description}


\chapter{Wahre Magieschulen}
Schon bevor die Drachen ihr Interdikt erließen, gab es neben den ihren den Augen dilettantischen Zauberern und ihrem Wunschkonzert, das mehr wie ein Crescendo aus Dissonanzen war, einige neue Realitätsmodelle, welche unverzichtbar waren und teilweise sogar geschätzt wurden. Sie alle erfüllen das Interdikt zu einem gewissen Grad oder ihr Schaden hält sich in einem tolerierbaren Rahmen.

\section{Elfen und Drachen}
Das Elfenblut ist selbst ein Zauber, um damit seine Macht in den Nachkommen weiterlebt. Ähnliche Magie hält die Drachen am Leben, daher war es natürlich unumgänglich, dass das Interdikt sie und ihre Kraft verschont, natürlich erfüllen sie gleichzeitig ihr eigenes Interdikt.

\subsection{Drachische Meisterschaften}
Die ersten Zauberschöpfer waren die Drachen selbst, die ihre Zauber jedoch nur in wenigen Ausnahmefällen weitergeben. Ihre Zauber zeichnen sich durch ihre den Drachen eigenen Effizienz aus und erfordern aufwendige und äußerst willkürliche Rituale mit Gesang, Runen und anderen Tamtam, um zu verhindern das jemand durch Zufall einen dieser Zauber entdecken kann. Auch in ihrer Zahl sind die Meisterschaften auf ein paar dutzend Beschränkt, da sie Magie nur nach teilweise Jahrhundertelanger Überlegung Einsetzen, um eine ihrer in dieser Zeit erstellten Thesen zu testen, auf der Suche nach der Wahrheit. Nur wenige eignen sich überdies hinaus für den allgemeinen Gebrauch:
\begin{description}
\item[Zeitenbrecher]Einer der ersten Zauber den die Drachen nach dem Interdikt wirkten, war dieser Zauber, dessen einziger Zweck es war die Spuren alter Zauber zu tilgen. Um dabei gründlich vorzugehen eliminiert dieser Zauber alle Einflüsse die nach ihrer Schöpfung durch die Götter auf die Realität eingewirkt hatten und damit auch göttliches Wirken. Dieser im großen Maßstab auf weite Teile von Aurum Orbis mit Ausnahme des Weltloches, sowie der beschädigten Anomalien, gewirkte Spruch fand seitdem keine Anwendung mehr und von allen Zaubern ist es auch gleichzeitig der Aufwendigste, da sein Ritual einen Monat zur Absolvierung dauert.
\item[Ewiges Siegel]Manchmal kommen die Drachen zu der Überlegung, dass bestimmte Dinge nicht in die Hände andere weniger weitsichtiger und vor allem umsichtiger Kreaturen gelangen dürfen oder sie ihre eigene Neugier bekämpfen müssen. Das Ewige Siegel ist nach dem Zeitenbrecher der mächtigste Bannerzauber, welcher das Ziel in eine mit allen Mittel unüberwindbare Barriere umgibt. Dieser Zauber existiert in verschiedenen Varianten, die teilweise passierbar sind oder sich bei Bedarf durch ein zweites Ritual lösen lassen. Die Rituale wurden in der Vergangenheit immer wieder herausgerückt um größere Übel zu versiegeln.
\item[Schleierbruch]Selbst nach vielen erfolglosen Versuchen konnten es sie Drachen nie lassen, einen Blick hinter Raum und Zeit zu wagen. Dieser Zauber kapselt eine Blase aus dem Raum ab und öffnet sie in das Nichts, was somit alles in dieser vernichtet. Die Drachen verzichten jedoch darauf diesen allzu oft zu benutzen, da mit diesem Zauber heftige Manaeruptionen einhergehen.
\end{description}

\subsection{Elfische Geheimnisse}\label{Elfische Geheimnisse}
Während die Drachenelfen selbst Gesegnete sind, sind weite Teile ihres Volkes auf ihre Zauber angewiesen. Gemäß den elfischen Kasten sind die neuen Zauber jeweils auf die Bedürfnisse und Traditionen dieser angepasst und aufgeteilt worden. Daher folgt nun eine kurze Auflistung nach Kasten.
\begin{description}
\item[Hain der Felsblume]
Als Zuständige für die Gärten und damit Nahrungsversorgung ihres Volkes sind ihre Zauber spezialisiert auf die Umformung von Fels in Fruchtbaren Boden, sowie die Manipulation von Pflanzen nach ihren Bedürfnissen. Dabei gehört zu jedem Zauber in der Regel ihr Band zu Natur, welches sie mit dem selbigen Zauber bei Initiation schaffen.
\begin{description}
\item[Band zur Natur]Dieses Ritual bildet die Basis aller Lehren und Traditionen dieser Kaste. Dabei bindet ein Elf einen Teil seiner Lebensessenz an den Keim einer Pflanze, die sie danach für den Rest ihres Lebens pflegen müssen, da ihr Tod mit dem Tod dieser Pflanze eintrifft. Die Pflanze entwickelt sich zu einem Ebenbild der Seele ihres verbunden Elfen und in der Regel robuster und langlebiger. Durch das Band erhält ein Elf die notwendige Einsicht um andere Zauber ihrer Kaste zu wirken und auch Charakterzüge, die an den des primitiven Pflanzenbewusstsein erinnert.
\item[Muttererde]Aus nahezu jedem Material, außer Gold und anderen Metallen, sowie lebender Materie und einigen anderen Dingen, kann der Elf fruchtbare Muttererde erschaffen. Einzig notwendige Komponente ist ein klumpen echten Mutterbodens.
\item[Wachstumsspurt]Innerhalb kürzester Zeit kann der Elf aus einem einfachen Keim eine ausgewachsene Pflanze heranzüchten, sofern ausreichend Wasser und Muttererde zur Verfügung steht. Bereits ausgewachsene Pflanzen können hiermit bis zu einem gewissem Zyklus heranzüchtet, z.b. von der Blüte bis zur Ernte.
\end{description}
\item[Stählerne Rosengarde]
Auch wenn die Elfen das Leben im Allgemeinen schätzen und es in der Elfengemeinschaft kein Bedarf zur Verbrechensbekämpfung gibt, so leben die Elfen nicht ohne Feinde. Die Rosen stehen dafür ein, die Sicherheit ihrer Haine und Geheimnisse zu bewahren. Dabei vertrauen sie auf schwere Rüstungen und über Generationen verfeinerte Kampftechniken, die sie dank ihrer Magie in einen Sturm aus Stahl verwandeln.
\begin{description}
\item[Bürde des Mantel]Mit einem Gedanken beugen sich Eisen und Stahl dem Willen des Kriegers und erlauben es die Belastung von Rüstungen zu erhöhen oder zu erhöhen.
\item[Winde des Eisen]
\end{description}
\end{description}

\section{Mönche}
Einst waren es die Drachen, denen es als erste in Jahrzehnte langer Arbeit in einem gemeinschaftlichen Akt gelang, genug Willenskraft zu bündeln, um Magie das erste Mal greifbar zu machen. Seitdem galt es für sie als ausgemachte Sache, dass kein Sterbliches Wesen ihnen dieses Kunststück nachmachen könnte. Viele Jahrhunderte später sollte es ein Eremit sein, der getrieben von seinen inneren Dämonen durch Meditation und Körperliches Training bis zur vollkommenen Erschöpfung sich von diesen zu befreien und seinen Körper bis in den letzten Winkel mit reinem Mana vollzupumpen. Von seiner Tat sollte zunächst nur wenige erfahren, denn Prahlerei war entgegen seiner neuen Natur. Es waren Wanderer und andere Aussätzige, die bei dem Eremiten in die Lehre seine Techniken adaptierten und mit ihm perfektionierten. Die Grundzüge ihrer Philosophie war das Chi, das in einem geeignetem Gefäß mit einem Ausgleich zwischen Körper und Geist, sich einnistet und dazu genutzt werden kann, um ein ungeahntes Potenzial zu entfalten. In diesen Grundzügen bestand auch Einigkeit unter den ersten Mönchen, doch vieles war Interpretationssache und so trennte man sich, um nicht den inneren Frieden der einzelnen zu gefährden. So zog jeder Mönch aus, um ihre eigene Orte der Meditation zu erbauen und neue Schüler zu finden. So erfuhren dann auch die Drachen von diesen Phänomen und es gab einige die selbst das Studium der Mönche aufnahmen und meisterten. Als schließlich das Interdikt kam, beschlossen die Drachen den Mönchen ihre Kunst zu lassen, im Wissen das es nur wenige geben würde, die den Weg des Chi meistern würden und das die strenge Disziplin und Tradition eines Mönches Dummheiten vermeiden würden.

\subsection{Kämpfer der Sieben Winde}
Meister des Kampfes. 4 elementare Schläge, 3 defensive Haltungen und die Achte verbotene Technik, die Flaute. Keine Fernkämpfer!

\subsection{Hand von Sonne und Mond}
Faustkämpfer, Ying-Yang-Like. Self-Buffer/Debuffer
Zwischen allen Dingen besteht eine Verbindung. Gemäß dieses Grundsatzes nutzen die Faustkämpfer dieser Schule ihre innere Balance 

\subsection{Seelenweber}
Die Webkunst kann eine langwierige und sehr monotone Aufgabe sein, dennoch kann die kleinste Unaufmerksamkeit die gesamte Arbeit ruinieren. Daher ist es nicht verwunderlich, dass es einen Mönchsorden gibt, der diese Kunst als Teil ihrer Meditation nutzt. Grundlage hierfür steht die Auffassung, dass jede Wunde eines Körpers sich als Störung im Geflecht der Seele wieder spiegelt und umgekehrt. Mithilfe ihres inneren Chis kann ein Seelenweber nun dieses Geflecht reparieren und damit dem Körper beim Heilungsprozess helfen. Auf der anderen Seite steht die Heilung des Geistes durch ihr Wissen über den Körper. Dieses sammeln sie bei ihren umfassenden Körpermeditationen, bei denen sie nicht nur ihre eigenen Grenzen erweitern, sondern vor allem die Geheimnisse jeder Faser und jedes Muskels zu ergründen. Seelenweber sehen sich nicht als Kämpfer und beherrschen nur eine Chi-Technik zu ihrer Verteidigung: Das Chi-Netz, welches den Getroffenen lähmt.

\subsection{Schattentänzer}
Assassinen oder Infiltratoren?

\chapter{Antimagie}\label{Antimagie}\index[Stichworte]{Antimagie}
Die Letzte Schlacht von Alpha und Omega aus den Tagen des großen Streites führte bei ihrer Vernichtung, aufgrund des Kontrastes zwischen den beiden Entitäten zur Entstehung einer ganz eigenen Anomalie: Antimagie. Es ist ein magisches Rauschen, welches so gerichtet ist, dass andere Zauber innerhalb dieser nicht zugelassen werden und sich einfach einordnen. Mana ergeht es ähnlich, es kehrt schlagartig in seinen Grundzustand zurück. Im Gegensatz zu gewöhnlichem Magierauschen, breitet sich Antimagie zum Glück nicht aus, sondern vernichtet einfach alle überschüssige Magie, die aus den übrigen Anomalien austritt, was sehr zur Stabilisierung der Welt beiträgt. Antimagie lässt sich nicht zu Mana destillieren oder in sonst irgendeiner Weise konserviert werden, um sie beispielsweise als Waffe gegen Magier einzusetzen und seit der Errichtung des \nameref{Tal des Zwielicht} ist der Zugang zu ihrer sehr erschwert worden. 

\chapter{Anomalie}
Von den Axiomaten sind die Anomalien wohl die deutlichsten, wenn auch für die Bewohner rätselhaftesten, Spuren. Es handelt sich um meist um Orte, wo die Gesetze der restlichen Welt ignoriert oder völlig verkehrt wurde. Manche von ihnen haben entweder nur geringen Nutzen oder sind gar gefährlich, aber sie alle sind Quellen von Magierauschen, weshalb sich strategisch wichtige Posten meist in der Nähe dieser befinden. Die wenigen wirklich nützlichen Anomalien sind in dieser Hinsicht oft mit ganzen Festungen und einer gewaltigen Metropolen umgeben.

\section{Übersicht der Anomalien}

\subsection{Das Arkeum}\index[Stichworte]{Arkeum}\label{Arkeum}
Geordnetes Chaos, das wäre wohl die eheste Beschreibung für die mächtigste den sterblichen Völkern bekannte Anomalie. An sich handelt es sich beim Arkeum um ein überdimensionales gläsernes Tetraeder, das zu jedem Zeitpunkt als Schatten einen perfekten Regenbogen wirft. Das besondere offenbart sich erst bei Nähere Betrachtung der Oberfläche hinter der man eine gewaltige Menge von Zeichen erkennt. Lange Zeit vermuteten die Völker, besonders die Drachen in diesen Zeichen eine Botschaft, ein gewaltiges Kompendium des Wissens zu erkennen und verbrachten lange Zeit mit dem Studium der Abschriften der Oberflächen, die ganze Bibliotheken füllten, doch wie man es auch drehte und wendete, niemanden sollte und würde es gelingen Sinn in diese Zeichen zu bringen, denn ihre Ordnung ist das Chaos. Sie enthalten gleichzeitig alles und nichts und nur ein Axiomat wäre in der Lage sie zu verstehen, wenn er ihre Botschaft auch scheut, da sie von der Ebenbürtigkeit des Chaos zur Ordnung singt. 
\\Bei den Völkern hingegen führte sie bei jenen, die sich zu lange mit ihnen beschäftigten entweder dazu, dass sie nach Jahren aufgaben oder sie entwickelten eine krankhafte Besessenheit, die sie meist in ihren Untergang führte, vor allem nach den Manakriegen. Während dieser hatte man das Arkeum umgebende Thalum, welches Hauptstadt des damaligen Reiches war, mit Manabomben attackiert. Eine der Manabomben verfehlte ihr Ziel und traf das Arkeum, wo die Explosion ein Loch in das ansonsten unzerstörbare Arkeum riss, mit fatalen Folgen: Das ausströmende Chaos verband sich mit der Magie und kontaminierte den gesamten Landstrich um das Arkeum mit verdrehtem magischen Rauschen, das alle Materie und alle Lebewesen einer Umstrukturierung und teilweise neuen Naturgesetzen unterwarf. Viele Lebewesen überlebten diesen Prozess nicht, andere sollten fortan als magische Monster die restliche Welt unsicher machen.
\\Noch immer kann sich seitdem den Bereich um das Arkeum niemand nähern, ohne zu riskieren, selbst ein Opfer einer solchen Verwandlung zu werden. Die Völker haben am Rand der Quarantäne gewaltige Manakristalle aufgestellt, um eine Ausbreitung zu verhindern. Dieser Ring stellt die größte Manareserve im gesamten Aurum Orbis da, dennoch steht das gesamte Gebiet unter einem Waffenstillstand, da niemand dafür verantwortlich sein möchte, dass sich das Chaos, welches sie im wahrsten Sinne des Wortes eindämmen, noch mehr ausbreitet. Mit der Vernichtung von Thalum, wo man die Abschriften aufbewahrt hatte, wurde das Projekt zur Übersetzung dieser eingestellt.

\subsection{Die Erzseen}
Aurum Orbis wurde in seinen Grundfesten als ein statisches Univerium gedacht, alles sollte nur bis zu einem gewissen Punkt verfügbar sein, wozu vor allem Metalle zählen. Der zweite Schöpfer brach diese Regel indem er diese unteriridische Magmaseen schuf, die stets bis zum Rand mit Erzen gefüllt sein sollte. Insgesamt gibt es 7 bekannte Erzseen, von denen 6 allerdings keine Edelmetalle, wie Gold und Silber und keine von den Göttern nachträglich eingeführten Stoffe produzieren. Nur der siebte, das Herz der Zwerge, wie sein elfischer Name übersetzt ist, hat diese Eigenschaft, seitdem die Götter selbst sich an dieser Quelle bedienten, um ihre Zwerge erschufen und ihnen einige Erzbrocken aus ihrer Schöpfung in diesen fielen. \\Um alle Erzseen hat sich eine rege Industrie mitsamt Städten und Militärischen Einrichtungen, zum sicheren Abschöpfen der Lava und Gewinnung von Erzen aus den Magmabrocken. Sie alle sind stehts heftig umkämpfte Postionen von besondere strategische Bedeutung, da die Kontrolle über einen Erzsee die Möglichkeit Waffen und Kriegsgerät in Massen zu produzieren eröffnet. Auch hier bildet das Herz der Zwerge eine Ausnahme, da die Elfen, hier in ihre erste Stadt nach dem Fall der Zwerge gründeten, den Standort und die Existenz vor anderen Völkern geheim halten, eine Unternehmung bei der eine Hundertschaft Drachenelfen und sogar ein Drache im Verborgenem mithilft.
\\Es gibt zwar noch vereinzelt in den unerforschten und vor allem nur schwer zugänglichen Teilen der Welt, über die auch eine ganze Reihe von Gerüchten und Legenden existieren, aber diese wurden noch nicht entdeckt.

\subsection{Das Weltherz}
Im Zentrum von Aurum Orbis, lediglich den Drachen und ihren engsten vertrauten Drachenelfen und dem ersten Götterpantheon, liegt das Herz der Welt. Es ist eine gewaltige Maschinerie aus einem unbekannten goldenem Material, welchen allen Versuchen es zu schmelzen oder herauszubrechen widersteht, die ununterbrochen tickt und arbeitet. Es handelt sich um den Ursprung der Zeit und reguliert ihren Fluss, auch wenn sich dies den Drachen, die ihre Horte hier anlegten und es seither studieren nur in Ansätzen erschlossen hat. Das einzige war sie mit absoluter Gewissheit sagen können ist, dass seine Beschädigung noch katastrophalere Auswirkung haben würde als der Schaden am Arkeum, weshalb sie das Herz mit einer ständigen Abschirmung ähnlich der, die einmal die Gegenwelt abschirmte, sodass nicht einmal die Götter Kenntnis von diesem Ort mehr erhalten konnten, geschweige denn ihn erreichen. Das erste Götterpantheon hat zwar durch ihre Zwerge Kenntnis von dem Herz erlangen können, allerdings haben sie sich stillschweigend mit den Drachen darauf geeinigt, dass sie dieses Geheimnis mit ins Grab nehmen würden.

\part{Götter}
\setcounter{chapter}{0}
\chapter{Einführung}

\section{Über die Natur der Götter}
Einst von einem Axiomat geschaffen, um ihm die kritische Arbeit an intelligenten Lebensformen und die Formung einer Geschichte für die kulturelle Entwicklung abzunehmen, sind Götter Wesen aus sog. Anima. In erster Linie kann diese Energie in Lebendes Fleisch und sog. Seelen, zum Beleben vorherigen mit einem Bewusstsein, benutzt werden. Seelen wiederum produzieren wiederum Anima, welches sich lose um die Seele sammelt. Durch Kontakt einer Seele mit einem Gott fließt loses Anima in Richtung des Gottes. Nach dem Tod des Körpers löst sich die Seele von diesem und treibt danach durch die Welt, bis sie durch Vergessen zerfällt und dabei Animus abgibt. Greift ein Gott nach einer solch freien Seele, um an das lose Anima zu kommen, wird aus der Seele ein neuer Gott. Nachkommen von beseelten Kreaturen kriegen wie von selbst eine neue Seele, weshalb über die Zeit ein Kräfteanwachsen zu Gunsten der Götter stattfindet.
\\Um ihren Schöpfungen auch was für ihre Verehrung zu bieten, verfügt jeder Gott über die Möglichkeit sog. Interventionen. Normalerweise sind Götter allgegenwärtige Wesen, die nahezu die gesamte Welt überblicken können, mit Ausnahme von anderen Göttern geweihten Gebieten, weshalb sie zu jedem Zeitpunkt in das Weltgeschehen eingreifen können. Dazu gehört an erster Stelle der Eingebung an einen Beseelten. Des weiteren können sie mit ihrer Macht auf den Corpus, das Fleisch, Einfluss nehmen. Zusätzlich kann jeder Gott eine sog. Domäne erwählen. Diese gewähren dem Gott weitere Kräfte mit denen er Einfluss auf das Weltgeschehen nehmen kann. Die Art bestimmt der Gott dabei je nach Vorlieben selbst, auch wenn jeder Gott darauf bestrebt ist einen einheitlichen und zu ihm passenden Stil zu finden. Außerdem ist es ungeschriebenes Gesetz unter Göttern, dass diese nicht in der Domäne eines anderen Gottes herum pfuschen. Ausnahme stellt dabei die Neuschöpfung eines Gottes, dem der Gott einen Teil seiner Domäne abtritt.

\section{Begriffe}
\begin{description}
\item[Laie]
Jeder mit einer Seele kann sich an die Götter oder Aspekte wenden und bei letzteren ist dies auch häufig der Fall. Für die Anbetung eines Aspektes reichen die Überlieferungen durch einen Mentor, doch für echte Götterkulte oder mehrere Aspekte reicht meist das Wissen des Laien nicht. Und da Unwissenheit nicht vor Fehler schützt und diese bei Göttlichen Interventionen gerne in Katastrophen enden, beschränkt sich ein Laie eben auf einen Aspekt.

\item[Priester]
In Jahrelangem Studium haben Priester das Wesen ihrer Gottheit oder der Gesamtheit der Aspekte ergründet. Sie haben vollen Zugriff auf alle Rituale.

\item[Anhänger]
Bezeichnung für den Laien oder Priester, welcher die Verantwortung für eine Zeremonie übernimmt.

\item[Meditationen]
Jede Beseelte Kreatur kann durch das Einstimmen auf die Ideale und Prinzipien eines Gottes eine Verbindung zu diesem herstellen über welche dann Anima Richtung Gottheit fließt. Damit die Gottheit diese Energien auch aufnehmen kann, muss sie sich der Anrufung bewusst werden, weshalb sie ihren Anhängern bestimmte Zeremonien lehrt, um auf sich aufmerksam zu machen.

\item[Anima]
Dies ist die Kraft welche Götter zum intervenieren einsetzten. Wird von Seelen generiert und in Meditationen übertragen.

\item[Gunst]
Letztendlich liegt die Entscheidung, inwiefern ein Gott interveniert, bei selbigen. Diese Meta-Währung wird beim Verwenden von Riten aufgebraucht.
Während man bei Aspekten für die bei der Meditation übertragene Anima 1:1 mit Gunst ausgezahlt wird, ist es bei gewöhnlichen Göttern so, dass diese eigene Kriterien zum Verdienen von Gunst besitzen.

\item[Ritual]
Ähnlich wie bei einer Meditation macht ein Anhänger sich durch eine Zeremonie bei seinem Gott bemerkbar und kann eine Intervention erfragen. Die Art dieser hängt von der jeweiligen Zeremonie ab.

\item[Gebote]
Jeder Gott und auch die Aspekte hat gewisse Grundregeln(Gebote) deren Bruch den Frevler etwas Gunst kostet. Je nach Schwere kann das Vergehen durch einen Anhänger durch die Auferlegung einer Zeichnung und anschließenden Bußqueste zur Beseitigung der Zeichnung geahndet werden. Nur in seltenen Fällen nimmt ein Gott oder ein Aspekt sich eines Frevels selbst an, doch falls dies geschieht sind die Auswirkungen meist gravierend.

\end{description}

\section{Rituale}
Aufgrund ihrer Natur stehen diese Rituale jedem Priester zu Verfügung, auch wenn manche Götter keine äquivalente Zeremonie aufgrund ihrer Natur zur Verfügung stellen.
\begin{description}
\item[Fremde Zungen]
Dieser Ritus gibt dem Anhänger für einen kurzen Zeitraum die Fähigkeit fremde Sprachen zu verstehen und zu sprechen. In bestimmten Domänen kommen auch die Sprachen von niederen Lebensformen(wie bestimmte Tiere oder Pflanzen) hinzu.
\item[Lebenskraft]
Durch seine Berührung kann der Anhänger die Lebengeister im Ziel erwecken. Wunden schließen sich oder Erschöpfung und ähnliche Einflüsse werden vermindert.
\item[Weihe]
Der Anhänger segnet ein Objekt oder ein Gebiet im Namen seiner Gottheit wodurch Anima fremder Gottheiten unterdrückt wird. Zermonien zu anderen Göttern sind erschwert und Kreaturen, die vom fremden Anima Leben(Untote), nehmen Schaden und meiden ggf. Geweihtes.
\item[Eidsegen]
Im Verlauf der Zeremonie wird ein fortan dauerhaftes im Fleisch verwobenes Gelübde abgelegt. Dieses zu brechen benötigt eine Selbstbeherrschungsprobe und führt anschließend dazu, dass der Eidbrecher physisch und dessen evt. vorhandene Seele gezeichnet wird.

\end{description}

\chapter{Die Aspekte}


\section{Ursprung}
Der Ursprung der Aspekte besteht aus dem Pantheon von drei Göttern, die im Nachhinein ihre Existenz als Individuen zu Teilen aufgaben, um besser auf die Anbetung durch ihre neuen Schöpfungen eingehen zu können. Aspekte besitzen für sich genommen keinen Charakter und auch keine Motivation, der sie zu eigenmächtigem Handeln befähigen würde, dennoch haben die Völker im Laufe der Zeit für jeden Aspekt eine Pseudoidentität mit Namen und Eigenschaften, sowie Legenden erschaffen, aus denen die Götter auch immer Inspiration für Meditationen und Rituale ziehen. Nur eine kleine Auswahl an Geheimkulten beten die ursprünglichen Götter an, diese Götter werden nun aufgelistet:

\subsection{Niju, Herrin über Land}\label{Niju}\index[Stichworte]{Niju}
Ihre Domäne ist die Natur und sie ist vor allem für die Flora und Fauna in Aurum Orbis zuständig. Sie ist launisch und hat sich nur selten für die Konflikte der anderen Götter interessiert. Sie ist die Schöpferin der \nameref{Dunkelelf}en, bei denen sie es eher aus Neugier, als aus Routine, gehandelt hat. Und ihre Entscheidung sich der Absprache mit ihren Mitgöttern, kein kriegerisches Volk zu erschaffen, war Folge ihres Wissens, dass sie später nicht mehr viel mit ihren animalischen oder pflanzlichen Schöpfungen spielen dürfen würde. Auch wenn die anderen Götter ihren Vorstoß wohl oder übel hinnahmen mussten, wollten sie nicht in ihre alten Muster zurückfallen, sorgten sie dafür, dass die Flora und Fauna von den anderen Völkern anvisiert wurde und man den Dunkelelfen keinen Zugriff auf die später entstehenden Aspekte gewährte, was diese aber nicht wirklich bemerkten oder interessierten.
\\\\\large Niju-Kulte
\\\normalsize Niju selbst wird von ihrer eigenen Schöpfung nicht verehrt, allerdings haben über die Jahre einige, in die Wildnis verbannte, Individuen zu ihr gefunden. Niju selbst kann ihren wenigen Anhängern nur eines bieten, freies Geleit durch selbst die tiefste und feindlichste Wildnis. Im Austausch hierfür verlangt sie lediglich, dass ihre Anhänger ein kurzes Dankesgebet für jedes erlegte Wildtier und für jedes Hindernis(wie Flüsse oder Pässe), welches sie überwunden haben.

\subsection{Ak-Mol, Hüter von Esse und Webstuhl}\label{Ak-Mol}\index[Stichworte]{Ak-Mol}
Seine Domäne sind alle Formen des Handwerks. Er steht für Stabilität und liebt die künstlerische Gestaltung von Handwerksobjekten, auch wenn er dabei nie den eigentlichen Zweck aus den Augen verliert. Er hat ganz nach seinen Vorstellungen die\nameref{Menschen}en erschaffen, die mit seiner Zielstrebigkeit die Dominanz in den ersten Zeitaltern übernehmen konnten. Als Aurum Orbis später durch Kriege und Katastrophen schwer gebeutelt wurde trat Ak-Mol an einzelne Individuen an und gab ihnen in entschiedenen Momenten eine Eingebung. So sind einige der mächtigsten Artefakte aus den Gegenweltkriegen, durch sein Wirken entstanden. Als sein Eingreifen jedoch von den Gegenwelt-Göttern zu Bemerken begannen und ihre Kriegstreiberei verstärkten zog sich Ak-Mol in die Verborgenheit hinter den Aspekten zurück. Erst nach den Gegenwelt-Kriegen begann er erneut auf die sterbliche Welt hinab zu sehen. Wie auch die anderen des Göttertrios, vermied auch er es all zu sehr unter den sterblichen Völkern aufzufallen, sondern sie den Aspekten zu überlassen.  Und dennoch konnte er nicht umhin die kunstvollsten Schöpfungen zu bewundern und ihren Schöpfern Respekt zu Zollen. So entstand ein ausgewählter Kreis von Ak-Mol Geweihten.
\\\\\large Ak-Mol-Kult
\\\normalsize Ak-Mol macht sich nicht viel aus Verehrung und Gebeten sondern achtet nur handwerkliches Geschick. Seine Gaben bestehen auch nicht direkt aus Macht, sondern vielmehr aus Wissen. Einem Geweihtem des Ak-Mol kann von den Erfahrungen und Geheimnissen seiner Vorgängern profitieren, um seine Künste auf eine noch höhere Ebene zu bringen. In seltenen Fällen ist Ak-Mol von den draus resultierenden Objekten so begeistert, dass er ihnen einen finalen Schliff verleiht, wodurch ein Artefakt entsteht.

\section{Aufbau der Aspekt-Kulte}
An sich wird jedem beseelten Bewohner im Laufe seiner Ausbildung die Traditionen zu einem Aspekt gelehrt, auf welchen er sich besonders in Notzeiten immer berufen kann. Angeführt bzw. koordiniert werden sie unter Umständen durch einen Priester, da dieser gewohnheitsmäßig einen besseren Draht zu den Aspekten hat. Neben diesen losen Bünden gibt es außerdem noch die Orden, die jeweils zusätzliche Traditionen und Strukturen pflegen.

\section{Besonderheiten von Aspekten}
Da Aspekte durch die Trinität verkörpert werden und sich diese untereinander die Energie für die Aspekte teilen, kann jeder Priester eines Aspekts auf die anderen Rituale zurückgreifen, von denen er Kenntnis hat(Techniken von Orden muss er also aufwendig studieren). Bei den Meditationen beschränkt sich ein Priester jedoch in der Regel auf die seines Hauptaspektes. Andere Meditationen muss er ebenfalls gesondert lernen und seine Gunst-Generation für diese Zeremonien ist nur ein Viertel.

\section{Übersicht über die Aspekte}

\subsection{Aigis, der Schutzherr}
Er gehört wohl zu den am meisten verehrten Aspekten, denn er hält seine Hand schützend über jene, die sich nicht aus eigener Kraft verteidigen können. Seine Tempel, überall selbst an den entlegensten Orten vorhanden, sind nicht nur Orte der Andacht, sondern dienen auch der Ausbildung von Wächtern, sowie der Unterbringung von Schutzlosen und Wandernden. 
\subsubsection{Meditationen}
\begin{description}
\item[In Not geratene Beschützen]
Sei es durch Räuber oder Wölfe, wenn ein Geweihter einen Wanderer oder eine Karawane vor einem mehr oder weniger großem Übel beschützt wird dies Belohnt.
\item[Lebensband]
Der Geweihte bindet sein Leben an eine andere Kreatur, für deren Schutz er fortan zuständig ist. Stirbt sein Schützling eines nicht natürlichen Todes, so stirbt der Geweihte und der Schützling ersteht an einem sicherem Ort wieder auf. Der Schwur endet erst mit dem natürlichen Tod eines der Beiden und versorgt den Geweihten jeden Mond mit einer gewissen Menge Energie.
\item[Finales Opfer]
Im Angesichts der Entscheidung zwischen seinem Leben und dem vieler, kann ein Geweihter sein Leben über das der Übrigen stellen. Im Gegenzug für die völlige und endgültige Vernichtung seiner eigenen Existenz, erhält der Geweihte in seinen letzten Momenten eine gewaltige Menge an Energie.
\end{description}
\subsubsection{Rituale}
\begin{description}
\item[Geteilter Schild]
Der Geweihte verschanzt sich hinter seinen Schild und spricht eine kurzes Gebet. Daraufhin erscheint ein Geisterhafter Schutzschirm aus seinem Schild, der die neben ihm stehenden und ihm deckt, als trügen sie einen ähnlichen Schild der nächsthöheren Größenkategorie. Würde es sich hierbei um einen Turmschild handeln, so erhalten alle eine volle Deckung. 
\item[Ausfall]
Mit einem Schlachtruf an Aigis stürmt der Geweihte getragen von übernatürliche Stärke an die Seite von Schutzbedürftigen und erzeugt durch eine Energiewoge ein kurzen Moment der Ruhe, indem er alle Feinde die eine Stärkeprobe nicht schaffen zurück wirft.

\end{description}
\subsubsection{Gruppierungen und Kulte}
\begin{description}
\item[Die Wächter]\index[Stichworte]{Aigis!Wächter}
Aurum Orbis ist ein gefährlicher Ort, so war es schon immer und so wird es auch bleiben. Und genauso wird es auch immer die geben, die nicht über die Stärke verfügen sich selbst zu verteidigen. Die Wächter sind ein Orden von Kämpfern, die sich unerbittlich zwischen die Gefahren der Welt und den Bedürftigen stellen. Sie halten Wache an den entlegensten Pässen wichtiger Handelsrouten oder streifen in kleinen Gruppen durch die Welt, um nach heranziehenden Gefahren Ausschau zu halten. Reisende Wächter gelten als höchst respektabel und finden in der Regel bei einfachem Volk immer eine einfache Unterkunft und eine warme Mahlzeit. Zu ihren Spezialitäten gehört eine besondere Kampfaufstellung, die sie zu einer nahezu undurchdringlichen Mauer werden lässt, die durch den göttlichen Beistand ihr Defensivpotenzial noch einmal gewaltig steigern können.
\item[Die Letzte Front]\index[Stichworte]{Aigis!Letzte Front}
Bevor man im großen Stil mit dem Aufstellen der Wächter begann, kam es nach dem ersten Gegenweltkrieg zur Bildung der letzten Front, die heute als legendäre Behüter von Aurum Orbis gelten. Sie stehen ganz im Zeichen von Aigis und gelten als das letzte, ultimative Bollwerk. Jeder einzelne hat einen Schwur, gesegnet durch Aigis selbst, wie man sagt, der ihn über den Tod hinaus an seinen Dienst bindet, geleistet. Ihre erste und einzige Schlacht zu Beginn des zweiten Gegenweltkrieges, hat nur ein einziger Herold überlebt, der sein Vermächtnis, ein Banner und ein Schlachthorn an seine Nachkommen weitergereicht hat, auf das in Zeiten größter Gefahr, mit diesen Zeichen die letzte Front wiedererweckt wird, um die Gefahr abzuwenden. Doch bis dahin werden die Artefakte und das Erbe streng behütet vor den Augen der Welt, damit diese sich nicht auf der Gewissheit, im schlimmsten Fall errettet zu werden, ausruht.
\end{description}

\subsection{Ixania, das Licht der Welt}\index[Stichworte]{Ixania}\label{Ixania}
Sonne und Mond sind zwei Lebenswichtige Dinge für fast alle Bewohner von Aurum Orbis. Auch wenn das Mondlicht nur eine sekundäre Rolle für das Pflanzenwachstum spielt, so ist es dennoch Wegweisend und hat für viele Rituale eine hohe Bedeutung. Für die Geweihten von Ixania, besteht daher kein Unterschied zwischen beiden. Trotzdem gilt zu beachten, dass für bestimmte Rituale(z.B. Erschaffung von Zwielicht) doch eine Grenze zwischen Sonnen- und Mondlicht gezogen wird. 
 \subsubsection{Meditationen}
\begin{description}
\item[Morgentau fangen]
Wie der Name schon sagt, sammelt der Geweihte den Morgentau bei Sonnenaufgang in einem geeigneten Behältnis, anschließend wird er im Zuge eines kurzen Gebetes zu Ixania getrunken. Alternativ kann es zur Destillierung von Zwielicht verwendet werden oder zur späteren Vervollständigung der Weihe aufbewahrt werden. Morgentau muss dabei innerhalb eines Sonnenzyklus verbraucht werden.
\item[Segen der verbrannten Sonnen]
Dieses Ritual kann mit dem Untergehen der Sonne begonnen werden. Zunächst wird ein Feuer entzündet, dabei ist nur frischestes Holz zu verwenden. Sobald die Mitte der Nacht vorüber gezogen ist, kann man mit der Phase zwei beginnen. Die reine Asche, ohne Reste von Ruß oder verkohltem Holz, muss mit Wasser aus einer unterirdischen Quelle vermengt werden, bis eine silbrige Paste entsteht. Diese wird anschließend in einer Silberschüssel ausgestrichen und ins Mondlicht gehalten. Nach einer Stunde wird die Paste im Zuge eines Gebetes auf der Haut des Geweihten verrieben, worauf sich die Asche zu Ruß wandelt, den man erst nach Sonnenaufgang abwaschen darf. Alternativ kann man die Paste, sofern man sie nicht dem Sonnenlicht aussetzt für einen Mondzyklus aufbewahrt werden, um sie später zu benutzen oder sie mit geweihtem Morgentau zu Zwielicht zu destillieren.
\end{description}
%\\\\Rituale:
\subsubsection{Rituale}
\begin{description}
\item[Zweite Sonne/Mond]Der Geweihte kann durch ein kurzes Gebet jederzeit künstliches Sonnen-, bzw. Mondlicht erschaffen, welches seinen Händen entspringt. Im Regelfall handelt es sich dabei um ein schwaches Glühen, welches höchstens 3m weit reicht. Durch Falten der Hände zu einer Einheit kann der Geweihte das Licht zusätzlich in einen Strahl bündeln oder die Intensität und damit Reichweite erhöhen. Letzteres ist dabei mit einem höheren Energieaufwand verbunden. Während so geschaffenes Sonnenlicht, beispielsweise für Pflanzenwachstum ausreichend ist, kann es nicht als Werkzeug für magische oder göttliche Rituale verwendet werden, sofern nicht ausdrücklich erwähnt ist.
\item[Schattensinn]Nach einer kurzen Meditation sendet der Geweihte in einem Radius von mehreren Meilen einen schwachen, bei Sonnenlicht nur als Blitzen wahrnehmbaren, alles durchdringenden Lichtpuls aus, der alle erfassten Objekte anhand ihres Schatten lokalisiert. Zwar verfallen ein Großteil dieser Informationen innerhalb kurzer Zeit(ca. 5 Minuten) wieder, jedoch können die schattenlosen Vampire, bzw. ihr düstere Einfluss in Form ihres Fluches festgestellt werden können. Ähnliches gilt für alle Formen von schattenaffinen Präsenzen. Auf der anderen Seite bemerken diese in der Regel das Licht.
\item[Licht der Welt]Auf geweihtem Öl entzündet der Geweihte eine kalte Flamme, die abhängig von der Tageszeit, gelb wie die Sonne oder bläulich wie der Mond scheint. Es brennt, sofern es nicht durch einen Geweihten, künstliche Dunkelheit oder vergießen des Öls gelöscht wird, ewig weiter, ohne Material zu verbrauchen. Unter weiterem Kraftaufwand kann die Flamme kurzzeitig zu einer Flammensäule auflodern, dabei wird alles in ihr Restlos von niederen Einflüssen, wie dem vampirischen Gift, nekromantischen, sowie allen entweder rein der Dunklen oder Hellen Seite zugerechneten Effekten, gereinigt. Dieses Säuberungsritual kann unter Umständen tödlich enden, falls zu viel vom Körper/Seele korrumpiert wurde, weshalb es vor allem für die Erlösung von Leidenden verwendet wird. Ansonsten ist es eine geeignete stationäre Lichtquelle.
\end{description}
\subsubsection{Zwielicht}\index[Stichworte]{Ixania!Zwielicht}\label{Zwielicht}
Eines der ältesten und wahrscheinlich einfachsten Rituale für einen Geweihten von Ixania, ist die Destillierung von Mond und Sonnenlicht zu einer Schemenhaften Substanz, namens Zwielicht. Es verfügt über eine gewisse Abschirmwirkung gegen göttliche Interventionen, weshalb es oft zusammen mit magischen Barrieren zur Abschirmung von Quellen potenzieller Gefahr genutzt wurde. Allerdings hat Zwielicht eine berauschende Wirkung, bei Kontakt, die sogar abhängig machen kann, mit schrecklichen Entzugserscheinungen. Weil es inzwischen bessere Möglichkeiten gibt, um göttliche Intervention zu blockieren, wurde der allgemeine Gebrauch des Zwielicht-Ritus streng reglementiert. Einer der wenigen Orte, an denen noch Gebrauch von diesem Ritus gemacht wird, ist das \uline{\nameref{Tal des Zwielicht}}.
\subsubsection{Gruppen und Kulte}
\begin{description}
\item[Orden der Sterne]Das Meer ist schon immer ein gefährlicher Ort gewesen. Dazu gehörte nicht nur das, was in den Tiefen lauern möchte oder ein plötzliche Sturm, nein die wohl größte Gefahr bestand darin, sich zu verfahren. Zwar könnte man sich am Sternenhimmel orientieren, wäre er nicht übersät von seltsamen Phantomsternen, die entweder ihre Position ändern oder gar ganz verschwinden, um woanders wieder aufzutauchen. Zusätzlich kommt noch ein Phänomen, welches auf magisches Wirken zurückgeht, wodurch sich der Raum spontan verzerrt und das Schiff manchmal um mehrere Meilen mit neuer Blickrichtung versetzt. Mit gleichmäßigem Seegang und einer guten Beobachtungsgabe lässt sich ein solche Sprung rechtzeitig bemerken, bevor die eigene Kursberechnung vollständig zusammenbricht. Die magnetischen Felder in Aurum Orbis sind unstabil, weshalb auch dies keine effektive Orientierung ermöglicht. Eine Lösung waren die Signallichter des Ordens der Sterne. Dieser errichtete entlang der Küsten Kloster über denen Tag und Nacht Leuchtfeuer brennen, die jeweils einzigartig in Anordnung und Farbe der Lichtkugeln sind. Nachts weithin sichtbar und ansonsten durch ein Ordensmitglied als Teil einer eigenen Weihe wahrnehmbar. Entlang des von ihnen gespannten Netzes können Schiffe, ob auf dem Wasser oder in der Luft sicher navigieren. Zusätzlich sind die Leuchtfeuer Leuchttürme und damit fester Bestandteil eines jeden Hafens, worraus sich die Orden auch finanzieren, zusätzlich zu den Raststätten in ihrem inneren. Im Landesinneren existieren vor allem auf Pässen auf denen schlechte Sichtverhältnisse herrschen einzelne solcher Leuchtfeuer in einfacher Form meist ohne einen Tempel in der Nähe und ohne Teil des größeren Netzes zu sein. Realisiert werden Letztere durch einfache Ewige Flammen im großen Maßstab.
\end{description}

\subsection{Tholemäus, der Wissensozean}
Wissen ist Macht: Geschichte lehrt die Völker ihre Fehler nicht zu wiederholen. Zauber, Techniken und Rituale versorgen die Völker mit den Möglichkeiten neue Herausforderungen zu meistern und wie unbequem wäre das Leben ohne die vielen technischen Apparaturen, die auf teilweise jahrzehntelanger Forschungsarbeit beruhen. Doch auch wenn die Macht von Wissen ewig und beständig ist, so ist Wissen auf der anderen Seite sehr anfällig dem Fluch des Vergessens anheim zu fallen und für immer verloren zu gehen. Die Anhänger von Tholemäus sind daher stets versucht Wissen zu bewahren und altes Wissen auszugraben.
\subsubsection{Meditationen}
\begin{description}
\item[Wissen konservieren]
Die wichtigste Aufgabe eines Anhängers ist das Erhalt alten Wissens. Sie betreiben zu diesem Zweck viel Selbststudium in den großen Bibliotheken der Welt, wo sie Tage, wenn nicht Wochen mit der Kopie von Büchern beschäftigt sind.
\item[Geheimnis lüften]
Während sich ein Anhänger mit der Notation neuem Wissen beschäftigt, geht neben den Anima auch ein paar Informationen an Tholemäus über.
\item[Wissen aufarbeiten]
Nicht immer haben außenstehende die Zeit oder das Interesse sich durch Regale von Büchern zu wälzen. Die Anhänger sehen es daher als ihre Aufgabe Wissen entweder in Form eines Vortrages oder als Publikation in für interessierte geeignete Häppchen aufzuteilen.
\end{description}
\subsubsection{Rituale}
\begin{description}
\item[Roter Faden]
Trotz ihrer Jahrelangen Arbeit in Bibliotheken ist es für den Geweihten nicht immer möglich, sich an jedes Detail zu erinnern oder sie stehen vor der Aufgabe in einer unbekannten oder gar unsortieren Sammlung von Texten nach einer Information zu suchen. Mithilfe eine roten Bandes, ähnlichem einem Lesezeichen kann ein Geweihter nach dem rituellen Stellen seiner Frage dem Band zu dem passendem Buch folgen, wo das Band an die passende Stelle schlüpft. Ist das gesuchte Wissen nicht vorhanden reagiert das Band einfach nicht.
\item[Gedächtnissprung]
Tholemäus nimmt im Zuge einer Weihe nicht nur Energie, sondern auch einen Teil des verarbeiteten Wissens auf. Bruchstücke selbigen können im Zuge eines Gebetes und anschließender Meditation abgerufen werden, auch wenn es sich meisten nur auf Hinweise, wo die Antwort liegen kann. Nur bei besonders einfache Fragen, wie Informationen über ein bereits bekannte Spezies sind in der Regel klarer. Dieses Ritual erlaubt auch das Wiederholen einer Wissensprobe.
\end{description}
\subsubsection{Gruppen und Kulte}
\begin{description}
\item[Die ewigen Archivare]
Dieser Orden hält die Bewahrung von Wissen für die größte Tugend und Pflicht in der Welt. In gewaltigen Bibliotheken oder kleinen Archiven, außerhalb der Reichweite von Agenten des Vergessens, sammeln und konservieren sie Texte. Ihre Jahrelange Arbeit mit Texten haben ihnen neben großer Erfahrung im Recherchieren von Informationen, ebenfalls ein großes Allgemein- oder Spezialwissen, welches sie in der Regel auch bereitwillig mit allen interessierten Teilen. Sie organisieren sich in kleineren Orden, die sich durch Spenden und oder Unterricht finanzieren. Manche agieren dabei in den Schatten, um der Aufmerksamkeit ihrer Feinde zu entgehen, die sowohl aus Brandstiftern, aber auch restriktiven Gesellschaften, die Wissen unterbinden wollen. Untereinander kommunizieren sie in einer eigenen Geheimsprache und tauschen sich auf regelmäßigen Konventen über Neuaneignungen aus. Als Geweihte kopieren, übersetzten oder katalogisieren sie ständig ihre Texte, um den Zugang zu Wissen noch mehr zu erweitern.
\item[Weltenbummler]
Getrieben von Neugier zieht es die Anhänger dieses Ordens hinaus in die Welt um neues zu lernen oder den Geheimnissen der Vergangenheit auf die Spur zu kommen. Sie organisieren sich auf losen Konventen, da sie ähnlich wie ihre Brüder in den Archiven Verfolgung durch die Feinde von Wahrheit und Wissen fürchten müssen. Um ihrer Aufträge und Nachforschung zu tarnen treten sie daher als fahrender Lehrer an Reisegruppen oder Abenteurer heran um sie zu unterstützen und später auf die Spur eines von ihnen verfolgten Geheimnisses zu lenken. Zwar verfügt jeder Weltenbummler ebenfalls über eine gute Allgemeinbildung und fühlt sich in Horten des Wissens zuhause, doch ein Archivar ist ihnen hier meilenweit überlegen.
\end{description}

\subsection{Haliya, die ewige Blüte}
Wahrscheinlich der am meisten durch Laien verehrte Aspekt, da Haliya über die Ernte und das Vieh wacht. Vor allem in den kargen und unwirtlichen Regionen werden daher regelmäßig Feste ihr zu ehren gefeiert.
\subsubsection{Meditationen}
\begin{description}
\item[Tierbegräbnis]
Jedem gestorbenem Tier steht auch im Tod die Gnade eines Begräbnisses zu. Zwar darf Fleisch, Haut und anderes Gewebe verwendet werden, doch die Knochen seien der Erde in einer kleinen Zeremonie zu überantworten.
\item[Asche für das Land]
Der Anhänger häuft zusammen mit anderen einen Berg aus nicht verwendeten Schnittgut an und entzünden diesen schließlich bei Sonnenuntergang, während alle ein Gebet zur Haliya spricht. Am nächsten Morgen muss die Asche unter einem erneuten Gebet über die Umliegenden Felder verteilt werden. Dieser Brauch wird vor allem zu Frühlingsbeginn von ganzen Dörfern gemeinschaftlich zelebriert.
\item[Erntekranz] 
Nach Einfuhr der Ernte wird von jedem Feld und jedem Garten, den man selbst mitbewirtschaftet hat eine Pflanze(z.B. bei Getreide) oder ein Ast, entnommen und diese zu einem Ring, sofern möglich zusammengelegt. Diesen Kranz legt man anschließend zur freien Verfügung aus.
\\Auch diese Meditation führt häufig von gesamten Dorfgemeinschaften zelebriert, wobei man die Kränze entweder austauscht oder zusammenträgt und ein gemeinsames Mal abhält, an dem auch Tiere beteiligt werden.
\end{description}
\subsubsection{Gebote}
\begin{description}
\item[Schützte die Herde]
Kein Tier darf durch Faulheit oder Unaufmerksamkeit seines Hirten sterben.
\item[Teile mit den Bedürftigen]
\item[Das Feld ist heiliger Boden]
Auf einem Acker ist kein Blutvergießen oder gar eine Rohdung, zum Zweck der Schädigung gestattet.
\end{description}
\subsubsection{Rituale}
\begin{description}
\item[Weihe von Acker und Ernte]
Durch das Ziehen eines Schutzkreises kann ein Feld oder deren Früchte vor Ungeziefer geschützt werden. Dieser Schutz hält für einen Zyklus oder auf dem Feld ein Frevel geschieht.
\item[Sommerwind]
Der Anhänger hält eine kurze Ansprache an Haliya und macht danach eine Tiefen Atemzug. Beim Ausatmen entsteht um ihn ein warmer Sommerwind, der für die Dauer der Intervention alle im Umfeld das Anhänger vor dem Kältetod gefeit sind. Die Brise hält selbst einem Sturmwind stand.
\end{description}

\subsection{Umbra, der Gewitzte}
Als Patron von Dieben, Einbrechern, Fälschern und Schmugglern ist Umbra mehr berüchtigt als jeder andere Aspekt. Ihre Anhänger und sie bevorzugen es, sich im Schatten der Gesellschaft zu bewegen. Zwar ist Umbra perse nicht böse, da sie grundsätzlich Blutvergießen ächtet und auch sonst ihre Fokus eher auf lohnenden Zielen liegt, weshalb die einfache Bevölkerung in der Regel keine Schäden zu beklagen haben und man Umbra nicht völlig ächtet. Eine besondere Beziehung hegt dieser Aspekt zu Dinaris, seiner Schwester, mit der er über seine Anhänger einen Zwist austrägt. 
\subsubsection{Meditationen}
\begin{description}
\item[Tribut der Beute]
Nach einem erfolgreichem Raubzug, Betrügerei spendet der Anhänger einen Teil seiner Beute an einen Schrein andere Aspekte, mit Ausnahme der von Dinaris.
\item[Zeichen der Lilie]
Der Anhänger beweist sich, indem er unbemerkt in verbotenes Territorium eindringt und dort ein kleines Andenken in Form einer Lilie da lässt, dieses kann auch nur aus einer Skizze der Blüte bestehen.
\item[Wettstreit der Schatten]
Zwei sich begegnende Anhänger Umbras führen einen formellen Wettstreit, dessen Natur sie durch den Geheimcode Umbras vereinbaren. Ein Beispielhafter Wettkampf könnte die Jagd nach einer Trophäe für einen Tribut an Umbra sein.
\end{description}
\subsubsection{Gebote}
\begin{description}
\item[Gewaltlosigkeit]
Zu keinem Zeitpunkt darf ein Umbra-Anhänger Blut vergießen. Einzige Ausnahme ist falls der Anhänger sich zwischen eine Klinge und die Kehle von Unschuldigen stellt, um für letztere zu Kämpfen oder selbst außerhalb eines Raubzuges attackiert wird.
\item[Nehme nur von denen die genug haben]
Arme und Schutzlose sind für Umbra kein lohnendes Ziel, da sie keine Herausforderung darstellen. Ein Anhänger darf nur stehlen, was auch gut genug geschützt ist.
\item[Im Schatten Einigkeit]
Trotz ihrer Tücke darf ein Anhänger, während er sich auf einem Raubzug befindet, sowie den gesamten nächsten Mondzyklus, seine Kollegen nicht verraten und sich auch an die Aufteilung der Beute halten.
\end{description}
\subsubsection{Rituale}
\begin{description}
\item[Geleit von Nacht und Nebel]
Der Anhänger bedeckt sein Gesicht entweder mit Asche oder einer Maske, während er ein kurzes Gebet spricht. Anschließend scheint er mit Dunkelheit oder Nebelschwaden zu verschmelzen, was ihn nahezu unsichtbar macht.
\item[Sternenschlüssel]
Der Anhänger deutet auf die Schließmechanismen in einem Gegenstand oder einer Struktur, während er ein Gebet spricht. Anschließend materialisieren sich an den aufgezeigten Stellen passende Schlüssel oder womit auch immer die Mechanismen interagieren und entriegeln, sofern der Anhänger keine Mechanismen vergessen hat, alles. Auf dieses Ritual folgt in der Regel ein Tribut der Beute für den Inhalt, der jedoch keine Gunst einbringt.
\end{description}

\subsection{Dinaris, die Umtriebige}
Patronin von Händlern und Vagabunden genießt sie ähnlich wie ihr Bruder nicht den besten Ruf.
\subsubsection{Meditationen} 
\begin{description}
\item[FEHLER]
\end{description}

\chapter{Pantheon der Gegenwelt}

\section{Einführung}
Mit dem Fall der Götterwelt retteten sich viele Götter auf eine Reihe von Raumsplittern. Während der Größte von ihnen mit einem Göttertrio in Aurum Orbis landete, blieb als einzige andere Landeposition die löchrige Gegenwelt. Doch die Gegenwelt war ein rauer Ort an dem viele Götter am ewigen Chaos zugrunde gingen. Nur den härtesten, grausamsten oder listigsten Göttern gelang es hier Fuß zu fassen. Statt sich dabei in eine zweite Sphäre, dem Pantheon zurückzuziehen, ließen sie sich direkt in der materiellen Welt nieder, wo sie eine von ihnen gewählte Gestalt annahmen.

\section{Wirken außerhalb der Gegenwelt}
Das \uline{\nameref{Tal des Zwielicht}} leistet ganze Arbeit bei der Abschirmung göttlicher Interventionen aus der Gegenwelt. Wenn man von wenigen Ausnahmen absieht, die sich fest in Aurum Orbis niedergelassen haben, sind die Götter der Gegenwelt völlig machtlos, was vor allem ihren direkten Gesandten zu schaffen macht, deren Existenz bis zu einem gewissen Grad von der göttlichen Macht abhängig ist. Dennoch können Kulte in Aurum Orbis mit ihrer Verehrung die Gegenwelt erreichen und die Götter besitzen dabei zumindest ein Mindestmaß an Möglichkeiten um zu antworten, meist in Form von Visionen. Auf beiden Seiten der Barriere kämpfen Anhänger um die Zerstörung dieser, um erneut einen Kontakt mit der anderen Seite herzustellen. Doch zum Glück scheiterten all diese Versuche auf Seiten von Aurum Orbis an der Verworfenheit der einzelnen Kulte untereinander.

\section{Übersicht über die Götter}

\subsection{Schicksal}
Manche Götter lieben sich als ein Mysterium auszugeben, doch sie alle lassen von Zeit zu Zeit einen Teil ihrer Hüllen fallen. Und dann gibt es Schicksal. Ihr wahrer Name ist unbekannt, stattdessen benutzt jeder einen ihrer unzähligen Spitznamen. Auch ansonsten verhält sie sich zurückhaltend, schon in der Götterwelt. Ihrer Interventionen bestehen im wesentlichen aus kurzen Eingebungen oder physischen Eingriff in der Größenordnung von Luftzügen. Auch wenn es gegen die bekannten Regel verstoßen würde, vermuten manche, dass sie in die Zukunft blicken kann, angesichts der Tatsache, wie gut ihre Pläne laufen. Ansonsten ist eigentlich über sie nur bekannt, dass sie von Zeit zu Zeit gegen die größten Spieler, sie Betrüger hasst und keinen Kult hat, sondern ihre Schützlinge selbst erwählt und auch diese würden sich nie offen zu Schicksal bekennen. Es ist nämlich ein ungeschriebenes Gesetz, dass jeder der nach ihr ruft sich sicher sein kann bereits von ihr verlassen worden zu sein. Ein Schützling von Schicksal muss daher mehr auf sich selbst vertrauen und wird Hilfe nur bekommen, wenn der Zufall sich gegen einen verschwört. Ihre Macht bezieht Schicksal aus all jenen, die ihr für gute Zufälle mit denen sie nur selten selbst zu tun hat danken. Auch wenn Schicksal sehr gut gefallen hat in der Gegenwelt nutzte sie die erstbeste Gelegenheit, um in das weitaus zivilisierte Aurum Orbis zu reisen und dort ihr ewiges Spiel fortzusetzen.

\subsection{Umos, Wächter des Morgens}
Nicht jeder Gott sah sich durch das Vergessen gefährdet. Einer der wenigen, der geplagt wurde von einer viel schlimmeren Vision war Umos. Er sah ein, dass mit dem Untergang der Welt durch Kriege, Katastrophen oder andere Ursachen, selbst dem mächtigsten Gott die Vernichtung drohte. Er selbst sah daher seine Mission darin, für die Zukunft zu kämpfen. Seine Macht bezog er dabei aus jedem neuem Tag, an dem es Veränderung gab und diese immense Macht nutze er damals in der Götterwelt um mit gewaltigen Heeren gegen die mächtigsten Vorzugehen, um ihren Treiben Einhalt zu Gebieten. Schlussendlich ging diese Strategie nicht auf und mit Ragnaröck sah sich Umos mit seinen Fehlern seiner eigenen Logik konfrontiert. Er sah seine Ankunft in der Gegenwelt damit als zweite Chance an und beschloss diesmal mit Verstand vorzugehen. Seine Schöpfungen die Lorkin waren keine Eroberer sondern vielmehr der Sand im Getriebe großer Imperien, um sie zusammenbrechen zu lassen, sodass die Macht immer neu verteilt wird und damit diese in Bewegung bleibt. Gut und Böse verblassen vor diesem höheren Ziel, dass beide zum Stillstand führen können, weshalb Seitenwechsel an der Tagesordnung sind.
\subsubsection{Falkenorden}
 Auch wenn Umos nur begrenzt auf einen eigenen Kult angewiesen ist, hat er ausgewählte Individuen zu einer Geheimgesellschaft zusammengeschmiedet, die in seinem Sinne für Veränderungen sorgen. Keine Eroberer und Herrscher sondern vielmehr Berater oder Revolutionäre zweiter Reihe. Untereinander sind sie nur lose Verbunden und kommunizieren über Umos als Kontaktperson. Der Teil, welcher seinen Weg nach Aurum Orbis fand, ist etwas enger Zusammengewachsen, auch wenn Geheimhaltung noch immer zweitoberste Priorität nach ihrem göttlichen Auftrag hat. Neben ihren göttlichen Beistand, der natürlich in Aurum Orbis wegfällt, pflegen die Mitglieder besondere Techniken, um Teil eines Größeren werden kann und dieses unbemerkt manipulieren können.

\subsection{Eron, der Gerechte}\index[Stichworte]{Eron}\label{Eron}
Man sagt, dass Erons Streben nach Gerechtigkeit einst geprägt war von dem höherem Ziel, Frieden unter den Göttern zu schaffen, weswegen er oft ein Diplomat oder Berater war. Im Zuge von Ragnarök, in dem alle seine unschuldigen Schöpfungen umkamen, ging jedoch seine höhere Motivation allen Anschein verloren. In der Gegenwelt ist er zumindest nur noch als ein Fanatiker und Streiter für eine Ordnung geworden, deren Maßstäbe unvereinbar sind mit der Natur sterblicher Wesen. Er ist in der Gegenwelt das ultimative Gericht, der eine sehr überspitzte Form von Gerechtigkeit vertritt: Ein Mörder muss mit dem Tod bestraft werden, ein Geretteter steht stets in der Schuld seines Helfer und muss diesem im Zweifel bis zum Tod gehorsam dienen. Selbst eine im Spaß verteilte Schelle oder ein böses Wort werden streng geahndet. Nicht selten mussten sich sogar die Verteidigen, die nach Gerechtigkeit schrien, da sie in den Augen Erons, selbst Schuld an ihrem Elend sein, ja sogar sich sogar selbst mit Schuld beluden, weil sie voreilig den Finger erhoben. Und da die Gegenwelt eine Welt ständigen Chaos ist, hat seine persönliche Elite-Schöpfung die \uline{\nameref{Engel}} immer etwas zu tun und Erons Macht wächst mit jedem Tag. Bei dem Kontakt mit Aurum Orbis waren seine Diener unter den ersten, mit der Mission die Gerechtigkeit in diese neue Welt zu bringen. Er selbst ließ sich dabei stets von dieser neuen Welt berichten, die in seinen Augen, fast eine noch größere Katastrophe war als die Gegenwelt, da man ihm solchen Widerstand entgegen brachte und man sich einer Macht(Magie) bediente, die unverdient in jedermanns Kontrolle fallen konnte und als die Drachen den ersten Gegenweltkrieg mit ihrer Barriere beendeten, hat Eron Jahrzehntelang seine Engel auf der Innenseite patrouillieren lassen, auf der Suche nach einer Schwachstelle, während er die wenigen Agenten, die noch in Aurum Orbis waren antrieb mit ihrer Mission fortzufahren. Im zweitem Gegenweltkrieg sorgte er dafür, dass seine Engel wesentlich besser gerüstet in die Schlacht zogen und trieb von allen Göttern die Eroberungsversuche am Beständigsten fort. Nach seiner erneuten Abtrennung durch das \uline{\nameref{Tal des Zwielicht}}, kehrte er bis auf weiteres zurück zu seinem üblichen Kampf um Gerechtigkeit in der Gegenwelt und der Jagd auf seine gefallenen Kinder, die \uline{\nameref{Nephilim}}.
\subsubsection{Eron Verehrung}
Eron lebt vom Streben nach Gerechtigkeit. Jedes Urteil, welches nach den Maßstäben seiner Gerechtigkeit getroffen wurde, stärkt ihn und darüber hinaus verlangt es Eron nach keiner Verehrung. Wer ein von Eron oder seinen Gesandten gesprochenes Urteil vollzieht hat damit bei Eron etwas gut. Man kann dies entweder zum Begleichen einer Schuld die man später auf sich lädt geltend machen oder auf eine Reihe niederer Gaben Zugreifen:
\paragraph{Rituale}
\begin{description}
\item[Ruf nach Gerechtigkeit:]
Nach der Gerechtigkeit durch Eron oder seiner Engel zu rufen, ist eine gefährliche Sache, da man entweder selbst Opfer des Urteils wird oder einfach bestraft wird die Zeit und Energie für Nichtigkeiten verschwendet zu haben. Man kann jedoch einen Gefallen bei Eron einfordern, sodass man zumindest für letzteres nicht bestraft wird und sich die Engel wirklich Zeit für den Fall nehmen. 
\item[Hoher Eidsegen:]
Ein Schwur, welcher unter diesem Segen abgelegt wurde, bindet beide Seiten bis ans Ende aller Tage und führt beim Bruch zum Tod.
\item[Pflege der Gerechtigkeit:]
Mit einem Fingerzeig wird die angezeigte Kreatur von einem flammenden Peitschenhieb getroffen, der eine verfluchte Narbe hinterlässt. Engel können dieses Ritual mit ihrer Peitsche ausführen.
\item[Mantel des Henkers]
Der Betende erhält eine Panzerung, die seine Haut für kurze Zeit Hart wie Stahl werden lässt.
\end{description}

\subsection{Askon, der Gefallene Gott}\index[Stichworte]{Askon}\label{Askon}
Askons Geschichte reicht zurück bis zu den ersten Tagen der Götter, lange bevor Ragnarök ihre Welt zerstörte. Damals waren die Götter bereits in ihren ewigen Schlachten um die Position des obersten Gottes am Kämpfen und während die meisten von ihnen, dies mit purer Stärke suchten, war Askon wahrscheinlich der Listigste. Er hatte sich die Domäne der Nacht und Schatten erwählt und seine Kreaturen strichen dann durch die Welt, wenn andere schliefen. Zwar konnten sie nicht selbst die Macht an sich reißen oder ihre Widersacher ausschalten, aber in Ruhe ihrer Verehrung für ihren Vater nachgehen. Askon schließlich griff mit seiner Macht nach der Sonne selbst, um sie für immer zu zerstören und die Welt in eine ewige Nacht zu hüllen, die niemand, außer seinen Kreaturen, hätte überleben können. Nur durch Zufall und im allerletzten Moment erkannten die anderen Götter die drohende Gefahr und handelten sofort und ohne die geringste Uneinigkeit untereinander. Sie nutzten einen Großteil ihrer Macht und belegten Askon mit einem schrecklichen Fluch, der ihn auf alle Zeit in die Schatten verbannte, außer Stande sich der Sonne zu nähern und ihr Schaden zufügen zu können, der Fluch sollte sich auch an seine Schöpfungen übertragen und sie so zu einem ewigen Leben im Schatten zwingen. Außerdem einigte man sich darauf ein Auge auf Askon zu werfen und beim kleinsten Anzeichen einer Intrige, gemeinsam ihre Kräfte gegen ihn und seine Schöpfungen zu richten. Askon wurde regelrecht aus dem Pantheon verstoßen und musste Unterschlupf in den Schatten bei den Sterblichen zu suchen, wo er lauerte, wartend auf eine Gelegenheit sich zu rächen und alle zu übertrumpfen. Auch nach seiner Ankunft in der Gegenwelt sollte sich aber erst einmal nichts an seiner Situation ändern, denn das Pantheon der Gegenwelt achtete weiterhin sorgsam auf jeden seiner Schritte.\\Doch schließlich ergab sich eine Gelegenheit für ihn als Charon an ihn herantrat. Charon selbst war wenig erfolgreich mit seinen Untoten, was nicht zuletzt an seiner persönlichen Beschränktheit lag, durch die er als zu oft von seinen Widersachern ausmanövriert wurde. Er selbst war bei Charons Verbannung dabei gewesen, wenn er damals noch nur als Wächter für das Equilibrium gewesen war und war Zeuge von seiner Raffinesse geworden und erhoffte sich von dem in seinen Augen gebrochenem Askon einen wertvollen Verbündeten zu finden. So trat er, als es eigentlich seine Aufgabe sein sollte, ein Auge auf Askon zu werfen, an jenen heran und bot ihm an, hinter dem Rücken der anderen an einer neuen Schöpfung zu arbeiten. Askon nahm die Gelegenheit wahr und schuf mit Charon den Fluch des Vampirismus. Doch als Charon seine ersten Vampire schuf, um sie in einem Gefecht auszuprobieren, tat Askon das Selbe, nur das er seinen Vampiren mit Rat und Tat zur Seite stand. Was folgte war die langsame Abschlachtung von Charons Vampirbrut, bis nur noch Askons Schöpfungen verblieben, die, bevor die anderen Götter noch reagieren konnten, sich mit der Gesellschaft vermengten und sich dem direkten Zugriff durch die Götter entzogen. Charon wurde für seinen Fehler schwer bestraft, als man allgemein die Nekromantie aus allen Gesellschaften verbannte, doch Askon und seine Vampire konnten der Jagd nach ihnen entgehen und entkamen schließlich im Zuge der Gegenweltkriege nach Aurum Orbis. Askon, inzwischen Müde von den ganzen Götterkriegen beschränkte sich daraufhin nur noch mit der Pflege seiner letzten und einzigen Schöpfung, den Vampiren. 
\subsubsection{Askon Verehrung}
Askon selbst interessiert sich nicht für Verehrung abgesehen von denen der Vampire(siehe auf Seite \pageref{Vampire:AskonGaben}). Da diese manchmal aber selbst ihre Beute in ebensolche organisieren, hat Askon eine Reihe von seinen alten Weihen an die Vampire zur Weiterverbreitung gegeben, ein Geweihter, selbst ein Vampir kann für diese Form der Verehrung in der Regel keine Gegenleistung erwarten.
\paragraph{Meditationen}
\begin{itemize}
\item Kodex der Nacht
\\Der Geweihte, isst, arbeitet, sprich lebt nur zwischen Sonnenuntergang und Sonnenaufgang und verbringt die Zeit tagsüber nur innerhalb der Schatten, wo er meditiert oder schläft. Er darf während des Tages kein Wort sprechen. Eine Sonnenfinsternis gilt als Nachtzeit. Jede vollendete Woche in der der Kodex ungebrochen befolgt wurde, zählt als Verehrung.
\item Opfer für die Schatten
\\Der Geweihte verzichtet im Zuge eines feierlichen Schwurs, der in einer Neumondnacht abgelegt werden muss, auf seinen Schatten und ihren Schutz. Bis zur nächsten Neumondnacht darf sich der Geweihte nicht mehr im Schatten befinden, sondern stets von einer Lichtquelle beleuchtet werden, ansonsten beginnt er sich aufzulösen. Zusätzlich wirft er für diese Zeit keinen Schatten und verfügt auch über kein Spiegelbild. Löst sich der Geweihte aufgrund von Kontakt mit Schatten auf, ist er unwiederbringlich verloren und seine Essenz geht komplett zu Askon über. Dieser Schwur wurde vor seinem Fall häufig von seinen humanoiden Agenten abgelegt, bevor sie sich auf eine Infiltrationsmission unter die anderen Völker begeben haben, um bei Enttarnung sich selbst am Reden hindern zu können.
\end{itemize}


\subsection{Charon, der Fährmann}\index[Stichworte]{Charon}\label{Charon}
Die Seelen der von den Göttern geschaffenen Kreaturen enthalten ein gewaltiges Potenzial an Macht für einen Gott, allerdings hat dies eine entscheidende Konsequenz: Ein diamantene Kern der Seele wird unweigerlich zu einem eigenem neuem Gott, wodurch der Konkurrenzkampf im Pantheon nur noch mehr verstärkt wird. Damit die Praxis des Götterschaffens nicht Überhand nehmen konnte, erschufen die Götter einen Sammelplatz für die Seelen der Verstorbenen, wo sie, bis zu ihrem Zerfall durch Vergessen, gelagert werden sollten, dem Equilibrium, und postierten einen Wächter vor dessen Eingang, betraut mit der Aufgabe die Seelen der Verstorbenen aus der irdischen Welt dorthin zu tragen und dafür zu sorgen, dass niemand unbemerkt in das Equilibrium eindringen konnte. Doch der Plan sollte nicht ganz aufgehen. Die Götter übersahen, dass jede Seele bei ihrem Zerfall eine geringe Menge Energie abgab, nicht genug um auf den ersten Blick bemerkt zu werden. Charon selbst sammelte sie unwissentlich, weil er besorgt war sonst nicht seinem Auftrag gerecht zu werden. Schließlich aber bemerkte er seine neue Macht. Es sollte Geburtsstunde eines neuen Gottes sein, als Charon begann sich eingehender mit den ihn überantworteten Seelen beschäftigten. Und über die Zeit eignete er sich wertvolles Wissen an. Ein Gott vermochte vielleicht nicht eine Seele als Energiequelle zu nutzen ohne einen neuen Gott zu erschaffen, weil er aufgrund seiner Macht viel zu grob zu Werke ging, doch Sterbliche besaßen in dieser Hinsicht ein gewisse Geschicktheit, die sie zu einer solchen Leistung befähigte. Charon selbst war zwar nicht in der Lage wie andere Götter durch Verehrung Macht zu sammeln, aber er besaß nun seinen eigenen Weg um sich mit einer Stufe zu den Göttern zu gesellen. Er begann hinter ihrem Rücken widernatürliche Rituale zur Zersetzung von Seelen zu entwickeln, deren Energie direkt an ihn fließen sollte. Mit diesem Wissen trat er an jene heran, die ihre Götter verlassen hatten und wie Charon nach Macht strebten. Dies war die erste Stunde der Nekromantie, als die ersten Seelen Charons neuen Ritualen zum Opfer fielen und er mit der neu gewonnen Macht seine Helfer mit untoten Dienern ausstattete. Als die Götter sein Treiben bemerkten und ihn zur Rede stellen wollten, hatte Charon bereits genug Macht gesammelt, um sich gegen sie zu behaupten. Er sorgte dafür, dass ihnen das Geheimnis hinter seinen Ritualen verborgen blieb und verteidigte seinen Seelenhort, der ihn neben den Nekromanten, auf die die einzelnen weltlichen Helfer der übrigen Götter Jagd machten, mit Energie versorgten. Nach seiner Ankunft in der Gegenwelt gelang es ihm zwar nicht, ein eigenes neues Equilibrium zu erschaffen, aber seine Lehren der Nekromantie waren immer noch sehr begehrt. Doch Charon selbst konnte sich nie groß behaupten, was vor allem daran lag, dass man ihn bei seiner Schöpfung nur mit einem Mindestmaß an Intellekt geschaffen hatte, welche ihn sowohl als Strategen, noch als kreativen Schöpfer ausscheiden ließ und er auf das Potenzial seiner Diener angewiesen war. Nach der Episode mit \uline{\hyperref[Askon]{Askon}} geriet er zusehends in Bedrängnis und nutzte seine erste Gelegenheit, um aus der Gegenwelt nach Arum Orbis zu schlüpfen. Hier gelang es ihm schließlich, wieder ein Equilibrium zu errichten und die Lehren Nekromantie zu verbreiten. Seitdem genießt er mehr oder weniger seine neugewonnene Ruhe, um an sich selbst zu feilen und seinen Seelenhort zu pflegen.
\subsubsection{Nekromantie}\label{Nekromantie}\index[Stichworte]{Nekromantie}
Charon selbst kann nicht durch Verehrung an Macht gewinnen und es gibt keine Charon-Geweihten, stattdessen stellt Charon eine Reihe von Ritualen zur Verfügung und die Ausführenden dieser werden allgemein hin Nekromanten genannt. Zu Beginn seiner Karriere beginnt ein jeder Nekromant mit einem Ritual der Opferung. Anschließend kann er in einer Reihe anderer Rituale totem Fleisch ein neues Leben einhauchen.

\paragraph{Rituale}
\begin{description}
\item[Ritual der Opferung]\label{Nekromantie:Ritual der Opferung}\index[Stichworte]{Nekromantie!Ritual der Opferung}
\begin{itemize}
\item Vorbereitung auf das Ritual, der Seelenstein
\\Um die Seele aus dem Körper zu ziehen und sie für die Weiterverarbeitung zu konservieren, benötigt es einen zur Seele affinen Seelenstein. Dabei vermengt man 5 Unzen Blut vom Opfer oder eines anderen Angehörigen seines Volkes mit einer Unze Materials, welches dem Schöpfer des Opfers geweiht ist. Einfaches Weihwasser ist dabei schon ausreichend. Die Mischung muss dann bei Mondlicht, welches Brücke für Charon zwischen Irdischer Welt und seinem Equilibrium ist, zusammen mit einem Metall, für ungefähr eine Münze, in einen Schmelztiegel gegeben werden. Das geschmolzene Metall wird daraufhin in einem Wasserbecken ausgekühlt. Der Klumpen wird anschließend noch mit einer dünnen Schicht Blut vom Beschwörer bestrichen, während er einen Treueschwur zu Charon leistet. Dies verhindert, dass andere Götter sich an der Energie aus der Seele des Opfers vergreifen können. 
\item Beschaffung der anderen Utensilien:
\\Für das Ritual werden außerdem noch folgende Dinge benötigt: Ein Pinsel, ein Messer oder vergleichbare Klinge, einen Eimer mit Wasser und einen Lappen, sowie Nadel und Faden. Außerdem sollte eine Esse mit Amboss und Hammer bereit stehen.
\item Das Opfer wird zunächst mit ausgestreckten Gliedmaßen auf dem Rücken fixiert. Ein Knebel oder eine Betäubung werden empfohlen, um das Opfer am Schreien zu hindern, allerdings ist dies nicht notwendig.
\item Anschließend muss aus dem Blut des Opfers ein Kreis gezogen werden. Ein zweiter Kreis sollte um Esse und Amboss gezogen werden, falls man diese nicht in den ersten Kreis mit einschließen kann. Sobald das Blut vollständig getrocknet ist und keine Lücken  mehr nachgezogen werden müssen, kann mit dem zweiten Schritt angefangen werden.
\item Mit dem Messer wird nun der Brustkorb aufgeschnitten und der Seelenstein auf das Herz gelegt, alles bis auf das Herz selbst kann dabei ruhig zerstört werden. Anschließend muss man den Brustkorb wieder verschließen, indem man das Fleisch mit Nadel und Faden zusammenflickt. Dieser Schritt muss beendet werden, solange das Opfer noch lebt!
\item Nun muss man ohne den Kreis zu verlassen auf den Tod des Opfers warten, sollte man zu gut gearbeitet haben, darf Gift gespritzt werden oder das Opfer erwürgt werden, allerdings sollte möglichst wenig Blut vergossen werden und das wenige darf unter keinen Umständen den Kreis durchbrechen.
\item Sobald das Opfer verstorben ist kann der Brustkorb wieder geöffnet werden und der Seelenstein entnommen werden.
\item Anschließend muss der Seelenstein komplett vom Blut des Opfers befreit werden und erneut mit dem Blut des Nekromanten bestrichen werden, der Treueschwur an Charon sollte wiederholt werden. Unter Nekromanten ist man sich über diesen Schritt nicht ganz einig, aber doppelt hält besser.
\item Anschließend muss der Seelenstein in der Esse zum Glühen gebracht werden.
\item Der Stein muss anschließend mit einem Hammer zerteilt werden, um die Seele restlos zu spalten, sodass sie in ihre Essenz zerfällt.
\item Die einzelnen Bruchstücke müssen nun abkühlen, bevor sie aus dem Blutkreis entfernt werden dürfen.
\item Von diesem Punkt kann der Nekromant die einzelnen Bruchstücke(im folgendem als Seelensplitter bezeichnet) zum Bezahlen der Seelenkosten für andere Nekromantie-Rituale verwenden.  
\end{itemize}
\item[Ruf des Ewigen Herzens] Eine der häufigsten Gründe, weshalb Menschen mit dem Studium und anschließenden Ausführung der Nekormantie beginnen ist, der Wunsch einen verstorbenen Geliebten wieder zu sehen. Ein wie sich heraus stellen sollen unmögliches Unterfangen, dennoch gefiel Charon die Assoziation, die der Name Ewiges Herz\label{Ewiges Herz}\index[Stichworte]{Ewiges Herz}, bei solchen hervorrief. Im Grunde handelt es sich nur um leicht abgewandelte göttliche Essenz, mit dem Ziel totes Fleisch einen Willen aufzuzwingen und es dabei zu animieren. Der Gebrauch des Herzen beschränkt sich dabei echt auf totes Fleisch und für Golems oder Konstrukte sind andere Energiequellen von Nöten.
Anbei eine kurze Beschreibung über dieses Ritual:
\begin{itemize}
\item Das Ritual muss in einer Vollmondnacht im Mondlicht ausgeführt werden, da dieses eine Brücke zu Charon in die irdische Welt darstellt.
\item Zu Beginn stellt der Nekromant eine silberne Platte an einen Ort, wo er das Mondlicht einfängt.
\item Anschließend muss mit Blut ein Kreis auf den Rand gezeichnet werden, wobei man ein kurzes Gebet an Charon spricht.
\item Nun müssen je nach gewünschter Stärke des später wiedererweckten Fleisches  Seelensplitter auf der Silberplatte angehäuft werden.
\item Anschließend müssen die Worte ''Höre mich an Charon, sieh mein Opfer!'' gesprochen werden. Was folgt ist eine kurzfristige Materialisierung von Charon beim Nekromanten und begrüßt diesen mit den Worten ''Dein Opfer wurde akzeptiert, was ist dein Begehr?'', worauf der Nekromant sein Anliegen darlegen muss. Wenn es innerhalb der Möglichkeiten eines ewigen Herzens liegt wird Charon dieses anbieten. Nimmt der Nekromant an wird Charon fortfahren, ansonsten wird er einfach verschwinden.
\item Falls angenommen werden im folgendem Dialog mit dem Gott, sofern erforderlich weitere Rituale zu Nutzung einen ewigen Herzes erörtert, bevor der Gott die Seelensplitter entgegen nimmt und stattdessen auf dem Silberteller eine schwarze kristalline Kugel zurück lässt, ein ewiges Herz.
\end{itemize}
\item[Erschaffung eines Zombies] Das einfachste und doch vielseitigste Ritual ist das Reanimieren toten Fleisches. Häufig wird mit dieser Wiedererweckung auch die Rückkehr der Seele verbunden, doch diese ist für immer verloren, stattdessen wird das Tote Fleisch durch einen Willen, entweder in Form einer einzelnen Anweisung oder durch dauerhafte Bindung an ein Bewusstsein, in Bewegung gesetzt. Anstatt einer Leiche kann ein Nekromant auch eigene Schöpfungen aus zusammengenähten Fleisch in Bewegung setzten. Ein Zombie bleibt bestehen, bis ihm entweder das ewige Herz herausgeschnitten wird oder dessen Energie verbraucht ist. Der Körper eines Zombies, einschließlich Knochen trägt in sich einen Fluch, der wenn ein Teil des Zombies in das Herz eines lebenden Wesens, mit einer göttlichen Seelen, gelangt, diese beginnt aufzulösen, wodurch der Infizierte stirbt und in einer spontanen Wiedererweckung, durch die Freisetzung der Seelenenergie als Zombie, nur mit dem Auftrag den Fluch weiterzutragen, aufersteht. Das Ritual zur Erweckung besteht eigentlich nur aus einem Schritt: In das Fleisch muss das ewige Herz eingenäht werden und anschließend der Befehl gesprochen werden oder alternativ ein Tropfen Blut zur Bindung an den Besitzer auf den Körper geträufelt werden, wobei der zukünftige Besitzer die Worte ''Unterwirf dich mir!'' spricht.
\item[Erweckung als Ghul] Niemals kann eine Seele nach dem Tod in den Körper zurückkehren, weshalb man niemals mittels Nekromantie einen geliebten Menschen von den Toten auferstehen lassen kann. Auf der anderen Seite ist ein ewiges Herz aber das perfekte Behältnis für eine Seele, ein Umstand den es seinem göttlichem Ursprung zu verdanken hat. Ein williges Opfer kann also seine Seele in ein solches Herz sperren und so bis zum Verbrauch des Herzen in diesem weiterexistieren und von dort aus die Gewalt über totes Fleisch übernehmen. Bei einem Ritual nicht unähnlich dem der Opferung, nur dass man ein ewiges Herz in sein eigenes rammt und dann auf den Tod wartet, wonach man direkt als Ghul erwacht. In seiner neuen Form lebt der Nekromant von Seelensplittern mit denen er sein ewiges Herz befüllen kann, verfügt aber ansonsten über die gleichen Eigenschaften wie ein Zombie.
\end{description}
\subparagraph{Nekropole}\label{Nekropole}
Während viele Nekromanten auf der Suche nach Perfektion ihrer Untoten Schöpfungen oder weil sie die Verfolgung der Gesellschaft fürchten in die Wildnis zurück ziehen, entstanden unter verschiedensten Umständen in direkter Nähe zu den Lebenden Zentren der Nekromantie. Diese Orte werden als Nekropolen genannt. Hier kommen Ghule, Nekromanten und andere Anhänger Cahrons zusammen, handeln mit Wissen über Rituale, Konservierungsformen und Dienstleistungen für Außenstehende gegen Gold, auch wenn Hauptzahlungsmittel innerhalb einer Nekropole Seelensplitter sind. Wie bereits erwähnt haben Nekropolen immer einen sehr speziellen Existenzgrund, angefangen bei einem Pseudo-Kult, der eine ganze Stadt in seinen Fängen hält, bis hin zu einem stillschweigendem Abkommen. Zwar gehören\uline{\hyperref[Vampir]{Vampire}} zu den Untoten, aber werden in einer Nekropole nicht geduldet. Vampire haben daher die Angewohnheit Nekropolen zu meiden, weshalb diese neben Sonnenlicht die effektivste Abwehrmethode gegen Vampire ist, was für einige ein guter Grund ist den Nekromanten Tür und Tor zu öffnen.

\subsection{Die Lady des Nichts}
Jeder Gott hatte seit jeher Ambitionen seine Macht zu vergrößern und dafür zu sorgen, dass sein Name niemals in Vergessenheit gerät, da das Vergessen der einzige Weg für einen Gott ist aus einer Welt zu verschwinden.
\\Doch zurück in der Götterwelt gab es eine Kriegerin. Ihr Name ist nicht mehr überliefert und auch den Göttern ist der Name über die Jahre entfallen. Sie führte einen Krieg gegen die Götter selbst, da sie diese für den Tod ihres Geliebten verantwortlich machte. Nach ihrem Tod sollte es ironischerweise ihre Seele sein, die durch die Gier der Götter ihren Reihen beitrat. Beraubt all ihrer Erinnerung war das einzige was ihr blieb eine tiefe Sehnsucht, die ständig an ihr nagte. So war es ihr Ziel vergessen zu werden, was sich allerdings als unmöglich heraus stellen sollte. Nach ihrem Tod wurden ihr zum Gedenken überall Denkmäler errichtet und viele eiferten ihr nach. Beides führte zum ungewolltem Anhäufen von Macht, wodurch sie auch in den Fokus andere Götter geriet, welche sich häufig mit ihr zu verbünden suchten. Auch wenn die Lady ihre Macht gegen ihre eigenen Anhänger einsetzte, sorgte das nur für eine weitere Anhäufung neuer Anhänger. Mit Ragnarök hoffte die Lady endlich auf Erlösung, scheiterte allerdings und landete in einer Scholle.
\\Wo andere Götter die Erfahrung in der Scholle, konfrontiert mit der höheren Wahrheit, verdrängten oder sich selbst mit Lügen täuschten, war die Lady versucht, ihre Existenz durch die Sprengung ihrer Raumzeit-Scholle zu beenden. Doch als sich ihre Essenz im Nichts zu lösen begann, spürte sie, dass ein Teil von ihr auch nach der Auslöschung ihrer Entität im Bewusstsein der Götter gefangen bleiben würde, vielleicht für immer. Verzweifelt griff die Lady um sich und versuchte ihren Fehler zu korrigieren, was ihr erstaunlicherweise gelang und sie für immer verändern sollte. Sie materialisierte einen Körper und umgab ihn mit einer Hülle aus den Resten ihres Splitters. Fortan war sie an diese Form gebunden konnte auf er anderen Seite jedoch in der Leere existieren und sie lernte, wie sie die Löcher der Gegenwelt nach ihren Belieben erweitern konnte. Diese Fähigkeit nutze sie nach ihrer Ankunft um sich einen Palast in mitten einer unüberwindbaren Sphäre aus Nichts, welche nur sie durchschreiten konnte, zu errichten. Von hier aus sollte ihr neuer und letzter Feldzug beginnen, gegen alle Götter und jene die in ihrem Geist ein Stück von ihr in Form einer Erinnerung trugen. Sie schuf das Volk der Leeresänger: Konturlose Kreaturen aus einem nebelhaftem Stoff. Diese Emsigen Wesen lebten in ihrem Palast und sortierten dort die verlorene Bibliothek und den Abgrund der Macht. Die Lady selbst streift, unfähig von ihrer Allgegenwärtigkeit Gebrauch zu machen, durch die Welt auf der Suche nach potenziellen Agenten für ihren neuen Krieg, einem Krieg des Vergessen. Dabei sucht sie sich die Fallengelassen heraus, solche die Leiden und kurz davor stehen ihre Existenz zu beenden. An solche tritt sie heran und flüstert ihnen die Wahrheit über das Jenseits, in welchem sie Gefangen blieben, bis die Welt ihre Existenz vergessen würde, nur dass die Lady ihnen diese Gnade zu verweigern drohte und ihre Namen ihren Leeresängern zuträgt, bis ihre Agenten ihren Zweck vollführt hätten. Diese Armen Seelen gehen in der Regel ''freiwillig'' den Weg des Geweihten der Lady.
\subsubsection{Weihen}
Die Lady verfügt dank ihrer Leeresänger und ihrer Vergangenheit über gewaltige Reserven an Energie und verzichtet darauf ihren Geweihten irgendwelche hinderlichen Weihen aufzuerlegen. Vielmehr erlangt man ihre Gunst durch die Ausführung von Aufträgen.
\subsubsection{Rituale und Gaben}
Die Lady ist in ihrer physischen Form gefangen und kann daher nicht beliebig intervenieren, sondern muss selbst präsent sein. Vor und nach einem Auftrag besucht die Lady daher persönlich ihre Geweihten an vereinbarten Treffpunkten und erteilt dort Gaben, um sie vorzubereiten oder zu belohnen. Diese bestehen in der Regel in unauffällige Veränderungen in der Leistung oder kleineren Artefakten mit begrenzten Ladungen. 
\subsubsection{Der Palast der Leere}
Der Sitz der Lady ist ein gewaltiger und imposanter Palast der inmitten einer Kuppel aus weißer Leere schwebt. Hier ist der Kult der Leeresänger, der die Lady permanent mit neuer göttlicher Macht versorgt, die verlorene Bibliothek, wo die letzte Ausgabe von ansonsten restlos vernichteten Schriften befindet und der Abgrund der Macht, einem Lagerhort für Artefakte und Schätze, die die Lady entweder als Belohnung für ihre Agenten oder zur weiteren Schwächung ihrer Feinde, da sie oft einen beträchtlichen Teil ihrer Macht in diese gesteckt haben. Die Lady ist die einzige, die den Palast betreten oder verlassen kann, auch wenn sie rein theoretisch in der Lage wäre einen Zugang für Andere zu erschaffen.

\subsection{Valery, die Prophetin des Chaos}
Auf den ersten Blick mag man es nicht für möglich halten, dass eine so zierliche und friedliche Göttin, wie Valery, es geschafft hat, in der rauen Gegenwelt Fuß zu fassen. Einst war sie die Patronin von natürlicher Schönheit in der Natur und Kunst, weshalb jede Form von Gewalt ihr zuwiderlief. Stattdessen leitete und schützte sie jene, die ihre Vision teilten. Als die Welt im Ragnaröck zusammenbrach und sie wenig später sich umgeben von Chaos und Zerstörung wiederfand, errichtete sie spontan aus ihren Erinnerung ein Reich von Stabilität und Harmonie. Es war ihr Versuch vor der Realität zu fliehen, einen Traum zu leben, dessen Schönheit jedes empfindsame Wesen vor Freude vergehen lässt. Ihre ersten Kreationen in der Gegenwelt waren die Sirenen, welche als einzige sich nicht im Traum verlieren konnten, da ihre Herzen kalt waren, so Valery Gesellschaft leisten. Eine Gruppe von Lorkin sollte auf der Suche nach einer Herausforderung, Valery's Sirenen zu einem Wettkampf, um die Gunst der Göttin herauszufordern. Die Lorkin nutzen diese Gelegenheit um die Herzen der Sirenen durch ihre Schauspielkunst und List zu brechen. Nicht länger willig den Alptraum, in den sich für sie die Illusion verwandelt hatte, zu leben, stürzten sich die Sirenen in den Tod oder verließen Valery. Für letzter brach erneut die Welt zusammen und gebrochen erwartete sie ihr Ende. Doch die Gegenwelt sollte ihr auch nach Jahrzehnten, in denen sie durch die tiefsten und gefährlichsten Winkel streifte mit der Absicht sich bei erstbester Gelegenheit in den Tod zu stürzen. Bei ihrer Wanderschaft entdeckte sollte sie jedoch eine neue Form der Schönheit, nämlich die des Chaos, finden. Schließlich legte sie ihre Vorherige Domäne und Göttlichkeit ab, um sie gegen die Rolle als Prophetin und Wegbereiterin der Gegenwelt einzutauschen. Um für Fortwährende Veränderung und damit Chaos zu sorgen, führte sie regelmäßig eine eigene Armee, welche sie aber bald sich selbst überließ, nachdem sie die Generäle abgeschlachtet hatte, um anschließend sich an den Machtkämpfen und der Unordnung ihrer früheren Soldaten zu ergötzen. Trotz ihrer Seitensprünge konnte jedoch ihr nie etwas anhängen, sondern huldigte ihr im Prinzip immer wieder.
\\Im zweiten Gegenweltkrieg trug sie entscheidend dazu bei, dass Aurum Orbis nicht überrannt wurde. Ihr Motiv war der ordnende Einfluss den Aurum Orbis dank seiner Geheimnisse auf die Gegenwelt ausübte. Daher führte sie ein Herr der schlimmsten und wildesten Kreaturen den Gegenwelt zum Weltspalt, fiel den Invasoren in den Rücken und schaffte so einen Atempause, in der sie an das Heer auf der anderen Seite herantrat und mit ihnen aushandelte, dass sie den Zugang auf Seiten der Gegenwelt halten würde, während man versuchen sollte, die Verbindung ein für alle Mal zu kappen oder zumindest so gut wie es ging abzuschotten. Auch wenn sich der Spalt nicht schließen lassen wollte, gelang es das Tal des Zwielicht zu errichten, welches aufgrund seiner natürlichen Magie, keine Ordnung in der Gegenwelt erzwang, was für Valery gut genug war.

%ABSCHNITT KREATUREN---------------------------------------
\part{Kreaturen}
\setcounter{chapter}{0}
\chapter{Flora und Fauna von Aurum Orbis}

\begin{description}

\item[Jungdrache,]der \index[Kreaturen]{Jungdrache}
\newline Nicht alle Drachen haben damals nach ihrer Entdeckung der Magie, sich diese neue Kraft angeeignet und verfolgten das Ziel, die Wahrheit über das Wesen der Welt zu ergründen. Sie genossen ihr einfaches Leben, weshalb sie in den Götterkriegen auch von den Zwergen verschont wurden. Über die Jahrhunderte, in denen sie sich paarten und wenn sie ihre Zeit für gekommen sahen, ihr Leben beendeten, degenerierte dieser Schlag von Drachen mit jeder Generation, wodurch sie kleiner und vor allem weniger göttlich wurden, als die Drachen bei ihrer Schöpfung waren. Dennoch sind sie immer noch imposante und mächtige Kreaturen, deren Atem alles zu Asche verbrennen kann und dem kein natürliches Material der Welt stand halten könnte. Doch ähnlich wie ihre philosophischen Verwandten tief unter der Erde, meiden Jungdrachen den Kontakt mit der restlichen Zivilisation. Sie leben dabei entweder in einem Hort, um interessante Objekte, wozu meist Schätze gehören, aufzutürmen oder streifen durch die Wildnis, auf der Suche nach Abenteuern, einen guten Kampf oder um die Schönheit der Welt zu erkunden.

\item[Zwerg,]der\index[Kreaturen]{Zwerg}
\\Eine Waffe der Götter, die einst im Krieg gegen die Drachen geschmiedet hatte, als diese versuchten hinter den Göttern nach der tieferen Wahrheit zu suchen. Erschaffen im Herz der Zwerge aus von den Göttern geschaffenen Materialien, vermehrten sich die Zwerge durch den Zugang zu unerschöpflichem Erz wie die Fliegen und schwärmten bald aus, um den rebellischen Drachen ein Ende zu bereiten. Ihr Siegeszug sollte enden als sie aufgrund ihrer Programmierung, andere Lebewesen als ihre Ziele zu ignorieren und nicht zu verletzten, überlistet werden konnten. Eigentlich sollte es keine Zwerge mehr geben, aber bei ihren Streifzügen durch die jungen Reiche anderer Völker hinterließen diese Titanen eine Spur der Verwüstung, etwas was diese nicht hinnehmen konnten, weshalb sie ihren eigenen sehr einseitigen Kampf mit den Zwergen führten. Diese waren ausgestattet mit ungeheurer Stärke und lediglich das durch Magie verstärkte Feuer war in der Lage ihre Panzerung zu zerschmelzen, dafür waren sie nur mit einem Minimum an Intelligenz ausgestattet, da sie über eine telepathische Verbindung im Kollektiv dachten. Gemeinsam schaffte man es ein paar der Zwerge unter großen Felslawinen zu begraben oder man Übergoss sie mit flüssigem Metall und stellte sich die im Erz gefangenen Zwerge als Statuen in den Palast. Doch natürlich starben diese wenigen Exemplare, etwas unter 1000, nie und über die Jahrhunderte standen sie still, während in ihrem inneren langsam die Programmierung zerbröselte. Als man schließlich bei einer Ausgrabung einen der verschütteten Zwerge befreite, begann er als Mordmaschine über alles und jeden herzufallen. Anschließend begab sich dieser auf den Weg seine Geschwister zu befreien, um ihren neuen Auftrag, die Auslöschung aller Lebewesen zu beginnen. Zwar blieb den Göttern diese Entwicklung nicht verborgen, doch sie waren inzwischen nicht mehr in der Lage etwas dagegen zu unternehmen. Die Drachen hingegen beschlossen abzuwarten, um sich die Genugtuung zu gewähren, den Göttern ihre Unzulänglichkeit vor die Augen zu führen und nur im Notfall zu handeln. Die Zwerge haben, nachdem sie alle beisammen waren sich irgendwo in die Wildnis zurück gezogen, wo sie auf der Suche nach Erzen für weitere Zwerge sind nur unglückliche Abenteurer überraschen und angreifen.

\item[Wolf,]der\index[Kreaturen]{Wolf}

\end{description}

\chapter{Magische Monster}

\begin{description}

\item[Frosttroll,]der
\\Einst geboren aus den Sturmwinden von Frostzahn gelten diese bis zu 6 Schritt Hohen Kreaturen genauso unbeugsam und rau, wie ihre Heimat. Gepanzert durch dickes Fell, welches zu einem Flexiblen Eispanzer gefroren ist und ihre fast noch dickere Haut bedeckt, ist die einzige Möglichkeit ihnen spürbaren Schaden hinzuzufügen sehr heißes Feuer oder spitzen Waffen. Auch ohne improvisierte Keulen, wie sie manche Trolle führen, genügt ein einzelner Schlag ihrer Pranken um einen Kämpfer in voller Rüstung zu zermatschen. Außerdem kann ein Frosttroll seinen Feinden denselben Sturmwind entgegen werfen, aus dem sie entstanden sind. Dieser lässt alles schlagartig gefrieren.
\\Einst bevölkerten sie die gesamten Frostzähne und fielen regelmäßig über die benachbarten Regionen her, bis ein Vorstoß der Zivilisation in den legendären Trollkriegen ihresgleichen bis an die Ufer von Trollheim zurückdrängte. Seitdem versuchen diese Bestien ihre Heimat zurückzuerobern, wozu sie jedoch nur jeweils nur zu den sog. Trollwintern, wenn der Fluss, der Trollheim vom übrigen Frostzahn trennt, zufriert, Gelegenheit erhalten.

\item[Doppelschatten,] der\index[Kreaturen]{Doppelschatten}
\\Dieser äußerst hinterlistiger Jäger tritt mit zwei Gesichtern auf. Um ihre Beute, wozu vor allem Wehrlose, wie Kinder oder unbewaffnete, zählen, zu jagen, nehmen sie die Gestalt eines niedlichen Tieres, wie einem Kaninchen oder Schmetterlings, an. Haben sie Beute isoliert und sich selbst in eine gute Position gebracht, erwacht das Biest in ihnen und sie mutieren zu einem Ungetüm mit Ähnlichkeiten zu einem Raubtier, aus welchen sie wohl hervorgehen. Im Gegensatz zu normalen Raubtieren legt ein Doppelschatten besonderen Wert darauf seine Spuren zu verwischen. Das einzige was von ihrem Überfall übrigbleibt, ist ein schwacher Schwefelgeruch, der ihre Verwandlung hinterlässt. Das einzige Erkennungsmerkmal eines Doppelschatten ist ihr Schatten, der jeweils die Form ihrer gerade nicht genutzten Form hat. Deshalb halten sich Doppelschatten vor allem in schattigen Wäldern auf, meiden jedoch Nachtaktivität, da sie in absoluter Dunkelheit ihre Fähigkeit verlieren, die Gestalt zu wechseln.

\item[Blutweide,] die\index[Kreaturen]{Blutweide}
\\Trotz ihres abschreckenden Namens und der Gefahr, die von diesem rot-blättrigem Gewächs ausgeht, hält das viele nicht davon ab, sich in die unmittelbare Gefahrenzone zu begeben. Grund hierfür ist das Rubinchylus, welches aus den Wunden der Rinde austritt. Dieses dient als halluzinogene Droge mit hoher Suchtgefahr. In der Regel wird die Droge durch gesunde Dealer geerntet und anschließend weiterverkauft. Wer Abhängig ist, läuft nämlich Gefahr sich bereits unter den Zweigen der Weide sich dem Rausch hinzugeben. Die Weide jedoch sondert Pollen ab, die sich im Körper des wehrlosen festsetzten und ihn beginnend im Gehirn langsam von innen heraus zersetzen. Über einen Zeitraum weniger Tage verliert das Opfer Erinnerungsvermögen und die Fähigkeit Bewusste Gedanken zu formulieren. Am Ende eines Mondes siecht man dahin und aus dem Leichnam sprießt ein Sprößling der Weide, welche dann über den Zeitraum einer Woche die gesamte Leiche aushöhlt und die Nährstoffe der Mutterpflanze zuführen.

\item[Entfesselter Dämon,] der\index[Kreaturen]{Entfesselter Daemon}
\\Diese Varianten von Dämonen können einem Phänomen zugeordnet werden, welches erst nach dem Exodus erstmals auftrat und Folge magischer Eruptionen war. Beraubt jedweden Mitgefühls, meist sogar einem Großteil ihres Verstandes streifen sie als Jäger und Bestien durch die Welt.
\\ Entfesselte Geoden erwachen bei ihrer Geburt mit ausgewachsenen Klingen, gefangen im Rausch des Berserkers, jedoch ohne selbst am Gift zu verenden. Sie töten alles und jeden, der sich ihnen in den Weg stellt und hinterlassen eine Schneise der Zerstörung. In manchen Fällen schart ein Geodum auch Anhänger in Form andere Bestien um sich, womit sich ihr Gefahrenpotenzial um ein vielfaches steigern kann.
\\Cheylin wiederum haben einen Adäquaten Ersatz für ihre Fähigkeit sich durch ihren Duft unwiderstehlich für andere zu machen gefunden und können nun Problemlos ihre Gestalt ändern und sich dabei den Wünschen und Träumen ihrer Opfer anpassen. Dabei handelt es sich jeweils um Angehörige des anderen Geschlechts, welche sie verführen und dann im Liebesspiel bis zur Erschöpfung treiben, um es anschließend restlos zu verschlingen. Sie gehen dabei sehr subtil und gerissen vor und lassen es oft zu Aussehen, als sei ihr Opfer mit ihrer vermeintlichen Liebe durchgebrannt.
\\Am gefürchtetsten jedoch sind die entfesselten Snagger. Noch verstohlener und ausgestattet mit einem Gift, welches alle organische Materie zersetzt. Bleiben von ihren Opfern nur ihre Eiserne Ausrüstung und eine Pfütze übriggebliebener Nährflüssigkeit, welche im Licht schlagartig entflammt und als ätzende Säurewolke in den Köpfen der unglücklichen, die diese einatmen, das Gehirn angreifen und in den Wahnsinn treiben, weshalb in der Regel niemand von einem Snaggerangriff berichten kann. Dies zusammen mit der Tatsache, dass sich normale Snagger im Hintergrund halten, führt dazu, dass man diese Monster nur als namenlosen Schrecken der Finsternis kennt, der unerwartet und völlig willkürlich zuschlägt.
\\Wie alle Dämonen tragen auch Entfesselte eine Duftnote, können aber nicht darüber kommunizieren. Allgemein gehen sich beide Dämonengruppen aus dem Weg, da sie in einer Konfrontation auf mehr oder weniger Ebenbürtige Gegner treffen, die jedoch ihr Wissen um die Schwachpunkte des anderen mit Verstand nutzen können.

\item[Toprazyx,] der\index[Kreaturen]{Toprazyx}
\\Diese Wesen gehören wohl zu der größter Plage von magischen Monstern. Ihre langezogenen Körper, die hinter ihrem Kopf zum größten Teil in einen Panzer aus meist angelegten kristallinen Stachel ausläuft, erinnern oft an Wölfe. Sie leben von Mana und können dabei gegen sie gerichtete Zauber, wie aus den arkanen Künsten, zum Teil absorbieren. Im Gegenzug können sie, die in ihnen gespeicherten Energien zu kinetischen Sprüngen benutzen. Diese erlauben es ihnen Wände und Körper passieren, letztere nehmen Schaden vor allem wenn sie Manablütige sind. Toprazyx werden von Manakristallen, Manablütigen und Zaubern angelockt, worunter am meisten jene leiden, die unwissentlich von Zaubern profitieren, wie Bauern, denen ein Magier einen Zauber verkauft.

\end{description}

\chapter{Plagen aus der Gegenwelt}
Die Gegenwelt ist das Herz von Chaos und Gegensätzen. In dieser unwirtlichen Welt werden neue Arten so schnell geboren, wie sie auch wieder aufgrund der Radikalität ausgelöscht werden. Nur die härtesten und anpassungsfähigen schaffen dauerhaft Fuß zu fassen, auch wenn ihr Kampf ums Überleben nie aufhört. Mit der Ankunft in Aurum Orbis ging für einige der Traum nach einem ruhigen Leben in Erfüllung. Doch die Mehrzahl waren Bestien, die alle ihre eigenen Ziele verfolgen, denen meistens die angestammten Bewohner im Weg stehen, wenn nicht sie selbst das Ziel sind. 
\begin{description}
\item[Werkreatur,]die\index[Kreaturen]{Werkreatur}
\\Als Werkreaturen werden die Opfer einer Krankheit, welche man in Aurum Obis Therianthropie nennt und die aus den Phfulen der Gegenwelt stammt. Sie erzeugt ein Mutagen, welches den Infizierten in unregelmäßigen Abständen, einer rapiden Mutation in eine schrecklich deformierte Bestie unterwirft. Bei dieser Transformation wird der Körper stark beansprucht und verbrennt einen großen Teil der Nährstoffe, wodurch das entstehende Monstrum einen gewaltigen ungezügelten Hunger auf Fleisch verspürt, der, wenn es ihm nicht schnell nachkommt, tötet. Nach einer gewissen Dauer, meistens 6-8 Stunden findet eine schlagartige Rückverwandlung in die Ursprungsform statt, dabei werden verlorene oder vom Kampf und Leben erlittene Lädierungen rückgängig gemacht(Also sofern sie nicht bereits ohne Arme geboren wurden, werden diese Nachwachsen). Auch sonst ist eine Werkreatur immun gegen Krankheiten und seine Lebenserwartung kann sich teilweise verdoppeln, da bei jeder Verwandlung ein zeitweiliger Verjüngungsprozess eintritt.
Mit der Ankunft der ersten Werkreaturen auf Aurum Orbis unterlag ihre Erkrankung einer gewissen Veränderung. Zum einem ereigneten sich jetzt die Transformationen zum großen Teil in Zyklen, die sich sehr am Mond orientieren. In Neumondnächten ist eine Werkreatur sicher vor der Verwandlung, um dieses Datum herum steigt die Wahrscheinlichkeit in jeder Woche um 25\%, sofern der Mond für die Werkreatur sichtbar ist und führt in einer Vollmondnacht immer zu einer Verwandlung. Die zweite große Veränderung ist, dass die Bestiengestalten nicht bei jeder Verwandlung unterschiedlich ist, sondern nach ihrer ersten Manifestation bleibt, zudem haben die Bestiengestalten von Neuinfizierte oft große Ähnlichkeit zu Raubtieren(Wölfe) aus Arum Orbis und werden anhand dieser in Subkategorien(z.B. Werwölfe) eingeordnet.
\\\textbf{Heilung}
\\Obwohl es schon oft versucht wurde, der Therianthropie Herr zu werden und sie zu behandeln, gelang es noch niemanden diesen Erreger aus einem Infizierten auszutreiben. Sogar Magie und göttliche Intervention scheiterten an dieser Aufgabe und hatten meist nur eine außerplanmäßige Transformation zur Folge.
Lediglich die Erweckung als Untoter oder die Verwandlung in einen Vampir kann der Therianthropie Einhalt gebieten, da diese im Totem Leib verendet.
\\\textbf(Verbreitung)
\\Zum Glück stehen sich Verbreitung und Natur bis zu einem gewissem Punkt im Wege: Therianthropie, wird nur durch Körperflüssigkeiten, die in der Bestiengestalt produziert werden, weitergegeben. Dies ist meistens der Speichel während die Bestie ihr Opfer frisst, nur diejenigen, die diese Begegnung überleben sind danach ebenfalls Werkreaturen. Ansonsten setzten sich nur alle, die mit den Überresten einer frisch erlegten Bestie agieren dem Risiko aus, infiziert zu werden. Kinder von Infizierten neigen dazu sich teilweise im Mutterleib zu transformieren und ihre Mutter oder ihr Ei(Je nach Art) von innen aufzufressen. Bei einer Rückverwandlung sterben sie dann meistens.
\\\textbf{Beziehungen mit Anderen}
\\Aufgrund ihrer brutalen und mörderischen Natur sind Werkreaturen Einzelgänger untereinander. Von der restlichen Gesellschaft werden sie gefürchtet und verfolgt, weshalb sie sich entweder in die Wildnis zurückziehen oder sich in den Schatten der Gesellschaft zu bewegen. Eine ganz besondere Beziehung hegen ''zivilisierte'' Werkreaturen mit den Vampiren, da sich beide ihr Jagdgebiet teilen müssen. Durch die unkontrollierten Ausbrüche sehen die Vampire in Werwölfen eine Bedrohung ihrer eigenen Sicherheit und versuchen sie durch Intrigen auffliegen zu lassen, während letztere die Vampire einfach nur zu töten suchen, was ihnen in ihrer Bestiengestalt, in der sie gegen die betäubende Wirkung des Vampirismus immun sind, sogar gelingt. 
\\\textbf{Eigenschaften}
\\Es folgt eine Liste der Standardmäßigen Fertigkeiten einer Werkreatur. Diese Liste kann je nach Sub-Typ noch um natürliche Eigenschaften der assoziierten Gestalt ergänzt werden(Ein Wervogel kann beispielsweise Fliegen). Bezieht sich eine Eigenschaft auf die Bestie, so gilt diese nur wenn die Werkreatur transformiert ist.
\begin{description}
\item[Instinkte der Bestie]
Eine Werkreatur entwickelt bei seiner ersten Transformation einen schärferen Geruchs- und Hörsinn. Sie kann essbares Fleisch auf 150m erschnüffeln und kann blind nur auf Basis ihres Gehörs kämpfen, sofern ihr Ziel Geräusche macht. Proben auf Wahrnehmung(Gehör) sind entsprechend erleichtert. Nach ihrer Rückverwandlung behält die Werkreatur ihre neue Sinnesschärfe. Außerdem entwickelt die Bestie einen 6. Sinn für Gefahr, der ihr eine 15\% Chance darauf gibt, gegen sie gerichteten Angriffen oder Umgebungszaubern auszuweichen, wenn sie die Quelle nicht erkannt hat. Alle Talente die auf dem Einsatz von Intuition basieren erhalten einen bestialischen Bonus.
\item[Unstet]
Der Trieb zu Fressen überlagert in einer Werkreatur alle anderen Empfindungen. Eine Bestie ist immun gegen Betäubung, Angst, Gedankenkontrolle aller Art oder Erschöpfung, sofern sie nicht natürlichen Ursprungs ist. Auf der anderen Seite verliert die Bestie jedwede Möglichkeit von Sozialen Fertigkeiten Gebrauch zu machen. Alle Aktionen die Konzentration benötigen sind ebenfalls unmöglich.
\item[Metabolismus]In ihrer Bestiengestalt werden Toxine oder Krankheiten innerhalb kürzester Zeit verarbeiten und ausgeschieden, weshalb Werkreaturen gegen diese immun sind(außer sie sind explizit gegen Werkreaturen). Außerdem werden bei der Mutation in die Bestie fehlende Gliedmaßen(auch wenn sie sich sofort der Bestie anpassen) ersetzt, diese fallen bei Rückverwandlung ab, sofern sie bereits seit der Geburt fehlen oder aufgrund eines Fluches fehlen.
\end{description}

\item[Quen,]die
\\Dieses ursprünglich mächtige Volk spinnenähnlicher Kreaturen gehörte zur Oberschicht der Gegenwelt, bis sie aus ihre Heimat durch Eron's Ankunft vertrieben wurden, worauf es ihnen nicht mehr gelang ein neues Königreich zu errichten. Wie vieles kamen diese Kreaturen in den Gegenweltkriegen nach Aurum Orbis, wo sich ihre Stämme in allen Winkeln der Welt verstreuten, um sich ein neues Reich aufbauen. Dabei setzen sie vornehmlich nicht auf ihre körperliche Überlegenheit, die sie im Kollektiv besitzen, sondern vielmehr auf subtile Unterwanderung und Ränkespiele, um ihre Beute nicht zu verschrecken.
\\Quen treten in drei Formen auf: Als einzelne Spinnen von der Größe einer Männerfaust, als Zusammenschluss vieler Spinnen zu einer komplexen Gestalt oder unter der Haut eines willigen Wirtes, da sie selbst keine überzeugende Tarngestalt annehmen können. Ihr widerstandsfähiger Körper verfügt über ein breites Arsenal von Giften, die sie entweder zum Töten verwenden oder um ihren Wirt zu stärken, bzw. im Zaum zu halten. Sie kommunizieren über Telepathie, die jedoch sehr rudimentär gegenüber ungeübten ist. Ein Stamm wird in der Regel vom ältesten weiblichen Exemplar geführt und hat immer eigene Jagdmethoden, von der einfachen Hatz, über das Einnisten in einem unterirdischen Komplex, wo sie mithilfe ihrer Gifte und Telepathie ihre Opfer hineinlocken und festhalten, bis hin zum Aufbau eines kleinen Imperiums unter dem Deckmantel eines Wirtes.
\\\\\textbf{Eigenschaften}
\begin{description}
\item[Gift der 100 Spinnen]Jede Quen produziert eine von vielen chemischen Substanzen. In ihrer normalen Dosis sind diese immer tödlich Toxine, doch in geringen Mengen haben sie die verschiedensten Wirkungen. Manche induzieren bestimmte Emotionen, andere können Hunger, Durst und Schmerz unterdrücken, während andere das Fleisch konservieren und so einer anderen Kreatur zur vermeintlich ewigen Jugend verhelfen können. In der Regel kommt nur ein Wirt in den Genuss dieser sekundären Wirkungen, was viele Verzweifelte bereits in die Arme dieser Bestien getrieben hat, um beispielsweise dem Tod durch altern zu entkommen.
\item[Telepathie]Innerhalb weniger Meter können Quen Gedanken untereinander oder mit anderen intelligenten Lebensformen austauschen. Letztere können zunächst nur abstrakte und sehr grobe Gedanken wahrnehmen, erlernen aber als Wirt mit der Zeit auf das Niveau von Quen zu kommen. 
\item[Körperverbund]Dank besonderer anatomischer Merkmale und jahrhundertelanger Übung sind Quen in der Lage aus ihren Körpern eine formlose Masse zu bilden, die als großer unzerstörbarer Muskel mit Giftstacheln beschrieben werden kann. Sie können zusätzlich mit Hilfe eines Wirtes, in dessen Körper schlüpfen und diesen am Leben erhalten, während der Geist des Wirtes die Koordination der Bewegungen übernimmt. Für Außenstehende ist eine solche Übernahme ohne genaue medizinische Obduktion nicht feststellbar.
\end{description}

\item[Harpie], die
Ehemalig waren diese Mischungen aus Schlangen und Vogel, die berüchtigten herzlosen Sirenen. Als ihnen die Lorkin die Augen öffneten entschieden sie sich dafür, sich restlos von ihrer vorherigen Existenz zu befreien. 

\end{description}

\chapter{Schöpfungen fremder Götter}
\begin{description}

\item[Engel,] der\label{Engel}\index[Kreaturen]{Engel}\index[Stichworte]{Eron!Engel}
\\Um seine Urteile durchzusetzen, hat \uline{\hyperref[Eron]{Eron, der Gerechte}}, eine sehr stark an ihn gebundene Art geschaffen, die Engel. Ausgestattet mit Flügeln und dem Insignium von Eron auf der Stirn streifen sie unermüdlich durch die Welt um Recht zu sprechen. Dabei sind sie ebenso kurzsichtig und hochmütig wie ihr Schöpfer und schlagen ohne zu zögern gegen jeden, der sich ihren Urteilen zu widersetzen sucht oder Kritik an ihrem Gerechtigkeitssinn übt. Neben ihrem engen Kontakt zu Eron, der ihren Körpern die Möglichkeit gegeben hat, dass sie selbst eine geringe Menge göttlicher Macht für eigene Interventionen speichern können, weshalb sie auch trotz der Barriere weiterhin Interventionen ausführen können. Durch die Ausführung ihrer Urteile erhalten sie dabei einen Teil der göttlichen Energie, der ihren Vorrat auffüllt. 
\\\textbf{Eigenschaften von Engeln}
\begin{description}
\item[Gesandte Erons:]
Engel besitzen eine natürliche Verbindung zu Eron, dem sie stets Geweiht sind, über die sie mit diesem über ihre Urteile diskutieren und er ihr Verhalten überwacht. Gleichzeitig speichert ihr Körper einen Teil der göttlichen Energie, die bei der Ausführung von Erons Urteilen entsteht zur eigenen Verwendung in ihren Körpern. Diese können sie ähnlich wie andere Geweihte für Rituale und Gaben verwenden, nur dass sie selbst Ursprung dieser Intervention sind, weshalb sie auch in Aurum Orbis diese durchführen können.
\item[Flügel:]
Jeder Engel trägt ein paar Flügel aus Energie auf den Rücken, mit diesen können Engel unabhängig der Umgebung frei fliegen(Perfekte Manövrierbarkeit).
\end{description} 

\item[Vampir,] der\label{Vampir}\index[Kreaturen]{Vampir}\index[Stichworte]{Askon!Vampir}
\\Nur wenige Kreaturen werden so sehr gefürchtet wie der Vampir. Nicht nur gelten sie als perfekte Jäger der Nacht, sondern verfügen in der Regel auch über einen Intellekt, der es ihnen erlaubt sich relativ frei unter ihrer Beute zu bewegen. Dazu bringt jeder Vampir noch eine Reihe von Möglichkeiten mit, seine Beute in williges Vieh zu verwandeln und so seine Herrschaft über die Lebenden noch weiter auszubauen. Lediglich ein alter Fluch, der auf einem ihrer finsterem Erschaffer, \uline{\hyperref[Askon]{Askon, dem gefallenem Gott}}, lastet und das Licht der Sonne für sie zur Todesfalle werden lässt, verhindert, dass sie ihr Imperium auf die gesamte lebendige Welt ausbreiten können. Über ihre Herkunft wissen nur die ersten und ältesten Vampire Bescheid und sie hüten dieses Wissen wie einen Schatz, genauso wie die geheimen Techniken die sie bei ihrer Schöpfung gelehrt bekommen haben.
\\\\\textbf{Vampirismus}
\\Der Fluch des Vampir(Vampirismus) ist das Ergebnis einer unheiligen Verbindung zwischen \uline{\hyperref[Charon]{Charon, dem Fährmann}} und \uline{\hyperref[Askon]{dem gefallenem Askon}}, in der sie versuchten die Schwachpunkte ihrer Schöpfungen auszugleichen. Und auch wenn Askon seinem Partner später um die Kreation betrog, konnte Charon sich der Vampire nicht mehr entledigen. Der Fluch des Vampirismus tötet seine Opfer, durchwirkt sie mit seinem Fluch, bevor er sie als Untote Kreaturen wiederauferstehen lässt. Fortan müssen sie sich von der göttlichen Essenz aus dem Blut andere Schöpfungen ernähren. Im Gegenzug winkt ihnen ein Leben in Unsterblichkeit und Macht, sofern sie das Sonnenlicht meiden. Denn Askon verleiht im Gegenzug für die Opferung von der gesammelten Lebenskraft eines Vampires diesen außergewöhnliche Fähigkeiten. Auch wenn Vampirismus einen de Fakto zu einem Untoten macht, interessieren sie sich nicht für Nekromantie, ja verachten diese sogar. 
\\\\\textbf{Eigenschaften und Fähigkeiten eines Vampirs}\\
\begin{description}
\item[Vampirblut:]
Mit der Transformation zu einem Vampir, zerfrisst der Fluch jeden Tropfen von Restblut und einen Großteil der inneren Organe und füllt die Blutbahnen mit einer Tinten-ähnlichen Flüssigkeit, die bei Kontakt mit der Luft jedoch sofort Blutrot wird. Dieses Blut verliert alle Eigenschaften bezüglich der Ursprungsrasse, wie die Affinität zu Mana von Drachenblut. Zwar gehört der Vampir immer noch seiner ursprünglichen Rasse an, wenn es um die Benutzung von göttlichen Artefakten geht, allerdings verdirbt ihr Einfluss diese, wodurch ihre Wirkung oft ins Gegenteil verkehrt werden.
\\Zwar ist der Vampirismus eine göttliche Schöpfung, was viele Vampire dennoch nicht davon abhält sich die Magie anzueignen. Aufgrund ihrer Fähigkeit sich sehr schnell von Wunden zu erholen und sie selbst de Fakto tot sind, verträgt sich ihr Körper außergewöhnlich gut mit Mana aller Art. Sie sind doppelt so Resistent gegen Manabrand, wie vergleichbare nicht-vampirische Manablütige(Bei einem Elfischen Vampir betrachtet man das menschliche Gegenstück).
\\Einem Vampir wird bei seiner Transformation automatisch zu einem Askon-Geweihten und kann dies auch nicht wieder ändern. Er erhält darüber hinaus Zugriff auf eine Reihe nur Vampiren vorbehaltenen Ritualen, die er in der Regel mit seinem eigenem Blut bezahlen muss. Hierzu erhält jeder Vampir einen sog. Pool aus Blutpunkten, dessen maximale Größe der Konstitution + 15 entspricht.
\item[Der Biss:]
Der Biss eines Vampirs dient diesem zum Aussagen seines Opfers. Um den Prozess dabei zu vereinfachen, injiziert der Vampir eine sehr, sehr geringe Menge seines eigenen Blutes, wodurch sein Opfer in einen Zustand purer Euphorie und Entspannung verfällt und nach Beendigung des Absaugen das Opfer ohne Erinnerung an dieses Ereignis zurück lässt. Je nachdem unter welchen Umständen der Biss eingeleitet wurde(z.B. vor dem geplanten Akt oder bei einem amourösen Kuss), kann das Opfer danach mit einer emotionalen Bindung zum Vampir zurück gelassen werden, was spätere Bisse wesentlich einfacher macht. Durch dieses Absaugen nimmt das Opfer 10 Schadenspunkte(untypisch) pro Runde wohingegen der Vampir einen Blutpunkt erhält. Der Vampir erhält für die Dauer des Bisses Einsicht über die verbleibenden Blutpunkte, bevor sein Opfer dahinscheidet.
\item[Der Kuss:]
Ein Vampir kann nicht nur sein Opfer blutleer saugen, sondern ihm anschließend einen Teil seines eigen vom Vampirismus verdorbenen Blutes schenken(Die Hälfte seines eigenen Blutpools, mindestens 5), wodurch dieses sich wenig später(1 Tag) ebenfalls als Vampir erheben wird. Durch den Kuss entsteht ein Band zwischen beiden Vampiren, ähnlich der Bindung zwischen Eltern und ihrem Kind, auch wenn vor allem bei korrumpierten Seelen, dieses Überwunden werden kann. Es gilt außerdem zu beachten, dass der Fluch mit jeder Generation eines neuen Vampir schwächer wird, weshalb ältere Vampire in der Regel über den jüngeren stehen.
\item[Der niedere Kuss:]
Ein Vampir kann, wenn sein Opfer noch genug eigenes Blut in seinen Adern hat, eine Portion, die wesentlich geringer als beim Kuss ist, seines Blutes injizieren(1 Blutpunkt). Dies hat zur Folge dass der Fluch des Vampirismus das Opfer nicht komplett tötet, allerdings die eigene Lebenskraft langsam verzehrt. Das Opfer, welches unter Vampiren als Famulus bezeichnet wird, fühlt sich zunächst ähnlich wie beim Biss euphorisch, was durch ein Gefühl von Macht, da man einen Teil der Stärke und Ausdauer eines Vampirs erhält, noch verstärkt wird. Doch gegen Ende des ersten Mondes nach der ersten Injektion werden erste Nebenwirkungen auftreten, rapide wird das Opfer altern und gebrechlich werden, sowie einen schrecklichen Hunger verspüren, den nichts, auch nicht der Genuss von gewöhnlichem Blut stillten kann. Der Vampir kann in diesem Zustand sich seines Famulus annehmen und ihn durch eine erneute Injektion wiederherstellen. Das Opfer wird dabei für die Zukunft durch das Wissen von Schmerz bei Ausbleiben einer neuen Injektion und des alles übersteigenden Glückes nach einer Injektion zu einem Sklaven seines Gönners. Oft geht dieses Verhältnis mit einer abgöttischen Verehrung für diesen einher, auch wenn es Ausnahmen gibt, die bis hin zu abgrundtiefen Hass reichen können. Ein Famulus, der für zwei Monde kein neues Blut von seinem Gönner bekommt, geht unweigerlich zu Grunde, wobei Körper und Seele vollständig vernichtet werden.
\item[Schatten ihrer Selbst:]
Eine der größten Schwächen eines Vampirs ist das Sonnenlicht. Sobald sie in Kontakt mit diesem kommen, brennt das Sonnenlicht in einem Augenblick Löcher dort wo es den Vampir berührt. Es bleibt nichts zurück, wenn ein Vampir komplett bestrahlt wird. Nur Dicke Mauern oder andere Lichtundurchlässige Barrieren können sie gegen dieses abschirmen. Kleidung, außer aus speziellem Material, kann sie nicht schützten.
\\Selbst wenn es ihnen gelingt sich selbst vollständig gegen das Sonnenlicht abzuschirmen, so schirmt Sonnenlicht alle Kräfte das Vampirs, wie seine Telepathie ab. Eine weitere Folge ihres Zustands ist das Fehlen eines Spiegelbildes oder Schattens, beides Merkmale an denen Kundige einen Vampir erkennen können. Durch Rituale erschaffenes Sonnenlicht zählt als vollwertiges Sonnenlicht, gleiches gilt für Morgentau aus der entsprechenden Meditation der Ixania.
\item[Perfekter Jäger:]
Vampire können wenn sie es wünschen, jedes Geräusch ihrer Schritte oder Ausrüstung eliminieren. Sie erhalten die Fähigkeit potenzielle Beute über ihren Geruchssinn aufzuspüren, wogegen nur sog. \uline{\nameref{Vampirbann}} hilft. Ein Vampir kann den Geruch studieren, um sich über den gesundheitlichen Zustand seines Opfers zu informieren, außerdem können besonders erfahrene Vampire, Informationen, wie Rasse, Geschlecht, Alter, Manablut und unter Umständen sogar Stand und Herkunft abschätzen. Vampire besitzen außerdem eine natürliche Begabung, im direkten Handgemenge einen Biss zu platzieren und ihr Opfer als Schild gegen Angriffe einzusetzen. Sie erhalten einen Bonus auf das Einleiten eines Handgemenge, zum Beginn eines Bisses und auf ihre Verteidigung während sie ihr Opfer aussaugen.
\item[Schattenaffin:]
Ein Vampir ist eine Kreatur der Dunkelheit und erhält in magischer oder völligen Dunkelheit eine perfekte Wahrnehmung von allem innerhalb dieser Dunkelheit, solange es sich nicht mehr als 30m von ihm entfernt befindet. Er kann auch Lichtquellen innerhalb dieses Radius lokalisieren, wenn auch nichts innerhalb ihrer Lichtkreise. Der Körper des Vampir scheint, mit der Dunkelheit selbst eins zu werden. Auch Kreaturen mit Dunkelsicht haben daher eine 50\% Chance den Vampiren zu verfehlen. Ein Vampir hat außerdem eine doppelte Bewegungsreichweite innerhalb von Dunkelheit und seine Sprünge haben immer die doppelte Reichweite und zählen immer so, als hätte der Vampir Anlauf genommen.
\item[Band des Geistes:]
Ein Vampir kann durch Blickkontakt mit einem Opfer ein telepathisches Band zwischen beiden knüpfen. Über dieses können beide miteinander kommunizieren, wobei der Vampir seine Ausführungen mit anderen Sinneseindrücken unterlegen kann, weshalb es ihm möglich ist, soziale Fertigkeiten einzusetzen, wie Überreden oder Einschüchtern, wogegen sich selbst ein unwilliger Geist nicht wehren kann. Ein Vampir kann auf diese Weise nur eine Konversation gleichzeitig führen, auch wenn er auf Wunsch jederzeit, auch ohne Blickkontakt das Band zwischen sich und einem seiner Famuli oder Kinder/Erzeuger, aufnehmen kann. Ein anderer Vampir kann diese Form des geistigen Kontaktes unterbinden. 
\item[Totes Fleisch:]
Der Körper eines Vampirs kann sich nicht auf natürlichem Wege heilen, allerdings behindern Wunden einen Vampir nicht, sofern der Körper noch immer ausreichend zusammengehalten wird. Um einen Vampiren zu vernichten, muss er daher dem Sonnenlicht ausgesetzt werden, ansonsten kann der Vampir, wenn er noch über ausreichend Energie verfügt selbst wiederherstellen, selbst als verbrannter Aschehaufen, oder mit der Hilfe durch einen anderen Vampiren oder Askon selbst über einen Zeitraum von 1 Woche wiederhergestellt werden.
\end{description}
\textbf{Askons Gaben}\label{Vampire:AskonGaben}
\\Wie bereits zuvor erwähnt hat jeder Vampir Zugriff auf eine eigene Reihe von Ritualen und Gaben, die der Vampir mit einem kleinem Blutopfer bezahlen muss.
\begin{description}
\item[Stimme des Verführers:]
Ein Vampir kann Askon, der selbst ein Meister der Lügen, falschen Versprechungen und schöner Worte ist, an seiner Stelle reden lassen, wodurch der Vampir einen ungemeinen Bonus auf eine Lügen, Verführen, Überreden, Überzeugen oder vergleichbare Probe erhält. Er kann diesen Bonus auch über sein Band des Geistes zur Geltung bringen.
\item[Vision der ewigen Nacht:]
Der Vampir umgibt sich mit einer Aura von Dunkelheit, die seinen Gegnern einen Eindruck des ihnen bevorstehendes Untergangs vermittelt. Alle nicht-schattenaffine Kreaturen werden Opfer eines Furchteffektes. Der Vampir kann diese Aura nicht für seine Schattenaffinität nutzen, allerdings kann er durch Aufbringen zusätzlicher Blutopfer Sonnenlicht negieren.
\item[Umarmung der Nacht:]
Mit Askons Segen vervollständigt der Vampir seine Verbindung mit der Dunkelheit und kann für eine gewisse Zeit zwischen einer körperlosen aus Dunkelheit bestehenden, die es ihm ermöglicht innerhalb eines Schattens jede beliebige Form anzunehmen, um beispielsweise Gitter zu passieren oder sich an Wänden und Decken zu bewegen, und seiner gewöhnlichen Form wechseln. Er kann in seiner Formlosen Gestalt durch physische Effekte keinen Schaden nehmen und auch die meisten Zauber die auf den Körper zielen, verlieren für die Dauer seiner Transformation ihre Wirkung. Endet die Wirkung von Umarmung der Nacht oder wird der Schatten, in dem sich der Vampir befindet, ausgeleuchtet, während er sich in seiner Schattenform befindet, nimmt er im nächsten geeignetem Raum wieder seine physische Gestalt an.
\item[Schatten zu Fleisch:]
Der Körper eines Vampirs kann bekanntermaßen nicht heilen. Um sich dennoch störender Wunden oder von Sonnenlicht gebrannte Löcher zu entledigen, kann ein Vampir sein Blut für die Heilung verwenden. Dabei kann jeder Punkt einen Punkt Vitalität wiederherstellen. Erreichen sie das theoretische Maximum, können sie für 5 Punkte eine theoretische Wunde wiederherstellen. Werden diese vollständig von einer fehlenden Extremität entfernt, so gilt diese als wiederhergestellt. Der Prozess der Heilung erfolgt dabei über den Zeitraum einer Rast, die Wiederherstellung eines Körperteils nimmt hingegen einen Tag im Anspruch, allerdings begrenzt von dem Zeitraum einer Woche.
\end{description}


\end{description}

\phantomsection
\addcontentsline{toc}{chapter}{Kreaturenindex}
\printindex[Kreaturen]
%TECHNOLOGIE!!!!!!!!!!!!!!!!!!!
\part{Technologie}
\setcounter{chapter}{0}
Hier fehlt noch alles\newpage
%Professionen !!!!!
\part{Professionen\&Spezialisierungen}
\setcounter{chapter}{0}
\chapter{Einführung}
Unter einer Profession versteht man einen Beruf, der einer Figur Zugriff auf bestimmte Fertigkeiten-Spezialisierung, Kampftalenten oder passiven Boni gibt, diese Orientieren sich oft an dem Tätigkeitsfeld der Profession.
\\Spezialisierungen auf der anderen Seite sind entweder stark zugespitzte Professionen oder es handelt sich um eine Reihe eigener Techniken und Dinge, die auf ein sehr eng begrenztes Aufgabenfeld beschränkt sind.
Es folgt eine Liste aller Professionen\&Spezialisierungen, bzw. Verweise auf die Stellen in vorhergehenden Texten.
\chapter{Offene Professionen}
In diese Kategorie fallen alle Professionen und Spezialisierungen, die in der Regel jeder erlernen kann, auch wenn kulturelle Differenzen das Erlernen erschweren können.
\section{Alchemist}
Meister der Alchemistischen Ordnung und Zerstörer von Stadtvierteln.
\subsection{Fertigkeiten}
+: Alchemie, Alchemistische Ordnung, Kräuterkunde
-: Alle Waffentalente
\subsection{Stufen}
\begin{description}
\item[Stufe 0:]
\end{description}

\section{Weltenbummler}
siehe Tholemäus, Weltenbummler.
\subsection{Anforderungen}
Göttliche Verbindung
\subsection{Fertigkeiten}
+: Alle Wissenfertigkeiten
-: Ale Waffentalente außer Schwerter und Duellierwaffen, 
\section{Bauer}
\section{Fischer}
\section{Bote}
\section{Matrose}
Ob auf See oder hoch in den Lüften, bewegen sich Matrosen wie emsige Ameisen über das Deck, durch die Takelage oder den Maschinenraum ihres Vehikels um es in Fahrt zu bringen und den Gewalten der rauen See oder hohen Lüfte zu trotzen. Diese Profession hat 2 Spezialisationen, die beim Erwerben dieser Profession gewählt werden muss und danach nicht mehr geändert werden kann.
\subsection{Stufen}
\begin{description}
\item[Stufe 0:]Seemann(Immunität gegen Seekrankheit)/Flieger()
\end{description}
\section{Pirat}
Nicht nur leben in den tiefen der Meere und den Gefilden der Lüfte teilweise namenlose Schrecken, nein darüber hinaus kommen jedoch aber auch die Ruchlosen Männer und Frauen, die unter der Schwarzen Flagge der Piraterie segeln. Sie gelten als ruchlos
\section{Mechaniker}
\section{Arzt}
\section{Soldat}
Ob in einer Armee oder als Wächter in einer Stadt, Soldaten sind ausgebildete Milizen, die aus verschiedensten Gründen sich für den Dienst mit der Waffe verpflichtet haben. In der Regel verpflichten sie sich für eine gewisse Zeit einem bestimmten Regiment oder einer Stadt zu dienen, wobei natürlich bei außergewöhnlichen Leistungen eine Beförderung zum Bleiben locken kann.
\subsection{Fertigkeiten}
+: Waffen, Schmieden(Waffen/Rüstungen)
-: Wissen(außer Kriegskunst, Staatskunst, Rechtskunde), Bailieren, Feinmechanik, Schneider, Gaukelein, Tanzen
\subsection{Stufen}
\begin{description}
\item[Stufe 0:]Kampfdrill(2 kostenlose grundlegende Kampftechniken, für die die Bedingungen erfüllt werden);
\item[Stufe 1:];Feldarzt(+6 Heilkunde:Wunden)
\end{description}
\section{Barde}
\subsection{Stufen}
\begin{description}
\item[Stufe 0:](Barden haben eine 50\% Chance unbemerkten Angriffen ausweiche zu dürfen); 
\item[Stufe 1:](Barden können beim Vortragen eines Bardenstücks ausweichen, ohne ihren Vortrag unterbrechen zu müssen);
\item[Stufe 2:](Bardenstücke können selbst den Lärm von Stürmen oder Schlachten durchdringen);
\item[Stufe 3:]();
\end{description}
\section{Gladiator}
\section{Knecht/Magd}
\section{Gaukler}
\subsection{Stufen}
\begin{description}
\item[Stufe 0:]Eure Aufmerksamkeit bitte!(Mit Worten und Gesten lenkt der Gaukler die Aufmerksamkeit aller Umstehenden auf ein Ziel seiner Wahl)
\item[Stufe 1:]
\end{description}
\section{Trickbetrüger}
\section{Fälscher}
\section{Gelehrter}
\section{Priester}
\section{Koch}
\section{Faustkämpfer}
Während hinter der typischen Kneipenschlägerei oft keinerlei Technik steckt, sind es vor allem bestimmte Arenen und Box-Ringe aus denen eine ganz eigene Art von Kämpfern stammen. Die Faustkämpfer, geübt im Umgang im Waffenlosen Kampf sind sie vor allem auf engem Raum und in Duell selbst einem Bewaffnetem ebenbürtig, wenn nicht sogar überlegen.
\subsection{Fertigkeiten}
+:Faustkampf, Athletik, Akrobatik
\\-:Alle anderen Waffentalente 
\subsection{Stufen}
\begin{description}
\item[Stufe 0:]Kampfstil(Zur Berechnung des Angriffswertes bei der Benutzung von Faustkampf darf ein Attribut gegen ein anderes körperliches, welches noch nicht zur Berechnung gehört, getauscht werden, diese Wahl kann nicht geändert werden); 
\item[Stufe 1:]Einstecken 1(Im Faustkampf kann bei Kampfmanövern die Probe erleichtert werden, indem man freiwillig Schaden nimmt);
\item[Stufe 2:]Kampfstil 2(Ein weiteres Grund-Attribut darf gegen ein beliebiges Körperliches Attribut getauscht werden.);
\item[Stufe 3:]Einstecken 2(Im Faustkampf kann man beim einstecken von physischen Schaden seinen Angriffswert bis zu seinem übernächsten Zug senken, um einen Teil des Schadens zu negieren);
\item[Stufe 4:]Kampfstil 3(Ein weiteres Grundattribut darf gegen ein beliebiges Geistiges Attribut getauscht werden);
\end{description}

\section{Jäger}
Das Gebiet eines Jägers ist die gemäßigte Wildnis, wo er mit Fallen und Fernkampfwaffen hinter dem Wild her sind, um ihre Felle und ihr Fleisch zu erbeuten und zu verkaufen. Dabei nutzen sie ihre Jahrelange Erfahrung, sowie ihre Instinkte, um der Beute immer einen Schritt voraus zu sein und um zu verhindern, selbst zum Gejagten zu werden.
\subsection{Voraussetzungen}
Volk: Jedes, Kultur: Jede, auch wenn Stadtbewohner die doppelte Menge an Arbeit in die Grundausbildung stecken müssen
\subsection{Fertigkeiten}
+:Schleichen, Verstecken, Überleben, Naturkunde, Tierkunde, Handwerk(Fallen), Orientieren, Fleischer, Gerber, Spuren lesen
-:Soziale Kategorie, Akademische Wissenfertigkeiten(Geschichte, Heraldik, Magie, Religion, etc.)
\subsection{Stufen}
\begin{description}
\item[Stufe 0:] Fokus(Schleichen, Pirschen); Fokus(Handwerk(Fallen), Schlingen), Feindklasse 1(Wild), Fokus(Spuren lesen, Fährtensuche), Ausschlachten(+3 Fleischer und Gerber)
\item[Stufe 1:] Geduld 1(Der Jäger kann stundenlang ohne Nahrung oder Trinken auf der Stelle verharren ohne Mali zu nehmen); Feindklasse 2(Wild), Schnelles Schleichen, Eiserne Miene(+3 Feilschen und Lügen), Improvisiertes Werkzeug(Fleischer und Gerber)
\item[Stufe 2:] Geduld 2(Ein Jäger riskiert sich keine Krankheiten durch schlechtes Wetter zuzuziehen und erhält nur den halben Malus auf Angriffe durch diese); Feindklasse 3(Wild), Jagdlager(Der Jäger kann aus einfachstes Mitteln ein gegen schwere Unwetter gefeites Lager errichten für bis zu 3 Personen errichten, dieses gewährt vollen Sichtschutz und verwendet den Verstecken-Wert des Jäger, sowie seinen Pirschen-Wert gegen Geruchswahrnehmung durch Wild), Improvisiertes Werkzeug(Fallen(Schlingen)), Jäger unter Jägern 1(Der Jäger hat gelernt selbst der Jagd durch stärkere Kreaturen zu entgehen und erhält einen Bonus von 6 auf Verstecken-Proben in der Wildnis), Spurlos.
\item[Stufe 3:] Geduld 3(Ein Jäger kann selbst in feindlichem Klima mit normalen Rationen auskommen und erhält keinen Zuschlag auf seine Erschöpfungsrate durch diese. Mali durch schlechtes Wetter wird vollständig ignoriert); Feindklasse 4(Wild), tierischer Gefährte(Der Jäger kann sich gegen einen kleinen Sagenpunktenzuschlag ein kleines Tier, meist einen Falken, Mader oder Wolf als Jagdgefährten besorgen), Trophäensammler(Fokus(Fleischer, Trophäen). Der Jäger kann auch ohne handwerkliches Hintergrundwissen wertvolle Bestandteile seiner Beute entnehmen), Stählerne Miene(+3 Feilschen und Lügen)
\item[Stufe 4:] Geduld 4(Gewaltmarsch durch unwirtliches Gelände oder bei schlechtem Wetter gibt keinen Zuschlag auf die Erschöpfung); Jäger unter Jägern 2(Der Jäger hat sich über die Zeit den Respekt animalischer Jäger verdient und wird daher niemals von solchen attackiert, natürlich nur wenn er es selbst nicht tut, und kann sich ihnen sogar bei einer Jagd anschließen, um einen Teil der Beute abzubekommen), Meisterlicher Schuss, Feidklasse 5(Wild), Nähe zum Tier(+3 Pirschen und Verstecken in der Wildnis)
\end{description}
\section{Söldner}
Neben den in den Kasernen oder speziellen Kampfschulen ausgebildeten Soldaten und Kämpfer sind Söldner Männer fürs Grobe. Ihre Ausbildung basiert mehr auf eigener Erfahrung in einfachen Kampfstilen und sie bedienen sich häufig unkonventioneller Kampftechniken. Sie werden meist als billige Schläger oder Leibwächter angeworben und sind dafür bekannt in der Regel nicht über ihren ursprünglichen Auftrag hinaus unnötig gewalttätig oder freundlich zu sein.
\subsection{Voraussetzung}
Volk: keine Elfen, Kultur: Jede
\subsection{Fertigkeiten}
+:Einschüchtern, einfache Kampfwaffen, Gassenwissen, Zechen
-: Etikette, Überreden, Überzeugen, Akademische Wissenfertigkeiten(Geschichte, Heraldik, Magie, Religion, etc.)
\subsection{Stufen}
\begin{description}
\item[Stufe 0:] Abgehärtet(+HP); Kampftrick(Konstenloses Kampfmanvöer), Straßennase(+3 Einschüchtern, +3 Gassenwissen), 
\item[Stufe 1:] Dicker Schädel(+HP, + 3 Zechen); Defensive Taktiken,
\item[Stufe 2:] Stahlmagen(+HP, +3 gegen Übelkeit); Offensive Taktiken, Gefahrensinn 1(Ein Söldner verliert nicht seine Verteidigung wenn er flankiert wird, auch wenn er weiterhin flankiert werden kann), 
\item[Stufe 3:] Überlebender(+HP, +3 gegen kritische Treffer); Gefahrensinn 2(Ein Söldner erhält stets einen zweiten Versuch auf seine Wahrnehmungsproben, um einen getarnten Feind wahrzunehmen)
\item[Stufe 4:] Veteran(+HP, +3 gegen Furcht); Gefahrensinn 3(Ein Söldner nimmt jeden auf ihn gerichteten Angriff als unbestimmtes Kribbeln wahr, sodass ihm die Gelegenheit bleibt diesem auszuweichen)
\end{description}
\section{Gauner}
Gauner sind die Finsteren Gestalten einer jeden Stadt, sei es Erpressung, Brandstiftung oder Drohungen, Gauner sind für solch grobe Verbrechen genau die Richtigen. Sie verlassen sich nicht nur auf ihre Schlagfertigkeit, sondern vor allem auf hinterlistige Tricks und einen ''bleibenden'' Eindruck. Dazu bringen viele Gauner noch ein gewisses Maß an Ambitionen für Organisation mit, weshalb diese häufig die allgemeine Kriminalität einer Stadt organisieren und einzelne Spezialisten zusammenführen.
\subsection{Voraussetzungen}
Volk: Jedes, Kultur: Jede die organisiertes Verbrechen kennt.
\subsection{Fertigkeiten}
+: Gassenwissen, Lügen, Einschüchtern, Waffenloser Kampf, Improvisierte Waffen, Dolche, Keulen, Zechen
-: Wissen, Handwerk
\subsection{Stufen}
\begin{description}
\item[Stufe 0:] Nachhaltige Drohung 1(Einschüchterungen eines Gauners haben die doppelte Wirkdauer); Schmutzige Tricks, Straßenköter(+3 Zechen und Gassenwissen), Mimik(+3 Lügen und Einschüchtern) 
\item[Stufe 1:] Nachhaltige Drohung 2(Nach einem Erfolgreichem Einschüchterungsversuch sind alle Einschüchterungen gegen das Ziel um 3 erleichtert, bis ein Versuch fehlschlägt);
\item[Stufe 2:] Präsenz der Angst 1(Eingeschüchterte Figuren sind für die Dauer der Einschüchterung unfähig Pläne gegen den Gauner zu schmieden. Ihre Lügenversuche ihm gegenüber sind um 3 erschwert);
\item[Stufe 3:] Präsenz der Angst 2(Auch nach dem Auslaufen eines Einschüchterungseffekt bleiben die Effekte von der Vorstufe erhalten);
\item[Stufe 4:] Ewige Drohung(Ein Gauner kann seinen Einfluss auch durch seine Handlanger gelten machend. Ein Gauner überträgt seinen Bonus von Nachhaltige Drohung auf alle, die in seinem Namen handeln, oder ihr Opfer von diesem Fakt überzeugen können.)
\end{description}
\section{Schurke}
Ob als Beutelschneider, Einbrecher oder Attentäter. Schurken sind die Kriminellen für feinfühlige Aktionen und bevorzugen es in den Schatten hinter den Kulissen zu arbeiten. Wo ihre Gegenstücke die Schurken oft als Gruppen agieren, sind Diebe in der Regel Einzelgänger. Natürlich gibt es auch Gilden aus Schurken, die oft einen strengen Arbeitskodex verfolgen und beispielsweise Attentate links liegen lassen, aber diese können meist nicht gegen reine Spezialistengruppierungen bestehen. Dennoch sind Schurken aufgrund ihrer Vielseitigkeit eine essenzielle Ergänzung für alle Arten des Organisierten Verbrechens.
\subsection{Voraussetzungen}
Volk: Jedes, Kultur: Jede die Verbrechen kennt.
\subsection{Fertigkeiten}
+: Gassenwissen, Schätzen, Lügen, Dolche, Boden, Wurfwaffen, Ringen, Zechen, Schlösser knacken, Feinmechanik, Schleichen, Verstecken, Klettern, Wahrnehmung, Gaukeln
-: Einschüchtern, Handwerk, Wissen, Alle übrigen Kampftalente
\subsection{Stufen}
\begin{description}
\item[Stufe 0:] Fokus(Verstecken, Mengen); Fokus(Feinmechanik, Fallen entschärfen), Hehler(+3 auf Schätzen und Feilschen), Taschenspieler(+3 Gaukeleien, +3 Gaukeleien(Taschendiebstahl)), Schatten(+3 Schleichen, +3 Verstecken)
\item[Stufe 1:] Mengengewandheit(+3 auf Verstecken in Mengen und keine Behinderung durch Mengen); Giftnutzung
\item[Stufe 2:] Pacour(+3 auf Klettern und Akrobatik)
\item[Stufe 3:] Wäsche(Wertvolles Diebesgut wird eher von gewöhnlichen Händlern abgenommen)
\end{description}
%
\chapter{Geschlossene Professionen}
Diese Professionen stehen entweder nur bestimmten Völkern zur Verfügung oder werden nur in entsprechenden Ausbildungsorten erlernt werden können und daher nicht als Startprofessionen gewählt werden können.
\section{dunkelelfischer Krieger}
Gefürchtet als erbarmungslose Jäger von allem mit Fleisch auf den Rippen, was ihnen begegnet. Unsichtbar, mit Schleudern, Bögen oder Speeren streifen sie über das Land, mal erlegen sie ihre Beute mit einem gezielten Schuss, mal fangen sie es auch nur ein, um ihren Jungen im Stamm die Kunst des Tötens am Lebendem Subjekt zu demonstrieren. Im jedem Fall sind sie als Gruppe ein Todesurteil, selbst für eine Übermacht.
%
\chapter{Weiterentwickelte Professionen}
Viele offene Professionen decken ein breites Aufgabenfeld ab, die folgenden Professionen dienen der Spezialisierung auf ein bestimmtes Aufgabenfeld. Um sie zu nutzten benötigt es daher eine gewisse Erfahrung mit der Grundprofession.
\section{Besitenjäger}
Nach Jahren in der Wildnis gehen viele Bewohner der Wildnis darin über sich nicht länger mit dem Wild zu messen und bei Gefahr durch einen größeren Jäger den Kopf einzuziehen, sondern vielmehr diese Bestien zu jagen. Mit größeren Waffen, kreativen Fallen oder anderen unkonventionellen Methoden, die Erfahrung eines Bestienjägers sind von unschätzbarem Wert um selbst unverwundbar scheinende Geschöpfe zu Fall zu bringen.
\subsection{Voraussetzungen}
Stufe 2 Jäger
\subsection{Fertigkeiten}
+: Schleichen, Handwerk(Fallen), Bogen, Speer, Armbrust, Wildniskunde, Spuren lesen, 
-: Gesellschaft, Andere Waffentalente, Andere Handwerkstalente
\subsection{Stufen}
\begin{description}
\item[Stufe 0:] Schwachstellen 1(Durch längere Beobachtung werden für einen Bestienjäger Schwachstellen sichtbar, die ansonsten verborgen blieben.); Fokus(Handwerk Fallen, Großwild), Tröphensammler(Fokus(Schlachter, Tröphen)), Fokus(Spurenlesen, Wildnis)
\item[Stufe 1:] Schwachstellen 2()
\end{description}
\section{Kopfgeldjäger}
\section{Einbrecher}
\section{Attentäter}
\newpage

\part{Zu Sortierendes}

%Geht in Geographie
\chapter{Tal des Zwielichts}\label{Tal des Zwielicht}\index[Geographie]{Tal des Zwielicht}
Nachdem man feststellen musste, dass magische Barrieren um das Weltenloch niemals einen dauerhaften Schutz vor der Gegenwelt bieten würden, wandten sich die Bewohner von Aurum Orbis der zweiten großen Machtquelle zu, den Aspekten. Sie waren bereits zuvor von großer Hilfe im Kampf gewesen und nun war man soweit, dass man eine Lösung für einen dauerhaften, göttlichen Schutzschirm um den Zugang zur Gegenwelt spannen konnte. Das Weltloch lag verborgen in einem, von einem Gebirge abgeschlossenen, Tal, das den Gegenweltlern zuvor eine ideale Verteidigung und Deckung geboten hatte. Als man mit den neuen Verbündeten aus der Gegenwelt die feindlichen Kräfte zurück in jenes trieben konnte, begann eine ganze Reihe von Geweihten der Ixania mit einem gewaltigen Ritual, an dessen Ende das gesamte Tal in \uline{\nameref{Zwielicht}} getaucht war. Daraufhin konnte keine Armee aus der Gegenwelt das Tal verlassen, ohne wegen des Entzugs innerhalb weniger Tage zugrunde zu gehen, da der Ritus diesen nicht bekannt war. In den nachfolgenden Jahren, nach Ende des zweiten Gegenweltkrieges, errichteten die Ixania-Geweihten und viele Helfer, einen Tempel zum Auffangen des ersten Morgengrauen auf der Außenseite der Berge und einen Tempel zum Einfangen von Mondlicht auf der Innenseite. Dazu kamen eine Reihe von kleineren Schreinen, die jeweils die Tempel bei ihrer Aufgabe unterstützen. Zusammen entsteht ein konstanter Strom von Zwielicht, der das Tal flutet und welches, zusammen mit der \uline{\nameref{Antimagie}}, ein Durchbrechen dieser Barriere unmöglich macht und die Welt schützt.


\chapter{Erze\&Metalle}
\section{Mundane Erze}
\begin{description}
\item[Eisen]Dieses Erz findet aufgrund seiner nahezu unbegrenzten Verfügbarkeit durch die Erzseen, in allen Bereichen die auf standhafte Materialien setzen Anwendung. Eisen aus den Erzseen, weist zudem die besondere Eigenschaft auf, dass sie mit magischem Summen interagieren können und daher zur Destillierung von Mana eine essenzielle Rolle spielen.
\item[Gold]Glänzend und Wertvoll ist es bei allen Völkern sehr begehrt. 
\item[Silber]Geschätzt von Alchemisten, aber auch Gläubigen und Zauberern aus verschiedensten Gründen, ist es nahezu gleichwertig mit Gold und Edelsteinen. Ursprünglich war Silber dabei nur eine Schöpfung der Götter, um ihren Einfluss in der sterblichen Welt besser zu kanalisieren. Die daraus resultierenden Sagen und Legenden färbten dann auch auf die übrigen Gebiete ab. Jeder Silberader wird einer eigenen Art zugeordnet und in der Regel gelten sie untereinander als unvereinbar, siehe Viersielber.    
\end{description}
\section{Göttliche Erze}
Das einzige, was bis heute einen eindeutigen Beweis auf das ursprüngliche Pantheon liefert, sind diese aus dem Herz der Zwerge stammenden Erzbrocken, deren übernatürliche Eigenschaften sie auszeichnen.
\begin{description}
\item[Artheum]In seiner Reinform ist Artheum weicher als Gold und bildet merkwürdige, verästelte Strukturen, die sich auch im flüssigen Zustand nicht auflösen, sondern ein merkwürdiges Eigenleben entwickeln. Dieser Zustand vergeht jedoch schlagartig, wenn Artheum mit einem Mundanem Erz in Berührung kommt. Dabei ist irrelevant in welchem Zustand sich letzteres befand. Es verflüssigt sich automatisch und geht eine Legierung mit Artheum ein, die daraufhin erst bei Temperaturen nahe dem Gefrierpunkt aus kristallisiert. Eine kristallisierte Legierung aus Artheum weist dabei folgende Eigenschaften auf: Erst bei Temperaturen, die denen von Drachenfeuer entsprechen, geht die Legierung wieder in ihren Flüssigen Zustand über. Zusätzlich weist die Oberfläche dieser Legierung alle nicht gleichen Legierungen ab und wird sich daher niemals mit diesen homogen vermengen können, dies hat zur Folge, dass Verunreinigung im Erz beim Vermengen mit Artheum das Metallobjekt brüchig wie Glas machen können, weshalb sich nur unter Zuhilfenahme von Magie oder Götterkraft beständige Objekte aus Artheum schaffen lassen. Reine Legierungen sind hingegen unzerbrechlich und unterscheiden sich ansonsten nur im Aussehen. Der Panzer von Zwergen besteht aus diesem Metall, weshalb sie nahezu unzerstörbar sind.
\item[Nihilim]Ein Metall, dessen ganze Natur die Verzweiflung wiederspiegelt, mit der die Götter gegen die Drachen anzukämpfen versuchten. Nihilim wurde am Rande des Weltenspaltes geschaffen und trägt in sich den Keim der Antiemagie. In seiner Reinform ist dieser jedoch nicht bedrohlich für den Erzsee, sondern wird erst durch einen aufwendigen Raffinerieprozess, bei dem Drachenblut eine entscheidene Rolle spielt und damals aus den wenigen von Götterhand selbst erschlagenden Drachen stammt, sein wahres Potenzial entfalten und kann bis zu einem gewissem Grad jede Form von Magie absorbieren. Für eine Anomalie und andere von einem Weltenbauer erschaffene System, wie göttliches Wirken, ist Nihilim selbst in großen Mengen immer noch ungefährlich, jedoch ist es vor allem für die Drachen, die mehr auf einen Dominoeffekt, als pure Stärke setzten, nahezu unmöglich im Umkreis von wenigen Metern einen Zauber zustande zu bringen. Allerdings ist dieser eigentlich effektive Schutz ein zweischneidiges Schwert, da der Effekt, wenn sich Zauber in einer gewissen Grauzone um Nihilim befinden nur partiell negiert werden können und der anschließende Backslash weitaus verheerender sein, als beabsichtigt. Die wenigen Stücke Nihilim die ihren Weg in die Hände Sterblicher gefunden haben, werden entweder in Rüstungen oder Waffen verbaut und haben einen legendären Ruf. Vor allem im Angesicht dunkler Zaubermächte ist es oft eine dieser Waffen mit denen ein tödlicher Schlag erst möglich wurde.
\item[Senterra]
\end{description}
\section{Metalle}
\begin{description}
\item[Stahl]Härter und widerstandsfähiger als gewöhnliches Eisen verliert es alle magischen Eigenschaften, die dem Eisen inne gewohnt haben mag. Vor allem in weiter entwickelten Zivilisationen ersetzt Stahl, das einfache Eisen in allen Anwendungsgebieten.
\item[Bronze]Weniger als Material für zweckmäßige Objekte im Einsatz, wird dieses Metall eher für Verzierungen verwendet.
\item[Viersilber]Normalerweise gilt Silber aus verschiedenen Adern als Unvermengbar, zumindest wenn es um die Reinheit und Potenz und damit Wert des Gemisches geht. Eine Ausnahme hierbei gilt das beinahe legendäre Viersilber. Sein Ursprung liegt bei einer kleinen Gruppierung von Handwerkern, die wohl mit Akmol in Verbindung stehen. Vielsilber ist in jeder Hinsicht reiner als jede seiner Komponenten für sich genommen. Außerdem ist es widerstandsfähiger und läuft nicht mehr an, womit es jeder Laie identifizieren kann. Diese Eigenschaften machen es auch geeignet für die Herstellung von Waffen und Rüstungsteilen, auch wenn in der Regel nie genug Viersilber hergestellt wird, um eine ganze Rüstung zu schmieden.
\end{description}


\chapter{Maße und Einheiten}
Sonnezyklus=1 Jahr
\\1 Sonnenumlauf=1 Tag
\\1 Mondzyklus=1 Monat
\\1 Meile=1 Kilometer
\\1 Fuß =1 Meter

\chapter{Ring der Tränen}
Ein uraltes Verbrecherkartell, welches sein Unwesen an der goldenen Küste, welches unter der Schirmherrschaft der Familie Rhone steht. Sie organisieren viele der dort ansässigen Söldnerbanden, haben diverse Drogenküchen und stehen auch teilweise mit dem Hinterland im Kontakt, falls dieses besonders delikate Waren anfordert. Hinter dieser Kulisse steckt ein Kult, der durch Vampire kontrolliert wird?!

\chapter{Hexen}
Die Schwesternschaft der Hexen war ursprünglich nicht mehr als ein Bund der Dorfältesten und Heiler, die ihre Geheimnisse über Natur und Mensch teilten. Über die Jahre verfeinerten sie ihre Kenntnisse und verfestigten das Fundament ihrer Gemeinschaft durch Regeln und Traditionen. Viele Jahre brachte die Schwesternschaft damit Eintracht und Sicherheit für ihre Dörfer. Doch mit der Zeit der großen Kriege sollte die Schwesternschaft wie so viele Dinge ins Unglück stürzen. Angesichts der Grausamkeit mit der man aufeinander losging kam ein Teil der Schwesternschaft zu dem Schluss, dass, um eine solche Tragödie in Zukunft zu verhindern, man die großen Reiche zerschlagen und zu dem einfachen Leben in kleinen Dorfgemeinschaften zurückkehren sollte. Dieser Zirkel berief sich auf die Magie und schaffte es trotz des Interdikts eine eigene Magie zu schaffen, die Kunst des Verfluchens. Es war dieser Akt, der das Schisma bringen sollte. Fortan existierten zwei Hexenzirkel, die sich aus dem Weg gingen und ihre eigenen Geheimnisse und Techniken entwickelte. Die Spaltung wurde über die Jahre immer tiefer und riss auch die zwei großen Zirkel auseinander und inzwischen finden sich auf beiden Seite Aufopferung und Selbstsucht.   
\section{Kesselhexen}
Dieser Zirkel ist seinen Traditionen treu geblieben und hat sich natürlicher Kräuterkunde und der Kunst der richtigen Worte gewidmet. Sie selbst sehen sich in der Regel als Begleiter des einfachen Volkes und Diener für die Gemeinschaft. Dennoch kommen sie nicht umhin für ihren Dienst einen gewissen Respekt einzufordern, wozu sie im Laufe der Zeit den Weg der Seelen entworfen haben. Er gibt ihnen Macht über Tier und Mensch nach dem Prinzip, Glaube versetzt Berge. Ein einfaches Gift, dass Kribbeln verursacht und ein paar Worte und ihrem Gegenüber wird der Körper versagen.
\section{Ritualhexen}
Im Gegensatz zu den Kesselhexen, denen jede Frau beitreten kann, ist man Ritualhexe vom Blute her. Das sog. Hexenblut, dass das Manablut der Ritualhexen ist wird von Mutter an Tochter weitergeben. Im Gegensatz zu Kesselhexen benötigt ein Hexenblut keine Ausbildung in ihren Gaben, dafür gilt ihre Macht als wesentlich unberechenbarer und fordern immer ihren Tribut. Diese Kräfte einer Hexe erwachen mit ihrer Weiblichkeit und gewähren ihnen, die Fähigkeit ihren Wünschen Wirklichkeit zu verleihen. Doch jedem dieser Wünsche, selbst dem mit Guter Absicht, ist es vorherbestimmt ihre Opfer ins Verderben zu stürzen und sie schlussendlich all ihrer Lebenskraft berauben. Doch jeder dieser sog. Flüche kann gebrochen werden, da ihre Macht immer an einen Makel in der Welt gebunden sind. Übersteigt ein Fluch die Kraft einer Hexe muss sie den ultimativen Preis in Form ihres eigenen Lebens bezahlen.    

\chapter{Däumling}
Klein wie eine einfache Schachfigur werden sie in der Regel leicht übersehen und noch viel leichter unterschätzt. Doch trotz oder vielleicht gerade wegen dieser Unzulänglichkeiten stehen sie mit den übrigen Völkern auf gleicher Augenhöhe, was man ihrer Kreativität, Fingerfertigkeit und Fähigkeit zur Zusammenarbeit zurechnet. Naturgemäß neigen Däumlinge dazu ihre Probleme durch selbstgebaute Hilfsmittel auf ein für sie Überschaubares Maß herunter zu brechen, auch wenn es oft eine ganze Sippschaft erfordert selbige zu bedienen. Diese familiären Verbünde stammen aus den Anfängen dieses relativ jungen Volkes, in denen man sich organisieren musste, um die Herausforderungen als potenzielle Beute für größere Tiere zu meistern und sich einen neuen Platz in der Welt zu erarbeiten. Däumlinge sind in ihrer Natur Miniaturversionen der übrigen Völker, Nachfahren von Stadtbewohnern die Opfer einer Manaeruption waren. Neben ihrer Größe ist die auffälligste Veränderung, dass Artdifferenzen ausgeräumt wurden und sich starke soziale Bande entwickeln. Auf letztere wird auch äußerster Wert gelegt, was sich vor allem in ihren strengen Traditionen widerspiegelt.

\chapter{Anomalien des Manas}
Manablut ist eine seltsame Angelegenheit, da ihre Realität nach ihrer Erschaffung durch den jeweiligen Zirkel nicht mehr erforscht wurde und es im Zuge von Manaeruptionen genauso anfällig für spontane Veränderungen, wie alles andere war. Meist sind diese Veränderungen tödlich oder so geringfügig, dass sie nicht als solche erkannt werden und dann gibt es jene Phänomene, die man als Krankheiten oder Flüche klassifiziert und bekanntermaßen nur jene mit Manablut infizieren können.
\section{Drittes Auge}
Diese willkürlich bei der Geburt oder Initiation als Manablut auftretende Anomalie sorgt dafür, dass dieses beim Zugriff auf seine magischen Reserven immer von Visionen heimgesucht werden. Über die Art dieser lässt sich nur munkeln. Manch einer sagt, es offenbare sich die Wahre und Vollständige Natur der Existenz, während andere von der Konfrontation mit Alptraumgestalten sprechen. Was auch immer sich im Geist der Unglückseligen abspielt, wenn sie zu lange im Kontakt mit ihrem Mana bleiben, bleibt von ihnen nur noch eine Seelenlose Hülle zurück, die zwar noch atmet, aber ansonsten keine Notiz von ihrer Umwelt mehr nimmt. Geraten diese Hüllen in Gefahr entfesseln sie reflexartig gewaltige Zauberkraft und entledigen sich aller Sorgen, ohne den Auswirkungen ihrer Zauber Beachtung zu schenken. Eine gewisse Zeit beschäftigten sich auch die Drachen mit dem Phänomen, der sog. Geläuterten, mussten aber feststellen, dass der Kontakt mit dem Geist den Zustand auf den Beobachter überträgt, worauf man auf weitere Versuche verzichtete.
\section{Blaues Fieber}
Diese durch einen mutierten Wurm verursachte Krankheit macht aus einem Magier kurzzeitig ein Wesen ähnlichen den Drachen oder einem Mönch, indem dieser die gesamten Manareserven miteinander verknüpft, wodurch Zauber mit dem Gesamten Körper gewirkt werden. Der Preis für diese Macht ist ein unzähmbarer Hunger, der bei Missachtung zum Tod führt, ausgelöst durch das Fressen des Wurm, welcher die so aufgenommene Materie in Mana umwandelt. Übertragen kann sich der Wurm durch die Eier, welche er in das Blut seines Wirtes pumpt und die über den Tod des Wirtes hinaus im Körper warten, bis sie in Kontakt mit neuem Manablut kommen, um dort zu schlüpfen. In der Regel versucht man die Ausbreitung dieses Wurmes einzudämmen, jedoch gibt es auch Zirkel und Sekten, die sich seine besonderen Eigenschaften zu nutze machen, um kurzzeitig ihre Kräfte zu steigern. Geschlüpfte Manawürmer können nur durch eine Kältekur behandelt werden, bei dem die Körpertemperatur des Wirts weit genug gesenkt werden muss. Die Eier sterben bei diesem Prozess zwar nicht und können sich später auf neue Wirte ausbreiten. Seinen Namen hat es aufgrund des blauen Leuchtens, welche das vom Wurm erzeugte Mana abgibt und selbst im Sonnenlicht unter der Haut zu erkennen ist.
\section{Freies Blut}
Ein Phänomen, welches seit den hellen Zeitaltern immer wieder Magier aus aller Welt verblüfft hat, sind jene Individuen, welche ohne entsprechende Abstammung oder Ritual von Geburt an Mana im Blut tragen.  
\\Darüber hinaus jedoch nicht genug. Zum Schrecken der Drachen gelten für ein freies Blut nicht die Regel des Interdikt, weshalb sie in der Lage sind, frei zu zaubern. Auf der anderen Seite, und das ist auch der Grund weshalb die Drachen derart Gesegnete nicht kurz nach ihrer Geburt bereits vernichtet, kommt ihr Blut in der Regel zusammen mit einer geistigen Beeinträchtigung, weshalb ihr Repotier an Zaubern auf wenige verschiedene Anwendung und auch in ihrer Häufigkeit eingeschränkt ist. Außerdem können andere diese Zauber nicht nachmachen, wodurch der Schaden beschränkt bleibt.
\\Im Regelfall ist jedoch jedes hunderte Freiblut mental komplett gesund, wenn nicht sogar etwas überbegabt. Hält es sich bedeckt so droht auch diesem keine Gefahr, andernfalls räumen die Drachen und die Elfen auf und schaffen bei dieser Gelegenheit auch Zauber früher Freiblüter aus dem Weg. Ein solches Ereignis trägt unter den Elfen die Bezeichnung Blutwinter, da bei dieser Jagd leicht ein kämpferischer Konflikt zwischen dem Freiblut und Elfen entstehen kann, der teilweise sogar nachhaltig die Beziehungen mit anderen Völkern stören kann.  

\chapter{Nirgendwo}
Als das Interdikt der Drachen erlassen wurde und man in der Welt mit den alten Zaubern aufräumte kam der berühmte Zeitenbrecher zum Einsatz. Ein Zauber, der im Gegensatz zu den üblichen Machwerken der Drachen, einen entscheidenden Fehler aufwies. Man stahl Aurum Orbis einen Moment, in welchem fortan die alten Zauber weiterexistieren und aus welchem sich eine alternative Zeitlinie entwickelte. Hier gab es kein Interdikt und keine Drachen und Elfen, womit dieses Sibro Murua seine Untergang entgegensteuerte und schließlich unterging und zerbarst. Die Splitter bohrten sich auf einer Seite in Aurum Orbis, wo sich an manchen Stellen natürliche Tore in dieses Scherbenmeer entstanden, aber auch zum Leidwesen aller in das Equilibrium. Jeder auch die Drachen wurden wurden von dieser Entwicklung überrascht und es stellte sich heraus, dass diese Entwicklung unmöglich umzukehren war, weshalb man zunächst angesichts der Bedrohung die von manchen Geistern die aus dem Equilibirum ihren Weg nach Aurum Orbis suchten, ebenfalls eine Apokalypse befürchtete.

\section{Seelenläufer}
Geboren in Aurum Orbis zieht es einige in das für Lebewesen eigentlich feindliche Nirgendwo. Als Gründe nennt jeder Seelenläufer etwas anderes, auch wenn die vorherrschende Meinung ist, dass es der Wunsch der ruhelosen Seelen nach Erlösung ist, welcher diese Entwicklung antreibt. Im Gegensatz zu Gewöhnlichen kann ein Seelenläufer sich selbst eine Passage ins Nirgendwo und heraus öffnen, anstatt nur durch zufällig verteilte Passagen zu stolpern. Dazu kommen noch eine ganze Reihe weitere Fähigkeiten, die ihnen ermöglichen sich relativ gefahrlos im Nirgendwo zu bewegen. Zusätzlich eignet sich jeder Seelenläufer über die Zeit neue Möglichkeiten Kraft aus dem Nirgendwo, den Geistern und was sich sonst noch dort bewegt. Zu Bemerken sei noch, dass es unter den Seelenläufer auch Vertreter von Spezies, sogar Pflanzen, befinden, die eigentlich kein höheres Bewusstsein besitzen und schon gar nicht die Fähigkeit mit anderen zu kommunizieren. Beides kommt jedoch innerhalb kürzester Zeit mit ihrem ersten Besuch im Nirgendwo. Angesichts von größeren Bedrohungen stehen sich die Seelenläufer ungewöhnlich nahe, auch wenn es drei große Strömungen gibt.
\subsection{Schamanen}
Immer wieder gibt es Seelen die an ihrer Existenz hängen, doch die Isolation des Equilibrium hat früher oder später dennoch ihnen Ruhe geschenkt, doch durch die Brücke, welche das Nirgendwo bildet, versuchen sie nun ihren Weg nach Aurum Orbis zu finden, manche Getrieben von einer unerledigten Angelegenheit, andere können sich einfach noch nicht mit dem Tod abfinden. Die Schamanen sehen es als ihre Pflicht an, diesen Geistern Ruhe zu schenken. Dazu können sie rudimentär mit einem Geist kommunizieren, indem sie in den stärksten Erinnerungen wühlen können. Alternativ sind sie auch in der Lage die Verbindung des Geistes zur sterblichen Welt materialisieren, was den Geist nach zerstören seiner physischen Form zurück ins Equilibrium zieht. Manchmal kann sich ein Geist auch dafür entscheiden seine letzten Momente vor ihrem Vergehen aus Dankbarkeit bei ihrem Retter zu bleiben und ihm mit ihren ungewöhnlichen Kräften zu dienen. Allerdings soll es auch Schamanen geben, die eine befreite Seele für sich selbst beanspruchen, um ihre Macht mit Gewalt an sich reißen. 

\section{Trinität des Bösen}
Im Namen der Namenlosen, bei den Mächten der Machtlosen und im Gedenken an die Vergessenen, seid verdammt und bezahlt, Verräter, Lügner und die, die euren Taten nacheifern. (Klassische Anrufungsformel)
\\An erste Stelle standen jene, deren Erfindungen und Ideen von ihnen Gestohlen und unter anderem Namen anderen zu Ruhm und Macht brachten. Ihr Wirken beeinflusst bis heute die Welt auch wenn niemand ihren Namen kannte. Ihnen folgten die größten Herrscher ihrer Zeit, sowohl die Weisen, wie die Grausamen, die durch Verrat, List und Betrug ihres Thrones und der Früchte ihrer Harten Arbeit beraubt wurden. Und ihnen folgen jene denen ihr Gedenken verwehrt wurde, die verleugnet und ignoriert werden, trotz ihrer Größe. Sie alle wandten sich in den Letzten Stunden von jedem Rest Mitgefühl oder Reue ab und beschlossen, sich alles von der Welt zurückzuholen, was ihnen genommen wurde, ungeachtet des Preises den die Welt für ihre Taten zahlen würde.
\\Im Gegensatz zu gewöhnlichen Seelen fanden diese wenigen nie Frieden im Equilibrium und im Laufe der Zeit schlossen sich die Seelen der drei Gruppen zu eigenen Entitäten des Bösen zusammen. Sie standen zwischen der Existenz als sterblichem Wesen und den allmächtigen Göttern und als das Equilibrium sich öffnete schlossen sich die drei Entitäten zur Trinität des Bösen zusammen, um Gemeinsam Rache an der Welt zu nehmen. Jede Entität überragt an Grausamkeit und Macht eine Hundertschaft an Wiedergängern. Sie vermochten fremde Körper, ja sogar gezielt unbelebte Materie zu übernehmen und das zu einem Maße, dass sie selbst von den Seelenläufern und deren Kräften niemals vollständig ins Equilibrium gebannt werden konnten.
\\Jene die ihre Namen rufen und die Begegnung überleben treten als sog. Avatare gegen die Welt an, ausgestattet mit der Befehlsgewalt über eine Hundertschaft an Geistern, denen sie sich nach ihrem Tod anschließen, wartend auf den nächsten, der sie ruft oder zu den Unglücklichen zählt, die durch Kontakt mit einem von einer Entität verfluchten Gegenstand besessen wird. Und schließlich existieren Legenden über einen Avatar, der alle 100 Generationen entsteht, welcher die gesamte Trinität des Bösen unter sich vereint und der nur in einem Bündnis aller Völker und Götter, wie zu den Gegenweltkriegen, gestoppt werden kann.

\chapter{Untergründige}
Bei der Erschaffung der drei Menschen schenkte Ak-Mol jedem der Stämme eine besondere Affinität zu einem Aspekt von Handwerksarbeit, allerdings unter der Auflage, dass sie ihre Gaben nicht gegen ihr eigenes Blut zu verwenden hätten. In den ersten Kriegen verloren daher viele dieses Geschenk und ihrer Nachkommen sollten nie in den Genuss dieser Gaben kommen. Es war der westliche Teil der Sommerfeldischen Front, die mit Eintreffen der Nachricht vom Kommen der Zwerge den Weg unter die Erde suchten, mit der sie sich aufgrund ihrer Affinität besonders verbunden fühlten. Dank der Gnade Ak-Mols, der dieses unschuldige Menschengeschlecht weiter bestrafen wollte als es durch die Zwerge sowieso schon war, fanden sie Zuflucht und später eine Neue Heimat tief unter der Erde. Über die Generationen erschufen sie ihren eigenen Kult um Mutter Erde, repräsentiert durch Ak-Mol und trugen mit ihren Legenden über die Schrecken der Oberwelt zur vollständigen Isolation vom Rest der Welt. Die Gegenweltkriege, in denen die Götter ihr Wirken beschränken musste, stellte für sie eine Harte Prüfung ihres Glaubens und Überlebenwillens dar, nach deren Bestehen Ak-Mol diesen Menschen das Geschenk einer seiner geheiligten Werkstätte machte. Hier entstanden heilige Artefakte, die seitdem das Fortbestehen der Untergründigen garantieren würden. Am bemerkenswerten war dabei wohl der Stab der elementaren Erde, mit dessen Macht die Wächter in Form der Goblins usw. erschaffen wurde. 

\cleardoublepage
\phantomsection
\addcontentsline{toc}{part}{Stichwortverzeichnis}
\printindex[Stichworte]
\phantomsection
\addcontentsline{toc}{part}{Geographieglossar}
\printindex[Geographie]


\end{document}
