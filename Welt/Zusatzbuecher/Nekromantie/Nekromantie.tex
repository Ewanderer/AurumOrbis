\documentclass[a4paper,12pt,oneside]{book}
\usepackage[ngerman]{babel}
\usepackage[utf8]{inputenc}
\usepackage{imakeidx}
\usepackage[hypertexnames=false]{hyperref}
\usepackage[all]{hypcap}
\usepackage{nameref}
\usepackage{ulem}


\hypersetup{
	bookmarks=true,
    colorlinks,
    citecolor=black,
    filecolor=black,
    linkcolor=black,
    urlcolor=black
}

\title{Nekromantie}
\author{Jordan Eichner}
\date{}
\setcounter{secnumdepth}{-2}
\setcounter{tocdepth}{1}

\begin{document}

\maketitle
\tableofcontents
\subsubsection{Nekromantie}\label{Nekromantie}\index[Stichworte]{Nekromantie}
Charon selbst kann nicht durch Verehrung an Macht gewinnen und es gibt keine Charon-Geweihten, stattdessen stellt Charon eine Reihe von Ritualen zur Verfügung und die Ausführenden dieser werden allgemein hin Nekromanten genannt. Zu Beginn seiner Karriere beginnt ein jeder Nekromant mit einem Ritual der Opferung. Anschließend kann er in einer Reihe anderer Rituale totem Fleisch ein neues Leben einhauchen.
\paragraph{Rituale}
\begin{description}
\item[Ritual der Opferung]\label{Nekromantie:Ritual der Opferung}\index[Stichworte]{Nekromantie!Ritual der Opferung}
\begin{itemize}
\item Vorbereitung auf das Ritual, der Seelenstein
\\Um die Seele aus dem Körper zu ziehen und sie für die Weiterverarbeitung zu konservieren, benötigt es einen zur Seele affinen Seelenstein. Dabei vermengt man 5 Unzen Blut vom Opfer oder eines anderen Angehörigen seines Volkes mit einer Unze Materials, welches dem Schöpfer des Opfers geweiht ist. Einfaches Weihwasser ist dabei schon ausreichend. Die Mischung muss dann bei Mondlicht, welches Brücke für Charon zwischen Irdischer Welt und seinem Equilibrium ist, zusammen mit einem Metall, für ungefähr eine Münze, in einen Schmelztiegel gegeben werden. Das geschmolzene Metall wird daraufhin in einem Wasserbecken ausgekühlt. Der Klumpen wird anschließend noch mit einer dünnen Schicht Blut vom Beschwörer bestrichen, während er einen Treueschwur zu Charon leistet. Dies verhindert, dass andere Götter sich an der Energie aus der Seele des Opfers vergreifen können. 
\item Beschaffung der anderen Utensilien:
\\Für das Ritual werden außerdem noch folgende Dinge benötigt: Ein Pinsel, ein Messer oder vergleichbare Klinge, einen Eimer mit Wasser und einen Lappen, sowie Nadel und Faden. Außerdem sollte eine Esse mit Amboss und Hammer bereit stehen.
\item Das Opfer wird zunächst mit ausgestreckten Gliedmaßen auf dem Rücken fixiert. Ein Knebel oder eine Betäubung werden empfohlen, um das Opfer am Schreien zu hindern, allerdings ist dies nicht notwendig.
\item Anschließend muss aus dem Blut des Opfers ein Kreis gezogen werden. Ein zweiter Kreis sollte um Esse und Amboss gezogen werden, falls man diese nicht in den ersten Kreis mit einschließen kann. Sobald das Blut vollständig getrocknet ist und keine Lücken  mehr nachgezogen werden müssen, kann mit dem zweiten Schritt angefangen werden.
\item Mit dem Messer wird nun der Brustkorb aufgeschnitten und der Seelenstein auf das Herz gelegt, alles bis auf das Herz selbst kann dabei ruhig zerstört werden. Anschließend muss man den Brustkorb wieder verschließen, indem man das Fleisch mit Nadel und Faden zusammenflickt. Dieser Schritt muss beendet werden, solange das Opfer noch lebt!
\item Nun muss man ohne den Kreis zu verlassen auf den Tod des Opfers warten, sollte man zu gut gearbeitet haben, darf Gift gespritzt werden oder das Opfer erwürgt werden, allerdings sollte möglichst wenig Blut vergossen werden und das wenige darf unter keinen Umständen den Kreis durchbrechen.
\item Sobald das Opfer verstorben ist kann der Brustkorb wieder geöffnet werden und der Seelenstein entnommen werden.
\item Anschließend muss der Seelenstein komplett vom Blut des Opfers befreit werden und erneut mit dem Blut des Nekromanten bestrichen werden, der Treueschwur an Charon sollte wiederholt werden. Unter Nekromanten ist man sich über diesen Schritt nicht ganz einig, aber doppelt hält besser.
\item Anschließend muss der Seelenstein in der Esse zum Glühen gebracht werden.
\item Der Stein muss anschließend mit einem Hammer zerteilt werden, um die Seele restlos zu spalten, sodass sie in ihre Essenz zerfällt.
\item Die einzelnen Bruchstücke müssen nun abkühlen, bevor sie aus dem Blutkreis entfernt werden dürfen.
\item Von diesem Punkt kann der Nekromant die einzelnen Bruchstücke(im folgendem als Seelensplitter bezeichnet) zum Bezahlen der Seelenkosten für andere Nekromantie-Rituale verwenden.  
\end{itemize}
\item[Ruf des Ewigen Herzens] Eine der häufigsten Gründe, weshalb Menschen mit dem Studium und anschließenden Ausführung der Nekormantie beginnen ist, der Wunsch einen verstorbenen Geliebten wieder zu sehen. Ein wie sich heraus stellen sollen unmögliches Unterfangen, dennoch gefiel Charon die Assoziation, die der Name Ewiges Herz\label{Ewiges Herz}\index[Stichworte]{Ewiges Herz}, bei solchen hervorrief. Im Grunde handelt es sich nur um leicht abgewandelte göttliche Essenz, mit dem Ziel totes Fleisch einen Willen aufzuzwingen und es dabei zu animieren. Der Gebrauch des Herzen beschränkt sich dabei echt auf totes Fleisch und für Golems oder Konstrukte sind andere Energiequellen von Nöten.
Anbei eine kurze Beschreibung über dieses Ritual:
\begin{itemize}
\item Das Ritual muss in einer Vollmondnacht im Mondlicht ausgeführt werden, da dieses eine Brücke zu Charon in die irdische Welt darstellt.
\item Zu Beginn stellt der Nekromant eine silberne Platte an einen Ort, wo er das Mondlicht einfängt.
\item Anschließend muss mit Blut ein Kreis auf den Rand gezeichnet werden, wobei man ein kurzes Gebet an Charon spricht.
\item Nun müssen je nach gewünschter Stärke des später wiedererweckten Fleisches  Seelensplitter auf der Silberplatte angehäuft werden.
\item Anschließend müssen die Worte ''Höre mich an Charon, sieh mein Opfer!'' gesprochen werden. Was folgt ist eine kurzfristige Materialisierung von Charon beim Nekromanten und begrüßt diesen mit den Worten ''Dein Opfer wurde akzeptiert, was ist dein Begehr?'', worauf der Nekromant sein Anliegen darlegen muss. Wenn es innerhalb der Möglichkeiten eines ewigen Herzens liegt wird Charon dieses anbieten. Nimmt der Nekromant an wird Charon fortfahren, ansonsten wird er einfach verschwinden.
\item Falls angenommen werden im folgendem Dialog mit dem Gott, sofern erforderlich weitere Rituale zu Nutzung einen ewigen Herzes erörtert, bevor der Gott die Seelensplitter entgegen nimmt und stattdessen auf dem Silberteller eine schwarze kristalline Kugel zurück lässt, ein ewiges Herz.
\end{itemize}
\item[Erschaffung eines Zombies] Das einfachste und doch vielseitigste Ritual ist das Reanimieren toten Fleisches. Häufig wird mit dieser Wiedererweckung auch die Rückkehr der Seele verbunden, doch diese ist für immer verloren, stattdessen wird das Tote Fleisch durch einen Willen, entweder in Form einer einzelnen Anweisung oder durch dauerhafte Bindung an ein Bewusstsein, in Bewegung gesetzt. Anstatt einer Leiche kann ein Nekromant auch eigene Schöpfungen aus zusammengenähten Fleisch in Bewegung setzten. Ein Zombie bleibt bestehen, bis ihm entweder das ewige Herz herausgeschnitten wird oder dessen Energie verbraucht ist. Der Körper eines Zombies, einschließlich Knochen trägt in sich einen Fluch, der wenn ein Teil des Zombies in das Herz eines lebenden Wesens, mit einer göttlichen Seelen, gelangt, diese beginnt aufzulösen, wodurch der Infizierte stirbt und in einer spontanen Wiedererweckung, durch die Freisetzung der Seelenenergie als Zombie, nur mit dem Auftrag den Fluch weiterzutragen, aufersteht. Das Ritual zur Erweckung besteht eigentlich nur aus einem Schritt: In das Fleisch muss das ewige Herz eingenäht werden und anschließend der Befehl gesprochen werden oder alternativ ein Tropfen Blut zur Bindung an den Besitzer auf den Körper geträufelt werden, wobei der zukünftige Besitzer die Worte ''Unterwirf dich mir!'' spricht.
\item[Erweckung als Ghul] Niemals kann eine Seele nach dem Tod in den Körper zurückkehren, weshalb man niemals mittels Nekromantie einen geliebten Menschen von den Toten auferstehen lassen kann. Auf der anderen Seite ist ein ewiges Herz aber das perfekte Behältnis für eine Seele, ein Umstand den es seinem göttlichem Ursprung zu verdanken hat. Ein williges Opfer kann also seine Seele in ein solches Herz sperren und so bis zum Verbrauch des Herzen in diesem weiterexistieren und von dort aus die Gewalt über totes Fleisch übernehmen. Bei einem Ritual nicht unähnlich dem der Opferung, nur dass man ein ewiges Herz in sein eigenes rammt und dann auf den Tod wartet, wonach man direkt als Ghul erwacht. In seiner neuen Form lebt der Nekromant von Seelensplittern mit denen er sein ewiges Herz befüllen kann, verfügt aber ansonsten über die gleichen Eigenschaften wie ein Zombie.
\end{description}
\subparagraph{Nekropole}\label{Nekropole}
Während viele Nekromanten auf der Suche nach Perfektion ihrer Untoten Schöpfungen oder weil sie die Verfolgung der Gesellschaft fürchten in die Wildnis zurück ziehen, entstanden unter verschiedensten Umständen in direkter Nähe zu den Lebenden Zentren der Nekromantie. Diese Orte werden als Nekropolen genannt. Hier kommen Ghule, Nekromanten und andere Anhänger Cahrons zusammen, handeln mit Wissen über Rituale, Konservierungsformen und Dienstleistungen für Außenstehende gegen Gold, auch wenn Hauptzahlungsmittel innerhalb einer Nekropole Seelensplitter sind. Wie bereits erwähnt haben Nekropolen immer einen sehr speziellen Existenzgrund, angefangen bei einem Pseudo-Kult, der eine ganze Stadt in seinen Fängen hält, bis hin zu einem stillschweigendem Abkommen. Zwar gehören\uline{\hyperref[Vampir]{Vampire}} zu den Untoten, aber werden in einer Nekropole nicht geduldet. Vampire haben daher die Angewohnheit Nekropolen zu meiden, weshalb diese neben Sonnenlicht die effektivste Abwehrmethode gegen Vampire ist, was für einige ein guter Grund ist den Nekromanten Tür und Tor zu öffnen.



\end{document}