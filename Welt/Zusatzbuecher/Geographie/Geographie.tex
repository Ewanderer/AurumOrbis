\documentclass[a4paper,12pt,oneside]{book}
\usepackage[ngerman]{babel}
\usepackage[utf8]{inputenc}
\usepackage{imakeidx}
\usepackage[hypertexnames=false]{hyperref}
\usepackage[all]{hypcap}
\usepackage{nameref}
\usepackage{ulem}


\hypersetup{
	bookmarks=true,
    colorlinks,
    citecolor=black,
    filecolor=black,
    linkcolor=black,
    urlcolor=black
}

\title{Geographie}
\author{Jordan Eichner}
\date{}
\setcounter{secnumdepth}{-2}
\setcounter{tocdepth}{1}

\begin{document}

\maketitle
\tableofcontents

\part{Sechsstein}

\chapter{Sommerfeld}

\section{Baronie Lethe}

\subsection{Region Ghuile}

\subsubsection{Stadt Ghuile}
\begin{description}
\item[Beschreibung:]Die Stadt wurde um eine Erhebung errichtet, auf welcher sich das Fort Demur, welches als Rastplatz diente, befindet. Das umliegende Gebiet ist relativ flach und Baumfrei. 
\item[Aufbau:]Die Stadt ist in wesentlich 3 Teile aufgeteilt, wobei die Grenzen durch die Straßen aus der Stadt bestimmt sind. \\Im Osten sind die Marktviertel, die drei kleine Märkte beherbergen. Dazwischen stehen Lagerhäuse, einige Ställe, sowie eine ganze Reihe von Herbergen. Am Aufgang zum Fort steht außerdem die Zollbehörde, welche im ehemaligen Rathaus untergebracht ist. Die wenigen Wohnhäuser gehören meist den stationären Händlern. 
\\Im Nordwesten haben sich die zugezogenen Handwerker und anderen Stadtbewohner niedergelassen. Es gibt hier ebenfalls einen Marktplatz, der allerdings mehr für lokale Produkte genutzt wird. Tavernen ohne Schlafmöglichkeit sind hier die Regel, wenn es auch ein gut besuchtes Bordell gibt. Es gibt keine Slums, auch wenn sich vor den Stadtmauern eine kleinere Zeltstadt, siehe unten, befindet.
\\Der neuste Zuwachs zur Stadt bilden im Südwesten die Quartiere der Stadträte aus den anderen Teilen der ehemaligen Baronie. Dabei hat jeder Teil sein eigenes Repräsentantenhaus, welches inmitten von gewöhnlichen Reihenhäusern für die Bediensteten steht, im Zentrum befindet sich das Parlament und der Richtplatz.
\\Wenn sie auch nicht zur eigentlichen Stadt gehört, so hat doch Ghuile, als Stadt des Aufbruchs für viele Abenteurer und andere Elemente angezogen. Aus Platzgründen und der Natur dieses Menschenschlages hat sich vor dem Bürgerviertel ein sich immer in Veränderung befindendes Zeltlager aufgetan, wo spezialisierte Händler und Handwerker, sowie Söldnergruppen ihre Dienste anbieten. Das Zentrum des ganzen Treibens bildet die nachträglich dort errichtete Taverne ''Heldenhumpen'', auch wenn es bei weitem nicht das einzige Etablissement dort ist, dafür jedoch das sauberste. 
\item[Architektur:]Die Gebäude sind sehr schlichte Fachwerkhäuser, wenn man vom Fort, dem Parlament und einigen Repräsentantenhäusern absieht, die alle ihren eigenen Stil haben.
\item[Besondere Gebäude:]Die Stadt verfügt im Marktviertel über einen Dinaris Tempel, sowie einen Kurier.  
\\Fort Demur - Ehemals eine Festung des Aigis, als die goldene Straße noch in ihrer Anfangsphase war. Seit jenen ersten Tagen befand sich das Fort in der Hand der Herzogs, bis es die Stadtverwaltung in die Hände einer Gilde übergeben hat, welche seitdem dort residiert und über die Stadt wacht.
\\Das Parlament - Dieser runde Komplex aus hellem Stein und einer beeindruckenden Kuppel, in deren Innern, wie zum Hohn ein Schrein zu Umbra liegt, gilt als das politische Zentrum der Stadt wenn nicht der ganzen Baronie, da hier nicht nur der Stadtrat seine Tagungen abhält, sondern auch die Versammlung aller freien Stadtstaaten.
\\Die Sternwarte - Die Akademie zu Kaltweihe hat es sich erlaubt sich als Residenz eine Sternwarte zu errichten. Der untere Teil dieses massiven Gebäudes hat eine eigene Ratskammer, da die Akademie eine ganze Delegation zu ihrer Vertretung unterhält.
\\Die Herzogliche Residenz - Als ein Andenken an alte Zeiten steht am Rand des Hügels mit eigenen Gärten. Da die Vertreter von Ghuile eigene Häuser in der Stadt haben, hat das Gebäude lang leer gestanden, bis es für gesellschaftliche Zusammenkünfte der reicheren Bevölkerung(Im Ballsaal), sowie Theater(im Garten) neu genutzt wurde.
\end{description}

\subsection{Freital}
Erst nach dem Fall der Baronie wurde diese Region der Küstenregion hinzugefügt. Bekannt ist das größtenteils auf den ausufernden Gebirgshängen erbaute Reich ist bekannt für seine dichten Wälder, die mit ihren Wurzeln die ansonsten von Erdrutschen und Gerölllawinen bedrohten Erdschichten schützen. Dieser Schutz kommt jedoch mit einem Preis, den die Wälder bieten unzähligen Räuberbanden und anderen unerwünschten Elementen eine Heimat nahe der goldenen Straße. Der einzige Grund weswegen diese Region überhaupt an den Zöllen beteiligt wird, ist die Tatsache, dass die Hauptstadt Carnes einen Großteil ihrer Zolleinnahmen auf die Säuberung der Wälder von den schlimmsten Kriminellen einsetzt. Da es immer etwas zu tun gibt und die Wälder mit unzähligen Wildtieren gefüllt sind, hat Freital eine ähnliche Ausstrahlung auf junge Abenteurer und Söldner, wie die Stadt Ghuile.

\subsubsection{Carnes}
\begin{description}
\item[Beschreibung:]Die kleine Stadt Carnes ist auf dem Pass zum Mondtal, wo sie auf das restliche Freital herunter blicken kann. Als nächste Stadt zum Hinterland wurde sie im Laufe der Zeit etwas befestigt für den seltenen Fall einer Invasion aus dieser Richtung.
\item[Aufbau:]Aus der Luft kann man leicht den ehemaligen Stadtkern mit hoher Mauer, einer kleinen Garnison des Aigis, sowie einem dutzend Fachwerkhäusern, in denen sich in diesen Tagen die Oberschicht aufhält, wenn man von wenigen Tavernen für Durchreisende und dem Richtplatz vor der Garnision, der jedoch mehr zu einem Markt verkommen ist, absieht.
\\Im Laufe der Zeit hat sich außerdem in Richtung des übrigen Freitals die Stadt ausgedehnt, was den Reisenden, sowie auch den Söldnern geschuldet ist, die von hier aus ihre Operationen im restlichen Freital koordinieren. Hier gibt es weitere Märkte, das eine oder andere Haus einer kleinen Gilde, sowie einige Herbergen, Ställe und Handwerker.
\item[Architektur:]Da die Winter relativ frisch werden können auf dem Pass sind alle Häuser sehr solide und mit dicken Mauern gebaut. Eigentlich alle Gebäude im Stadtkern verfügen über die sog. Doppelwand, einer Bautechnik, wie sie eigentlich nur im deutlich kälteren Frostzahn üblich ist. Als Material für die Fassaden wird eigentlich hauptsächlich Stein verwendet.
\item[Besondere Gebäude:]Trotz seiner Zentralen Lage und der geringen Größe verfügt die Stadt über einen voll gerüsteten Tempel des Aigis in Form der Garnison. Grund hierfür sind die bereits erwähnten Banditen, gegen die Aigis die Söldner organisiert, da sie selbst nicht genug Kapazitäten für diese Aufgabe bereitstellen können.
\end{description}

\section{Hinterland}

\subsection{Das Mondtal}

\subsubsection{Öllepo}
\begin{description}
\item[Beschreibung:]Im Zuge des Auflebens des Alten Pfades, der Alternativroute zur goldenen Straße, errichtete man hier am Mondfluss für Zeiten in denen Dunkelheit über dem restlichen Tal liegt eine Festung der Zivilisation. Auch wenn natürlich unter dem Druck der Herrscher des Mondtals die Bewohner nicht völlig deren Geboten und Veränderungen widerstehen können, so ist die Stadt seit jeher in der Hand vieler Banden, deren ständige Kämpfe es einem Herrscher nahezu unmöglich machen absolute Herrschaft über diese Stadt zu erlangen, weshalb durchreisende Händler relativ unbehelligt bleiben können.
\item[Aufbau:]Die Stadt selbst scheint von oben ein Gewirr ohne feste Struktur in Form von physischen Abtrennung zu sein. Wer sich lange genug umhört wird jedoch feststellen, dass eine Zahl von Banden die Stadt unter sich aufgeteilt hat, auch wenn die genauen Grenzen, sowie die Anzahl der Gebiete stets zu schwanken scheint, zwischen einem und zwei dutzend.
\item[Architektur:]Ohne eine federführende Hand sind die Straßen mehr oder weniger verschlungen. Die Gebäude stehen teilweise krumm und schief und scheinen um Platz zu konkurrieren. Als Material dient unter anderem Holz, Lehmziegel und bei den größeren Häusern sogar Stein. Aufgrund ständiger Rivalitäten und äußerer Einflüsse gibt es immer wieder Ruinen, in verschiedenen Stadien des Wiederaufbaus. Die Stadt selbst wird von einer Mauer begrenzt, ohne sich jemals irgendwelche Vorstädte gebildet haben.
\item[Besondere Gebäude:]Die Stadt selbst weist keine Bauwerke auf, die auf den ersten Blick herausstechen würden, das Religion, Politik und andere ein Stadtbild prägende Parteien so häufig wechseln. Jedoch wurden im Laufe der Zeit unzählige Tunnel im felsigen Fundament der Stadt gegraben, die inzwischen teils Kanalisation, teils Territorium für die Banden sind. Womit die Stadt sich zumindest nicht so sehr um Seuchen kümmern muss. 
\end{description}


\end{document}