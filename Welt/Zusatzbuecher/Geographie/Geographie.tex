\documentclass[a4paper,12pt,oneside]{book}
\usepackage[ngerman]{babel}
\usepackage[utf8]{inputenc}
\usepackage{imakeidx}
\usepackage[hypertexnames=false]{hyperref}
\usepackage[all]{hypcap}
\usepackage{nameref}
\usepackage{ulem}


\hypersetup{
	bookmarks=true,
    colorlinks,
    citecolor=black,
    filecolor=black,
    linkcolor=black,
    urlcolor=black
}

\title{Geographie}
\author{Jordan Eichner}
\date{}
\setcounter{secnumdepth}{1}
\setcounter{tocdepth}{10}

\begin{document}

\maketitle
\tableofcontents

\part{Sechsstein}

\chapter{Feuerland}

\section{Schwarze Küste}
Lange Zeit der unbewohnte Landstrich an der südlichen Küste. Grund hierfür ist ein fast durchgehender Ascheregen aus den nordwestlichen Vulkanen. Nach der Übernahme des Erzsees von Lethe begannen diejenigen, welche keinen Platz unter den Tunnelratten fanden, hier sich eine Gesellschaft nach dem Prinzip des Stärkeren zu errichten. 
\\Während Land- und Forstwirtschaft aufgrund des Klimas nahezu keine Erträge bringen, müssen sie sich ihre Vorräte anderweitig beschaffen. Da kam der rege Warenaustausch nördlicher Städte mit Mephus durch das Labyrinth aus Tunneln und die diversen Karawanen in der Perlwüste gerade Recht.
\\Fortan waren sie eine Plage, die sich allerdings dank ihrer blutigen und rücksichtslosen Strategien, sowie der Tatsache, dass sie selbst in dutzende, wenn nicht hunderte kleine Organisationen gespalten sind, aller Versuche, sie zu beseitigen, widersetzen konnte.

\subsection{Schädelinsel}
Vormahls ein Teil des Feuerland näherte sich diese Landmasse über die Jahre der Perlwüste an. Ihren Namen hat sie sowohl aufgrund ihrer Form, welcher der eines 

\subsection{Rauchendes Archipel}


\section{Unterland}

\subsection{Die ehemalige Schnelltrasse}
Der Weg von Mephus nach Ammat ist lang und die Metalltransporte sind seit jeher beliebtes Ziel von Räubern, da man in den labyrinthartigen Stollen seinen Verfolgern stets entkommen kann. Schließlich wurde es Lethe zu bunt und sie gruben einen direkten Schacht, in welchen man Schienen legte. Betrieben mit Kristallen fuhren die Züge mit unglaublichen Geschwindigkeiten. Doch man hatte die Hartnäckigkeit der kriminellen unterschätzt, die begannen die Strecken zu beschädigen, um die Züge zum entgleisen zu bringen. Nachdem man sechs mal den Tunnel aufbereitet hatte und mit immer neuen Sicherheitsvorkehrungen gesichert hatte, war es schließlich Lethe zu dumm und sie stellten das Projekt ein, worauf man wieder zu den langsamen, schwer bewachten Karawanen zurück kehrte. Die Überreste des Tunnels werden daher immer noch benutzt und die ehemaligen Wartungsstationen wurden zu Rastplätzen oder kleinen Garnisonen umfunktioniert.

\subsection{Stadt Ammat}
\begin{description}
\item[Beschreibung:]Die Stadt stellt das Bindeglied zwischen der tiefergelegenden Stadt Mephus und der Oberfläche dar. Dazu hat sich im Laufe der Zeit ein reger Bergwerksbetrieb entwickelt. Die Stadt und ihre Bewohner kommen dabei immer wieder in Konflikt mit den Elfen, wenn es um den Ausbau der Mienen geht.
\item[Aufbau:]Im Tal der Langezogenen Kaverne liegen die Wohnviertel für die einfachen Arbeiter. Im Norden beginnt die künstliche Einhöhlung, in welcher das Schienennetz für die Mienen, sowie die Hütten befinden. Im Westen befinden sich erhöht, die Handwerksstätten, über denen ein ewiger Nebel liegt. In einer südlichen Einbuchtung, so hoch gelegen, dass sich hier die Dämpfe der Stadt sammeln und abregnen, wurde eine Pilzfarm errichtet, die mit Spiegelsystem Licht von der Oberfläche erhält. Schließlich befinden sich am Aufgang im Westen und beim Tunnel nach Mephus jeweils eine Festung.
\item[Architektur:]Während alles im Bereich der Mienen, sowie die beiden Festungen, künstlich aus dem Fels gehauen wurde, sind die restlichen Gebäude aus Abraum und Restmetall als Mörtel errichtet. Da die Luft mit zunehmender Höhe immer feuchter, aber auch dreckiger wird, pressen sich alle Gebäude an den Boden.
\item[Besondere Gebäude:] Die Nebelfarmen im Süden gilt als die tiefste, nicht mit Magie belichtete Farm, auch wenn die unglücklichen Bauern es aufgrund der schlechten Luft als Hölle bezeichnen.
\end{description}

\subsection{Der Erzsee}
Der einzige Grund, weshalb das Gebirge der Feuerlande mit Blut und Schweiß erschlossen wurde, ist der gewaltige Lavasee, welcher an der Grenze zum brennenden Land und tief unter dem Gebirge liegt. Der nördliche Kontinent, sowie die ätherischen Splitter beziehen fast alle einfachen Metalle aus den unerschöpflichen Tiefen dieser Anomalie. An einem Ausläufer im Südosten befand sich daher seit jeher eine gewaltige Maschinenstadt, Mephus.
\subsubsection{Kolonie der Baronie Lethe}
\begin{description}
\item[Beschreibung:]Mit ihrem technologischen Aufstieg, nahm der Hunger der Baronie Lethe nach Metall so zu, dass sie sich nicht länger auf die Bewohner Mephus für billigen Nachschub verlassen wollte. So marschierte man ein, riss in einem blutigen Krieg am nördlichen Ufer die Stadt nieder und errichte auf den Ruinen eine große befestigte Kolonie.
\item[Aufbau:]Hinter einem großen Doppelwall befinden sich im Norden die Wohnviertel, wo die Arbeiter untergebracht sind.
\\Ein Großteil des Südens ist hingegen der gesamten Metallindustrie und den dazugehörigen Werkstätten gewidmet, wobei erstere durch einen weiteren Wall gesichert werden.
\\Im Osten wacht ein Fort über die Gebiete in Richtung des großen Erzsees. In seinem Schatten befinden sich die wenigen edleren Quartiere für Verwalter und Magier, welche bei der Erzschöpfung helfen.
\\Der Westen befindet sich ein zweites Fort in dessen Schatten sich die Reste des Verladebereichs Richtung Ammat befinden.
\\Darunter schließlich eine Reihe von Baracken, in denen verzweifelte auf den Schutz vor Ungeheuern aus den Erzseen durch die Soldaten hoffen.
\item[Architektur:]Man merkt sofort, dass die gesamte Kolonie auf dem Reißbrett geplant wurde. Alle Straßen sind gepflastert und gerade. Die Häuser aus besonderen Felsblöcken, um den Bewohnern das Kühlen zu erleichtern, sind alle gerade und glatt geschliffen. Lediglich bei den Fabrikhallen und massiven Wällen wurde anstatt auf einen polierten Stil zu setzten, mehr die Funktionalität und Beständigkeit in den Vordergrund gestellt.
\end{description}

\subsubsection{Altstadt}
\begin{description}
\item[Beschreibung:]Nach dem Krieg mit Lethe haben sich die Überlebenden in den südlichen Teil zurück gezogen und betreiben dort ebenfalls Erzförderung. Ohne ihren größten Abnehmer und das Aufkommen diverser Gefahren befindet sich das gesamte Gebiet in einer Abwärtsspirale, weshalb weite Teile der ehemaligen Arbeiter sich den Räubern und Schmugglern anschließen.
\item[Aufbau:]
\end{description}

\chapter{Sommerfeld}
Geteilt in die drei Baronien und das Hinterland, steht es dank der großen Handelsstraße und Felder, bildet diese Region für viele den Mittelpunkt der Zivilisation. Die Bezeichnung der Regionen als Baronien, kommt aus der Geschichte des Landes, als noch die Magister regierten. Zwar sind sie gestürzt, aber die neuen Herrscher mochten den Titel und so wurde er wieder eingerichtet, als sich nach dem Wiederaufbau die drei großen Herrscherhäuser Lethe, Rhone und Hortens hervortaten. Durch den warmen Strom vor der Küste, welcher seinen Ursprung im heißen Feuerland hat, herrscht hier immer ein relativ warmes Klima, wenn man von den südlichen Zipfeln zum Frostzahngebierge absieht. 

\section{Baronie Lethe}

\subsection{Stadt Maa}
\begin{description}
\item[Beschreibung:]
\item[Aufbau:]Im Stadtkern befindet sich die beeindruckende Luftschiffwerft mit ihren Riesigen Hallen, deren Dächer nach oben geöffnete werden, sobald ein neues Schiff fertiggestellt wurde. Das Straßennetz ist hier mehr ein Irrgarten aus Gerüsten, Aufgängen, Brücken und Leitern, der sich zwischen den Außenmauern und den Hallen aufspannt. Im östlichen Bereich befindet sich die Garnison, die für den Schutz der Wert und Schiffe zuständig ist.
\\Darum befindet sich von einer zweiten Außenmauer die restliche Stadt eingefasst, die in 4 Viertel eingeteilt ist, welche durch kleinere Mauern abgetrennt werden:
\\Im Norden befinden sich die Werkshallen, wo größere Bauteile vorgefertigt werden, sowie massenhaft Standardbauteile, wie Schrauben und Nägel. Die Lagerhäuser sind ebenfalls hier untergebracht.
\\Im Osten, wohin die Fabrikwinde nicht hin wehen, liegen die Wohnviertel der Arbeiter. Die besonders besessenen Bürger haben hier zusätzlich private Werkstätten, ansonsten finden sich hier die Herbergen für Reisende.
\\Im Süden befinden sich die wesentlich spezialisierteren Werkstätten, der Handwerker, wo kleine besondere Komponente für Luftschiffe gefertigt werden. Es gibt vereinzelte Marktplätze, wo man kleinere Produktionen verkauft.
\\Die Ausbildung neuer Ingenieure, sowie die Grundausbildung der Bürger wird im Westen auf dem Campus der Akademie zu Maa ausgeführt. Neben den Lehrgebäuden hat sich hier eine eigene Stadt mit Tavernen, Wohnhäusern, sogar einigen frei zugänglichen Werkstätten gebildet, da der Zugang zur Restlichen Stadt und auf den Campus zurück streng überwacht wird aus Angst vor Spionen. Zutritt haben somit nur Studenten, Bedienstete, sowie Professoren und natürlich das Militär.
\item[Architektur:]Wo früher eine einfache Stadt aus Stein stand, sind mit der technischen Revolution, alle Gebäude und Mauern in die Höhe gewachsen. 10 Stockwerke, sind nicht mehr die Ausnahme sondern Regel. Damit nicht die ganze Stadt auf den Straßen feststeckt, existieren 2 zusätzliche Ebenen, aus Brücken und Gerüsten. Sämtliche Dächer in den Wohnviertel sind teil eines großen Gartenkomplexes, in dem man perfekt ausspannen kann.
\end{description}

\section{Baronie Hortens}

\subsection{Region Ghuile}

\subsubsection{Stadt Ghuile}
\begin{description}
\item[Beschreibung:]Die Stadt wurde um eine Erhebung errichtet, auf welcher sich das Fort Demur, welches als Rastplatz diente, befindet. Das umliegende Gebiet ist relativ flach und Baumfrei. 
\item[Aufbau:]Die Stadt ist in wesentlich 3 Teile aufgeteilt, wobei die Grenzen durch die Straßen aus der Stadt bestimmt sind. \\Im Osten sind die Marktviertel, die drei kleine Märkte beherbergen. Dazwischen stehen Lagerhäuser, einige Ställe, sowie eine ganze Reihe von Herbergen. Am Aufgang zum Fort steht außerdem die Zollbehörde, welche im ehemaligen Rathaus untergebracht ist. Die wenigen Wohnhäuser gehören meist den stationären Händlern. 
\\Im Nordwesten haben sich die zugezogenen Handwerker und anderen Stadtbewohner niedergelassen. Es gibt hier ebenfalls einen Marktplatz, der allerdings mehr für lokale Produkte genutzt wird. Tavernen ohne Schlafmöglichkeit sind hier die Regel, wenn es auch ein gut besuchtes Bordell gibt. Es gibt keine Slums, auch wenn sich vor den Stadtmauern eine kleinere Zeltstadt, siehe unten, befindet.
\\Der neuste Zuwachs zur Stadt bilden im Südwesten die Quartiere der Stadträte aus den anderen Teilen der ehemaligen Baronie. Dabei hat jeder Teil sein eigenes Repräsentantenhaus, welches inmitten von gewöhnlichen Reihenhäusern für die Bediensteten steht, im Zentrum befindet sich das Parlament und der Richtplatz.
\\Wenn sie auch nicht zur eigentlichen Stadt gehört, so hat doch Ghuile, als Stadt des Aufbruchs für viele Abenteurer und andere Elemente angezogen. Aus Platzgründen und der Natur dieses Menschenschlages hat sich vor dem Bürgerviertel ein sich immer in Veränderung befindendes Zeltlager aufgetan, wo spezialisierte Händler und Handwerker, sowie Söldnertruppen ihre Dienste anbieten. Das Zentrum des ganzen Treibens bildet die nachträglich dort errichtete Taverne ''Heldenhumpen'', auch wenn es bei weitem nicht das einzige Etablissement dort ist, dafür jedoch das sauberste. 
\item[Architektur:]Die Gebäude sind sehr schlichte Fachwerkhäuser, wenn man vom Fort, dem Parlament und einigen Repräsentantenhäusern absieht, die alle ihren eigenen Stil haben.
\item[Besondere Gebäude:]Die Stadt verfügt im Marktviertel über einen Dinaris Tempel, sowie einen Kurier.  
\\Fort Demur - Ehemals eine Festung des Aigis, als die goldene Straße noch in ihrer Anfangsphase war. Seit jenen ersten Tagen befand sich das Fort in der Hand der Herzogs, bis es die Stadtverwaltung in die Hände einer Gilde übergeben hat, welche seitdem dort residiert und über die Stadt wacht.
\\Das Parlament - Dieser runde Komplex aus hellem Stein und einer beeindruckenden Kuppel, in deren Innern, wie zum Hohn ein Schrein zu Umbra liegt, gilt als das politische Zentrum der Stadt wenn nicht der ganzen Baronie, da hier nicht nur der Stadtrat seine Tagungen abhält, sondern auch die Versammlung aller freien Stadtstaaten.
\\Die Sternwarte - Die Akademie zu Kaltweihe hat es sich erlaubt sich als Residenz eine Sternwarte zu errichten. Der untere Teil dieses massiven Gebäudes hat eine eigene Ratskammer, da die Akademie eine ganze Delegation zu ihrer Vertretung unterhält.
\\Die Herzogliche Residenz - Als ein Andenken an alte Zeiten steht am Rand des Hügels mit eigenen Gärten die ehemalige Sommerresidenz der alten Herzogsfamilie. Da die Vertreter von Ghuile eigene Häuser in der Stadt haben, hat das Gebäude lang leer gestanden, bis es für gesellschaftliche Zusammenkünfte der reicheren Bevölkerung(Im Ballsaal), sowie Theater(im Garten) neu genutzt wurde.
\end{description}

\subsection{Region Kaltweihe}

\subsubsection{Akademie zu Kaltweihe}
Begründet durch einen Sprössling aus dem Haus Hortens, welcher sich für die Natur begeisterte, ist die Akademie das Zentrum für Naturwissenschaften. Sie sagen mit Stolz, jede Pflanze und jedes Tier in der Region katalogisiert zu haben und inzwischen auf eine ganze Reihe nicht-magischer Erfindungen blicken. Mit Magie hat das Institut aber wenig am Hut, lediglich Alchemisten werden dort als Nebenwissenschaft gelehrt. Die Akademie ist keine Zentrale Einheit, sondern hat neben ihren an Städte gebundenen Instituten(Tier- und Pflanzenkunde, Medizin, Alchemie und Mineralogie, Geschichte), sehr viele kleine Außenstellen in Form von Gehöften, die einem Professor unterstellt sind und auf welchen er im stillen an seinen persönlichen Projekten arbeiten kann.

\subsubsection{Stadt Stork}
\begin{description}
\item[Beschreibung:]Hafen von Kaltweihe, welcher in einer Bucht mit spitzer Landzunge im Süden errichtet wurde.
\item[Aufbau:]Im Wesentlichen ist die Stadt in Hafen, Campus und Wohnviertel unterteilt. Die Trennung findet dabei durch einen Deich, zwischen Hafen und restlicher Stadt statt. Der Campus liegt auf einer Anhöhe, um auch im Falle eines Deichbruches erhalten zu bleiben.
\item[Architektur:]Holzbauten mit Doppelwänden. Straßen gepflastert, damit sie auch bei Regen passierbar bleiben.
\item[Besondere Gebäude:]Der ganze stolz der Stadt bildet ist der erste gasbetriebene Leuchtturm auf der Welt, dessen Linsen und sonstige Bauweise aus der Akademie stammen.
\\Der bereits erwähnte Campus gehört zum Institut für Medizin, welches im wesentlichen ein großes Krankenhaus ist, dessen Patienten von der Seeluft profitieren. In der Regel werden nur gut betuchte oder ungewöhnliche Patienten aufgenommen.
\end{description}

\subsubsection{Stadt Blyte}
\begin{description}
\item[Beschreibung:]Errichtet um den Blumenberg, welcher ehemals eine Residenz von Hortens war und seit jeher für ihre Gärten bekannt ist. Hier wurde damals die Akademie begründet und es ist seit jeher das Zentrum ihrer Verwaltung, neben der Funktion als Institut für Tier- und Pflanzenkunde.
\item[Aufbau:]Das Zentrum bildet der Blumenberg, auf welchem sich der Campus befindet. Es gibt eine kleine Anzahl von zusätzlichen Gebäuden für Dinge, wie Archive oder Ausstellungen im Norden davon. Die übrige Stadt streckt sich jedoch entlang der Hauptstraße, von der goldenen Straße zu Küste nach Stork.
\item[Architektur:]Die Akademie und ihre Gebäude sind meisterliche Steinbauten. Der Rest sind allesamt weiße Fachwerkhäuser.
\end{description}

\subsection{Freital}
Erst nach dem Fall der Baronie wurde diese Region der Küstenregion hinzugefügt. Bekannt ist das größtenteils auf den ausufernden Gebirgshängen erbaute Reich ist bekannt für seine dichten Wälder, die mit ihren Wurzeln die ansonsten von Erdrutschen und Gerölllawinen bedrohten Erdschichten schützen. Dieser Schutz kommt jedoch mit einem Preis, den die Wälder bieten unzähligen Räuberbanden und anderen unerwünschten Elementen eine Heimat nahe der goldenen Straße. Der einzige Grund weswegen diese Region überhaupt an den Zöllen beteiligt wird, ist die Tatsache, dass die Hauptstadt Carnes einen Großteil ihrer Zolleinnahmen auf die Säuberung der Wälder von den schlimmsten Kriminellen einsetzt. Da es immer etwas zu tun gibt und die Wälder mit unzähligen Wildtieren gefüllt sind, hat Freital eine ähnliche Ausstrahlung auf junge Abenteurer und Söldner, wie die Stadt Ghuile.

\subsubsection{Carnes}
\begin{description}
\item[Beschreibung:]Die kleine Stadt Carnes ist auf dem Pass zum Mondtal, wo sie auf das restliche Freital herunter blicken kann. Als nächste Stadt zum Hinterland wurde sie im Laufe der Zeit etwas befestigt für den seltenen Fall einer Invasion aus dieser Richtung.
\item[Aufbau:]Aus der Luft kann man leicht den ehemaligen Stadtkern mit hoher Mauer, einer kleinen Garnison des Aigis, sowie einem dutzend Fachwerkhäusern, in denen sich in diesen Tagen die Oberschicht aufhält, wenn man von wenigen Tavernen für Durchreisende und dem Richtplatz vor der Garnision, der jedoch mehr zu einem Markt verkommen ist, absieht.
\\Im Laufe der Zeit hat sich außerdem in Richtung des übrigen Freitals die Stadt ausgedehnt, was den Reisenden, sowie auch den Söldnern geschuldet ist, die von hier aus ihre Operationen im restlichen Freital koordinieren. Hier gibt es weitere Märkte, das eine oder andere Haus einer kleinen Gilde, sowie einige Herbergen, Ställe und Handwerker.
\item[Architektur:]Da die Winter relativ frisch werden können auf dem Pass sind alle Häuser sehr solide und mit dicken Mauern gebaut. Eigentlich alle Gebäude im Stadtkern verfügen über die sog. Doppelwand, einer Bautechnik, wie sie eigentlich nur im deutlich kälteren Frostzahn üblich ist. Als Material für die Fassaden wird eigentlich hauptsächlich Stein verwendet.
\item[Besondere Gebäude:]Trotz seiner Zentralen Lage und der geringen Größe verfügt die Stadt über einen voll gerüsteten Tempel des Aigis in Form der Garnison. Grund hierfür sind die bereits erwähnten Banditen, gegen die Aigis die Söldner organisiert, da sie selbst nicht genug Kapazitäten für diese Aufgabe bereitstellen können.
\end{description}

\chapter{Hinterland}
Eigentlich noch ein Teil von Sommerfeld, wird es aufgrund seiner geographischen Isolierung, häufig aus dem Gedächtnis der übrigen Bewohner verdrängt. 

\section{Das Mondtal}
Bezeichnet nach dem klaren Mondfluss, über dessen Quell in der Regel der Mond aufgeht, bevor er später an seiner Mündung wieder untergeht und welcher das Tal in der Mitte teilt. Als Durchgangsstrecke für den alten Pfad, ist es noch am besten besucht. Die Forste aus Freital habe hier ebenfalls ihre Forstsetzung und Zusammen mit dem Fischfang und ihrer Landwirtschaft, sind die Bewohner relativ autonom.

\subsection{Pavells Rast}
Auf dem Pass zu den grauen Marschen steht seit jeher ein altes Kloster, dessen Kapelle auf den Kavernen gebaut wurde, wo der Held Pavell nach seiner Niederlage begraben wurde. Man nennt ihn heute einen Helden, weil keiner weiß, dass er damals eine Sklavenarmee gegen die Grauen Marschen zu führen suchte. Heutzutage ist es allerdings mehr ein Rastplatz für Karawanen auf dem alten Pfad, als für Pilger. Auch wenn Pavell schon seit Jahrzehnten Tod ist, so scheint ein unheimlicher Fluch über dem Ort zu liegen, denn er zieht oft Nekromanten und andere düstere Gestalten an, die sich anschließend der Herrschaft über das Mondtal zuwenden.

\subsection{Öllepo}
\begin{description}
\item[Beschreibung:]Im Zuge des Auflebens des Alten Pfades, der Alternativroute zur goldenen Straße, errichtete man hier am Mondfluss für Zeiten in denen Dunkelheit über dem restlichen Tal liegt eine Festung der Zivilisation. Auch wenn natürlich unter dem Druck der Herrscher des Mondtals die Bewohner nicht völlig deren Geboten und Veränderungen widerstehen können, so ist die Stadt seit jeher in der Hand vieler Banden, deren ständige Kämpfe es einem Herrscher nahezu unmöglich machen absolute Herrschaft über diese Stadt zu erlangen, weshalb durchreisende Händler relativ unbehelligt bleiben können.
\item[Aufbau:]Die Stadt selbst scheint von oben ein Gewirr ohne feste Struktur in Form von physischen Abtrennung zu sein. Wer sich lange genug umhört wird jedoch feststellen, dass eine Zahl von Banden die Stadt unter sich aufgeteilt hat, auch wenn die genauen Grenzen, sowie die Anzahl der Gebiete stets zu schwanken scheint, zwischen ein oder zwei dutzend.
\item[Architektur:]Ohne eine federführende Hand sind die Straßen mehr oder weniger verschlungen. Die Gebäude stehen teilweise krumm und schief und scheinen, um den wenigen Platz zu konkurrieren. Als Material dient unter anderem Holz, Lehmziegel und bei den größeren Häusern sogar Stein. Aufgrund ständiger Rivalitäten und äußerer Einflüsse gibt es immer wieder Ruinen, in verschiedenen Stadien des Wiederaufbaus. Die Stadt selbst wird von einer Mauer begrenzt, ohne dass sich jemals irgendwelche Vorstädte gebildet haben.
\item[Besondere Gebäude:]Die Stadt selbst weist keine Bauwerke auf, die auf den ersten Blick herausstechen würden, das Religion, Politik und andere ein Stadtbild prägende Parteien so häufig wechseln. Jedoch wurden im Laufe der Zeit unzählige Tunnel im felsigen Fundament der Stadt gegraben, die inzwischen teils Kanalisation, teils Territorium für die Banden sind. Womit die Stadt sich zumindest nicht so sehr um Seuchen kümmern muss. 
\end{description}

\chapter{Frostzahn}

\section{Küste der Magister}
Die Küstengebiet von Frostzahn werden durch die Überreste des zweiten großen Magisterimperiums dominiert. Während es letzteres anderswo aufgrund der Kluft zwischen den Magiern und der versklavten Bevölkerung restlos zusammenbrach, als die Magie den Herrschern den Dienst versagte, waren die Herrscherhäuser des unwirtlichen Südens, entweder Verlierer der großen Ränkespiele, die ebenfalls ein schnelles Ende fanden, oder aber die wenigen Individuen, die aus verschiedenen Gründen der blutigen Praxis anderer Magister abgeschworen haben und auch der Magie nur wenig Beachtung schenkten. Da sie stattdessen mit dem Volk zusammenarbeiteten, statt es auszubeuten, war ihr Ruf wesentlich besser und man überließ ihnen freiwillig die Aufgabe, wieder für Ordnung und den Wiederaufbau zu sorgen. Als hier die Magie versagte, ging als eines der wenigen Gebäude, das große fliegende Archiv verlorenen, während andere Gebäude aus Mangel an magischen Reserven echte Handwerksarbeiten waren und damit bestand hatten.

\subsection{Region Tess}

\subsubsection{Festung Nimmermorgen}

\subsection{Region Stahlheim}

\subsubsection{Akademie zu Stahlheim}

\subsection{Region Arthas}

\subsubsection{Akademie zu Arthas}

\subsection{Nutzloser Platzhalter}

\subsection{Region Schwarzmarschen}

\subsubsection{Stadt Aschenfeld}

\end{document}