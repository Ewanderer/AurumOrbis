\documentclass[a4paper,12pt,oneside]{book}
\usepackage[ngerman]{babel}
\usepackage[utf8]{inputenc}
\usepackage{imakeidx}
\usepackage[hypertexnames=false]{hyperref}
\usepackage[all]{hypcap}
\usepackage{nameref}
\usepackage{ulem}


\hypersetup{
	bookmarks=true,
    colorlinks,
    citecolor=black,
    filecolor=black,
    linkcolor=black,
    urlcolor=black
}

\title{Aurum Orbis - Arkane Künste}
\author{Jordan Eichner}
\date{}
\setcounter{secnumdepth}{-1}
\setcounter{tocdepth}{10}

\begin{document}

\maketitle
\tableofcontents

\part{Übersicht}
\section{Einführung}
Die Arkanen Künste gelten weithin als das Werkzeug aller großen Magien innerhalb der sterblichen Welt. Ihr Zentrum bilden die sog. Runen, deren Sprache und Grammatik zur vollständigen Beschreibung aller bekannten Vorgänge und Objekte und damit Replikationen ersterer gemacht werden können. Im Laufe der Zeit kamen immer wieder weitere Anwendungen hinzu.

\section{Geschichte}


\part{Anwendungen}

\chapter{Zauber}
Durch die Aussprache eines arkanen Ausdrucks, die Beisteuerung von Gesten, das Bereitlegen von materiellen Komponenten oder Foki und abschließend die Entrichtung der Manakosten, kann ein magischer Effekt produziert werden. Die Möglichkeiten dieser Effekte sind eigentlich nur durch das Verständnis der Runensprache, einigen wenigen Ultimativen Grenzen, sowie der Kreativität des Zauberers begrenzt. Größte Schwäche bei Zaubern ist, dass die Ausdrücke sehr schnell, sehr lang werden und damit im spontanem Kampf nur wenig Verwendung haben. Man unterscheidet zwischen zwei Arten von Zauberwirkern:
\\Diejenigen, welche die an Akademien oder aus Büchern gelernten Zaubern benutzen und lediglich minimal Anpassen und
anderen, die jeden ihrer Zauber handfertigen.

\chapter{Verzauberung}
Bereits in den ersten Tagen der Arkanen Künste konnten bestimmte Zaubereffekte, wie eine Lichtkugel mit Energie angereichert werden, wodurch diese auch ohne die Konzentration des Magiers aufrecht erhalten wurden. Darüber hinaus bestand natürlich das Interesse für verschiedene Anwendungen, wie den Kampf oder die Produktion von Gebrauchsgegenständen, Objekte zu erschaffen, die innerhalb kürzester Zeit ihre Magie hervorbringen konnten. Die einfachsten Verzauberungen erlauben es hierbei, dass bei Zufuhr von Mana durch einen Magier oder eine bei der erschaffende Manaquelle sich ein Zauber aktiviert. Nach einer gewissen Nutzungszeit wird die Energie, welche den Zauber zusammenhält, verbraucht, woraufhin das Objekt erneut verzaubert werden muss.
\\Spezialisten auf diesem Gebiet haben allerdings ebenfalls Wege gefunden, solche Verzauberungen selbsterhaltend zu machen oder allgemein die Lebensdauer dieser Objekte zu verlängern, auch wenn der Prozess aufwendig und damit kostspielig ist.


\chapter{Beschwörung}
Die Schule der Beschwörer ist ursprünglich aus den Arkanen Künsten hervorgegangen, auch wenn beide klar voneinander getrennt sind. Eine der wenigen Disziplinen, welche ebenfalls einem Träger der arkanen Künste möglich ist, ist die Anrufung eines Elementars, um es für eine kurze Zeit in seine Dienste zu bringen. Der Vorgang ist jedoch wesentlich komplexer und gefährlicher und die Ergebnisse im Vergleich zu einem echten Beschwörer ernüchternd.

\chapter{Sonstiges}


\part{Die Runensprache}

\chapter{Der magische Ausdruck}

\chapter{Vokabular}

\section{Nomen}
Worte, bzw. Ausdrücke sind in ihrer Komplexität in der Regel in folgende Stufen eingeteilt.
\begin{description}
\item[1 Elementare und Anorganische Materialien]
\item[3 Organische Materialien, Gegenstandklasse(z.B. Schwerter)]
\item[7 Geistige Attribute, Volk/Rasse, spezifisches Objekt*]
\item[15 Einzelnes Individuum*]
\end{description}
*Natürlich sind diese Ausdrücke nicht irgendwo geschrieben, sondern müssen aufwendig erforscht werden, wenn dies überhaupt möglich ist.

\section{Verben}
\begin{description}
\item[Bewegen]Die Basisoperation, mit welcher erfasste Ziele bewegt werden können. Alternativ kann eine Richtung für einen selbsttätigen Vorgang eingefügt werden. Kann einzeln verwendet werden.
\item[Erschaffen]Kann nur mit Elementaren und Anorganischer Materie verwendet werden, wobei letztere nach Ablauf der Dauer, verfallen. In keinem Fall lassen sich die erschaffenen Materialien für andere Zwecke, als die Umformung zu Objekten verwendet werden(also keine Extraktion von alchemischen Komponenten etc.).
\item[Umformen]Kann Elemente in ihr Gegenteil(2:1) oder ein anderes Element(4:1) überführt werden. Dieser Transfer ist dauerhaft. Anorganische Materie kann in ihrer Gestalt angepasst werden mit Beachtung zum Masseerhaltungssatz. Dieser Transfer wird nach Ablauf der Zauberdauer umgekehrt.
\end{description}

\section{Formworte}
Umfassen die einfache geometrischen Gestalten(Kreis, Kugel, Rechteck) und den Richtungsvektor. Letztere können verwendet werden, um beispielsweise Spiralen aus einem Würfel zu machen oder eine Pfad zu erstellen. 

\section{Regulatoren}
Elementare können mittels dieser in ihrer Intensität reguliert werden. Es gibt 4 Wörter, mit denen 5 Stufen erzielt werden können.

\chapter{Grammatik}

\part{Binden von Elementaren}

\part{Institutionen}

\chapter{Die Akademie zu Arthas}

\section{Geschichte}
Nach der Offenlegung der Archive in Eisheim, der Entdeckung der Runensprache, errichtete König Arthas dort einen Ort des Lernes und der Forschung. Aufgrund ihres exklusiven Zugriffes auf umfassende Informationen zur Runensprache, hat sie seitdem eine Vormachtstellung auf diesem Gebiet. Durch die strenge und lang widrige Ausbildung von Moral und Begabung ist die Anzahl der Absolventen gering, dafür sehr begehrt. Die Schule der Beschwörung und auch Alchemie wurden beide durch von hier aus begründet und alle Techniken und Erweiterungen der Arkanen Künste über die Jahre gesammelt und dem Lehrplan hinzugefügt.

\section{Tradition}
Gemäß dem Erlass von Arthas sollten Magier ihre Kräfte zu schätzen wissen und stets darauf bedacht sein, ihren Mitmenschen keinen unnötigen Schaden zuzufügen. Eine Konfrontation mit einem anderen Magier darf nicht mit Soldaten ausgetragen werden, sondern in einem speziellen Zaubererduell entschieden werden. Wer nicht in der Lage ist, sich dieser Idee zu beugen hat sein Recht auf das Manablut oder falls er zu viel Wissen angesammelt hat, sein Leben selbst verwirkt. Die Praxis des Tribunals der Akademie für Magier mit ihrem Sigel ist inzwischen weltweit anerkannt und ausgeführt. Auch der Ascheorden akzeptiert die Existenz von Akademikern.

\section{Lehrplan}
Nach der Einführungsphase, die ein Charaktertest ist, wird einem Manablut die Grundlagen der Runensprache vermittelt, sowie die ersten fertigen Arkanen Ausdrücke. Auch in der Verzauberung, sowie nicht magiebezogenen Fächern, deren Kenntnis sich bei dem Entwerfen von Techniken häufig als nützlich erwiesen haben, werden die Grundlagen vermittelt. Anschließend kann sich jeder Student seinen Lehrplan gemäß seinen Vorstellungen mit Vorlesungen über weiterführende Themen füllen, bis er genug Erfahrung für den Rang eines Magisters hat. Anschließend kann man sich natürlich mit weiteren Kursen fortbilden.

\section{Spezialisierung}
Die Akademie zu Arthas legt ihren Fokus auf die Weiterentwicklung arkaner Künste in nahezu allen Gebieten, allerdings ist ein Großteil ihrer Forschung mit der Optimierung von Fertigzaubern, beschäftigt. Aus diesem Grund fällt es Magistern wesentlich einfacher, ihre Zauber zu modifizieren.

\chapter{Ascheorden}

\section{Geschichte}

\part{Anhang}

\chapter{Fertigzauber}

\chapter{Vokabeln}





\end{document}