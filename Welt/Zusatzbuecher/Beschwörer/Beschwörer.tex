\documentclass[a4paper,12pt,oneside]{book}
\usepackage[ngerman]{babel}
\usepackage[utf8]{inputenc}
\usepackage{imakeidx}
\usepackage[hypertexnames=false]{hyperref}
\usepackage[all]{hypcap}
\usepackage{nameref}
\usepackage{ulem}


\hypersetup{
	bookmarks=true,
    colorlinks,
    citecolor=black,
    filecolor=black,
    linkcolor=black,
    urlcolor=black
}

\title{Regeln für Beschwörer}
\author{Jordan Eichner}
\date{}
\setcounter{secnumdepth}{-2}
\setcounter{tocdepth}{1}

\begin{document}

\maketitle
\tableofcontents

\part{Grundlagen}

\chapter{Die Profession des Beschwörers}

\section{Stufen}
\begin{description}
\item[Stufe 0:] Beschwörungen(Einsatz von offensiven und defensiven Beschwörungen ohne Möglichkeit der Selektion)
\item[Stufe 1:] Selektive Beschwörungen 1(Ziele können gegen Aufschlag in Höhe der Selektionsgröße ausgewählt werden, der Rahmen sei dabei zu Ungunsten des Beschwörers zu wählen)
\item[Stufe 2:] Novize des fünftes Element(Erschaffung niederer Elementare und Herbeirufung selbiger, sowie großer Elementare);
\item[Stufe 3:] Selektive Beschwörungen 2(Rahmen wird zugunsten des Beschwörers gelegt. Alternativ kann der Beschwörer einen beliebigen Bereich(mit Aufschlag von 5 einen weiteren und so fort) aus der Gleichung für Selektion und Arealberechnung herausnehmen);
\item[Stufe 4:] Meister des Fünften Elements(Erschaffung großer Elementare, sowie Modifikation niederer Elementare);
\end{description}


\part{Offensive Beschwörungen}

\chapter{Formen}
\section{Detonation}
Wirkung breitet sich von einem Fixpunkt in alle Richtungen aus und nimmt mit zunehmender Reichweite(jeweils in fixen Intervallen) an Stärke ab. Als Stufe 4 Beschwörer kann anstatt eines Ausgangspunkte, ein Bereich gewählt werden. Mögliche Modifikationen sind:
\begin{description}
\item[Gepresste Schockwelle]Anstatt einer Kugel, breitet sich die Energie primär entlang einer Achse aus(z.B. als Welle). Durch die Bündelung werden die Reichweitensegmente verdoppelt. 
\item[Streckung]Durch einen Mehraufwand an Energie wird ein Segment in seiner Größe erweitert.
\item[Implosion]Effekt beginnt als Sphäre, die in sich zusammenfällt.
\end{description}

\chapter{Liste der Modifikatoren}

\part{Defensive Beschwörungen}

\part{Autonome Beschwörungen}

\chapter{Einführung}
Neben den vier Grundelementen, welche jeder Beschwörer zu Beginn seiner Ausbildung verwenden kann. Existiert für jene mit mehr Erfahrung die Möglichkeiten das Fünfte Element, Lyrium, zu benutzen um ihrer Magie Verstand zu geben. Dabei kann der Beschwörer entweder aus Lyrium, elementarer Energie und einem Teil seines eigenen Verstandes einen Elementaren Erschaffen, oder einen bereits vorhandenen mittels Lyrium in seinen Dienst zwingen.

\chapter{Niedere Elementare}
\section{Erschaffung}
\subsection{Regeln, Eigenschaften}
\begin{itemize}
\item Neugeschaffene Elementare haben keine eigene Persönlichkeit und folgen blind jedem Befehl, auch solche die ihren eigenen Tod bedeuten würden.
\item Ein Beschwörer kann, maximal Komplexitätsklasse Charisma * 2 niedere selbstgemachte Elementare an sich binden.
\item Elementare verbrauchen, sofern sie sich nicht in der Nähe ihres Elementes befinden, Energie aus dem Fokuskristall, bis dieser gelehrt ist. Beachte, dass das Primärelement ebenfalls geladen werden muss.
\item Alle Versuche die Verbindung zwischen Elementar und Beschwörer zu lösen sind um 5*(Stufe des Elementars + 1) erschwert. 
\end{itemize}
\section{Anrufung}
\subsection{Regel, Eigenschaften}
\begin{itemize}
\item Ein Elementar muss für seine Dienste mit Lyrium bezahlt werden und wenn er gegen seine Interessen handelt muss der Beschwörer eine Probe ablegen.
\item Angerufene Elementare muss der Beschwörer entweder an sich binden, wobei er Komplexitätsklasse Charisma * 4 Elementare an sich binden kann, wodurch diese ebenfalls von der Energie des Fokuskristalls leben, oder sie können nicht die Nähe ihres Elementes verlassen. Durch Erschaffung an den Beschwörer gebundene Elementare zählen nicht gegen dieses Limit. 
\end{itemize}
\section{Klassen von Elementaren}
\subsection{Feuer}
\begin{description}
\item[0 - Glühwürmchen:]Eine kleine schwebende Lichtquelle von der Stärke einer Fackel.
\item[1 - Irrlicht:]Ein kleiner Ball aus Licht und Feuer, welcher Dinge entflammen kann.
\end{description}
\subsection{Wind}
\begin{description}
\item[0 - Echo:]Ein Lufthauch, welcher ein Geräusch der Länge 10 Sekunden einfangen und später mit der Lautstärke einer normalen Stimme wiedergeben kann.
\item[1 - Briesling]Ein Luftwirbel, ein Objekt mit der Masse eines halben Pfunds bewegen kann.
\end{description}
\subsection{Wasser}
\begin{description} 
\item[0 - Gischt]Eine Gestalt aus wässrigem Dunst, die kleine Feuer um sich löschen kann oder einem Gegner Wasser in die Augen sprühen kann.
\item[1 - Eisblume]Bedeckt alles in einem Radius von 15cm mit einer dünnen Eisschicht, die nach einer Runde soweit angewachsen ist, dass beim Passieren man auszurutschen droht.
\end{description}
\subsection{Erde}
\begin{description} 
\item[0 - Kiesel]Winzige Gestalt aus Geröll, die durch Erde traversieren kann und eine passierende Figur stolpern lassen kann 
\item[1 - Staubteufel]Ein Sandwirbel, der durch einen Sprung in die Augen blenden kann.
\end{description}


\chapter{Große Elementare}

\chapter{Modifikation niederer Elementare}
Niedere Elementare sind aufgrund ihrer einfachen Wesens wesentlich einfacher zu kontrollieren als ihre großen Geschwister. Ein Stufe 4 Beschwörer kann die relativ geringe Stärke dieser Wesen in gewissen Aspekten verstärken, sodass sie in diesen ihren Großen Geschwistern ähneln. Beispielsweise kann ein Briesling mit genügend Stärke ausgestattet werden, dass er auch sehr viel schwerer Objekte tragen kann.

\part{Effekte}

\chapter{Feuer}
\section{Licht}
\begin{description}
\item[0] Erzeugt Helligkeit in der Größenordnung einer Fackel.
\item[1] Ein Taghelles Licht.
\item[2] Blendende Helligkeit, die für W4 Runden jeden der seine Augen nicht bedeckt hat, die Sicht raubt.
\item[3] Magische Dunkelheit und mit ihr assoziierte Kreaturen werden negiert, bzw. nehmen Schaden.     
\end{description}
\section{Hitze}
\begin{description}
\item[0] Körpertemperatur eines normalen Menschen.
\item[1] Bringt Wasser zum Kochen. Schwere Verbrennungen
\item[2] Heiß genug um Holz sofort zu entflammen.
\item[3] Eisen schmilzt.
\item[4] Gestein schmilzt. Eisen verdampft
\item[5] Gestein wird verdampft. Wolfram schmilzt.
\end{description}

\chapter{Erde}
\section{Stabilität}
\begin{description}
\item[0]Härte einer 3cm Schieferplatte
\item[1]Härte eines Ziegels
\item[2]Härte eines Eisenbarrens
\item[3]Härte eines Stahlbarrens
\item[4]Härte von Kohlenstoffnanoröhren.
\item[6]Hält einem Meteor stand.
\end{description}
\section{Masse}
\begin{description}
\item[0]Staub
\item[1]Ziegel
\item[2]Halbe Tonne
\item[3]5 Tonnen
\item[6]Nicht anhebar.
\end{description}

\chapter{Wasser}
\section{Kälte}
\begin{description}
\item[0]Wasser gefriert
\item[2]Fleisch gefriert.
\item[4]Luft wird flüssig.
\item[6]Absoluter Nullpunkt
\end{description}
\section{Feuchtigkeit}
\begin{description}
\item[0]Leichte Gischt oder Benetzung der Umgebung.
\item[1]10L Wasser oder leichter Nebel
\item[2]100L Wasser oder schwerer Nebel.
\item[3]1000L Wasser
\item[4]10000L Wasser
\end{description}

\chapter{Wind}
\section{Elektrizität}
\begin{description}
\item[0]Balloon-Haar Interaktion
\item[1]Spannung lässt Haut kribbeln.
\item[2]Elektrische Entladung, die kleine Verbrennung hervorruft
\item[4]Blitzbogen
\item[5]Echter Blitz
\end{description}
\section{Windkraft}

\end{document}