\documentclass[a4paper,12pt,oneside]{book}
\usepackage[ngerman]{babel}
\usepackage[utf8]{inputenc}
\usepackage{imakeidx}
\usepackage[hypertexnames=false]{hyperref}
\usepackage[all]{hypcap}
\usepackage{nameref}
\usepackage{ulem}


\hypersetup{
	bookmarks=true,
    colorlinks,
    citecolor=black,
    filecolor=black,
    linkcolor=black,
    urlcolor=black
}

\title{Aurum Orbis - Alchemistische Ordnung}
\author{Jordan Eichner, Christoph Schmidt}
\date{}
\setcounter{secnumdepth}{-1}
\setcounter{tocdepth}{10}

\begin{document}

\maketitle
\tableofcontents

\part{Alchemistische Ordnung}

\chapter{Sternenkarte}

\section{Extraktor}


\section{Essenzknacker}
Zerlegt alle Familienessenzen, sodass die Bestandteile mit anderen Elementaren interagieren können.

\section{Essenzfalter}
Essenzen könne durch

\chapter{Essenztypen}

\section{Elementare}
Elementare Essenzen sind spezifische Energien, zu winzigen Teilen der Schöpfung. Licht, Feuer, Harmonie, Vitalität - Sie alle lassen sich nicht weiter spalten. 

\section{Familienessenz}
Eine Familienessenz fasst unter sich mehrere Elementare zusammen, nicht nur auf dem Papier, sondern auch Energietechnisch. So stören sich die in Sonnenessenzen befindlichen Feuerteile an evt. freien Wasseressenzen. Einmal aufgespaltene Familien können nicht wieder vereint werden und auch nicht künstlich zusammengesetzt werden. Daher sind reine Familienessenzen noch unmöglicher als Elementare.

\part{Komponenten-Index}

\chapter{Pflanzen}

\section{Glaziale}
Geboren in den eisigen Winden, auf jedem Untergrund sind diese Pflanzen bekannt für 
\\Enthält Frost, Wind, Wasser

\chapter{Minerale}

\chapter{Luft und Gas}

\chapter{Fleischliches}

\section{Blut}
Enthält immer die Essenz der Lebensform und ggf. Lebensessenz. Jäger und Beutebeziehungen unter Ausschluss von Völkern bilden hierbei einen geeignetes Gegenelement.

\chapter{Rituelles}

\section{Kohle}
Enthält Feuer, Zerstörung, Tod, 

\section{Morgentau}
Frisch gesammeltes Kondenswasser auf Pflanzen in der Stunde des Sonnenaufgang.
\\Enthält Wasser, Wind, Sonne


\part{Essenzen}

\chapter{Energie}
\begin{description}

\end{description}

\chapter{Geistliches}

\chapter{Metaphysisches}

\end{document}