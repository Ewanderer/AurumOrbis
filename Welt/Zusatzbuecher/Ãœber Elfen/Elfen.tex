\documentclass[a4paper,12pt,oneside]{book}
\usepackage[ngerman]{babel}
\usepackage[utf8]{inputenc}
\usepackage{imakeidx}
\usepackage[hypertexnames=false]{hyperref}
\usepackage[all]{hypcap}
\usepackage{nameref}
\usepackage{ulem}


\hypersetup{
	bookmarks=true,
    colorlinks,
    citecolor=black,
    filecolor=black,
    linkcolor=black,
    urlcolor=black
}

\title{Elfen}
\author{Jordan Eichner}
\date{}
\setcounter{secnumdepth}{-2}
\setcounter{tocdepth}{1}

\begin{document}

\maketitle
\tableofcontents

\part{Einführung}
Trotz unzähliger Versuche und jahrelanger Missionsarbeit konnten die Elfen den Makel ihrer Vorfahren, den heimtückischen Dunkelelfen, nie ganz beseitigen, weshalb dieses so friedfertige und hochentwickelte Volk mit Missachtung und Feindseligkeit betrachtet wird. Diese Isolation hat bei den Elfen nie  

\part{Körper und Geist}
\chapter{Elfische Anatomie}
Im wesentlichen erinnern die 
\chapter{Natur des Elfischen Gemüts}
Der Charakter eines Elfen ließe sich am besten mit dem Wort Rein beschreiben. In den Herzen ihrer Vorfahren war nie Platz für etwas anderes als Hass und den Sinn für Gemeinschaft gewesen. Nach ihrer Läuterung durch die Drachen ist nur letzter erhalten geblieben und an die Stelle des Hasses, traten nur die reinsten Gefühle und Charakterzüge. Elfenkinder, aber auch die Elfen der Städte kann man daher regelmäßig dabei beobachten, wie sie sich in den Anblick einfachster Wunder der Natur versenken können, wie dem Tau eines Spinnennetzes oder einfach allen Anstand fallen lassen, um das freie Leben zu genießen.    

\part{Leben in den Städten}
\chapter{Struktur der Städte}
Jede elfische Stadt folgt in ihrem Aufbau einem ähnlichen Schema: 
\\Zwischen dem künstlich angelegten Gärten und Hainen gibt es vereinzelte Lichtungen, die ein Elf sein eigenen nennt und auf der er gemäß seinen Vorstellungen sein Leben führt. Ein Elf gibt durch die Art, wie das Unterholz um seine Lichtung gestaltet ist, zu verstehen, inwiefern er ungestört bleiben möchte. Diese Markierungen reichen von verflochten Zäunen für Isolation, über einfache Blumenringe, die jeden Willkommen heißen. Auch in der Bebauung der Lichtung ist jede einzigartig. Einige Elfen bevorzugen das Leben in einem prächtigem Baum, während andere auf der nackten Erde schlafen. Dies sagt natürlich nur wenig über den Stand oder die Kaste eines Elfen aus.
\\Zusätzlich hat jede Kaste noch einen eigenen Abschnitt des großen Garten für sich, um dort ihre Riten zu zelebrieren oder ihre Traditionen und Geheimnisse weitergeben. Diese Gebiete sind zwar auf einzelnen mit Lampen ausgeleuchteten Pfaden zu durchqueren, aber in bestimmten Fällen kann der Zutritt auch völlig verwehrt sein.
\\Alles wird durch ein Geflecht aus Pfaden verbunden, deren Gestaltung immer in die Hände der Bewohner fällt, auch wenn es brauch ist, dass jeder Elf, der die Wege benutzt, einen kleinen Akzent setzt.
\chapter{Alltag}
Der Tag eines Elfen besteht im wesentlichen aus 3 Teilen. Je nach Kaste gehen sie unterschiedlichen Pflichten in der Stadt nach, auch wenn man hier nicht wirklich von Arbeit sprechen sollte. Zwar macht sich jeder Elf die Hände schmutzig, doch empfindet jeder Elf diese Arbeit als Entspannung und festen Anker ihres Sozialen Lebens, da sie hier stets mit Gleichgesinnten in Kontakt kommen. Außerhalb ihrer Arbeitsphasen geht jeder Elf seinen persönlichen Interessen nach, diese reichen von Wanderungen im und um die Städte herum, der Ausübung eines Handwerks oder wonach ihnen sonst der Sinn steht. Aufgrund einer Sonne als Zeitgeber schlafen Elfen unregelmäßig, wenn sich auch Elfen je nach Bekanntheitsgrad in ihren Schlafrhythmen ähneln. Von außen wirken die Städte daher immer lebendig, was sich oft auf das Zeitgefühl von Besuchern auswirken kann, auch wenn das Treiben an sich nach reiner Willkür schreit.
\chapter{Gesetze und Rechtsprechung}
Der oberste Grundsatz ist das ungeschriebene Gesetz des Respekt vor seinem Gegenüber. Damit werden nicht nur andere Elfen oder Besucher eingeschlossen, sondern auch alle anderen Lebewesen, weshalb beispielsweise Holz nur durch Kaste der Gärtner im Einvernehmen mit dem Baum geerntet werden darf. Besuchern wird in diesem Punkt sehr großzügig verziehen, auch wenn man sie von Zeit zu Zeit auf ihre Fehler aufmerksam macht und ab einem bestimmten Grad der Unverschämtheit Einen zum Gehen auffordert. 
\\Gemäß der Traditionen der einzelnen Kasten habe diese in ihren Hoheitsgebieten das Sagen und ihren Anweisungen muss sich jeder Besucher beugen. Verstöße werden hier durch die Wächter geahndet, wenn auch dank der Drachenelfen und ihrer Kunst, den Geist von Erinnerungen zu befreien, auf solch morbide Strafen, wie den Tod verzichtet werden kann.
\chapter{Über die wahre Natur einiger Städte}
Die ersten großen Elfenreiche wurden errichtet auf den Ruinen der Zwergenstädte, welche diesen als Rückzugsort unter die Erde dienten, als die Drachen mit den Elfen die Oberhand gewannen. Um die Geschichte dieses Krieges vor anderen Völkern geheimzuhalten und die Welt vor den Überbleibseln zu beschützen wurde jede Elfenstadt stilistisch angepasst. Anbei eine Übersicht über diese Städte.
\section{Das Zwergengrab}
Bei der Belagerung der Drachen in den Feuerlanden stauten sich Zwergenleiber kilometerweit und als die Drachen mit ihrem Gegenangriffen begannen, tat sich ein See aus flüssigem Metall auf, den man loswerden musste. Nicht weil das Artheum ansonsten den Landstrich unbewohnbar machte, sondern auch, weil im Senterra das Kollektiv der Zwerge weiterlebte und ihre reine mentale Präsenz jeden Geist zu brechen mochte, wenn es nur lange genug Zeit erhielt. Da man das Kollektiv nicht einfach zerstören konnte, nutzte man die Schwäche des Metalls, dass es bei niedrigen Temperaturen seine göttlichen Eigenschaften verliert. Doch auch wenn das Kollektiv dadurch in einen Winterschlaf fiel, so produzierte das Metall eine eigene Abwärme die künstlich abgeführt werden musste, um den Schlaf aufrecht zu halten. Das Zwergengrab war eine unterirdische Stadt auf einem See mit künstlichem Wasserkreislauf, um jedes Stück Wärme, welches das schlafende Metall am Grund dieses Sees abgab, in die umliegenden Höhlen zu verteilen.

\section{Das Herz der Zwerge}
Da man fürchtete, die überlebenden Zwerge könnten mit genügend göttlichem Metall erneut ihre Armee errichten und so dem Land großen Schaden zufügen, wollte man den übrigen Völkern, das Geheimnis des Sees vorzuenthalten, damit es nicht über die ganze Welt verteilt werden würde. Die Stadt gleicht daher einem gewaltigen Labyrinth von Häusern, welche zusätzlich verzaubert ist, um gewöhnliches Volk um den See und den unter der Stadt befindlichen Ruinen der Zwergenwerkstätte unbemerkt herumzuleiten. 


\part{Beziehungen mit der Außenwelt}
\chapter{Diplomatie}
In ihren Beziehungen mit der Oberwelt sind es stets die Elfen gewesen, die die Initiative ergriffen haben. Sie reichen ihren Nachbarn steht die Hand, in Form von freiem Verkehr in ihren Städten und dem Angebot sie in Notzeiten mit Nahrung oder anderen Lebenswichtigen Gütern zu versorgen unter der Bedingung, dass man sie ansonsten unbehelligt lässt und sich bei einem Besuch an ihre Gesetze halten. Auf dieses Angebot erhalten die Elfen allerdings häufig keine Antwort, außer von den verzweifelten. Da sie keine wertvollen Ressourcen oder Gebiete für sich beanspruchen, wird ihnen auch im Spiel der Macht keine Beachtung geschenkt, da die Elfen streng Neutral bleiben.
\chapter{Handel}
Die wenigen echten Waren, die in einer elfischen Stadt entstehen, sind ihre kunstfertigen Handwerksstücke. Diese einem Elfen abzukaufen erfordert jedoch in der Regel einige Mühen, da man diesen nur sehr wenig im Austausch bieten kann. Dazu zählen handgefertigte und mit viel Mühe erschaffene Kunstwerke, für manche Elfen eine gute Geschichte oder ein schöne Erinnerung, wie ein Abendessen oder die Einladung auf ein Fest. Auch wenn Diebstahl von der elfischen Gesellschaft nicht direkt geahndet wird, kann und wird ein einzelner Elf einen Dieb verfolgen und seinen Besitz zurückholen. 
\chapter{Besucher}
Jedem stehen die Tore in ein Elfenreich offen und es gibt im Gebiet der Kaste, Freunde in der Fremde, auch immer Platz in Form von Betten. In der Regel steht einer Gruppe von Besuchern immer ein Elf zu Seite, wenn sie Hilfe brauchen, sich zurecht zu finden. Die Gesetze und Tradition werden, sofern notwendig, beiläufig beim Essen und im Gespräch erläutert. 

\part{Außerhalb ihrer Reiche}
\chapter{Freunde in der Fremde}

\chapter{Elfen mit Stamm bei den andern}
Es kommt nur selten vor, dass ein Elf außerhalb der Gemeinschaft aufwächst und noch viel seltener, dass es ohne einen elfischen Elternteil aufwächst. Diesen Kindern fällt es in der Regel nicht schwer mit anderen Personen zurecht zu kommen, auch wenn ihnen immer wieder mit Vorurteilen und Anfeindungen begegnet wird. Elfische Kinder fallen außerdem dafür auf, dass sie sehr Naturverbunden sind und offen für neues sind. Erst mit der Zeit übernehmen sie die Charakterzüge ihrer Umwelt. Aufgrund ihrer langen Lebenszeit neigen Elfen allerdings dazu im Angesicht des Todes aller die sie kennen, regelmäßig in Trauerphasen zu verfallen. Nur in wenigen Fällen zieht es einen Elfen zurück in die Heimat, wo sie allerdings nie wirklich in die Gemeinschaft der Kasten zurückfinden.

\part{Halbelfen}

\part{Religion und Drachenelfen}
\chapter{Einführung}


\end{document}