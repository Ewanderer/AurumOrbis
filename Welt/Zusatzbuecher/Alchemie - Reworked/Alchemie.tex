\documentclass[a4paper,12pt,oneside]{book}
\usepackage[ngerman]{babel}
\usepackage[utf8]{inputenc}
\usepackage{imakeidx}
\usepackage[hypertexnames=false]{hyperref}
\usepackage[all]{hypcap}
\usepackage{nameref}
\usepackage{ulem}


\hypersetup{
	bookmarks=true,
    colorlinks,
    citecolor=black,
    filecolor=black,
    linkcolor=black,
    urlcolor=black
}

\title{Aurum Orbis - Alchemistische Ordnung}
\author{Jordan Eichner}
\date{}
\setcounter{secnumdepth}{-1}
\setcounter{tocdepth}{10}

\begin{document}

\maketitle
\tableofcontents

\part{Einführung}

\chapter{Die Profession des Alchemisten}

\chapter{Grundlegende Prinzipien}

\part{Die Sternenkarte}
Auch wenn sich Alchemisten nie als Magier bezeichnen würden, ist ihr zentrales Werkzeug ein Manakristall, eine polierte Metallplatte die sie mit Runenmustern beschreiben welche dann zum Leben erweckt werden. Auch wenn es Erzählungen über eine Zeit gibt, in welcher Alchemie mit einer ganzen Reihe von komplizierten und obskuren Werkzeugen und Ritualen zurechtkam, scheint sich diese viel billiger, schnellere und präzisere Arbeitsweise durchgesetzt zu haben. Der Manakristall ist für die gesamte Alchemie uninteressant, da sie selbst nach jahrelanger Verwendung nicht verbraucht werden, werden im folgendem Platte und Runen betrachtet.

\chapter{Die Platte}


\chapter{Runen}
Während die Platte lediglich als Medium dient, ist es die Beschriftung, welcher der Alchemist bei seinen Arbeiten wählt, die über die Funktion der Sternenkarte entscheidet. Dabei ist die Anzahl der verschiedenen Runen unbekannt, wenn diese auch in drei Funktionen(Auge/Extrator/Manipulator) und Stufen unterteilt(Novize/Adept/Meister) geordnet werden. Während sich Runen der Stufe Novize mit beliebigen Material einfach aufzeichnen lassen, benötigen höhere Stufen einen alchemistischen Fokus, welcher im Vorfeld hergestellt werden muss.

\section{Augen}
Auch wenn es ausführliche Aufzeichnungen über die Essenzen innerhalb verschiedenster Objekte gibt und ein Alchemist selbst eine Menge über derartiges von sich aus memoriert hat, beginnt alle Forschungsarbeit mit der Analyse, welche das Auge ermöglicht. Im Zentrum der Sternenkarte wird das zu untersuchende Objekt gelegt, worauf der Alchemist seine Hände drumherum platziert und das Auge aktiviert. Im Geist wird dann die Energiesignatur angezeigt, welche allerdings noch entschlüsselt werden muss. 

\subsection{Novizen}
Es lässt sich die stärkste Essenz innerhalb des Objektes bestimmen.

\subsection{Adept}
Es lassen sich alle Essenzen innerhalb des Objekts bestimmen

\subsection{Meister}
Funktion und Wirkungsweise alchemistischer Manipulationen lassen sich bestimmen.

\section{Extratoren}
Werden benutzt um die eigentlichen Essenzen gemäß der bereits erläuterten Prinzipien zu gewinnen. Manche Extratoren ermöglichen auch Familienessenzen aufzutrennen. Für die eigentliche Anwendung werden alle Komponenten in den Slots der Rune platziert, bevor durch eine Berührung, der Prozess in Gang gesetzt wird. Je nach Natur der Rune muss der Alchemist weitere mentale Impulse geben.

\subsection{Novizen}
Die Essenzen haben maximal Qualtität F.

\subsection{Adept}
Die Essenzen haben maximal Qualität C.

\subsection{Meister}
Die Essenzen haben maximale Qualität A.

\section{Manipulatoren}

\subsection{Novizen}

\subsection{Adept}

\subsection{Meister}

\part{Essenzen}

\chapter{Einzelessenz}

\chapter{Familienessenz}

\part{Ingredienzien}
\chapter{Pflanzen}

\chapter{Minerale}

\chapter{Lebendiges}

\chapter{Rituelles}



\part{Manipulationen}

\part{Beispielhafte Anwendungsfälle}

\end{document}