\documentclass[a4paper,12pt,oneside]{book}
\usepackage[ngerman]{babel}
\usepackage[utf8]{inputenc}
\usepackage{imakeidx}
\usepackage[hypertexnames=false]{hyperref}
\usepackage[all]{hypcap}
\usepackage{nameref}
\usepackage{ulem}


\hypersetup{
	bookmarks=true,
    colorlinks,
    citecolor=black,
    filecolor=black,
    linkcolor=black,
    urlcolor=black
}

\title{Waffenindex}
\author{Till Markusch}
\date{}
\setcounter{secnumdepth}{-2}
\setcounter{tocdepth}{1}

\begin{document}

\maketitle
\tableofcontents

\chapter{Nahkampf}

\section{Klingen}

\subsection{Something.Something}

\begin{description}
\item[Schwert]- PA+0,AT+0, INI+0
\\Die klassischste aller Waffen. Schnell, scharf und in den richtigen Händen durchaus tödlich.
\item[Säbel]- PA+1, AT+0, INI-1
\\Sie sind durch die generell breitere Klinge sehr wiederstandsfähig, aber somit auch ein Stück langsamer.
\item[Kurzschwert] - PA+0, AT-1, INI+1
\\Kurze Klingen sorgen für nicht so tiefe Wunden, dafür sind sie aber sehr schnell zu ziehen und zu verstecken.
\item[Zweihänder] - PA+0, AT+1, INI-1
\\Ein sehr großes Schwert mit mächtiger Klinge, das auch dementsprechend Schaden anrichtet, ist aber generell langsamer als andere Waffen.
\item[Dolche] - PA-1, AT+0, INI+1
\\Der Unterschied zum Kurzschwert ist das man den Dolch mit ner anderen Technick führt, für die das Kurzschwert aber zu schwer wäre.
\item[Äxte] - PA-1, AT+1, INI+0
\\Die Axt, ein Alltagsgegenstand den jeder Abenteurer dabei haben sollte, ob man sie nun zum Fuerholz hacken, oder zum Schädel spalten benutzen will ist jedem selbst überlassen.
\end{description}

\chapter{Fernkampf}

\end{document}