\documentclass[a4paper,12pt,oneside]{book}
\usepackage[ngerman]{babel}
\usepackage[utf8]{inputenc}
\usepackage{imakeidx}
\usepackage[hypertexnames=false]{hyperref}
\usepackage[all]{hypcap}
\usepackage{nameref}
\usepackage{ulem}


\hypersetup{
	bookmarks=true,
    colorlinks,
    citecolor=black,
    filecolor=black,
    linkcolor=black,
    urlcolor=black
}

\title{Waffenindex}
\author{Till Markusch}
\date{}
\setcounter{secnumdepth}{-2}
\setcounter{tocdepth}{1}

\begin{document}

\maketitle
\tableofcontents

\chapter{Nahkampf}

\section{Klingen}
\\ Waffen die viel Trefferpunkte Schaden anrichten und mit den richtigen Technicken auch Blutungsschaden Verursachen können,
   wenn man ein Meister dieser Waffen Art ist kann man auch Gliedmaßen abtrennen und Feinde so verkrüppeln.

\begin{description}
\item[Schwert]- PA+0,AT+0, INI+0
\\Die klassischste aller Waffen. Schnell, scharf und in den richtigen Händen durchaus tödlich.
\item[Säbel]- PA+1, AT+0, INI-1
\\Sie sind durch die generell breitere Klinge sehr wiederstandsfähig, aber somit auch ein Stück langsamer.
\item[Kurzschwert] - PA+0, AT-1, INI+1
\\Kurze Klingen sorgen für nicht so tiefe Wunden, dafür sind sie aber sehr schnell zu ziehen und zu verstecken.
\item[Zweihänder] - PA+0, AT+1, INI-1
\\Ein sehr großes Schwert mit mächtiger Klinge, das auch dementsprechend Schaden anrichtet, ist aber generell langsamer als andere Waffen.
\item[Dolche] - PA-1, AT+0, INI+1
\\Der Unterschied zum Kurzschwert ist das man den Dolch mit ner anderen Technick führt, für die das Kurzschwert aber zu schwer wäre.
\item[Äxte] - PA-1, AT+1, INI+0
\\Die Axt, ein Alltagsgegenstand den jeder Abenteurer dabei haben sollte, ob man sie nun zum Fuerholz hacken, oder zum Schädel spalten benutzen will ist jedem selbst überlassen.
\end{description}

\section{Stich Waffen}
\\Diese Art von Waffen ist für tiefe wunden ausgelegt, die daher auch Vitalitäs Schaden verursachen kann, und in höhren Stufen auch Technicken 
   enthält die es elaubt Innere Organe zu verletzen und so den Gegner auszuschalten.

\begin{description}
\item[Speer] - PA+1, AT+0, INI-1
\\Speere sind große Waffen, vielleicht nicht so langsam wie andere Stich Waffen, aber trotzdem noch ziemlich langsam, der große Griff ermöglicht aber gutes Blocken.
\item[Lanze] - PA+0, AT+1, INI-1
\\Die größte der Stich Waffen, tödlich wenn richtig eingesetzt, aber auch dementsprechend langsam und die Beschaffenheit der Waffe macht das Blocken schwer.
\item[Sai] - PA+0, AT-1, INI+1
\\Eine art Dolch mit einer sehr langen dünnen Klinge, die dazu benutzt werden kann um die Organe auf kurzer Reichweite zu verletzen.
\end{description}

\section{Stumpfe Waffen}
\\ Waffen, die weder Klingen noch Spitzen besitzen, noch Blutungsschaden anrichten. Diese Waffen nutzt man dann wenn man Knochen Knacken hören will. 

\begin{description}
\item[Knüppel] - PA-1, AT+1, INI+0
\\Die Grunlegendste Waffe, man haut damit Schädel ein, was auch sonst. Manchmal kommen sie auch mit Stacheln.
\item[Hammer] - PA+0, AT+1, INI-1
\\Ein großer Streithammer. Eigentlich nur zum Knochen und Schilde zu zertrümmern.
\item[Flegel] - PA-1, AT+1, INI-1
\\Ein Knüppel an einer Kette, er ist zwar langsam und macht das Blocken unmöglich, ist dafür aber länger da er an einer Kette hängt
\end{description}

\section{Sonder Waffen}
\\Waffen für die man meistens eine eigene Schule braucht.

\begin{description
\item[Degen] - PA+1, AT-1, INI+0
\\Im Prinzip eine Klingen Waffe, die aber jedoch Stich Schaden anrichtet. Hat eine ganz eigene Schule, die nur von gelernten Proffessionen ausgeübt werden kann.
\item[Katana] - PA-1, AT+0, INI+1
\\Eine unterart des Schwertes, die wie der Degen ne eigene Schule hat. Gut dazu wenn man Gliedmaßen abtrennen will.
\Item[Naginata] - PA+0, AT+1, INI-1
\\Ein Speer mit einer langen Klinge an der Spitze, langsamer als ein Speer, ist aber gut um Blutungsschaden auszuteilen.
\item[Peitsche] - PA-1, At+0, INI+1
\\Eine Waffe die zwar nur leichte wunden verursacht, aber mit einer enormen Reichweite glänzt und mit guten Fähigkeiten den Gegner auf Fesseln kann.
\item[Hellebarde] - PA-1, AT+1, INI+0
\\Eine Multifunktionswaffe, sie kann durch die Klinge zum Schneiden genutzt werden, und durch die Pike an der Spitze auch zum stechen genutzt werden kann. Aber sie benötigt durch den schweren Kopf relativ viel Stärke, was auch das Bkocken schwerer macht.
\Item[Karma] - PA-1, AT+0, INI+1 
\\Sehr schnelle Waffen die Effektiv genutzt werden können um dem Gegner gliedmaßen abzutrennen, aber dafür auch eine spezielle Schule benötigen.
\item[Human Messer Interface] - PA-1, AT+0, INI-1
\\Ein Lederhandschuh der dazu benutzt wird um Leute zu operieren, aber mit den richtigen Kenntnissen kann man ihn auch zum Kampf nutzen
\end {description}

\chapter{Fernkampf}

\end{document}