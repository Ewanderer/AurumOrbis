\documentclass[a4paper,12pt,oneside]{book}
\usepackage[ngerman]{babel}
\usepackage[utf8]{inputenc}
\usepackage{imakeidx}
\usepackage[hypertexnames=false]{hyperref}
\usepackage[all]{hypcap}
\usepackage{nameref}
\usepackage{ulem}


\hypersetup{
	bookmarks=true,
    colorlinks,
    citecolor=black,
    filecolor=black,
    linkcolor=black,
    urlcolor=black
}

\title{Gruppierungen, Geheimbünde und politische Parteien}
\author{Jordan Eichner}
\date{}
\setcounter{secnumdepth}{-2}
\setcounter{tocdepth}{1}

\begin{document}

\maketitle
\tableofcontents

\part{Gruppen}

\chapter{Kriminelle Verbindungen}

\section{Das Mevial-Kartell}
Ein Verbrecherring der innerhalb der Baronie Rhone agiert und unter anderem mit der Obrigkeit zusammen arbeiten, um deren Herrschaft zu sichern im Gegenzug für Sonderrechte. Von Raub, Drogenhandel bis hin zu Entführungen und Attentaten gibt es kein Verbrechen für welches sie nicht zu haben sind. Organisiert wird der Drogenring hierarchisch, wobei an der Spitze 9 aus den Schatten agierende Individuen stehen, alles Vampire unter der Führung eines Gegenwelter-Vampiren. Jede Führungsperson in allen Schichten gilt als besonders Grausam und ist dafür bekannt, dass ihnen ein Leben nichts wert ist. Mit Brutaler Hand versuchen sie regelmäßig ihren Herrschaftsbereich auszuweiten. Aufgrund ihrer Natur und Ruchlosigkeit werden sie auch von Ihresgleichen nicht geschätzt und all zu oft schließen sie sich gegen dieses Syndikat zusammen, um ihre Geschäfte zu behindern. Trotz vieler Rückschläge ist es jedoch nie gelungen ihren Kern in Rhone zu zerstören, wegen der Schützenden Hand der Obrigkeit. Ihr Name stammt aus der Gegenwelt und bezeichnete eine Blutlinie von Askons Vampiren.

\part{Orden}

\chapter{Politisch}

\section{Der Falkenorden}
Begründet durch Umos ist dieser Zirkel darauf bedacht das Rad der Zeit voran zu treiben. Sie sind Revolutionäre, Bewahrer von Geheimnissen über das Gute, wie das Böse und noch vieles mehr. Es gibt keine wirkliche Hackordnung, sondern vielmehr existieren unzählige Zellen, die zwischen einem Einzelne bis einem dutzend fassen. Untereinander kommunizieren sie über diverse Dialekte ihrer Geheimsprache oder besprechen sich, sofern es zum Kontakt kommt. Es ist daher nicht unüblich, dass auf beiden Seiten eines Krieges verschiedene Zellen des Falkenordens gegeneinander operieren. Mitglied wird man durch die Einführung eines Agenten oder nach direktem Kontakt mit den inneren Geheimnissen, wobei es nicht selten vorkommt das unbequeme Mitglieder schnell zum Schweigen gebracht werden. Zu der Ausbildung eines jeden Falken gehört die Fähigkeit sich in anderen Gruppen zu integrieren, sie teilweise über Jahrzehnte zu unterwandern, bis man sie für die eigenen Zwecke instrumentalisiert, bevor man sich abseilt. 

\end{document}