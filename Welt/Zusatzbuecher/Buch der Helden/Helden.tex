\documentclass[a4paper,12pt,oneside]{book}
\usepackage[ngerman]{babel}
\usepackage[utf8]{inputenc}
\usepackage{imakeidx}
\usepackage[hypertexnames=false]{hyperref}
\usepackage[all]{hypcap}
\usepackage{nameref}
\usepackage{ulem}


\hypersetup{
	bookmarks=true,
    colorlinks,
    citecolor=black,
    filecolor=black,
    linkcolor=black,
    urlcolor=black
}

\title{Buch der Helden}
\author{Jordan Eichner}
\date{}
\setcounter{secnumdepth}{-2}
\setcounter{tocdepth}{1}

\begin{document}

\maketitle
\tableofcontents

\part{Völker}

\chapter{Elfen}

\chapter{Menschen}
\begin{description}
\item[Fertigkeiten:] 4 Wahrnehmung, 2 Athletik, 2 Zechen, 2 Überreden
\end{description}

\chapter{Dunkelelfen}

\chapter{Mekka}

\chapter{Nephilim}

\chapter{Draklinge}

\chapter{Dämonen}

\chapter{Lorkin}




\part{Kulturen}




\part{Professionen\&Spezialisierungen}
\setcounter{chapter}{0}
\chapter{Einführung}
Unter einer Profession versteht man einen Beruf, der einer Figur Zugriff auf bestimmte Fertigkeiten-Spezialisierung, Kampftalenten oder passiven Boni gibt, diese Orientieren sich oft an dem Tätigkeitsfeld der Profession.
\\Spezialisierungen auf der anderen Seite sind entweder stark zugespitzte Professionen oder es handelt sich um eine Reihe eigener Techniken und Dinge, die auf ein sehr eng begrenztes Aufgabenfeld beschränkt sind.
Es folgt eine Liste aller Professionen\&Spezialisierungen, bzw. Verweise auf die Stellen in vorhergehenden Texten.
\chapter{Offene Professionen}
In diese Kategorie fallen alle Professionen und Spezialisierungen, die in der Regel jeder erlernen kann, auch wenn kulturelle Differenzen das Erlernen erschweren können.

\section{Alchemist}
Meister der Alchemistischen Ordnung und Zerstörer von Stadtvierteln.
\subsection{Fertigkeiten}
+: Alchemie, Alchemistische Ordnung, Kräuterkunde, Werfen
-: Alle Waffentalente, außer Werfen
\subsection{Stufen}
\begin{description}
\item[Stufe 0:]
\end{description}

\section{Weltenbummler}
siehe Tholemäus, Weltenbummler.
\subsection{Anforderungen}
Göttliche Verbindung
\subsection{Fertigkeiten}
+: Alle Wissenfertigkeiten
-: Alle Waffentalente außer Schwerter und Duellierwaffen, 
\section{Bauer}

\section{Fischer}

\section{Bote}

\section{Matrose}
Ob auf See oder hoch in den Lüften, bewegen sich Matrosen wie emsige Ameisen über das Deck, durch die Takelage oder den Maschinenraum ihres Vehikels um es in Fahrt zu bringen und den Gewalten der rauen See oder hohen Lüfte zu trotzen. Diese Profession hat 2 Spezialisationen, die beim Erwerben dieser Profession gewählt werden muss und danach nicht mehr geändert werden kann.
\subsection{Stufen}
\begin{description}
\item[Stufe 0:]Seemann(Immunität gegen Seekrankheit)/Flieger()
\end{description}

\section{Luftpirat}
Nicht nur leben in den tiefen der Meere und den Gefilden der Lüfte teilweise namenlose Schrecken, nein darüber hinaus kommen jedoch aber auch die Ruchlosen Männer und Frauen, die unter der Schwarzen Flagge der Piraterie segeln. Sie gelten als ruchlose Monster mit einem Auge für Schätze.
\subsection{Fertigkeiten}
+: Säbel, Handfeuerwaffen, Armbrüste, Flugdynamik, Orientieren, Zechen, Klettern, Großwaffenkunde
\\0: Kartografie, Schätzen
\\-: Alle übrigen Waffen, Schwimmen, Gerber, Kürschner
\subsection{Stufen}
Stufe 0: Profitgier(Der Pirat erhält einen Bonus auf alle Wissensproben(sofern es um Geld/Schätze geht), sowie einen Bonus auf seinen Schätzenwert in Höhe von 2*Stufen in dieser Profession, zusätzlich kann ein Pirat alle im Zusammenhang mit Schätzen stehenden Informationen(Wie Gesucht-Poster oder Schatzkarten) für 1 Woche/ 1 Monat/6 Monate/1 Jahr/ewig problemlos memorieren.);
Stufe 1: Profitgier 2();

\section{Mechaniker}
\subsection{Fertigkeiten}
+: Flugdynamik, Mechanik, Feinmechanik, Kristallkunde
-: Alle Kampftalente
0: Alle Handwerkstalente.

\section{Arzt}
\subsection{Fertigkeiten}
+: Heilkunde(Wunden), Heilkunde(Krankheiten), Heilkunde(Gift), Kräuterkunde, Anatomie, Chirurgie 
\\0: Mensch-Messer-Schnittstellen, 
\\-: Alle Kampftalente, mit Ausnahme der Mensch-Messer-Schnittstellen.
\subsection{Stufen:}
\begin{description}
\item[Stufe 0:]Ungewohnte Heilpraktiken(Nach einer erfolgreicher Diagnose kann der Arzt auch ohne geeignete Werkzeuge/Materialien/Arzneien eine Behandlung erschwert um 10 durchführen, wobei alle Punkte die bei der Diagnose nach Abzug des Modifikators ''vollständig'' übriggeblieben sind eingebracht werden dürfen); Angewandte Chirurgie(Arzt darf ihren Chirurgie-Wert zum Kämpfen mit der Mensch-Messer-Schnittstelle benutzen)
\item[Stufe 1:]Ungewohnte Heilpraktiken 2(Bei der Behandlung darf der Arzt seinen Anatomiewert(Probe) einbringen oder den Durchschnitt von HK(Wunden, Gift, Krankheiten) als Basiswert benutzen);
\item[Stufe 2:]Ungewohnte Heilpraktiken 3(Bei der Behandlung darf der Arzt entweder ein mit der Beschwerde oder Behandlungsmethode geeignetes Wissentalent(Probe) einbringen);
\end{description}

\section{Soldat}
Ob in einer Armee oder als Wächter in einer Stadt, Soldaten sind ausgebildete Milizen, die aus verschiedensten Gründen sich für den Dienst mit der Waffe verpflichtet haben. In der Regel verpflichten sie sich für eine gewisse Zeit einem bestimmten Regiment oder einer Stadt zu dienen, wobei natürlich bei außergewöhnlichen Leistungen eine Beförderung zum Bleiben locken kann.
\subsection{Fertigkeiten}
+: Waffen, Schmieden(Waffen/Rüstungen)
-: Wissen(außer Kriegskunst, Staatskunst, Rechtskunde), Bailieren, Feinmechanik, Schneider, Gaukelein, Tanzen
\subsection{Stufen}
\begin{description}
\item[Stufe 0:]Kampfdrill(2 kostenlose grundlegende Kampftechniken, für die die Bedingungen erfüllt werden);
\item[Stufe 1:];Feldarzt(+6 Heilkunde:Wunden)
\end{description}

\section{Barde}
\subsection{Stufen}
\begin{description}
\item[Stufe 0:](Barden haben eine 50\% Chance unbemerkten Angriffen ausweiche zu dürfen); 
\item[Stufe 1:](Barden können beim Vortragen eines Bardenstücks ausweichen, ohne ihren Vortrag unterbrechen zu müssen);
\item[Stufe 2:](Bardenstücke können selbst den Lärm von Stürmen oder Schlachten durchdringen);
\item[Stufe 3:]();
\end{description}
\section{Gladiator}

\section{Knecht/Magd}

\section{Gaukler}

\subsection{Stufen}
\begin{description}
\item[Stufe 0:]Eure Aufmerksamkeit bitte!(Mit Worten und Gesten lenkt der Gaukler die Aufmerksamkeit aller Umstehenden auf ein Ziel seiner Wahl)
\item[Stufe 1:]
\end{description}
\section{Trickbetrüger}

\section{Fälscher}

\section{Gelehrter}

\section{Priester}

\section{Koch}

\section{Faustkämpfer}
Während hinter der typischen Kneipenschlägerei oft keinerlei Technik steckt, sind es vor allem bestimmte Arenen und Box-Ringe aus denen eine ganz eigene Art von Kämpfern stammt. Die Faustkämpfer, geübt im Umgang im Waffenlosen Kampf, sind vor allem auf engem Raum und in Duell selbst einem Bewaffnetem ebenbürtig, wenn nicht sogar überlegen.
\subsection{Fertigkeiten}
+:Faustkampf, Athletik, Akrobatik
\\-:Alle anderen Waffentalente 
\subsection{Stufen}
\begin{description}
\item[Stufe 0:]Kampfstil(Zur Berechnung des Angriffswertes bei der Benutzung von Faustkampf darf ein Attribut gegen ein anderes körperliches, welches noch nicht zur Berechnung gehört, getauscht werden, diese Wahl kann nicht geändert werden); 
\item[Stufe 1:]Einstecken 1(Im Faustkampf kann bei Kampfmanövern die Probe erleichtert werden, indem man freiwillig Schaden nimmt);
\item[Stufe 2:]Kampfstil 2(Ein weiteres Grund-Attribut darf gegen ein beliebiges Körperliches Attribut getauscht werden.);
\item[Stufe 3:]Einstecken 2(Im Faustkampf kann man beim Einstecken von physischen Schaden seinen Angriffswert bis zu seinem übernächsten Zug senken, um einen Teil des Schadens zu negieren);
\item[Stufe 4:]Kampfstil 3(Ein weiteres Grundattribut darf gegen ein beliebiges Geistiges Attribut getauscht werden);
\end{description}

\section{Jäger}
Das Gebiet eines Jägers ist die gemäßigte Wildnis, wo er mit Fallen und Fernkampfwaffen hinter dem Wild her ist, um dessen Felle und Fleisch zu erbeuten und zu verkaufen. Dabei nutzen sie ihre jahrelange Erfahrung, sowie ihre Instinkte, um der Beute immer einen Schritt voraus zu sein und um zu verhindern, selbst zum Gejagten zu werden.
\subsection{Voraussetzungen}
Volk: Jedes, Kultur: Jede, auch wenn Stadtbewohner die doppelte Menge an Arbeit in die Grundausbildung stecken müssen
\subsection{Fertigkeiten}
+:Schleichen, Verstecken, Überleben, Naturkunde, Tierkunde, Handwerk(Fallen), Orientieren, Fleischer, Gerber, Spuren lesen
-:Soziale Kategorie, Akademische Wissenfertigkeiten(Geschichte, Heraldik, Magie, Religion, etc.)
\subsection{Stufen}
\begin{description}
\item[Stufe 0:] Fokus(Schleichen, Pirschen); Fokus(Handwerk(Fallen), Schlingen), Feindklasse 1(Wild), Fokus(Spuren lesen, Fährtensuche), Ausschlachten(+3 Fleischer und Gerber)
\item[Stufe 1:] Geduld 1(Der Jäger kann stundenlang ohne Nahrung oder Trinken auf der Stelle verharren ohne Mali zu nehmen); Feindklasse 2(Wild), Schnelles Schleichen, Eiserne Miene(+3 Feilschen und Lügen), Improvisiertes Werkzeug(Fleischer und Gerber)
\item[Stufe 2:] Geduld 2(Ein Jäger riskiert sich keine Krankheiten durch schlechtes Wetter zuzuziehen und erhält nur den halben Malus auf Angriffe durch diese); Feindklasse 3(Wild), Jagdlager(Der Jäger kann aus einfachstes Mitteln ein gegen schwere Unwetter gefeites Lager errichten für bis zu 3 Personen errichten, dieses gewährt vollen Sichtschutz und verwendet den Verstecken-Wert des Jäger, sowie seinen Pirschen-Wert gegen Geruchswahrnehmung durch Wild), Improvisiertes Werkzeug(Fallen(Schlingen)), Jäger unter Jägern 1(Der Jäger hat gelernt selbst der Jagd durch stärkere Kreaturen zu entgehen und erhält einen Bonus von 6 auf Verstecken-Proben in der Wildnis), Spurlos.
\item[Stufe 3:] Geduld 3(Ein Jäger kann selbst in feindlichem Klima mit normalen Rationen auskommen und erhält keinen Zuschlag auf seine Erschöpfungsrate durch diese. Mali durch schlechtes Wetter wird vollständig ignoriert); Feindklasse 4(Wild), tierischer Gefährte(Der Jäger kann sich gegen einen kleinen Sagenpunktenzuschlag ein kleines Tier, meist einen Falken, Mader oder Wolf als Jagdgefährten besorgen), Trophäensammler(Fokus(Fleischer, Trophäen). Der Jäger kann auch ohne handwerkliches Hintergrundwissen wertvolle Bestandteile seiner Beute entnehmen), Stählerne Miene(+3 Feilschen und Lügen)
\item[Stufe 4:] Geduld 4(Gewaltmarsch durch unwirtliches Gelände oder bei schlechtem Wetter gibt keinen Zuschlag auf die Erschöpfung); Jäger unter Jägern 2(Der Jäger hat sich über die Zeit den Respekt animalischer Jäger verdient und wird daher niemals von solchen attackiert, natürlich nur wenn er es selbst nicht tut, und kann sich ihnen sogar bei einer Jagd anschließen, um einen Teil der Beute abzubekommen), Meisterlicher Schuss, Feidklasse 5(Wild), Nähe zum Tier(+3 Pirschen und Verstecken in der Wildnis)
\end{description}

\section{Söldner}
Neben den in den Kasernen oder speziellen Kampfschulen ausgebildeten Soldaten und Kämpfer sind Söldner Männer fürs Grobe. Ihre Ausbildung basiert mehr auf eigener Erfahrung in einfachen Kampfstilen und sie bedienen sich häufig unkonventioneller Kampftechniken. Sie werden meist als billige Schläger oder Leibwächter angeworben und sind dafür bekannt in der Regel nicht über ihren ursprünglichen Auftrag hinaus unnötig gewalttätig oder freundlich zu sein.
\subsection{Voraussetzung}
Volk: keine Elfen, Kultur: Jede
\subsection{Fertigkeiten}
+:Einschüchtern, einfache Kampfwaffen, Gassenwissen, Zechen
-: Etikette, Überreden, Überzeugen, Akademische Wissenfertigkeiten(Geschichte, Heraldik, Magie, Religion, etc.)
\subsection{Stufen}
\begin{description}
\item[Stufe 0:] Tapferkeit(Vorstufe zu heldenhafter Tapferkeit. Mit einer MUT oder KON Probe, kann 1 VP in 5 TP umgewandelt werden.); Kampftrick(Konstenloses Kampfmanvöer), Straßennase(+3 Einschüchtern, +3 Gassenwissen), 
\item[Stufe 1:] Dicker Schädel(+HP, + 3 Zechen); Defensive Taktiken,
\item[Stufe 2:] Stahlmagen(+HP, +3 gegen Übelkeit); Offensive Taktiken, Gefahrensinn 1(Ein Söldner verliert nicht seine Verteidigung wenn er flankiert wird, auch wenn er weiterhin flankiert werden kann), 
\item[Stufe 3:] Überlebender(+HP, +3 gegen kritische Treffer); Gefahrensinn 2(Ein Söldner erhält stets einen zweiten Versuch auf seine Wahrnehmungsproben, um einen getarnten Feind wahrzunehmen)
\item[Stufe 4:] Veteran(+HP, +3 gegen Furcht); Gefahrensinn 3(Ein Söldner nimmt jeden auf ihn gerichteten Angriff als unbestimmtes Kribbeln wahr, sodass ihm die Gelegenheit bleibt diesem auszuweichen)
\end{description}

\section{Gauner}
Gauner sind die finsteren Gestalten einer jeden Stadt. Sei es Erpressung, Brandstiftung oder Drohungen, Gauner sind für solch grobe Verbrechen genau die Richtigen. Sie verlassen sich nicht nur auf ihre Schlagfertigkeit, sondern vor allem auf hinterlistige Tricks und einen ''bleibenden'' Eindruck. Dazu bringen viele Gauner noch ein gewisses Maß an Ambitionen für Organisation mit, weshalb diese häufig die allgemeine Kriminalität einer Stadt organisieren und einzelne Spezialisten zusammenführen.
\subsection{Voraussetzungen}
Volk: Jedes, Kultur: Jede die organisiertes Verbrechen kennt.
\subsection{Fertigkeiten}
+: Gassenwissen, Lügen, Einschüchtern, Waffenloser Kampf, Improvisierte Waffen, Dolche, Keulen, Zechen
-: Wissen, Handwerk
\subsection{Stufen}
\begin{description}
\item[Stufe 0:] Nachhaltige Drohung 1(Einschüchterungen eines Gauners haben die doppelte Wirkdauer); Schmutzige Tricks, Straßenköter(+3 Zechen und Gassenwissen), Mimik(+3 Lügen und Einschüchtern) 
\item[Stufe 1:] Nachhaltige Drohung 2(Nach einem Erfolgreichem Einschüchterungsversuch sind alle Einschüchterungen gegen das Ziel um 3 erleichtert, bis ein Versuch fehlschlägt);
\item[Stufe 2:] Präsenz der Angst 1(Eingeschüchterte Figuren sind für die Dauer der Einschüchterung unfähig Pläne gegen den Gauner zu schmieden. Ihre Lügenversuche ihm gegenüber sind um 3 erschwert);
\item[Stufe 3:] Präsenz der Angst 2(Auch nach dem Auslaufen eines Einschüchterungseffekt bleiben die Effekte von der Vorstufe erhalten);
\item[Stufe 4:] Ewige Drohung(Ein Gauner kann seinen Einfluss auch durch seine Handlanger gelten machend. Ein Gauner überträgt seinen Bonus von Nachhaltige Drohung auf alle, die in seinem Namen handeln, oder ihr Opfer von diesem Fakt überzeugen können.)
\end{description}

\section{Schurke}
Ob als Beutelschneider, Einbrecher oder Attentäter. Schurken sind die Kriminellen für feinfühlige Aktionen und bevorzugen es in den Schatten hinter den Kulissen zu arbeiten. Wo ihre Gegenstücke die Gauner oft als Gruppen agieren, sind Schurken in der Regel Einzelgänger. Natürlich gibt es auch Gilden aus Schurken, die oft einen strengen Arbeitskodex verfolgen und beispielsweise Attentate links liegen lassen, aber diese können meist nicht gegen reine Spezialistengruppierungen bestehen. Dennoch sind Schurken aufgrund ihrer Vielseitigkeit eine essenzielle Ergänzung für alle Arten des organisierten Verbrechens.
\subsection{Voraussetzungen}
Volk: Jedes, Kultur: Jede die Verbrechen kennt.
\subsection{Fertigkeiten}
+: Gassenwissen, Schätzen, Lügen, Dolche, Boden, Wurfwaffen, Ringen, Zechen, Schlösser knacken, Feinmechanik, Schleichen, Verstecken, Klettern, Wahrnehmung, Gaukeln
-: Einschüchtern, Handwerk, Wissen, Alle übrigen Kampftalente
\subsection{Stufen}
\begin{description}
\item[Stufe 0:] Fokus(Verstecken, Mengen); Fokus(Feinmechanik, Fallen entschärfen), Hehler(+3 auf Schätzen und Feilschen), Taschenspieler(+3 Gaukeleien, +3 Gaukeleien(Taschendiebstahl)), Schatten(+3 Schleichen, +3 Verstecken)
\item[Stufe 1:] Mengengewandheit(+3 auf Verstecken in Mengen und keine Behinderung durch Mengen); Giftnutzung
\item[Stufe 2:] Pacour(+3 auf Klettern und Akrobatik)
\item[Stufe 3:] Wäsche(Wertvolles Diebesgut wird eher von gewöhnlichen Händlern abgenommen)
\end{description}



\chapter{Geschlossene Professionen}
Diese Professionen stehen entweder nur bestimmten Völkern zur Verfügung oder werden nur in entsprechenden Ausbildungsorten erlernt werden können und daher nicht als Startprofessionen gewählt werden können.

\section{dunkelelfischer Krieger}
Gefürchtet als erbarmungslose Jäger von allem mit Fleisch auf den Rippen, was ihnen begegnet. Unsichtbar, mit Schleudern, Bögen oder Speeren streifen sie über das Land, mal erlegen sie ihre Beute mit einem gezielten Schuss, mal fangen sie es auch nur ein, um ihren Jungen im Stamm die Kunst des Tötens am Lebendem Subjekt zu demonstrieren. Im jedem Fall sind sie als Gruppe ein Todesurteil, selbst für eine Übermacht.

\section{Freunde in der Fremde}
\subsection{Fertigkeiten}
+: Alles Soziale, Sprachen, 



\chapter{Weiterentwickelte Professionen}
Viele offene Professionen decken ein breites Aufgabenfeld ab, die folgenden Professionen dienen der Spezialisierung auf ein bestimmtes Aufgabenfeld. Um sie zu nutzten benötigt es daher eine gewisse Erfahrung mit der Grundprofession.

\section{Besitenjäger}
Nach Jahren in der Wildnis gehen viele ihrer Bewohner darin über sich nicht länger mit dem Wild zu messen und bei Gefahr durch einen größeren Jäger nicht den Kopf einzuziehen, sondern vielmehr diese Bestien zu jagen. Mit größeren Waffen, kreativen Fallen oder anderen unkonventionellen Methoden, die Erfahrung eines Bestienjägers sind von unschätzbarem Wert um selbst unverwundbar scheinende Geschöpfe zu Fall zu bringen.
\subsection{Voraussetzungen}
Stufe 2 Jäger
\subsection{Fertigkeiten}
+: Schleichen, Handwerk(Fallen), Bogen, Speer, Armbrust, Wildniskunde, Spuren lesen, 
-: Gesellschaft, Andere Waffentalente, Andere Handwerkstalente
\subsection{Stufen}
\begin{description}
\item[Stufe 0:] Schwachstellen 1(Durch längere Beobachtung werden für einen Bestienjäger Schwachstellen sichtbar, die ansonsten verborgen blieben.); Fokus(Handwerk Fallen, Großwild), Tröphensammler(Fokus(Schlachter, Tröphen)), Fokus(Spurenlesen, Wildnis)
\item[Stufe 1:] Schwachstellen 2()
\end{description}

\section{Kopfgeldjäger}

\section{Einbrecher}

\section{Attentäter}



\end{document}