\documentclass[a4paper,12pt,oneside]{book}
\usepackage[ngerman]{babel}
\usepackage[utf8]{inputenc}
\usepackage{imakeidx}
\usepackage[hypertexnames=false]{hyperref}
\usepackage[all]{hypcap}
\usepackage{nameref}
\usepackage{ulem}


\hypersetup{
	bookmarks=true,
    colorlinks,
    citecolor=black,
    filecolor=black,
    linkcolor=black,
    urlcolor=black
}

\title{Über die Staaten}
\author{Jordan Eichner}
\date{}
\setcounter{secnumdepth}{-2}
\setcounter{tocdepth}{1}

\begin{document}

\maketitle
\tableofcontents

\part{Sechsstein}

\chapter{Sommerfeld}
Es sollte bereits bekannt sein, dass Sommerfeld aktuell in drei sog. Baronien unterteilt ist. Die Bezeichnung der Baronie stammt hierbei noch aus den Tagen eines in den zweiten Manakriegen untergegangen Großreiches. Im Zuge des Wiederaufbaus etablierten sich aus den Gelehrten, Großhändlern und teilweise Klerikalen eine Schicht von Adeligen, aus deren Rängen schließlich drei Familien hervorgingen, welche nun als neue Barone über ihren Teil von Sommerfeld herrschen.

\section{Baronie Lethe}
Im Angesichts der Entdeckung von Kristallenergie sollte man denken, dass es Ingenieure sein sollten, die auf dem Thron von Höhenstolz sitzen sollten. Doch es sind die Nachfahren jenen Investors und Großhändler, der das Forschungsgeld zur Verfügung gestellt hatte, die dank ihres politischen Geschicks und guter Öffentlichkeitsarbeit sich den Weg zur Herrschaft ebneten. Sie sind auch seitdem Förderer von technologischen Entwicklungen und dieser Vorsprung ist es der ihnen Dominanz über ganz Sechsstein verschafft hat. Zum Glück für den Rest der Welt ist die Familie Lethe klug genug es nicht auf eine direkte Konfrontation anzusetzen, sofern sie nicht von der Gegenseite provoziert wurde, da es nur dem Geschäft und damit ihren Forschungen schaden würde. Außerdem sichert sich die Baronie durch ihre Freizügigkeit mit ihren Endprodukten, ein Monopol auf technische Entwicklungen, was ihre außenpolitische Position zusätzlich sichert. Innenpolitisch wurden alle Konkurrenten über die Jahre heruntergewirtschaftet und die Ränge des Adels durch die Staatstechniker ersetzt, welche sich aus den fähigsten Technikern rekrutieren, die ihre Arbeit in der Regel über persönliche Machtinteressen stellen. Die wenigen Ausnahmen spinnen zwar Intrigen, die aber vom Herrscherhaus eher belustigt abgewehrt werden und von ihm als Übungen für ihre Jüngsten gesehen werden, um sie auf ihre eigene Herrschaft vorzubereiten.

\section{Baronie Rohne}
Ähnlich wie die Herrschaft der Familie Rohne über ihre Baronie, ist auch ihre Geschichte sehr blutig. In den frühen Tagen des Adels, regierte man gemeinsam in etwas, was man wohl Demokratie nennen konnte, da sich keines der Häuser besonders hervorgetan hatte und daher auch keine Mehrheit finden konnte. Haus Rohne verwaltete zu diesem Zeitpunkt die nordöstliche Grenzregion, wo die Goldene Straße weit weg war, während die Bevölkerung ständig von Monstern aus dem verlorenen Land belästigt wurde. Verzweifelt und später auch frustriert über die miserable finanzielle Lage, die drohte das Haus zu zerstören, war der Familie jede Hilfe und jedes Mittel recht. Und so schlug man sofort an, als ein mächtiger Vampir unter der Maske eines Söldneranführers die Hand reichte. In den darauffolgenden Jahren korrumpierten die Einflüsterungen des Vampirs jedes Mitglied der Familie, bis er für sie bei einer Versammlung der Adeligen, selbige ermordete und damit den Thron für Haus Rohne ebnete. Um die Ordnung innerhalb der Bevölkerung und des Militär zu wahren, agierte das Mevial-Syndikat, welcher der Vampir aufgebaut und mit Seinesgleichen anführte, als Geheimpolizei, bis man den innenpolitischen Widerstand gebrochen hatte und man das Tagesgeschäft dem Haus überantwortete. Dessen Anhänger werden inzwischen von Kindesbeinen an mit den Einflüsterungen des Mevial-Syndikat gefüttert und ihr Geist ist bis in den Kern verdreht, sodass sie ganz im Sinne ihrer Unheiligen Berater handeln und das Volk unterdrücken, sowie dem Mevial-Syndikat freie Hand lassen, wenn sie sich von Zeit zu Zeit ein Blutopfer holen. Um nicht weiter den Verdacht auf sich zu lenken, werden Rohne alle Expansionsgedanken ausgetrieben, weshalb diese im Laufe der Zeit verarmende Region, Außenpolitisch nahezu keine Beachtung erhält, auch wenn die Kontrolle über einen Teil der Goldenen Straße viele Händler und vor allem Schmuggler auf Alternativrouten treibt.

\section{Baronie Hortens}
Wo heute die Baronie in viele kleine Stadtstaaten zersplittert ist, über die später Berichtet wird, stand eins das Haus Hortens, welches heute auch nur noch Asche ist. Schon immer war die Grenze zwischen Adel und dem Rest der Bevölkerung in dieser Baronie sehr dünn gewesen, da die Bewohner hier außergewöhnliche Startbedienungen, nahe der Goldenen Straße und von keiner Seite bedroht, durch irgendwelche Monster. Noch mehr als im nördlichen Rohne dominierten hier ein Parlamentarischer Verbund, der allerdings hunderte Mitglieder hatte. Erst nach einem Krieg mit den Hinterlande, tat sich Zidan von Hortens als Kriegsheld hervor und seine Familie wurde gemeinschaftlich zum Baron erhoben. Seine Herrschaft endete mit seinem Tod knapp zwei Jahre später und seinen Thron hinterließ er seinen 4 Nachkommen, die das Reich nach einem Streit über den letzten Willen unter sich aufteilten, zum Unmut aller. Letztere nahm über die Jahre zu, da sich keiner der neuen Herrscher, bis auf die jüngste Tochter Iurka, nicht besonders geschickt herausstellten. Schließlich wurde es den übrigen Adeligen und auch dem Militär zu bunt und man stürzte das Haus Hortens und tötete alle Mitglieder. Man sagt, dass lediglich die uneheliche Tochter Iurkas als Säugling überlebt haben soll und damit als letzte Recht auf die Herrschaft hätte. Nach dem Fall des Hauses, zerstritt man sich ebenfalls und die Baronie zerfiel in die bereits erwähnten Stadtstaaten, die aufgrund ihrer geringen Größe mehr mit sich selbst als mit Außenpolitik beschäftigt sind.
\subsection{Kaltweihe}
 

\end{document}