\documentclass[a4paper,12pt,oneside]{book}
\usepackage[ngerman]{babel}
\usepackage[utf8]{inputenc}
\usepackage{imakeidx}
\usepackage[hypertexnames=false]{hyperref}
\usepackage[all]{hypcap}
\usepackage{nameref}
\usepackage{ulem}


\hypersetup{
	bookmarks=true,
    colorlinks,
    citecolor=black,
    filecolor=black,
    linkcolor=black,
    urlcolor=black
}

\title{Götter - Regeltechnischer Kram}
\author{Jordan Eichner}
\date{}
\setcounter{secnumdepth}{-2}
\setcounter{tocdepth}{1}

\begin{document}

\maketitle
\tableofcontents

\part{Allgemeine}

\part{Aspekte}

\chapter{Aigis}
\section{Meditationen}
\begin{description}
\item[In Not geratene Beschützen]
Sei es durch Räuber oder Wölfe, wenn ein Geweihter einen Wanderer oder eine Karawane vor einem mehr oder weniger großem Übel beschützt wird dies Belohnt.
\item[Lebensband]
Der Geweihte bindet sein Leben an eine andere Kreatur, für deren Schutz er fortan zuständig ist. Stirbt sein Schützling eines nicht natürlichen Todes, so stirbt der Geweihte und der Schützling ersteht an einem sicherem Ort wieder auf. Der Schwur endet erst mit dem natürlichen Tod eines der Beiden und versorgt den Geweihten jeden Mond mit einer gewissen Menge Energie.
\item[Finales Opfer]
Im Angesichts der Entscheidung zwischen seinem Leben und dem vieler, kann ein Geweihter sein Leben über das der Übrigen stellen. Im Gegenzug für die völlige und endgültige Vernichtung seiner eigenen Existenz, erhält der Geweihte in seinen letzten Momenten eine gewaltige Menge an Energie.
\end{description}
\section{Liturgien}
\begin{description}
\item[Geteilter Schild]
Der Geweihte verschanzt sich hinter seinen Schild und spricht eine kurzes Gebet. Daraufhin erscheint ein Geisterhafter Schutzschirm aus seinem Schild, der die neben ihm stehenden und ihm deckt, als trügen sie einen ähnlichen Schild der nächsthöheren Größenkategorie. Würde es sich hierbei um einen Turmschild handeln, so erhalten alle eine volle Deckung. 
\item[Ausfall]
Mit einem Schlachtruf an Aigis stürmt der Geweihte getragen von übernatürliche Stärke an die Seite von Schutzbedürftigen und erzeugt durch eine Energiewoge ein kurzen Moment der Ruhe, indem er alle Feinde die eine Stärkeprobe nicht schaffen zurück wirft.
\item[Deny Death] Der Anhänger strahlt Lebensenergie aus, die sowohl Freund als auch Feind, welche im Sterben liegen, stabilisieren und im Reich der Lebenden halten.
\item[Fels in der Brandung]
Der Anhänger kanalisiert göttliche Energie in einen von ihm geführten Schild oder eine Barriere, welcher daraufhin an Stärke gewinnt und ihren Widerstand verdoppelt. Für zusätzliche Opfer kann der Schild Einschläge abfangen, die den Widerstand überwinden.
\end{description}


\chapter{Ixania}
 \section{Meditationen}
\begin{description}
\item[Morgentau fangen]
Wie der Name schon sagt, sammelt der Geweihte den Morgentau bei Sonnenaufgang in einem geeigneten Behältnis, anschließend wird er im Zuge eines kurzen Gebetes zu Ixania getrunken. Alternativ kann es zur Destillierung von Zwielicht verwendet werden oder zur späteren Vervollständigung der Weihe aufbewahrt werden. Morgentau muss dabei innerhalb eines Sonnenzyklus verbraucht werden.
\item[Segen der verbrannten Sonnen]
Dieses Ritual kann mit dem Untergehen der Sonne begonnen werden. Zunächst wird ein Feuer entzündet, dabei ist nur frischestes Holz zu verwenden. Sobald die Mitte der Nacht vorüber gezogen ist, kann man mit der Phase zwei beginnen. Die reine Asche, ohne Reste von Ruß oder verkohltem Holz, muss mit Wasser aus einer unterirdischen Quelle vermengt werden, bis eine silbrige Paste entsteht. Diese wird anschließend in einer Silberschüssel ausgestrichen und ins Mondlicht gehalten. Nach einer Stunde wird die Paste im Zuge eines Gebetes auf der Haut des Geweihten verrieben, worauf sich die Asche zu Ruß wandelt, den man erst nach Sonnenaufgang abwaschen darf. Alternativ kann man die Paste, sofern man sie nicht dem Sonnenlicht aussetzt für einen Mondzyklus aufbewahrt werden, um sie später zu benutzen oder sie mit geweihtem Morgentau zu Zwielicht zu destillieren.
\end{description}
\section{Liturigien}
\begin{description}
\item[Zweite Sonne/Mond]Der Geweihte kann durch ein kurzes Gebet jederzeit künstliches Sonnen-, bzw. Mondlicht erschaffen, welches seinen Händen entspringt. Im Regelfall handelt es sich dabei um ein schwaches Glühen, welches höchstens 3m weit reicht. Durch Falten der Hände zu einer Einheit kann der Geweihte das Licht zusätzlich in einen Strahl bündeln oder die Intensität und damit Reichweite erhöhen. Letzteres ist dabei mit einem höheren Energieaufwand verbunden. Während so geschaffenes Sonnenlicht, beispielsweise für Pflanzenwachstum ausreichend ist, kann es nicht als Werkzeug für magische oder göttliche Rituale verwendet werden, sofern nicht ausdrücklich erwähnt ist.
\item[Schattensinn]Nach einer kurzen Meditation sendet der Geweihte in einem Radius von mehreren Meilen einen schwachen, bei Sonnenlicht nur als Blitzen wahrnehmbaren, alles durchdringenden Lichtpuls aus, der alle erfassten Objekte anhand ihres Schatten lokalisiert. Zwar verfallen ein Großteil dieser Informationen innerhalb kurzer Zeit(ca. 5 Minuten) wieder, jedoch können die schattenlosen Vampire, bzw. ihr düstere Einfluss in Form ihres Fluches festgestellt werden können. Ähnliches gilt für alle Formen von schattenaffinen Präsenzen. Auf der anderen Seite bemerken diese in der Regel das Licht.
\item[Licht der Welt]Auf geweihtem Öl entzündet der Geweihte eine kalte Flamme, die abhängig von der Tageszeit, gelb wie die Sonne oder bläulich wie der Mond scheint. Es brennt, sofern es nicht durch einen Geweihten, künstliche Dunkelheit oder vergießen des Öls gelöscht wird, ewig weiter, ohne Material zu verbrauchen. Unter weiterem Kraftaufwand kann die Flamme kurzzeitig zu einer Flammensäule auflodern, dabei wird alles in ihr Restlos von niederen Einflüssen, wie dem vampirischen Gift, nekromantischen, sowie allen entweder rein der Dunklen oder Hellen Seite zugerechneten Effekten, gereinigt. Dieses Säuberungsritual kann unter Umständen tödlich enden, falls zu viel vom Körper/Seele korrumpiert wurde, weshalb es vor allem für die Erlösung von Leidenden verwendet wird. Ansonsten ist es eine geeignete stationäre Lichtquelle.
\end{description}

\chapter{Tholemäus}
\section{Meditationen}
\begin{description}
\item[Wissen konservieren]
Die wichtigste Aufgabe eines Anhängers ist das Erhalt alten Wissens. Sie betreiben zu diesem Zweck viel Selbststudium in den großen Bibliotheken der Welt, wo sie Tage, wenn nicht Wochen mit der Kopie von Büchern beschäftigt sind.
\item[Geheimnis lüften]
Während sich ein Anhänger mit der Notation neuem Wissen beschäftigt, geht neben den Anima auch ein paar Informationen an Tholemäus über.
\item[Wissen aufarbeiten]
Nicht immer haben außenstehende die Zeit oder das Interesse sich durch Regale von Büchern zu wälzen. Die Anhänger sehen es daher als ihre Aufgabe Wissen entweder in Form eines Vortrages oder als Publikation in für interessierte geeignete Häppchen aufzuteilen.
\end{description}
\section{Liturgien}
\begin{description}
\item[Roter Faden]
Trotz ihrer Jahrelangen Arbeit in Bibliotheken ist es für den Geweihten nicht immer möglich, sich an jedes Detail zu erinnern oder sie stehen vor der Aufgabe in einer unbekannten oder gar unsortieren Sammlung von Texten nach einer Information zu suchen. Mithilfe eine roten Bandes, ähnlichem einem Lesezeichen kann ein Geweihter nach dem rituellen Stellen seiner Frage dem Band zu dem passendem Buch folgen, wo das Band an die passende Stelle schlüpft. Ist das gesuchte Wissen nicht vorhanden reagiert das Band einfach nicht.
\item[Gedächtnissprung]
Tholemäus nimmt im Zuge einer Weihe nicht nur Energie, sondern auch einen Teil des verarbeiteten Wissens auf. Bruchstücke selbigen können im Zuge eines Gebetes und anschließender Meditation abgerufen werden, auch wenn es sich meisten nur auf Hinweise, wo die Antwort liegen kann. Nur bei besonders einfache Fragen, wie Informationen über ein bereits bekannte Spezies sind in der Regel klarer. Dieses Ritual erlaubt auch das Wiederholen einer Wissensprobe.
\end{description}


\chapter{Haliya}
\section{Meditationen}
\begin{description}
\item[Tierbegräbnis]
Jedem gestorbenem Tier steht auch im Tod die Gnade eines Begräbnisses zu. Zwar darf Fleisch, Haut und anderes Gewebe verwendet werden, doch die Knochen seien der Erde in einer kleinen Zeremonie zu überantworten.
\item[Asche für das Land]
Der Anhänger häuft zusammen mit anderen einen Berg aus nicht verwendeten Schnittgut an und entzünden diesen schließlich bei Sonnenuntergang, während alle ein Gebet zur Haliya spricht. Am nächsten Morgen muss die Asche unter einem erneuten Gebet über die Umliegenden Felder verteilt werden. Dieser Brauch wird vor allem zu Frühlingsbeginn von ganzen Dörfern gemeinschaftlich zelebriert.
\item[Erntekranz] 
Nach Einfuhr der Ernte wird von jedem Feld und jedem Garten, den man selbst mitbewirtschaftet hat eine Pflanze(z.B. bei Getreide) oder ein Ast, entnommen und diese zu einem Ring, sofern möglich zusammengelegt. Diesen Kranz legt man anschließend zur freien Verfügung aus.
\\Auch diese Meditation führt häufig von gesamten Dorfgemeinschaften zelebriert, wobei man die Kränze entweder austauscht oder zusammenträgt und ein gemeinsames Mal abhält, an dem auch Tiere beteiligt werden.
\end{description}
\section{Liturgien}
\begin{description}
\item[Weihe von Acker und Ernte]
Durch das Ziehen eines Schutzkreises kann ein Feld oder deren Früchte vor Ungeziefer geschützt werden. Dieser Schutz hält für einen Zyklus oder auf dem Feld ein Frevel geschieht.
\item[Sommerwind]
Der Anhänger hält eine kurze Ansprache an Haliya und macht danach eine Tiefen Atemzug. Beim Ausatmen entsteht um ihn ein warmer Sommerwind, der für die Dauer der Intervention alle im Umfeld das Anhänger vor dem Kältetod gefeit sind. Die Brise hält selbst einem Sturmwind stand.
\end{description}

\chapter{Umbra}
\section{Meditationen}
\begin{description}
\item[Tribut der Beute]
Nach einem erfolgreichem Raubzug, Betrügerei spendet der Anhänger einen Teil seiner Beute an einen Schrein andere Aspekte, mit Ausnahme der von Dinaris.
\item[Zeichen der Lilie]
Der Anhänger beweist sich, indem er unbemerkt in verbotenes Territorium eindringt und dort ein kleines Andenken in Form einer Lilie da lässt, dieses kann auch nur aus einer Skizze der Blüte bestehen.
\item[Wettstreit der Schatten]
Zwei sich begegnende Anhänger Umbras führen einen formellen Wettstreit, dessen Natur sie durch den Geheimcode Umbras vereinbaren. Ein Beispielhafter Wettkampf könnte die Jagd nach einer Trophäe für einen Tribut an Umbra sein.
\end{description}
\section{Liturigen}
\begin{description}
\item[Geleit von Nacht und Nebel]
Der Anhänger bedeckt sein Gesicht entweder mit Asche oder einer Maske, während er ein kurzes Gebet spricht. Anschließend scheint er mit Dunkelheit oder Nebelschwaden zu verschmelzen, was ihn nahezu unsichtbar macht.
\item[Sternenschlüssel]
Der Anhänger deutet auf die Schließmechanismen in einem Gegenstand oder einer Struktur, während er ein Gebet spricht. Anschließend materialisieren sich an den aufgezeigten Stellen passende Schlüssel oder womit auch immer die Mechanismen interagieren und entriegeln, sofern der Anhänger keine Mechanismen vergessen hat, alles. Auf dieses Ritual folgt in der Regel ein Tribut der Beute für den Inhalt, der jedoch keine Gunst einbringt.
\end{description}

\chapter{Dinaris}
\section{Meditationen} 
\begin{description}
\item[Buchführung]Der Anhänger zählt sein Geld und erstellt eine exakte Buchführung, die für ein Jahr aufbewahrt werden muss.
\end{description}


\part{Gegenwelt Pantheon}

\chapter{Eron}
\section{Liturigen}
\begin{description}
\item[Ruf nach Gerechtigkeit:]
Nach der Gerechtigkeit durch Eron oder seiner Engel zu rufen, ist eine gefährliche Sache, da man entweder selbst Opfer des Urteils wird oder einfach bestraft wird die Zeit und Energie für Nichtigkeiten verschwendet zu haben. Man kann jedoch einen Gefallen bei Eron einfordern, sodass man zumindest für letzteres nicht bestraft wird und sich die Engel wirklich Zeit für den Fall nehmen. 
\item[Hoher Eidsegen:]
Ein Schwur, welcher unter diesem Segen abgelegt wurde, bindet beide Seiten bis ans Ende aller Tage und führt beim Bruch zum Tod.
\item[Pflege der Gerechtigkeit:]
Mit einem Fingerzeig wird die angezeigte Kreatur von einem flammenden Peitschenhieb getroffen, der eine verfluchte Narbe hinterlässt. Engel können dieses Ritual mit ihrer Peitsche ausführen.
\item[Mantel des Henkers]
Der Betende erhält eine Panzerung, die seine Haut für kurze Zeit Hart wie Stahl werden lässt.
\end{description}



\part{Übergelaufende Götter}



\end{document}