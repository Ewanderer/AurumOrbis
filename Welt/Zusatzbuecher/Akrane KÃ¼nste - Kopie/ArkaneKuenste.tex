\documentclass[a4paper,12pt,oneside]{book}
\usepackage[ngerman]{babel}
\usepackage[utf8]{inputenc}
\usepackage{imakeidx}
\usepackage[hypertexnames=false]{hyperref}
\usepackage[all]{hypcap}
\usepackage{nameref}
\usepackage{ulem}


\hypersetup{
	bookmarks=true,
    colorlinks,
    citecolor=black,
    filecolor=black,
    linkcolor=black,
    urlcolor=black
}

\title{Aurum Orbis - Arkane Künste}
\author{Jordan Eichner, Christoph Schmidt}
\date{}
\setcounter{secnumdepth}{-1}
\setcounter{tocdepth}{10}

\begin{document}

\maketitle
\tableofcontents

\part{Grundlagen}

\chapter{Komponente}
\begin{description}
\item[S]omatisch
\\Hierzu zählen Handbewegungen, weshalb beim Zaubern mindestens eine Hand frei bleiben muss. S* bedeutet, dass eigentlich zwei Hände benötigt werden. Einhändig solche Zauber auszuführen erschwert die Zauberprobe um 2.
\item[G]eistig
\\Der Magier muss in den Zauber mentales Input, wie Gedankenbilder oder dauerhafte Konzentration. Der Prozess des Zaubers kann daher durch mentale Ablenkung abgebrochen oder verfälscht wurde. 
\end{description}


\part{Runen}

\chapter{Elementare}
Alle Runen aus dieser Kategorie sind für sich schon einfache Zauber, erfordern sofern nicht deklariert keine Zauberprobe und sind Magiern aller Schulen und Zirkel bekannt. Eine Sammlung von Elementaren wird zur Definition von komplexen Objekten verwendet werden. 

\begin{description}
\item[Lux,] Lucis; 1M/h; Komponente: -
\\Die Hand des Magiers wird in einen Lichtschein mit der Helligkeit einer Fackel gehüllt. Durch Berührung kann dieser Lichtschein auf ein berühtes Objekt übertragen werden. Der Magier kann die Farbe bei Erschaffung anpassen.
\item[Ignis,] Ignis; 5M; Komponente: S
\\Durch einen Fingerzeig wird ein Objekt mit der Brennbarkeit von trockenen Holzscheiten entflammt.
\item[Aqua,] Aquae; 3M; Komponente: S*
\\In einer Handschale sammelt der Magier Feuchtigkeit aus der Umgebung. Das Wasser kann zwischen 5 und 20 Grad Celsius Temperatur haben, je nach Wunsch des Magiers,
\item[Ventus,] Venti; 3M; Komponente: S
\\Kurzzeitig wird die Luft um den Magier gebündelt und anschließend als Böhe entlang seines ausgestreckten Armes ausgesendet. Dieser Wind zerfällt über den Verlauf von 10 Fuß und kann dabei nur Objekte bis zu einem Gewicht von 1kg aufnehmen.
\item[Nox,] Noctis; 5M/min/10m Radius; Komponente: -
\\Schlagartig werden in der vom Magier deklarierten Umgebung Licht gedämmt bis hin zur absoluten Dunkelheit. Sonnenlicht verdoppelt den Energieaufwand.
\end{description}

\chapter{Hilfswörter}
Diese teilweise immer noch grundlegenden Begriffe sind für sich keine alleinstehenden Zauber.
\begin{description}
\item[Imago,] Imaginis; Komponente: S, G
\\Ein geistiges Bildnis oder die Details einer Realen Vorlage werden auf das gekoppelte Element projiziert.
\item[Globus,] Globi; Komponente: S*
\\Diese Rune bannt einen Zauber in eine Kugel die durch Gesten in Größe und Position angepasst werden kann. Die Zauberkraft wird dabei über die gesamte Kugel verteilt.

\end{description}

\chapter{Beschreibungswörter}
Hilft bei der Zuordnung von Eigenschaften zu magischen Konstrukten.
\begin{description}
\item[Asper,]Asperis;
\\Abweisung, Vertreibung.
\end{description}

\chapter{Zielwörter}
Werden benutzt um die Ziele eines Zaubers zu spezifizieren, wie auch die Hilfswörter sind diese Runden keine eigenen Zauber.
\begin{description}
\item[Soma,]Somatis;
Bund von Fleisch, Knochen, Blut, Haut einer jeden Lebensform.
\end{description}


\part{Zauber}

\chapter{Eigenschaften}
\begin{}
\\Kosten: Repräsentiert die Manakosten, berechnet sich aus der Effekt-Komponente * (Anzahl der Ziele/Größenmodifikator der Ziele)
\\ 

\chapter{Regeln zur Komplexität}

\section{Zielen}
Berührung - Kein Aufschlag zur Zauberzeit, Zauberprobe um 4 erleichtert.
\\Fingerzeig - Kein Aufschlag zur Zauberzeit, Zauberprobe um 2 erleichtert.
\\Zielsuchender Fingerzeit - Keine Modifikation.
\\Areal - 1 



\end{document}